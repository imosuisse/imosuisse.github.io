Let $n \in \mathbb{N}$, $n \geq 2$ and let $t_1,t_2,\ldots,t_k$ be different divisors of $n$.
An identity of the form $n=t_1+t_2+ \ldots +t_k$ is called a \emph{representation} of $n$ as a
sum of different divisors. Two such representations are considered equal if they differ only in
the order of the summands (for example, $20=10+5+4+1$ and $20=5+1+10+4$ are twice the
same representation of $20$ as the sum of different divisors).
Let $a(n)$ be the number of different representations of $n$ as the sum of different divisors.
Determine whether there is an $M \in \mathbb{N}$ with $a(n)\leq M$ for all $n \in \mathbb{N}$, $n\geq 2$.
