If $p\geq 5$ is a prime number, let $q$ denote the smallest prime number such that
$q>p$ and let $n$ be the number of positive divisors of $p+q$ ($1$ and $p+q$ included).
\begin{enumerate}[a)]
    \item Prove that no matter the choice of $p$, the number $n$ is always at least $4$.
    \item Find the actual minimal value $m$ that $n$ can reach among all possible choices for $p$. That is:
    \begin{itemize}
        \item Give an example of a prime number $p$ for which the value $m$ is reached.
        \item Prove that there is no prime number $p$ for which $n$ is smaller than $m$.
    \end{itemize}
\end{enumerate}
