\documentclass[11pt,a4paper]{article}

\usepackage[italian]{babel}
\usepackage[T1]{fontenc}
\usepackage[latin1]{inputenc} 

\usepackage{amsfonts}
\usepackage{amsmath}
\usepackage{amsthm}
\usepackage{amssymb}

\oddsidemargin=10pt
\textwidth=15cm

\begin{document}

\pagestyle{empty}

\begin{center}
{\huge OSM - Turno preliminare} \\
\medskip Lugano, Losanna, Zurigo - 12 gennaio 2013
\end{center}
\vspace{8mm}
Durata: 3 ore\\
Ogni esercizio vale 7 punti.

\vspace{15mm}

\begin{enumerate}

\item[\textbf{1.}] 
Un gruppo di $2013$ persone siedono uniformemente distribuiti ad un tavolo rotondo.
Dopo che si sono seduti si accorgono che ad ogni piatto vi � collocato un cartellino con nome e nessun commensale � seduto ad un posto con cartellino indicante il suo nome. Dimostrare che si pu� girare il tavolo in modo che ci siano almeno due persone con il loro nome scritto davanti. 

\bigskip

\item[\textbf{2.}] 
Siano $M_1$ ed $M_2$ i centri di due cerchi, rispettivamente $k_1$ e $k_2$. Supponiamo che i due cerchi si taglino perpendicolarmente in un punto $P$, sia inoltre $Q$ l'intersezione di $k_1$ con $M_1M_2$. Dimostrare che l'intersezione della perpendicolare al segmento $M_1M_2$ passante per $M_2$ e della retta retta $PQ$ si trova su $k_2$.

\bigskip

\item[\textbf{3.}]
Diciamo che un numero � "simpatico" se le cifre della sua rappresentazione decimale soddisfano le seguenti condizioni:
\begin{enumerate}
\item[a)] Ogni cifra $0,1,\ldots, 9$ appare al massimo una volta.
\item[b)] Sia  $A$ una cifra pari e  $B$ una cifra dispari, allora tra $A$ e $B$ troviamo esattamente $\frac{A+B-1}{2}$ altre cifre.
\end{enumerate}
Si trovi il numero di numeri simpatici.

\bigskip

\item[\textbf{4.}] 
Trova tutte le paia $(m,n)$ di numeri naturali che soddisfano:
\[(m+1)! + (n+1)! = m^2n^2\]

\bigskip

\item[\textbf{5.}] 
Trova il pi� piccolo numero naturale  $n$, per il quale ogni sottoinsieme $S$ di $n$ elementi di $\left\{1,2,\ldots,100\right\}$ contiene almeno un numero scrivibile come somma di 3 altri elementi di $S$.

\bigskip

\end{enumerate}

\vspace{1.5cm}

\center{Buon lavoro!}
\end{document}
