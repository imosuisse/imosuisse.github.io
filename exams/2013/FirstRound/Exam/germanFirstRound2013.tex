\documentclass[11pt,a4paper]{article}

\usepackage[german]{babel}
\usepackage[T1]{fontenc}
\usepackage[latin1]{inputenc} 

\usepackage{amsfonts}
\usepackage{amsmath}
\usepackage{amsthm}
\usepackage{amssymb}

\oddsidemargin=10pt
\textwidth=15cm

\begin{document}

\pagestyle{empty}

\begin{center}
{\huge SMO - Vorrunde} \\
\medskip Z�rich, Lausanne, Lugano - 12. Januar 2013
\end{center}
\vspace{8mm}
Zeit: 3 Stunden\\
Jede Aufgabe ist 7 Punkte wert.

\vspace{15mm}

\begin{enumerate}

\item[\textbf{1.}] 
Eine Gruppe von $2013$ Leuten setzt sich gleichm�ssig verteilt an einen runden Tisch. Nachdem sie sich hingesetzt haben, bemerken sie, dass an jedem Platz ein Namens\-schild steht und dass sich niemand an den Platz mit seinem Namen gesetzt hat. Zeige, dass sie den Tisch so drehen k�nnen, dass mindestens zwei Personen das richtige Namens\-schild vor sich haben.

\bigskip

\item[\textbf{2.}] 
Seien $M_1$ und $M_2$ die Mittelpunkte der Kreise $k_1$ resp. $k_2$, welche sich im Punkt $P$ senkrecht schneiden. Ferner schneide $k_1$ die Strecke $M_1M_2$ in $Q$.
Zeige, dass sich die Senkrechte zur Strecke $M_1M_2$ durch den Punkt $M_2$ und die Gerade $PQ$ auf $k_2$ schneiden.


\bigskip

\item[\textbf{3.}]
Wir nennen eine nat�rliche Zahl sympathisch, falls die Ziffern ihrer Dezimaldarstellung die folgenden beiden Bedingungen erf�llen:
\begin{enumerate}
\item[a)] Jede der Ziffern $0,1,\ldots, 9$ kommt h�chstens einmal vor.
\item[b)] Ist $A$ eine gerade und $B$ eine ungerade Ziffer, so liegen genau $\frac{A+B-1}{2}$ andere Ziffern zwischen $A$ und $B$.
\end{enumerate}
Bestimme die Anzahl sympathischer Zahlen.

\bigskip

\item[\textbf{4.}] 
Finde alle Paare $(m,n)$ nat�rlicher Zahlen, f�r die gilt:
\[(m+1)! + (n+1)! = m^2n^2\]

\bigskip

\item[\textbf{5.}] 
Bestimme die kleinste nat�rliche Zahl $n$, sodass jede $n$-elementige Teilmenge $S$ von $\left\{1,2,\ldots,100\right\}$ mindestens eine Zahl enth�lt, welche sich als Summe von drei anderen, verschiedenen Elementen aus $S$ schreiben l�sst.

\bigskip

\end{enumerate}

\vspace{1.5cm}

\center{Viel Gl�ck!}
\end{document}
