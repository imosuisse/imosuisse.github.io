\documentclass[11pt,a4paper]{article}

\usepackage[utf8]{inputenc}
\usepackage[french]{babel}

\usepackage{amsfonts}
\usepackage{amsmath}
\usepackage{amsthm}
\usepackage{amssymb}

\oddsidemargin=10pt
\textwidth=15cm


\begin{document}

\pagestyle{empty}

\begin{center}
{\huge OSM - Examen préliminaire} \\
\medskip Lausanne, Lugano, Zurich - le 12 janvier 2013
\end{center}
\vspace{8mm}
Durée: 3 heures\\
Chaque exercice vaut 7 points.

\vspace{15mm}

\begin{enumerate}

\item[\textbf{1.}] 
Un groupe de $2013$ personnes s'assied autour d'une table ronde, en se répartissant de manière régulière. Une fois assises, ces personnes constatent qu'un carton indiquant un nom est posé à chacune des places et que personne ne s'est assis à la place où son nom figure. Montrer qu'elles peuvent tourner la table de sorte qu'au moins deux personnes se retrouvent avec le carton correct devant elles.

\bigskip

\item[\textbf{2.}] 
Soit $M_1$ et $M_2$ les centres de deux cercles $k_1$ et $k_2$ respectivement. Supposons que les deux cercles se coupent de manière perpendiculaire en un point $P$. De plus soit $Q$ l'intersection de $k_1$ avec le segment $M_1M_2$. Montrer que l'intersection de la perpendiculaire au segment $M_1M_2$ passant par le point $M_2$ et de la droite $PQ$ se trouve sur le cercle $k_2$.

\bigskip

\item[\textbf{3.}]
Un nombre naturel est appelé sympathique si les chiffres de sa représentation dans le système décimal satisfont les deux conditions suivantes:
\begin{enumerate}
\item[a)] Chacun des chiffres $0,1,\ldots, 9$ apparait au plus une fois.
\item[b)] Si $A$ est un chiffre pair et $B$ est un chiffre impair, alors il y a exactement $\frac{A+B-1}{2}$ autres chiffres entre $A$ et $B$.
\end{enumerate}
Combien y a-t-il de nombres sympathiques?

\bigskip

\item[\textbf{4.}] 
Déterminer toutes les paires $(m,n)$ de nombres naturels satisfaisant
\[(m+1)! + (n+1)! = m^2n^2.\]

\bigskip


\item[\textbf{5.}] 
Trouver le plus petit nombre naturel $n$ satisfaisant la condition suivante: chaque sous-ensemble $S$ à $n$ éléments de l'ensemble $\left\{1,2,\ldots,100\right\}$ contient au moins un nombre qui est la somme de trois autres éléments distincts de $S$.

\bigskip

\end{enumerate}


\vspace{1.5cm} 

\center{Bonne chance!}
\end{document}
