\documentclass[11pt,a4paper]{article}

\usepackage{amsfonts}
\usepackage[latin1]{inputenc}
\usepackage[centertags]{amsmath}
\usepackage{amsthm}
\usepackage{amssymb}

\leftmargin=0pt \topmargin=0pt \headheight=0in \headsep=0in \oddsidemargin=0pt \textwidth=6.5in
\textheight=8.5in






% Schriftabk�rzungen

\newcommand{\eps}{\varepsilon}
\renewcommand{\phi}{\varphi}
\newcommand{\Sl}{\ell}    % sch�nes l
\newcommand{\ve}{\varepsilon}  %Epsilon

\newcommand{\BA}{{\mathbb{A}}}
\newcommand{\BB}{{\mathbb{B}}}
\newcommand{\BC}{{\mathbb{C}}}
\newcommand{\BD}{{\mathbb{D}}}
\newcommand{\BE}{{\mathbb{E}}}
\newcommand{\BF}{{\mathbb{F}}}
\newcommand{\BG}{{\mathbb{G}}}
\newcommand{\BH}{{\mathbb{H}}}
\newcommand{\BI}{{\mathbb{I}}}
\newcommand{\BJ}{{\mathbb{J}}}
\newcommand{\BK}{{\mathbb{K}}}
\newcommand{\BL}{{\mathbb{L}}}
\newcommand{\BM}{{\mathbb{M}}}
\newcommand{\BN}{{\mathbb{N}}}
\newcommand{\BO}{{\mathbb{O}}}
\newcommand{\BP}{{\mathbb{P}}}
\newcommand{\BQ}{{\mathbb{Q}}}
\newcommand{\BR}{{\mathbb{R}}}
\newcommand{\BS}{{\mathbb{S}}}
\newcommand{\BT}{{\mathbb{T}}}
\newcommand{\BU}{{\mathbb{U}}}
\newcommand{\BV}{{\mathbb{V}}}
\newcommand{\BW}{{\mathbb{W}}}
\newcommand{\BX}{{\mathbb{X}}}
\newcommand{\BY}{{\mathbb{Y}}}
\newcommand{\BZ}{{\mathbb{Z}}}

\newcommand{\Fa}{{\mathfrak{a}}}
\newcommand{\Fb}{{\mathfrak{b}}}
\newcommand{\Fc}{{\mathfrak{c}}}
\newcommand{\Fd}{{\mathfrak{d}}}
\newcommand{\Fe}{{\mathfrak{e}}}
\newcommand{\Ff}{{\mathfrak{f}}}
\newcommand{\Fg}{{\mathfrak{g}}}
\newcommand{\Fh}{{\mathfrak{h}}}
\newcommand{\Fi}{{\mathfrak{i}}}
\newcommand{\Fj}{{\mathfrak{j}}}
\newcommand{\Fk}{{\mathfrak{k}}}
\newcommand{\Fl}{{\mathfrak{l}}}
\newcommand{\Fm}{{\mathfrak{m}}}
\newcommand{\Fn}{{\mathfrak{n}}}
\newcommand{\Fo}{{\mathfrak{o}}}
\newcommand{\Fp}{{\mathfrak{p}}}
\newcommand{\Fq}{{\mathfrak{q}}}
\newcommand{\Fr}{{\mathfrak{r}}}
\newcommand{\Fs}{{\mathfrak{s}}}
\newcommand{\Ft}{{\mathfrak{t}}}
\newcommand{\Fu}{{\mathfrak{u}}}
\newcommand{\Fv}{{\mathfrak{v}}}
\newcommand{\Fw}{{\mathfrak{w}}}
\newcommand{\Fx}{{\mathfrak{x}}}
\newcommand{\Fy}{{\mathfrak{y}}}
\newcommand{\Fz}{{\mathfrak{z}}}

\newcommand{\FA}{{\mathfrak{A}}}
\newcommand{\FB}{{\mathfrak{B}}}
\newcommand{\FC}{{\mathfrak{C}}}
\newcommand{\FD}{{\mathfrak{D}}}
\newcommand{\FE}{{\mathfrak{E}}}
\newcommand{\FF}{{\mathfrak{F}}}
\newcommand{\FG}{{\mathfrak{G}}}
\newcommand{\FH}{{\mathfrak{H}}}
\newcommand{\FI}{{\mathfrak{I}}}
\newcommand{\FJ}{{\mathfrak{J}}}
\newcommand{\FK}{{\mathfrak{K}}}
\newcommand{\FL}{{\mathfrak{L}}}
\newcommand{\FM}{{\mathfrak{M}}}
\newcommand{\FN}{{\mathfrak{N}}}
\newcommand{\FO}{{\mathfrak{O}}}
\newcommand{\FP}{{\mathfrak{P}}}
\newcommand{\FQ}{{\mathfrak{Q}}}
\newcommand{\FR}{{\mathfrak{R}}}
\newcommand{\FS}{{\mathfrak{S}}}
\newcommand{\FT}{{\mathfrak{T}}}
\newcommand{\FU}{{\mathfrak{U}}}
\newcommand{\FV}{{\mathfrak{V}}}
\newcommand{\FW}{{\mathfrak{W}}}
\newcommand{\FX}{{\mathfrak{X}}}
\newcommand{\FY}{{\mathfrak{Y}}}
\newcommand{\FZ}{{\mathfrak{Z}}}

\newcommand{\CA}{{\cal A}}
\newcommand{\CB}{{\cal B}}
\newcommand{\CC}{{\cal C}}
\newcommand{\CD}{{\cal D}}
\newcommand{\CE}{{\cal E}}
\newcommand{\CF}{{\cal F}}
\newcommand{\CG}{{\cal G}}
\newcommand{\CH}{{\cal H}}
\newcommand{\CI}{{\cal I}}
\newcommand{\CJ}{{\cal J}}
\newcommand{\CK}{{\cal K}}
\newcommand{\CL}{{\cal L}}
\newcommand{\CM}{{\cal M}}
\newcommand{\CN}{{\cal N}}
\newcommand{\CO}{{\cal O}}
\newcommand{\CP}{{\cal P}}
\newcommand{\CQ}{{\cal Q}}
\newcommand{\CR}{{\cal R}}
\newcommand{\CS}{{\cal S}}
\newcommand{\CT}{{\cal T}}
\newcommand{\CU}{{\cal U}}
\newcommand{\CV}{{\cal V}}
\newcommand{\CW}{{\cal W}}
\newcommand{\CX}{{\cal X}}
\newcommand{\CY}{{\cal Y}}
\newcommand{\CZ}{{\cal Z}}

% Theorem Stil

\theoremstyle{plain}
\newtheorem{lem}{Lemma}
\newtheorem{Satz}[lem]{Satz}

\theoremstyle{definition}
\newtheorem{defn}{Definition}[section]

\theoremstyle{remark}
\newtheorem{bem}{Bemerkung}    %[section]



\newcommand{\card}{\mathop{\rm card}\nolimits}
\newcommand{\Sets}{((Sets))}
\newcommand{\id}{{\rm id}}
\newcommand{\supp}{\mathop{\rm Supp}\nolimits}

\newcommand{\ord}{\mathop{\rm ord}\nolimits}
\renewcommand{\mod}{\mathop{\rm mod}\nolimits}
\newcommand{\sign}{\mathop{\rm sign}\nolimits}
\newcommand{\ggT}{\mathop{\rm ggT}\nolimits}
\newcommand{\kgV}{\mathop{\rm kgV}\nolimits}
\renewcommand{\div}{\, | \,}
\newcommand{\notdiv}{\mathopen{\mathchoice
             {\not{|}\,}
             {\not{|}\,}
             {\!\not{\:|}}
             {\not{|}}
             }}

\newcommand{\im}{\mathop{{\rm Im}}\nolimits}
\newcommand{\coim}{\mathop{{\rm coim}}\nolimits}
\newcommand{\coker}{\mathop{\rm Coker}\nolimits}
\renewcommand{\ker}{\mathop{\rm Ker}\nolimits}

\newcommand{\pRang}{\mathop{p{\rm -Rang}}\nolimits}
\newcommand{\End}{\mathop{\rm End}\nolimits}
\newcommand{\Hom}{\mathop{\rm Hom}\nolimits}
\newcommand{\Isom}{\mathop{\rm Isom}\nolimits}
\newcommand{\Tor}{\mathop{\rm Tor}\nolimits}
\newcommand{\Aut}{\mathop{\rm Aut}\nolimits}

\newcommand{\adj}{\mathop{\rm adj}\nolimits}

\newcommand{\Norm}{\mathop{\rm Norm}\nolimits}
\newcommand{\Gal}{\mathop{\rm Gal}\nolimits}
\newcommand{\Frob}{{\rm Frob}}

\newcommand{\disc}{\mathop{\rm disc}\nolimits}

\renewcommand{\Re}{\mathop{\rm Re}\nolimits}
\renewcommand{\Im}{\mathop{\rm Im}\nolimits}

\newcommand{\Log}{\mathop{\rm Log}\nolimits}
\newcommand{\Res}{\mathop{\rm Res}\nolimits}
\newcommand{\Bild}{\mathop{\rm Bild}\nolimits}

\renewcommand{\binom}[2]{\left({#1}\atop{#2}\right)}
\newcommand{\eck}[1]{\langle #1 \rangle}
\newcommand{\wi}{\hspace{1pt} < \hspace{-6pt} ) \hspace{2pt}}


\begin{document}

\pagestyle{empty}

\begin{center}
{\huge SMO Finalrunde 2013} \\
\medskip erste Pr�fung - 8. M�rz 2013
\end{center}
\vspace{8mm}
Zeit: 4 Stunden\\
Jede Aufgabe ist 7 Punkte wert.

\vspace{15mm}

\begin{enumerate}
\item[\textbf{1.}]
Bestimme alle Tripel $(a,b,c)$ nat�rlicher Zahlen, sodass die Mengen 
\[\Big\{\ggT(a,b),\ \ggT(b,c),\ \ggT(c,a),\ \kgV(a,b),\ \kgV(b,c),\ \kgV(c,a)\Big\}\]
und $$\big\{2, 3, 5, 30, 60\big\}$$ gleich sind.

\textit{Bemerkung: Zum Beispiel sind die Mengen $\left\{1,2013\right\}$ und $\left\{1,1,2013\right\}$ gleich.}


\bigskip

\item[\textbf{2.}] Seien $n$ eine nat�rliche Zahl und $p_1,...,p_n$ paarweise verschiedene Primzahlen. Zeige, dass gilt:
$$
p_1^2+p_2^2+\dots + p_n^2 > n^3.
$$


\bigskip

\item[\textbf{3.}] 
Sei $ABCD$ ein Sehnenviereck mit $\angle ADC=\angle DBA$. Ferner sei $E$ die Projektion von $A$ auf $BD$. Zeige, dass $BC=DE-BE$ gilt.


\bigskip

\item[\textbf{4.}]
Finde alle Funktionen $f: \BR_{>0} \to \BR_{>0}$ mit der folgenden Eigenschaft:
$$
f\left(\frac{x}{y+1}\right)=1-xf(x+y) \quad \text{f�r alle}\quad x>y>0\ .
$$



\bigskip

\item[\textbf{5.}]
Jeder von $2n+1$ Sch�lern w�hlt eine endliche, nichtleere Menge aufeinanderfolgender ganzer Zahlen. Zwei Sch�ler sind befreundet, falls sie eine gemeinsame Zahl ausgew�hlt haben. Jeder Sch�ler ist mit mindesten $n$ anderen Sch�lern befreundet. Zeige, dass es einen Sch�ler gibt, welcher mit allen anderen befreundet ist.

\end{enumerate}
\bigskip
\begin{center}
Viel Gl�ck !
\end{center}

\newpage

\begin{center}
{\huge SMO Finalrunde 2013} \\
\medskip zweite Pr�fung - 9. M�rz 2013
\end{center}
\vspace{8mm}
Zeit: 4 Stunden\\
Jede Aufgabe ist 7 Punkte wert.

\vspace{15mm}

\begin{enumerate}

\item[\textbf{6.}] 
Auf einem Tisch stehen zwei nichtleere Stapel mit $n$ respektive $m$ M�nzen. Folgende Operationen sind erlaubt:
\begin{itemize}
	\item Von beiden Stapeln wird jeweils die gleiche Anzahl M�nzen entfernt.
	\item Die Anzahl M�nzen eines Stapels wird verdreifacht.
\end{itemize}
F�r welche Paare $(n,m)$ ist es m�glich, dass nach endlich vielen Operationen keine M�nzen mehr vorhanden sind?

\bigskip

\item[\textbf{7.}] 
Sei $O$ der Umkreismittelpunkt des Dreiecks $ABC$ mit $AB\neq AC$. Ferner seien $S$ und $T$ Punkte auf den Strahlen $AB$ beziehungsweise $AC$, sodass $\angle ASO=\angle ACO$ und $\angle ATO= \angle ABO$ gelten. Zeige, dass $ST$ die Strecke $BC$ halbiert.

\bigskip

\item[\textbf{8.}] 
Seien $a,b,c > 0$ reelle Zahlen. Zeige die folgende Ungleichung:
$$
a^2\cdot\frac{a-b}{a+b}+b^2\cdot\frac{b-c}{b+c}+c^2\cdot\frac{c-a}{c+a} \ge 0\ .
$$
Wann gilt Gleichheit?


\bigskip

\item[\textbf{9.}]
Finde alle Quadrupel $(p,q,m,n)$ nat�rlicher Zahlen, sodass $p$ und $q$ Primzahlen sind und die folgende Gleichung erf�llt ist:
$$
p^m-q^3=n^3.
$$

\bigskip

\item[\textbf{10.}] 
Sei $ABCD$ ein Tangentenviereck mit $BC>BA$. Der Punkt $P$ liege so auf der Strecke $BC$, dass $BP=BA$ gilt. Zeige, dass sich die Winkelhalbierende von $\angle BCD$, die Senkrechte zu $BC$ durch $P$ und die Senkrechte zu $BD$ durch $A$ in einem Punkt schneiden.



\bigskip

\end{enumerate}
\bigskip
\begin{center}
Viel Gl�ck !
\end{center}
\end{document}
