Die Kanten eines Würfels werden mit den natürlichen Zahlen von 1 bis 12 durchnummeriert. Dann wird für jede Ecke die Summe der drei Zahlen ermittelt, die auf den von ihr ausgehenden Kanten stehen. Ermittle alle natürlichen Zahlen von 1 bis 12 mit folgender Eigenschaft: Ersetzt man nur diese Zahl durch 13, so gibt es eine Nummerierung der Kanten eines Würfels mit diesen zwölf Zahlen derart, dass alle acht Eckensummen gleich gross sind.

\bigskip

Les arrêtes d'un cube sont numérotés avec des nombres naturels de 1 à 12. On calcule alors pour chaque sommet la somme des trois nombres qui sont sur les arrêtes touchant ce sommet. Calculer tous les nombres de 1 à 12 avec la propriété suivant: Si on remplace uniquement ce nombre par 13, on obtient une numérotation des arrêtes du cube telle que les huit sommes calculées aux sommets sont égales.