\documentclass[12pt]{article}

\usepackage{amsfonts}
\usepackage{german}
\usepackage{amsthm}
\usepackage{amssymb}

\leftmargin=0pt \topmargin=0pt \headheight=0in \headsep=0in \oddsidemargin=0pt \textwidth=6.5in
\textheight=8.5in

\catcode`\? = \active \catcode`\? = \active \catcode`\? = \active \catcode`\? = \active \catcode`\? = \active
\catcode`\? = \active

\def?{"A}
\def?{"a}
\def?{"O}
\def?{"o}
\def?{"U}
\def?{"u}

\begin{document}

\pagestyle{empty}
\newcommand{\wi}{\hspace{1pt} < \hspace{-6pt} ) \hspace{2pt}}

\begin{center}
{\huge SMO - Tour pr\'{e}liminaire} \\
\medskip Berne, Zurich - le 10 janvier 2004
\end{center}
\vspace{8mm}
Dur\'{e}e : 2 heures\\
Chaque probl\`{e}me vaut 7 points.

\vspace{15mm}

\begin{enumerate}

\item[\textbf{1.}] Trouver tous les nombres naturels $a$, $b$ et $n$ tels que l'\'{e}quation suivante est satisfaite:
\[a!+b!=2^n\]

\bigskip

\item[\textbf{2.}] Il y a 17 tours sur un \'{e}chiquier ordinaire. Montrer que l'on peut toujours en choisir trois
qui ne se menacent pas entre elles. (Une tour peut se d\'{e}placer verticalement ou horizontalement d'autant de
cases qu'elle le veut. Une tour menace une autre au cas o\`{u} la premi\`{e}re peut se d\'{e}placer sur la case
de la deuxi\`{e}me.)

\bigskip

\item[\textbf{3.}] Soit $ABCD$ un parall\`{e}logramme. Les points $P$ et $Q$ se trouvent \`{a} l'int\'{e}rieur de
$ABCD$ sur la diagonale $AC$, o\`{u} $|AP|=|CQ|<{\textstyle \frac{1}{2}}|AC|$. La droite $BP$ coupe $AD$ en un
point $E$, la droite $BQ$ coupe $CD$ en $F$. Montrer que $EF$ est parall\`{e}le \`{a} la diagonale $AC$.

\bigskip

\item[\textbf{4.}] Trouver tous les nombres naturels $n$ ayant exactement 100 diviseurs positifs diff\'{e}rents,
tels que au moins 10 de ces diviseurs sont cons\'{e}cutifs.

\bigskip

\item[\textbf{5.}] $m\times n$ points forment un rectangle sur une grille quadratique. De combien de mani\`{e}res
diff\'{e}rentes peut-on colorer les points en blanc et en rouge si l'on veut que parmi les sommets d'un
carr\'{e} unitaire il y ait exactement deux rouges et deux blancs?\\[3cm]

\end{enumerate}


\begin{center}
Bonne chance!
\end{center}

\end{document}

