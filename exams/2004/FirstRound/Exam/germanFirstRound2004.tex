\documentclass[12pt]{article}

\usepackage{amsfonts}
\usepackage{german}
\usepackage{amsthm}
\usepackage{amssymb}

\leftmargin=0pt \topmargin=0pt \headheight=0in \headsep=0in \oddsidemargin=0pt \textwidth=6.5in
\textheight=8.5in

\catcode`\� = \active \catcode`\� = \active \catcode`\� = \active \catcode`\� = \active \catcode`\� = \active
\catcode`\� = \active

\def�{"A}
\def�{"a}
\def�{"O}
\def�{"o}
\def�{"U}
\def�{"u}

\begin{document}

\pagestyle{empty}
\newcommand{\wi}{\hspace{1pt} < \hspace{-6pt} ) \hspace{2pt}}

\begin{center}
{\huge SMO - Vorrunde} \\
\medskip Bern, Z�rich - 10. Januar 2004
\end{center}
\vspace{8mm}
Zeit: 2 Stunden\\
Jede Aufgabe ist 7 Punkte wert.

\vspace{15mm}

\begin{enumerate}

\item[\textbf{1.}] Finde alle nat�rlichen Zahlen $a$, $b$ und $n$, sodass die folgende Gleichung gilt:
\[a!+b!=2^n\]

\bigskip

\item[\textbf{2.}] Auf einem gew�hnlichen Schachbrett stehen 17 T�rme. Zeige, dass man stets drei T�rme
ausw�hlen kann, die sich gegenseitig nicht bedrohen. (Ein Turm kann in einem Zug beliebig viele Felder nach
links, rechts, oben oder unten ziehen. Ein Turm bedroht einen anderen, falls er in einem Zug auf das Feld des
anderen Turmes ziehen kann.)

\bigskip

\item[\textbf{3.}] Sei $ABCD$ ein Parallelogram. Die Punkte $P$ und $Q$ liegen im Innern von $ABCD$ auf der
Diagonalen $AC$, dabei gilt $|AP|=|CQ|<{\textstyle \frac{1}{2}}|AC|$. Die Gerade $BP$ schneidet $AD$ im Punkt
$E$, die Gerade $BQ$ schneidet $CD$ in $F$. Zeige, dass $EF$ parallel zur Diagonalen $AC$ ist.

\bigskip

\item[\textbf{4.}] Bestimme alle nat�rlichen Zahlen $n$ mit genau 100 verschiedenen positiven Teilern, sodass
mindestens 10 dieser Teiler aufeinanderfolgende Zahlen sind.

\bigskip

\item[\textbf{5.}] $m\times n$ Punkte sind in einem quadratischen Gitter zu einem Rechteck angeordnet. Wieviele
M�glichkeiten gibt es, diese Punkte rot oder weiss zu f�rben, sodass unter je vier Punkten, die Ecken eines
Einheitsquadrates bilden, genau zwei weisse und zwei rote vorkommen?\\[3cm]


\end{enumerate}


\begin{center}
Viel Gl�ck!
\end{center}

\end{document}

