\documentclass[12pt,a4paper]{article}

\usepackage{amsfonts}
\usepackage[centertags]{amsmath}
\usepackage{german}
\usepackage{amsthm}
\usepackage{amssymb}
\usepackage{enumerate}

\leftmargin=0pt \topmargin=0pt \headheight=0in \headsep=0in \oddsidemargin=0pt \textwidth=6.5in
\textheight=8.5in

\catcode`\� = \active \catcode`\� = \active \catcode`\� = \active \catcode`\� = \active \catcode`\� = \active
\catcode`\� = \active

\def�{"A}
\def�{"a}
\def�{"O}
\def�{"o}
\def�{"U}
\def�{"u}




% Schriftabk�rzungen

\newcommand{\dreieck}{\bigtriangleup}

\newcommand{\eps}{\varepsilon}
\renewcommand{\phi}{\varphi}
\newcommand{\Sl}{\ell}    % sch�nes l
\newcommand{\ve}{\varepsilon}  %Epsilon

\newcommand{\BA}{{\mathbb{A}}}
\newcommand{\BB}{{\mathbb{B}}}
\newcommand{\BC}{{\mathbb{C}}}
\newcommand{\BD}{{\mathbb{D}}}
\newcommand{\BE}{{\mathbb{E}}}
\newcommand{\BF}{{\mathbb{F}}}
\newcommand{\BG}{{\mathbb{G}}}
\newcommand{\BH}{{\mathbb{H}}}
\newcommand{\BI}{{\mathbb{I}}}
\newcommand{\BJ}{{\mathbb{J}}}
\newcommand{\BK}{{\mathbb{K}}}
\newcommand{\BL}{{\mathbb{L}}}
\newcommand{\BM}{{\mathbb{M}}}
\newcommand{\BN}{{\mathbb{N}}}
\newcommand{\BO}{{\mathbb{O}}}
\newcommand{\BP}{{\mathbb{P}}}
\newcommand{\BQ}{{\mathbb{Q}}}
\newcommand{\BR}{{\mathbb{R}}}
\newcommand{\BS}{{\mathbb{S}}}
\newcommand{\BT}{{\mathbb{T}}}
\newcommand{\BU}{{\mathbb{U}}}
\newcommand{\BV}{{\mathbb{V}}}
\newcommand{\BW}{{\mathbb{W}}}
\newcommand{\BX}{{\mathbb{X}}}
\newcommand{\BY}{{\mathbb{Y}}}
\newcommand{\BZ}{{\mathbb{Z}}}

\newcommand{\Fa}{{\mathfrak{a}}}
\newcommand{\Fb}{{\mathfrak{b}}}
\newcommand{\Fc}{{\mathfrak{c}}}
\newcommand{\Fd}{{\mathfrak{d}}}
\newcommand{\Fe}{{\mathfrak{e}}}
\newcommand{\Ff}{{\mathfrak{f}}}
\newcommand{\Fg}{{\mathfrak{g}}}
\newcommand{\Fh}{{\mathfrak{h}}}
\newcommand{\Fi}{{\mathfrak{i}}}
\newcommand{\Fj}{{\mathfrak{j}}}
\newcommand{\Fk}{{\mathfrak{k}}}
\newcommand{\Fl}{{\mathfrak{l}}}
\newcommand{\Fm}{{\mathfrak{m}}}
\newcommand{\Fn}{{\mathfrak{n}}}
\newcommand{\Fo}{{\mathfrak{o}}}
\newcommand{\Fp}{{\mathfrak{p}}}
\newcommand{\Fq}{{\mathfrak{q}}}
\newcommand{\Fr}{{\mathfrak{r}}}
\newcommand{\Fs}{{\mathfrak{s}}}
\newcommand{\Ft}{{\mathfrak{t}}}
\newcommand{\Fu}{{\mathfrak{u}}}
\newcommand{\Fv}{{\mathfrak{v}}}
\newcommand{\Fw}{{\mathfrak{w}}}
\newcommand{\Fx}{{\mathfrak{x}}}
\newcommand{\Fy}{{\mathfrak{y}}}
\newcommand{\Fz}{{\mathfrak{z}}}

\newcommand{\FA}{{\mathfrak{A}}}
\newcommand{\FB}{{\mathfrak{B}}}
\newcommand{\FC}{{\mathfrak{C}}}
\newcommand{\FD}{{\mathfrak{D}}}
\newcommand{\FE}{{\mathfrak{E}}}
\newcommand{\FF}{{\mathfrak{F}}}
\newcommand{\FG}{{\mathfrak{G}}}
\newcommand{\FH}{{\mathfrak{H}}}
\newcommand{\FI}{{\mathfrak{I}}}
\newcommand{\FJ}{{\mathfrak{J}}}
\newcommand{\FK}{{\mathfrak{K}}}
\newcommand{\FL}{{\mathfrak{L}}}
\newcommand{\FM}{{\mathfrak{M}}}
\newcommand{\FN}{{\mathfrak{N}}}
\newcommand{\FO}{{\mathfrak{O}}}
\newcommand{\FP}{{\mathfrak{P}}}
\newcommand{\FQ}{{\mathfrak{Q}}}
\newcommand{\FR}{{\mathfrak{R}}}
\newcommand{\FS}{{\mathfrak{S}}}
\newcommand{\FT}{{\mathfrak{T}}}
\newcommand{\FU}{{\mathfrak{U}}}
\newcommand{\FV}{{\mathfrak{V}}}
\newcommand{\FW}{{\mathfrak{W}}}
\newcommand{\FX}{{\mathfrak{X}}}
\newcommand{\FY}{{\mathfrak{Y}}}
\newcommand{\FZ}{{\mathfrak{Z}}}

\newcommand{\CA}{{\cal A}}
\newcommand{\CB}{{\cal B}}
\newcommand{\CC}{{\cal C}}
\newcommand{\CD}{{\cal D}}
\newcommand{\CE}{{\cal E}}
\newcommand{\CF}{{\cal F}}
\newcommand{\CG}{{\cal G}}
\newcommand{\CH}{{\cal H}}
\newcommand{\CI}{{\cal I}}
\newcommand{\CJ}{{\cal J}}
\newcommand{\CK}{{\cal K}}
\newcommand{\CL}{{\cal L}}
\newcommand{\CM}{{\cal M}}
\newcommand{\CN}{{\cal N}}
\newcommand{\CO}{{\cal O}}
\newcommand{\CP}{{\cal P}}
\newcommand{\CQ}{{\cal Q}}
\newcommand{\CR}{{\cal R}}
\newcommand{\CS}{{\cal S}}
\newcommand{\CT}{{\cal T}}
\newcommand{\CU}{{\cal U}}
\newcommand{\CV}{{\cal V}}
\newcommand{\CW}{{\cal W}}
\newcommand{\CX}{{\cal X}}
\newcommand{\CY}{{\cal Y}}
\newcommand{\CZ}{{\cal Z}}

% Theorem Stil

\theoremstyle{plain}
\newtheorem{lem}{Lemma}
\newtheorem{Satz}[lem]{Satz}

\theoremstyle{definition}
\newtheorem{defn}{Definition}[section]

\theoremstyle{remark}
\newtheorem{bem}{Bemerkung}    %[section]



\newcommand{\card}{\mathop{\rm card}\nolimits}
\newcommand{\Sets}{((Sets))}
\newcommand{\id}{{\rm id}}
\newcommand{\supp}{\mathop{\rm Supp}\nolimits}

\newcommand{\ord}{\mathop{\rm ord}\nolimits}
\renewcommand{\mod}{\mathop{\rm mod}\nolimits}
\newcommand{\sign}{\mathop{\rm sign}\nolimits}
\newcommand{\ggT}{\mathop{\rm ggT}\nolimits}
\newcommand{\kgV}{\mathop{\rm kgV}\nolimits}
\renewcommand{\div}{\, | \,}
\newcommand{\notdiv}{\mathopen{\mathchoice
             {\not{|}\,}
             {\not{|}\,}
             {\!\not{\:|}}
             {\not{|}}
             }}

\newcommand{\im}{\mathop{{\rm Im}}\nolimits}
\newcommand{\coim}{\mathop{{\rm coim}}\nolimits}
\newcommand{\coker}{\mathop{\rm Coker}\nolimits}
\renewcommand{\ker}{\mathop{\rm Ker}\nolimits}

\newcommand{\pRang}{\mathop{p{\rm -Rang}}\nolimits}
\newcommand{\End}{\mathop{\rm End}\nolimits}
\newcommand{\Hom}{\mathop{\rm Hom}\nolimits}
\newcommand{\Isom}{\mathop{\rm Isom}\nolimits}
\newcommand{\Tor}{\mathop{\rm Tor}\nolimits}
\newcommand{\Aut}{\mathop{\rm Aut}\nolimits}

\newcommand{\adj}{\mathop{\rm adj}\nolimits}

\newcommand{\Norm}{\mathop{\rm Norm}\nolimits}
\newcommand{\Gal}{\mathop{\rm Gal}\nolimits}
\newcommand{\Frob}{{\rm Frob}}

\newcommand{\disc}{\mathop{\rm disc}\nolimits}

\renewcommand{\Re}{\mathop{\rm Re}\nolimits}
\renewcommand{\Im}{\mathop{\rm Im}\nolimits}

\newcommand{\Log}{\mathop{\rm Log}\nolimits}
\newcommand{\Res}{\mathop{\rm Res}\nolimits}
\newcommand{\Bild}{\mathop{\rm Bild}\nolimits}

\renewcommand{\binom}[2]{\left({#1}\atop{#2}\right)}
\newcommand{\eck}[1]{\langle #1 \rangle}
\newcommand{\gaussk}[1]{\lfloor #1 \rfloor}
\newcommand{\frack}[1]{\{ #1 \}}
\newcommand{\wi}{\hspace{1pt} < \hspace{-6pt} ) \hspace{2pt}}

\begin{document}

\pagestyle{empty}

\begin{center}
{\huge IMO s\'election 2004} \\
\medskip Premier test - 15. mai 2004
\end{center}
\vspace{8mm}
Temps: 4.5 heures\\
Chaque probl\`eme vaut 7 points.

\vspace{15mm}

\begin{enumerate}
\item[\textbf{1.}] Consid\'erons l'ensemble $S$ de tout les $n$-tuples $(X_1,\ldots,X_n)$, consistant des
sous-ensembles de $\{1,2,\ldots, 1000\}$. Pour chaque $a=(X_1,\ldots,X_n) \in S$ on d\'efinit
\[E(a)= \mbox{nombre d'\'el\'ements  de } X_1 \cup \ldots \cup X_n.\]
Calculer explicitement la somme suivante
\[\sum_{a \in S} E(a).\]

\bigskip
\bigskip

\item[\textbf{2.}] D\'eterminer le plus grand nombre naturel $n$ tel que
\[4^{995}+4^{1500}+4^n\]
est un carr\'e.

\bigskip
\bigskip

\item[\textbf{3.}] Soit $ABC$ un triangle isoc\`ele avec $|AC|=|BC|$ et $I$ le centre du cercle inscrit. Soit $P$ un
point sur le cercle circonscrit au triangle $AIB$, se trouvant \`a l'int\'erieur du triangle $ABC$. Les droites
passant par $P$ et parall\`eles \`a $CA$ et \`a $CB$ coupent $AB$ en $D$ et en $E$. La droite passant par $P$ et
parall\`ele \`a $AB$ coupe $CA$ en $F$ et $CB$ en $G$. Montrer que l'intersection des droites $DF$ et $EG$ se
trouve sur le cercle circonscrit au triangle $ABC$.

\end{enumerate}

\pagebreak


\begin{center}
{\huge IMO s\'election 2004} \\
\medskip Second test - 16. mai 2004
\end{center}
\vspace{8mm}
Temps: 4.5 heures\\
Chaque probl\`eme vaut 7 points.

\vspace{15mm}

\begin{enumerate}

\item[\textbf{4.}] Soit $a,b,c$ trois r\'eels positifs avec $abc=1$. Prouver l'in\'equation suivante:
\[\frac{ab}{a^5+ab+b^5}+\frac{bc}{b^5+bc+c^5}+\frac{ca}{c^5+ca+a^5} \leq 1.\]


\bigskip
\bigskip

\item[\textbf{5.}] Un \textit{bloc de construction}, constitu\'e de 7 cubes-unit\'e, a la forme d'une cube $2 \times 2 \times 2$ avec un des coins (une cube-unit\'e) manquant. Soit alors une cube de longueur de c\^ot\'e $2^n$, $n \geq 2$, dont on enl\`eve une cube-unit\'e arbitraire. Montrer que le corps restant peut \^etre reconstruit \`a partir de tels blocs de construction.

\bigskip
\bigskip

\item[\textbf{6.}] Trouver toutes les suites  r\'eelles finies $(x_0,x_1,\ldots,x_n)$ telles que le nombre $k$
appara\^it exactement $x_k$ fois dans la suite.
\end{enumerate}

\pagebreak


\begin{center}
{\huge IMO S\'{e}lection 2004} \\
\medskip troisi\`{e}me examen - le 12 juin 2004
\end{center}
\vspace{8mm}
Dur\'{e}e : 4.5 heures\\
Chaque probl\`{e}me vaut 7 points.

\vspace{15mm}

\begin{enumerate}
\item[\textbf{7.}] Pour quatre nombres r\'{e}elles $a,b,c,d$, les \'{e}quations suivantes sont satisfaites
\begin{eqnarray*}
a=\sqrt{45-\sqrt{21-a}} &,& \qquad b=\sqrt{45+\sqrt{21-b}},\\
c=\sqrt{45-\sqrt{21+c}} &,& \qquad d=\sqrt{45+\sqrt{21+d}}.
\end{eqnarray*}
Montrer que $abcd=2004$.

\bigskip
\bigskip

\item[\textbf{8.}] Soit $m$ un nombre naturel plus grand que 1. La suite $x_0, x_1, x_2,\ldots$ est d\'{e}finie par
\[x_i= \left\{
\begin{tabular}{lll}
$2^i$,& pour & $0 \leq i \leq m-1$;\\
$\sum_{j=1}^{m}x_{i-j}$ & pour & $i \geq m$.
\end{tabular}
\right.\] Trouver le plus grand nombre $k$ tel qu\'{}il existe $k$ termes cons\'{e}cutifs de la suite tous divisibles par $m$.

\bigskip
\bigskip

\item[\textbf{9.}] Soit $X$ un ensemble avec $n$ \'{e}l\'{e}ments et soient $A_1, A_2, \ldots A_n$ des sous-ensembles distincts de $X$. Montrer qu\'{}il existe un $x \in X$ tel que les ensembles
\[A_1 \setminus \{x\}, A_2 \setminus \{x\}, \ldots , A_n \setminus \{x\}\]
sont deux \`{a} deux distincts.

\end{enumerate}


\pagebreak


\begin{center}
{\huge IMO S\'{e}lection 2004} \\
\medskip quatri\`{e}me examen - le 13 juin 2004
\end{center}
\vspace{8mm}
Dur\'{e}e : 4.5 heures\\
Chaque probl\`{e}me vaut 7 points.

\vspace{15mm}

\begin{enumerate}
\item[\textbf{10.}] Soit $\dreieck ABC$ un triangle aigu avec les hauteurs $\overline{AU}$, $\overline{BV}$, $\overline{CW}$ et l'orthocentre
$H$. Soit $X$ un point sur $\overline{AU}$, $Y$ sur $\overline{BV}$, $Z$ sur $\overline{CW}$, avec $X, Y, Z$ tous
distincts de $H$.
\begin{enumerate}[a)]
\item Si $X,Y,Z,H$ se trouvent sur le m\^{e}me cercle, alors
$$ [ABC] = [ABZ] + [AYC] + [XBC]$$

o\`{u} $[PQR]$ d\'{e}signe l'aire du $\dreieck PQR$.
\bigskip
\item Montrer que l'inverse de a) est \'{e}galement vrai.
\end{enumerate}

\bigskip
\bigskip

\item[\textbf{11.}] Trouver toutes les fonctions injectives $f: \BR \rightarrow \BR$ tels que pour tout couple de nombres r\'{e}els $x \neq y$ l\'{}\'{e}quation suivante est satisfaite
\[f\left(\frac{x+y}{x-y}\right)=\frac{f(x)+f(y)}{f(x)-f(y)}.\]

\bigskip
\bigskip

\item[\textbf{12.}] Trouver tous les nombres naturels pouvant \^{e}tre \'{e}crits sous la forme
\[\frac{(a+b+c)^2}{abc}\]
avec $a,b$ et $c$ trois nombres naturels.

\end{enumerate}

\end{document}
