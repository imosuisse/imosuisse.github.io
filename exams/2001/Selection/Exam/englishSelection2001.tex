\documentclass[12pt]{article}
\usepackage{amsfonts}
\leftmargin=0pt \topmargin=0pt \headheight=0in \headsep=0in
\oddsidemargin=0pt \textwidth=6.5in \textheight=8.5in

\begin{document}

\pagestyle{empty}
\newcommand{\wi}{\hspace{1pt} < \hspace{-6pt} ) \hspace{2pt}}

\begin{center}
{\Large Swiss Mathematical Competition} \\
\medskip First Part - May 4, 2001
\end{center}
\vspace{4mm}
Time: 3 hours\\
Each problem is worth 7 points.


\begin{enumerate}
\item  In a park there is a square of 2001 $\times$ 2001 trees.
What is the largest number of trees that can be cut down, so that
from each stump you cannot see another stump?

\item Let $a, b$ and $c$ be the sides of a triangle. Prove the
inequality:

$$ \sqrt{a+b-c} + \sqrt{c+a-b} + \sqrt{b+c-a} \leq \sqrt{a} +
\sqrt{b} + \sqrt{c} $$

Under what conditions does equality hold?

\item A convex pentagon is given, in which each diagonal is
parallel to one side. Prove that the ratio between the lengths of
each diagonal and the side which is parallel to it, is the same
for every diagonal and determine the value of this ratio.

\item Let $n \in \mathbb{N}$, $n\geq2$ and $t_1,
t_2, \ldots, t_k$ be $k$ different divisors of $n$.\\
An identity $n=t_1+t_2+ \ldots +t_k$ is called representation of
$n$ as sum of divisors, while two such representations are called
equal if they differ only in the order of the summands (for
example: $20=10+5+4+1$ and $20=5+1+10+4$ are the same
representation of $20$ as sum of divisors). \\
Let $a(n)$ be the number of different representations of $n$ as
sum of divisors.\vspace {2mm}\\
\vspace {2mm}Prove or disprove:\\
There is an $M \in \mathbb{N}$ with $a(n)\leq M$ for all $n \in
\mathbb{N}, n\geq 2$

\item Let $a$ be a sequence of positive integers $a_1<a_2< \ldots
<a_n$ with the property, that for $i<j$, the decimal
representation of $a_j$ does not start with the one of the $a_i$
(for example: 137 and 13729 cannot both appear in the sequence).
Prove that:
$$\sum_{i=1}^n \frac{1}{a_i} \leq \frac{1}{1}+\frac{1}{2}+\frac{1}{3}+\ldots+\frac{1}{9}$$

\end{enumerate}
\pagebreak

\begin{center}
{\Large Swiss Mathematical Competition} \\
\medskip Second Part - May 19, 2001
\end{center}
\vspace{4mm}
Time: 3 hours\\
Each problem is worth 7 points.

\begin{enumerate}
\item A function $f:[0,1]\rightarrow\mathbb{R}$ has the
following properties:

\begin{enumerate}
  \item $f(x) \geq 0 \hspace{6mm} \forall \; x \in [0,1]$
  \item $f(1) = 1$
  \item $f(x+y) \geq f(x)+f(y) \hspace{6mm} \forall \; x, y, x+y \in [0,1]$
\end{enumerate}

Prove that: $f(x)\leq 2x \hspace{6mm} \forall x \in [0,1]$

\item Let $ABC$ be an acute-angled triangle with circumcenter $O$.
Let $S$ be the circle through $A, B$ and $O$. The straight lines
$AC$ and $BC$ meet $S$ in the
additional points $P$ and $Q$ respectively.\\
Prove that $CO \perp PQ$.

\item Find the two smallest positive integers $n$, such that the
fractions
$$ \frac{68}{n+70}, \; \frac{69}{n+71},\; \frac{70}{n+72},\; \ldots,
\; \frac{133}{n+135}$$ are irreducible.

\item In Geneva there are 16 secret agents at work. Each of them is
spying on at least one other agent, but no two agents are spying
on one another. Suppose that we can number every ten agents in a
way, that the first one is spying on the second, the second on the
third and so on and the
tenth is spying on the first.\\
Prove that also every 11 agents can be numbered in a way, so that
every agent is spying on the next one.

\item Prove that each 1000-element subset $M
\subset \{0, 1, \ldots, 2001 \}$ has the following property:\\
There exists one element in $M$ which is a power of two or there
are two distinct elements in $M$ whoze sum is a power of two.


\end{enumerate}




\end{document}
