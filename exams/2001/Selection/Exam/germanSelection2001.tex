\documentclass[12pt]{article}
\usepackage{amsfonts}
\leftmargin=0pt \topmargin=0pt \headheight=0in \headsep=0in \oddsidemargin=0pt \textwidth=6.5in
\textheight=8.5in

\usepackage{german}

\catcode`\� = \active \catcode`\� = \active \catcode`\� = \active \catcode`\� = \active \catcode`\� = \active
\catcode`\� = \active

\def�{"A}
\def�{"a}
\def�{"O}
\def�{"o}
\def�{"U}
\def�{"u}

\begin{document}

\pagestyle{empty}
\newcommand{\wi}{\hspace{1pt} < \hspace{-6pt} ) \hspace{2pt}}

\begin{center}
{\huge Schweizer IMO - Selektion} \\
\medskip erste Pr�fung - 4. Mai 2001
\end{center}
\vspace{8mm}
Zeit: 3 Stunden\\
Jede Aufgabe ist 7 Punkte wert.

\vspace{15mm}

\begin{enumerate}

\item[\textbf{1.}] In einem Park sind $2001 \times 2001$ B�ume in einem quadratischen Gitter angeordnet. Was ist
die gr�sste Zahl an B�umen, die man f�llen kann, sodass kein Baumstrunk von einem anderen aus sichtbar ist?\\
(Die B�ume sollen Durchmesser 0 haben)

\bigskip

\item[\textbf{2.}] Seien $a,b$ und $c$ die Seiten eines Dreiecks. Beweise die Ungleichung
\[\sqrt{a+b-c} + \sqrt{c+a-b} + \sqrt{b+c-a} \leq \sqrt{a} + \sqrt{b} + \sqrt{c}.\]
Wann gilt das Gleichheitszeichen?

\bigskip

\item[\textbf{3.}] In einem konvexen F�nfeck ist jede Diagonale parallel zu einer Seite. Zeige, dass das
Verh�ltnis zwischen den L�ngen der Diagonalen und der dazu parallelen Seite f�r alle Diagonalen dasselbe ist.
Bestimme den Wert dieses Verh�ltnisses.

\bigskip

\item[\textbf{4.}] Seien $n \in \mathbb{N}$, $n \geq 2$ und $t_1,t_2,\ldots,t_k$ verschiedene Teiler von $n$.
Eine Identit�t der Form $n=t_1+t_2+ \ldots +t_k$ heisst Darstellung von $n$ als Summe verschiedener Teiler. Zwei
solche Darstellungen gelten als gleich, wenn sie sich nur um die Reihenfolge der Summanden unterscheiden (zum
Beispiel sind $20=10+5+4+1$ und $20=5+1+10+4$ zweimal die gleiche Darstellung von 20 als Summe verschiedener
Teiler). Sei $a(n)$ die Anzahl verschiedener Darstellungen von $n$ als Summe verschiedener Teiler. Zeige oder
widerlege: \vspace {2mm}\\
Es gibt ein $M \in \mathbb{N}$ mit $a(n)\leq M$ f�r alle $n \in \mathbb{N}$, $n\geq 2$.

\bigskip

\item[\textbf{5.}] Sei $a_1<a_2< \ldots <a_n$ eine Folge positiver ganzer Zahlen mit der Eigenschaft, dass f�r
$i<j$ die Dezimaldarstellung von $a_j$ nicht mit jener von $a_i$ beginnt (zum Beispiel k�nnen die Zahlen 137 und
13729 nicht beide in der Folge vorkommen). Beweise, dass gilt
\[\sum_{i=1}^n \frac{1}{a_i} \leq \frac{1}{1}+\frac{1}{2}+\frac{1}{3}+\ldots+\frac{1}{9}\]

\end{enumerate}


\pagebreak


\begin{center}
{\huge Schweizer IMO - Selektion} \\
\medskip zweite Pr�fung - 19. Mai 2001
\end{center}
\vspace{8mm}
Zeit: 3 Stunden\\
Jede Aufgabe ist 7 Punkte wert.

\vspace{15mm}

\begin{enumerate}

\item[\textbf{6.}] Die Funktion $f:[0,1]\rightarrow\mathbb{R}$ habe die folgenden Eigenschaften:
\begin{enumerate}
  \item $f(x) \geq 0 \hspace{6mm} \forall \; x \in [0,1]$
  \item $f(1) = 1$
  \item $f(x+y) \geq f(x)+f(y) \hspace{6mm} \forall \; x, y, x+y \in [0,1]$
\end{enumerate}
Beweise: $f(x)\leq 2x \hspace{6mm} \forall x \in [0,1]$

\bigskip

\item[\textbf{7.}] Sei $ABC$ ein spitzwinkliges Dreieck mit Umkreismittelpunkt $O$. $S$ sei der Kreis durch
$A,B$ und $O$. Die Geraden $AC$ und $BC$ schneiden $S$ in den weiteren Punkten $P$ und $Q$. Zeige $CO \perp PQ$.

\bigskip

\item[\textbf{8.}] Finde die zwei kleinsten nat�rlichen Zahlen $n$, sodass die Br�che
\[\frac{68}{n+70}, \; \frac{69}{n+71},\; \frac{70}{n+72},\; \ldots, \; \frac{133}{n+135}\]
alle irreduzibel sind.

\bigskip

\item[\textbf{9.}] In Genf sind 16 Geheimagenten am Werk. Jeder Agent �berwacht mindestens einen anderen
Agenten, aber keine zwei Agenten �berwachen sich gegenseitig. Nehme an, dass je 10 Agenten so nummeriert werden
k�nnen, dass der erste den zweiten �berwacht, der zweite den dritten usw. und der zehnte den ersten. Zeige, dass
dann auch je 11 Agenten in dieser Art nummeriert werden k�nnen, dass jeder den n�chsten �berwacht.

\bigskip

\item[\textbf{10.}] Zeige, dass jede 1000-elementige Teilmenge $M \subset \{0, 1, \ldots, 2001 \}$ eine
Zahl enth�lt, die eine Zweierpotenz ist, oder zwei verschiedene Zahlen, deren Summe eine Zweierpotenz ist.

\end{enumerate}

\end{document}

