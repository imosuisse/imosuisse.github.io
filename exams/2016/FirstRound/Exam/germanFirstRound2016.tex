\documentclass[language=german,style=exam]{smo}

\title{SMO - Vorrunde}
\examdate{16. Januar 2016}
\place{Belinzona, Lausanne, Zürich}
\examtime{3 Stunden}

\begin{document}

\begin{enumerate}

\item[\textbf{1.}] 
Zwei Kreise $k_1$ und $k_2$ schneiden sich in den Punkten $A$ und $C$. Sei $B$ der zweite Schnittpunkt von $k_1$ und der Tangente an $k_2$ in $A$, und sei $D$ der zweite Schnittpunkt von $k_2$ und der Tangente an $k_1$ in $C$. Zeige, dass $AD$ und $BC$ parallel sind.

\bigskip

\item[\textbf{2.}] 
Quirin hat $n$ Klötze mit den Höhen 1 bis $n$ und möchte diese so nebeneinander aufstellen, dass sich seine Katze von links nach rechts über sie hinwegbewegen kann. Die Katze kann dabei jeweils auf den nächsten Klotz springen, falls dieser tiefer oder um 1 höher ist. Zu Beginn wird die Katze auf den Klotz am linken Ende gesetzt.\\
Wie viele Möglichkeiten hat Quirin, die Klötze in einer solchen Reihe aufzustellen?

\textit{Bemerkung: Für $n=5$ ist $3-4-5-1-2$ eine Möglichkeit, $1-3-4-5-2$ jedoch nicht.} 

\bigskip

\item[\textbf{3.}]
Bestimme alle natürlichen Zahlen $n$, sodass für alle positive Teiler $d$ von $n$ gilt:
\[
d+1 \div n+1.
\]

\bigskip

\item[\textbf{4.}] 
Bei 22 Mathematikwettbewerben werden jeweils 5 Preise verteilt. Nachdem alle Wettbewerbe durchgeführt sind, bemerken die Organisatoren, dass es für jede Kombination von zwei Wett\-bewer\-ben genau einen gemeinsamen Preisträger gibt. Zeige, dass ein Teilnehmer bei allen Wettbe\-wer\-ben einen Preis gewonnen hat.

\bigskip

\item[\textbf{5.}] 
Sei $ABC$ ein Dreieck mit $AB < AC$. Die Winkelhalbierende von $\angle BAC$ schneidet die Seite $BC$ im Punkt $D$. Sei $k$ der Kreis, der durch $D$ geht und die Seiten $AC$ und $AB$ in den Punkten $E$ respektive $F$ berührt. Sei $G$ der zweite Schnittpunkt von $k$ und $BC$ und sei $S$ der Schnittpunkt der Strecken $EG$ und $DF$. Zeige, dass $AD$ und $BS$ senkrecht aufeinander stehen.

\bigskip

\end{enumerate}

\vspace{1cm}

\center{\hspace{1cm} Viel Glück!}
\end{document}
