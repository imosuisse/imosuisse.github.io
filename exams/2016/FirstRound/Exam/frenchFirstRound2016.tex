\documentclass[language=french,style=exam]{smo}

\title{OSM - Tour préliminaire}
\examdate{16 janvier 2016}
\place{Bellinzone, Lausanne, Zürich}
\examtime{3 heures}

\begin{document}

\begin{enumerate}

\item[\textbf{1.}] 
%Zwei Kreise $k_1$ und $k_2$ schneiden sich in den Punkten $A$ und $C$. Sei $B$ der zweite Schnittpunkt von $k_1$ und der Tangente an $k_2$ in $A$, und sei $D$ der zweite Schnittpunkt von $k_2$ und der Tangente an $k_1$ in $C$. Zeige, dass $AD$ und $BC$ parallel sind.
Deux cercles $k_1$ et $k_2$ se coupent en deux points $A$ et $C$. Soit $B$ le deuxième point d'intersection de $k_1$ avec la tangente à $k_2$ passant par $A$. Soit $D$ le deuxième point d'intersection de $k_2$ avec la tangente à $k_1$ passant par $C$. Montrer que $AD$ et $BC$ sont parallèles.

\bigskip

\item[\textbf{2.}] 
%Quirin hat $n$ Klötze mit den Höhen 1 bis $n$ und möchte diese so nebeneinander aufstellen, dass sich seine Katze von links nach rechts über sie hinwegbewegen kann. Die Katze kann dabei jeweils auf den nächsten Klotz springen, falls dieser tiefer oder um 1 höher ist. Zu Beginn wird die Katze auf den Klotz am linken Ende gesetzt.\\
%Wie viele Möglichkeiten hat Quirin, die Klötze in einer solchen Reihe aufzustellen?
Cunégonde a $n$ blocs de hauteur 1 à $n$ et souhaiterait les agencer, les uns après les autres, de telle manière que son chat puisse se déplacer en sautant d'un bloc à l'autre, de la gauche vers la droite. Son chat peut sauter d'un bloc au suivant si celui-ci est soit moins haut, soit plus haut de 1 que le bloc précédent. Au début, son chat se trouve sur le bloc à l'extrémité gauche.\\
De combien de manières Cunégonde peut-elle agencer ses blocs, pour que son chat puisse franchir tous les blocs ?


\textit{Remarque: Pour $n=5$, $3-4-5-1-2$ est une possibilité, mais $1-3-4-5-2$ n'en est pas une.} 

\bigskip

\item[\textbf{3.}]
%Bestimme alle natürlichen Zahlen $n$, sodass für alle Teiler $d$ von $n$ gilt:
Déterminer tous les entiers naturels $n$ tels que pour chaque diviseur positif $d$ de $n$ on ait 
\[
d+1 \div n+1.
\]

\bigskip

\item[\textbf{4.}] 
%Bei 22 Mathematikwettbewerben werden jeweils 5 Preise verteilt. Nachdem alle Wettbewerbe durchgeführt sind, bemerken die Organisatoren, dass es für jede Kombination von zwei Wett\-bewer\-ben genau einen gemeinsamen Preisträger gibt. Zeige, dass ein Teilnehmer bei allen Wettbe\-wer\-ben einen Preis gewonnen hat.
Vingt-deux concours de mathématique ont eu lieu, et pour chacun d'entre eux on a remis cinq prix. Les organisateurs s'aperçoivent alors que pour chaque paire de concours, il y a exactement un participant qui a gagné un prix dans ces deux concours. Montrer qu'un des participants a gagné un prix dans chacun des concours.

\bigskip

\item[\textbf{5.}] 
%Sei $ABC$ ein Dreieck mit $AB < AC$. Die Winkelhalbierende von $\angle BAC$ schneidet die Seite $BC$ im Punkt $D$. Sei $k$ der Kreis, der durch $D$ geht und die Seiten $AC$ und $AB$ in den Punkten $E$ respektive $F$ berührt. Sei $G$ der zweite Schnittpunkt von $k$ und $BC$ und sei $S$ der Schnittpunkt der Strecken $EG$ und $DF$. Zeige, dass $AD$ und $BS$ senkrecht aufeinander stehen.
Soit $ABC$ un triangle avec $AB < AC$. La bissectrice de $\angle BAC$ coupe le côté $BC$ en $D$. Soit $k$ le cercle qui passe par $D$ et qui est tangent aux segments $AC$ et $AB$ en $E$, respectivement $F$. Soit $G$ le deuxième point d'intersection de $k$ avec $BC$. Soit $S$ le point d'intersection de $EG$ et $DF$. Montrer que $AD$ est perpendiculaire à $BS$.
\bigskip

\end{enumerate}

\vspace{1cm}

\center{\hspace{1cm} Bonne chance!}
\end{document}
