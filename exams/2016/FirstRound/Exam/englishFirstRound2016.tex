\documentclass[language=english,style=exam]{smo}

\title{SMO - First round}
\examdate{$16^{th}$ january 2016}
\place{Bellinzone, Lausanne, Zürich}
\examtime{3 hours}

\begin{document}

\begin{enumerate}

\item[\textbf{1.}] 
%Zwei Kreise $k_1$ und $k_2$ schneiden sich in den Punkten $A$ und $C$. Sei $B$ der zweite Schnittpunkt von $k_1$ und der Tangente an $k_2$ in $A$, und sei $D$ der zweite Schnittpunkt von $k_2$ und der Tangente an $k_1$ in $C$. Zeige, dass $AD$ und $BC$ parallel sind.
Two circles $k_1$ and $k_2$ intersect at points $A$ and $C$. Let $B$ be the second point of intersection of $k_1$ and the tangent line to $k_2$ through $A$, and let $D$ be the second point of intersection of $k_2$ and the tangent line to $k_1$ through $C$. Prove that the lines $AD$ and $BC$ are parallel.

\bigskip

\item[\textbf{2.}] 
%Quirin hat $n$ Klötze mit den Höhen 1 bis $n$ und möchte diese so nebeneinander aufstellen, dass sich seine Katze von links nach rechts über sie hinwegbewegen kann. Die Katze kann dabei jeweils auf den nächsten Klotz springen, falls dieser tiefer oder um 1 höher ist. Zu Beginn wird die Katze auf den Klotz am linken Ende gesetzt.\\
%Wie viele Möglichkeiten hat Quirin, die Klötze in einer solchen Reihe aufzustellen?
Quirin has $n$ blocks, with height 1 to $n$, and he wants to place them in a line in such a way that his cat can go through them from left to right. The cat can jump from one block to the next one if either the block from which it jumps is higher, or if the difference in height is at most 1. At the beginning his cat is on the leftmost block.\\
In how many ways can Quirin place his blocks in such a line?


\textit{Remark: for $n=5$, $3-4-5-1-2$ is possible, but $1-3-4-5-2$ is not.} 

\bigskip

\item[\textbf{3.}]
%Bestimme alle natürlichen Zahlen $n$, sodass für alle Teiler $d$ von $n$ gilt:
Find all natural numbers $n$ such that for every positive divisor $d$ of $n$ we have:
\[
d+1 \div n+1.
\]

\bigskip

\item[\textbf{4.}] 
%Bei 22 Mathematikwettbewerben werden jeweils 5 Preise verteilt. Nachdem alle Wettbewerbe durchgeführt sind, bemerken die Organisatoren, dass es für jede Kombination von zwei Wett\-bewer\-ben genau einen gemeinsamen Preisträger gibt. Zeige, dass ein Teilnehmer bei allen Wettbe\-wer\-ben einen Preis gewonnen hat.
22 mathematical competitions are organised, and for each competition five contestants are given an award. After the end of all the competitions, the organizers remark that for any two competitions exactly one contestant has received an award in both competitions. Prove that one of the contestants has received an award in all the competitions.

\bigskip

\item[\textbf{5.}] 
%Sei $ABC$ ein Dreieck mit $AB < AC$. Die Winkelhalbierende von $\angle BAC$ schneidet die Seite $BC$ im Punkt $D$. Sei $k$ der Kreis, der durch $D$ geht und die Seiten $AC$ und $AB$ in den Punkten $E$ respektive $F$ berührt. Sei $G$ der zweite Schnittpunkt von $k$ und $BC$ und sei $S$ der Schnittpunkt der Strecken $EG$ und $DF$. Zeige, dass $AD$ und $BS$ senkrecht aufeinander stehen.
Let $ABC$ be a triangle with $AB <AC$. The angle bisector of $\angle BAC$ intersects the side $BC$ at $D$. Let $k$ be the circle going through $D$ which is tangent to the sides $AC$ and $AB$ at the points $E$ respectively $F$. Let $G$ be the second point of intersection of $k$ and $BC$ and let $S$ be the point of intersection of the segments $EG$ and $DF$. Prove that $AD$ and $BS$ are perpendicular.
\bigskip

\end{enumerate}

\vspace{1cm}

\center{\hspace{1cm} Good Luck!}
\end{document}
