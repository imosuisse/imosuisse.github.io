\documentclass[12pt,a4paper]{article}

\usepackage{german}
\usepackage[utf8]{inputenc}
\usepackage{amsfonts}
\usepackage{amsthm}
\usepackage{amssymb}
\usepackage{enumerate}

\usepackage{graphicx}
\usepackage{float}
%\usepackage{floatflt}
\usepackage{subfigure}

\oddsidemargin=0cm \headheight=-2.1cm
\textwidth=16.4cm \textheight=26cm

\begin{document}
\thispagestyle{empty}
\begin{figure}[h]
\includegraphics{de-smo_logo}
\end{figure}

\vspace{1cm}

\begin{center}
\Huge{\textbf{Vorrundenprüfung SMO 2016}}\\[1.5cm]
\large{Lugano, Lausanne, Zürich - 16. Januar 2016}\\[2.5cm]
\end{center}


Bitte beachtet die folgenden Punkte:

\begin{itemize}
\item Benutzt nur das zur Verfügung gestellte Papier.

\item Da wir eure Lösungen einscannen, sollen alle Blätter nur einseitig beschrieben werden. 

\item Schreibt alles auf, was ihr bei einer Aufgabe herausfindet, auch scheinbar triviale Beobachtungen können Teilpunkte geben. Und gebt alle lesbaren Blätter ab, für falsche oder widersprüchliche Aussagen werden niemals Punkte abgezogen.

\item Die Aufgaben sind nach Schwierigkeit geordnet, es kommt aber sicherlich vor, dass jemand den Kniff einer einfachen Aufgabe nicht sieht, dafür eines der schwierigen Probleme löst. Es geben alle Aufgaben gleich viele Punkte. Ihr habt genügend Zeit, euch mit jeder Aufgabe eine Zeit lang auseinanderzusetzen, wir raten euch unbedingt dies zu tun.

\item Falls ihr Fragen zu der Aufgabenstellung habt, mehr Papier benötigt, auf die Toilette müsst, oder sonst ein Anliegen besteht: hebt die Hand und wir kommen vorbei.

\item Ihr könnt die Prüfung vorzeitig abgeben und den Raum leise verlassen, jedoch nicht in der letzten halben Stunde. Falls ihr nicht mehr wisst was zu machen, empfehlen wir euch, eure Resultate kritisch zu überprüfen.

\item Damit uns beim Korrigieren nichts verloren geht, soll am Schluss auf jedem Blatt euer Name (oder mindestens ein Kürzel), die Aufgabennummer, die Seitenzahl und die totale Anzahl Seiten zu dieser Aufgabe stehen. Also z.B. "`C.F.Gauss Nr. 3 S. 2/4"'.

\item Deine korrigierte Prüfung wird dir in der nachfolgenden Woche mit beiliegendem Couvert zugeschickt. Bitte adressiere es jetzt korrekt, gross und gut leserlich mit deiner Adresse.
\end{itemize}

\end{document}
