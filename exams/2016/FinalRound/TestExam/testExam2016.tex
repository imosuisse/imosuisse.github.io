%& -job-name=testpruefung_2016_fr

\documentclass[language=german,style=exam]{smo}

\StrRight{\jobname}{4}[\mystring]
\IfSubStr*{\mystring}{fr}{\germanfalse \frenchtrue \italianfalse}{}
\IfSubStr*{\mystring}{de}{\germantrue \frenchfalse \italianfalse}{}
\IfSubStr*{\mystring}{it}{\germanfalse \frenchfalse \italiantrue}{}

\examtime{\translation{4 Stunden}{4 heures}{???}}
\examdate{\translation{10. März 2016}{10 mars 2016}{}}

\title{\translation{SMO - Weisse Prüfung}{OSM - Test blanc}{???}}
\examplace{Wila}

\begin{document}

\begin{enumerate}[label=\textbf{\arabic*.}]

\item\translation%1
{Jede nichtnegative ganze Zahl ist entweder rot oder weiss gefärbt, sodass gilt:
\begin{itemize}
\item Es gibt mindestens eine rote und eine weisse Zahl.
\item Die Summe einer roten und einer weissen Zahl ist weiss.
\item Das Produkt einer roten und einer weissen Zahl ist rot.
\end{itemize}
Zeige, dass das Produkt und die Summe zweier roter Zahlen immer rot ist.}
{Les nombres entiers positifs ou nuls sont coloriés en rouge ou en blanc, de telle manière que:
\begin{itemize}
\item Il y a au moins un nombre rouge et un nombre blanc.
\item La somme d'un nombre rouge et un nombre blanc est blanche.
\item Le produit d'un nombre rouge et un nombre blanc est blanche.
\end{itemize}
Montrer que le produit et la somme de deux nombres rouges est toujours rouge.}
{}

\bigskip\bigskip

\item\translation%2
{Soit $ABC$ un triangle non-rectangle en $A$ et $M$ le milieu de $BC$. Les tangentes en $B$ et $C$ au cercle circonscrit du triangle $ABC$ se coupent en $D$. De plus, la droite symétrique à $BC$ par rapport à $AB$ coupe la droite $AM$ en $T$. Montrer que les triangles $ABT$ et $ACD$ sont semblables.}%Sei $ABC$ mit $\angle BAC \neq 90^\circ$ und $M$ der Mittelpunkt der Strecke $BC$. Die Tangenten in $B$ und $C$ an den Umkreis des Dreiecks $ABC$ schneiden sich im Punkt $D$. Weiter }
{Soit $ABC$ un triangle non-rectangle en $A$ et $M$ le milieu de $BC$. Les tangentes en $B$ et $C$ au cercle circonscrit du triangle $ABC$ se coupent en $D$. De plus, la droite symétrique à $BC$ par rapport à $AB$ coupe la droite $AM$ en $T$. Montrer que les triangles $ABT$ et $ACD$ sont semblables.}
{}

\bigskip\bigskip

\item\translation%3
{Seien $x,y$ und $z$ reelle Zahlen mit $x<y<z<6$. Finde alle Tripel $(x,y,z)$, sodass gilt:
\[
\frac{1}{6-z} + \frac{1}{z-y} + \frac{1}{y-x} \leq x.
\]}
{Soient $x,y$ et $z$ des nombres réels vérifiant $x<y<z<6$. Trouver tous les tels triplets $(x,y,z)$ qui satisfont:
\[
\frac{1}{6-z} + \frac{1}{z-y} + \frac{1}{y-x} \leq x.
\]}
{}

\bigskip\bigskip

\item\translation%4
{Sei $n$ eine natürliche Zahl. Annalena und Romina spielen das folgende Spiel:\\
Am Anfang sind $s$ Smarties auf dem Tisch. Die beiden machen abwechslungsweise ihren Zug, wobei Annalena beginnt. Ein Zug besteht aus einer der folgenden Aktionen:
\begin{enumerate}
\item[(i)] Esse ein Smarties.
\item[(ii)] Esse eine prime Anzahl Smarties.
\item[(iii)] Esse eine durch $n$ teilbare Anzahl Smarties.
\end{enumerate}
Diejenige, die das letzte Smarties isst, gewinnt. Für welche $s$ kann Romina einen Sieg erzwingen?}
{Soit $n$ un nombre naturel. Annalena et Romina jouent au jeu suivant:\\
Au début il y a $s$ smarties sur la table. Annalena commence à jouer, puis chacune effectue tour à tour son coup. Un coup consiste en l'une des actions suivantes:
\begin{enumerate}
\item[(i)] Manger un smarties.
\item[(ii)] Manger un nombre pair de smarties.
\item[(iii)] Manger un nombre de smarties divisible par $n$.
\end{enumerate}
La gagnante est celle qui mange le dernier smarties. Pour quels $s$ Romina peut-elle forcer la victoire?}
{}

\bigskip\bigskip

\item\translation%5
{Soit $n$ un nombre naturel ayant un nombre pair de chiffres. Écrivons $n=\overline{ab}$, où $\overline{ab}$ est le nombre décimal obtenu en accolant $a$ et $b$, avec $a$ et $b$ ayant le même nombre de chiffres.On définit $d(n)=a+b$, pour autant que $b$ ne commence pas par un zéro. Par exemple, pour $n=1729$, $a=17, b=29,d(n)=17+29=46$.
Trouver tous les entiers $n\geq 1000$ tels que le nombre de diviseurs de $n$ vaut $d(n)$.}
{Soit $n$ un nombre naturel ayant un nombre pair de chiffres. Écrivons $n=\overline{ab}$, où $\overline{ab}$ est le nombre décimal obtenu en accolant $a$ et $b$, avec $a$ et $b$ ayant le même nombre de chiffres.On définit $d(n)=a+b$, pour autant que $b$ ne commence pas par un zéro. Par exemple, pour $n=1729$, $a=17, b=29,d(n)=17+29=46$.
Trouver tous les entiers $n\geq 1000$ tels que le nombre de diviseurs de $n$ vaut $d(n)$.}
{}

\end{enumerate}

\bigskip

\vspace{1cm}

\center{\translation{Viel Glück!}{Bonne chance!}{Buona fortuna!}}

\end{document}
