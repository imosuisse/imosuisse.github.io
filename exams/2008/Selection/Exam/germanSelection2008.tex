\documentclass[12pt,a4paper]{article}

\usepackage{amsfonts}
\usepackage[centertags]{amsmath}
\usepackage{german}
\usepackage{amsthm}
\usepackage{amssymb}

\leftmargin=0pt \topmargin=0pt \headheight=0in \headsep=0in \oddsidemargin=0pt \textwidth=6.5in
\textheight=8.5in

\catcode`\� = \active \catcode`\� = \active \catcode`\� = \active \catcode`\� = \active \catcode`\� = \active
\catcode`\� = \active

\def�{"A}
\def�{"a}
\def�{"O}
\def�{"o}
\def�{"U}
\def�{"u}







% Schriftabk�rzungen

\newcommand{\eps}{\varepsilon}
\renewcommand{\phi}{\varphi}
\newcommand{\Sl}{\ell}    % sch�nes l
\newcommand{\ve}{\varepsilon}  %Epsilon

\newcommand{\BA}{{\mathbb{A}}}
\newcommand{\BB}{{\mathbb{B}}}
\newcommand{\BC}{{\mathbb{C}}}
\newcommand{\BD}{{\mathbb{D}}}
\newcommand{\BE}{{\mathbb{E}}}
\newcommand{\BF}{{\mathbb{F}}}
\newcommand{\BG}{{\mathbb{G}}}
\newcommand{\BH}{{\mathbb{H}}}
\newcommand{\BI}{{\mathbb{I}}}
\newcommand{\BJ}{{\mathbb{J}}}
\newcommand{\BK}{{\mathbb{K}}}
\newcommand{\BL}{{\mathbb{L}}}
\newcommand{\BM}{{\mathbb{M}}}
\newcommand{\BN}{{\mathbb{N}}}
\newcommand{\BO}{{\mathbb{O}}}
\newcommand{\BP}{{\mathbb{P}}}
\newcommand{\BQ}{{\mathbb{Q}}}
\newcommand{\BR}{{\mathbb{R}}}
\newcommand{\BS}{{\mathbb{S}}}
\newcommand{\BT}{{\mathbb{T}}}
\newcommand{\BU}{{\mathbb{U}}}
\newcommand{\BV}{{\mathbb{V}}}
\newcommand{\BW}{{\mathbb{W}}}
\newcommand{\BX}{{\mathbb{X}}}
\newcommand{\BY}{{\mathbb{Y}}}
\newcommand{\BZ}{{\mathbb{Z}}}

\newcommand{\Fa}{{\mathfrak{a}}}
\newcommand{\Fb}{{\mathfrak{b}}}
\newcommand{\Fc}{{\mathfrak{c}}}
\newcommand{\Fd}{{\mathfrak{d}}}
\newcommand{\Fe}{{\mathfrak{e}}}
\newcommand{\Ff}{{\mathfrak{f}}}
\newcommand{\Fg}{{\mathfrak{g}}}
\newcommand{\Fh}{{\mathfrak{h}}}
\newcommand{\Fi}{{\mathfrak{i}}}
\newcommand{\Fj}{{\mathfrak{j}}}
\newcommand{\Fk}{{\mathfrak{k}}}
\newcommand{\Fl}{{\mathfrak{l}}}
\newcommand{\Fm}{{\mathfrak{m}}}
\newcommand{\Fn}{{\mathfrak{n}}}
\newcommand{\Fo}{{\mathfrak{o}}}
\newcommand{\Fp}{{\mathfrak{p}}}
\newcommand{\Fq}{{\mathfrak{q}}}
\newcommand{\Fr}{{\mathfrak{r}}}
\newcommand{\Fs}{{\mathfrak{s}}}
\newcommand{\Ft}{{\mathfrak{t}}}
\newcommand{\Fu}{{\mathfrak{u}}}
\newcommand{\Fv}{{\mathfrak{v}}}
\newcommand{\Fw}{{\mathfrak{w}}}
\newcommand{\Fx}{{\mathfrak{x}}}
\newcommand{\Fy}{{\mathfrak{y}}}
\newcommand{\Fz}{{\mathfrak{z}}}

\newcommand{\FA}{{\mathfrak{A}}}
\newcommand{\FB}{{\mathfrak{B}}}
\newcommand{\FC}{{\mathfrak{C}}}
\newcommand{\FD}{{\mathfrak{D}}}
\newcommand{\FE}{{\mathfrak{E}}}
\newcommand{\FF}{{\mathfrak{F}}}
\newcommand{\FG}{{\mathfrak{G}}}
\newcommand{\FH}{{\mathfrak{H}}}
\newcommand{\FI}{{\mathfrak{I}}}
\newcommand{\FJ}{{\mathfrak{J}}}
\newcommand{\FK}{{\mathfrak{K}}}
\newcommand{\FL}{{\mathfrak{L}}}
\newcommand{\FM}{{\mathfrak{M}}}
\newcommand{\FN}{{\mathfrak{N}}}
\newcommand{\FO}{{\mathfrak{O}}}
\newcommand{\FP}{{\mathfrak{P}}}
\newcommand{\FQ}{{\mathfrak{Q}}}
\newcommand{\FR}{{\mathfrak{R}}}
\newcommand{\FS}{{\mathfrak{S}}}
\newcommand{\FT}{{\mathfrak{T}}}
\newcommand{\FU}{{\mathfrak{U}}}
\newcommand{\FV}{{\mathfrak{V}}}
\newcommand{\FW}{{\mathfrak{W}}}
\newcommand{\FX}{{\mathfrak{X}}}
\newcommand{\FY}{{\mathfrak{Y}}}
\newcommand{\FZ}{{\mathfrak{Z}}}

\newcommand{\CA}{{\cal A}}
\newcommand{\CB}{{\cal B}}
\newcommand{\CC}{{\cal C}}
\newcommand{\CD}{{\cal D}}
\newcommand{\CE}{{\cal E}}
\newcommand{\CF}{{\cal F}}
\newcommand{\CG}{{\cal G}}
\newcommand{\CH}{{\cal H}}
\newcommand{\CI}{{\cal I}}
\newcommand{\CJ}{{\cal J}}
\newcommand{\CK}{{\cal K}}
\newcommand{\CL}{{\cal L}}
\newcommand{\CM}{{\cal M}}
\newcommand{\CN}{{\cal N}}
\newcommand{\CO}{{\cal O}}
\newcommand{\CP}{{\cal P}}
\newcommand{\CQ}{{\cal Q}}
\newcommand{\CR}{{\cal R}}
\newcommand{\CS}{{\cal S}}
\newcommand{\CT}{{\cal T}}
\newcommand{\CU}{{\cal U}}
\newcommand{\CV}{{\cal V}}
\newcommand{\CW}{{\cal W}}
\newcommand{\CX}{{\cal X}}
\newcommand{\CY}{{\cal Y}}
\newcommand{\CZ}{{\cal Z}}

% Theorem Stil

\theoremstyle{plain}
\newtheorem{lem}{Lemma}
\newtheorem{Satz}[lem]{Satz}

\theoremstyle{definition}
\newtheorem{defn}{Definition}[section]

\theoremstyle{remark}
\newtheorem{bem}{Bemerkung}    %[section]



\newcommand{\card}{\mathop{\rm card}\nolimits}
\newcommand{\Sets}{((Sets))}
\newcommand{\id}{{\rm id}}
\newcommand{\supp}{\mathop{\rm Supp}\nolimits}

\newcommand{\ord}{\mathop{\rm ord}\nolimits}
\renewcommand{\mod}{\mathop{\rm mod}\nolimits}
\newcommand{\sign}{\mathop{\rm sign}\nolimits}
\newcommand{\ggT}{\mathop{\rm ggT}\nolimits}
\newcommand{\kgV}{\mathop{\rm kgV}\nolimits}
\renewcommand{\div}{\, | \,}
\newcommand{\notdiv}{\mathopen{\mathchoice
             {\not{|}\,}
             {\not{|}\,}
             {\!\not{\:|}}
             {\not{|}}
             }}

\newcommand{\im}{\mathop{{\rm Im}}\nolimits}
\newcommand{\coim}{\mathop{{\rm coim}}\nolimits}
\newcommand{\coker}{\mathop{\rm Coker}\nolimits}
\renewcommand{\ker}{\mathop{\rm Ker}\nolimits}

\newcommand{\pRang}{\mathop{p{\rm -Rang}}\nolimits}
\newcommand{\End}{\mathop{\rm End}\nolimits}
\newcommand{\Hom}{\mathop{\rm Hom}\nolimits}
\newcommand{\Isom}{\mathop{\rm Isom}\nolimits}
\newcommand{\Tor}{\mathop{\rm Tor}\nolimits}
\newcommand{\Aut}{\mathop{\rm Aut}\nolimits}

\newcommand{\adj}{\mathop{\rm adj}\nolimits}

\newcommand{\Norm}{\mathop{\rm Norm}\nolimits}
\newcommand{\Gal}{\mathop{\rm Gal}\nolimits}
\newcommand{\Frob}{{\rm Frob}}

\newcommand{\disc}{\mathop{\rm disc}\nolimits}

\renewcommand{\Re}{\mathop{\rm Re}\nolimits}
\renewcommand{\Im}{\mathop{\rm Im}\nolimits}

\newcommand{\Log}{\mathop{\rm Log}\nolimits}
\newcommand{\Res}{\mathop{\rm Res}\nolimits}
\newcommand{\Bild}{\mathop{\rm Bild}\nolimits}

\renewcommand{\binom}[2]{\left({#1}\atop{#2}\right)}
\newcommand{\eck}[1]{\langle #1 \rangle}
\newcommand{\gaussk}[1]{\lfloor #1 \rfloor}
\newcommand{\frack}[1]{\{ #1 \}}
\newcommand{\wi}{\hspace{1pt} < \hspace{-6pt} ) \hspace{2pt}}
\newcommand{\dreieck}{\bigtriangleup}

\parindent0mm

\begin{document}

\pagestyle{empty}

\begin{center}
{\huge IMO Selektion 2008} \\
\medskip erste Pr�fung - 17. Mai 2008
\end{center}
\vspace{8mm}
Zeit: 4.5 Stunden\\
Jede Aufgabe ist 7 Punkte wert.

\vspace{15mm}

\begin{enumerate}

\item[\textbf{1.}] Finde alle Tripel $(a,b,c)$ nat�rlicher Zahlen, sodass gilt:
\[a \div bc-1, \qquad b \div ca-1, \qquad c \div ab-1.\] 

\bigskip
\bigskip

\item[\textbf{2.}] Seien $m,n$ nat�rliche Zahlen. Betrachte ein quadratisches Punktgitter aus $(2m+1) \times (2n+1)$ Punkten in der Ebene. Eine Menge von Rechtecken heisst \emph{gut}, falls folgendes gilt:
\begin{enumerate}
\item[(a)] F�r jedes der Rechtecke liegen die vier Eckpunkte auf Gitterpunkten und die Seiten parallel zu den Gitterlinien.
\item[(b)] Keine zwei der Rechtecke haben einen gemeinsamen Eckpunkt.
\end{enumerate}
Bestimme den gr�sstm�glichen Wert der Summe der Fl�chen aller Rechtecke in einer guten Menge.\\

\bigskip
\bigskip

\item[\textbf{3.}]  Sei $ABC$ ein Dreieck mit $\angle ABC \neq \angle BCA$. Der Inkreis $k$ des Dreiecks $ABC$ ber�hre die Seiten $BC$, $CA$ bzw. $AB$ in den Punkten $D$, $E$ bzw. $F$. Die Strecke $AD$ schneide $k$ ein weiteres Mal in $P$. Sei $Q$ der Schnittpunkt von $EF$ mit der Rechtwinkligen zu $AD$ durch $P$. Sei $X$ bzw. $Y$ der Schnittpunkt von $AQ$ mit $DE$ bzw. mit $DF$. Zeige, dass $A$ der Mittelpunkt der Strecke $XY$ ist.

\end{enumerate}

\pagebreak


\begin{center}
{\huge IMO Selektion 2008} \\
\medskip zweite Pr�fung - 18. Mai 2008
\end{center}
\vspace{8mm}
Zeit: 4.5 Stunden\\
Jede Aufgabe ist 7 Punkte wert.

\vspace{15mm}

\begin{enumerate}

\item[\textbf{4.}] Zwei Kreise $k_1$ und $k_2$ schneiden sich in $A$ und $B$. Sei $r$ eine Gerade durch $B$, die $k_1$ in $C$ und $k_2$ in $D$ schneidet, so dass B zwischen $C$ und $D$ liegt.
Sei $s$ die Gerade parallel zu $AD$, die $k_1$ in $E$ ber�hrt und zu $AD$ den kleinstm�glichen Abstand hat. Die Gerade $AE$ schneidet $k_2$ in $F$. Sei $t$ die Tangente zu $k_2$ durch $F$. Beweise dass gilt:
\begin{enumerate}
\item[(a)] Die Gerade $t$ ist parallel zu $AC$.
\item[(b)] Die Geraden $r$, $s$ und $t$ schneiden sich in einem Punkt.\\
\end{enumerate}


\bigskip
\bigskip

\item[\textbf{5.}] Seien $a,b,c$ positive reelle Zahlen. Beweise die folgende Ungleichung:
\[\frac{a}{\sqrt{3a+2b+c}}+\frac{b}{\sqrt{3b+2c+a}}+\frac{c}{\sqrt{3c+2a+b}} \leq \frac{1}{\sqrt{2}}\sqrt{a+b+c}.\]

\bigskip
\bigskip

\item[\textbf{6.}] Ein regul�res $2008$-Eck wird irgendwie mit $2005$ sich nicht schneidenden Diagonalen in lauter Dreiecke zerlegt. Bestimme die kleinstm�gliche Anzahl nicht gleichschenkliger Dreiecke, die in einer solchen Zerlegung auftreten k�nnen.
 
\end{enumerate}


\pagebreak


\begin{center}
{\huge IMO Selektion 2008} \\
\medskip dritte Pr�fung - 24. Mai 2008
\end{center}
\vspace{8mm}
Zeit: 4.5 Stunden\\
Jede Aufgabe ist 7 Punkte wert.

\vspace{15mm}

\begin{enumerate}
\item[\textbf{7.}]  Seien $a,b$ nat�rliche Zahlen. Zeige, dass man die ganzen Zahlen mit drei Farben f�rben kann, sodass zwei ganze Zahlen mit Differenz $a$ oder $b$ stets verschieden gef�rbt sind. \\

\bigskip
\bigskip

\item[\textbf{8.}] Sei $ABC$ ein Dreieck und $D$ ein Punkt im Innern der Strecke $BC$. Sei $X$ ein weiterer Punkt im Innern der Strecke $BC$ verschieden von $D$ und sei $Y$ der Schnittpunkt von $AX$ mit  dem Umkreis von $ABC$. Sei $P$ der zweite Schnittpunkt der Umkreise von $ABC$ und $DXY$. Beweise, dass $P$ unabh�ngig von der Wahl von $X$ ist.\\

\bigskip
\bigskip

\item[\textbf{9.}] Sei $\BR^+$ die Menge der positiven reellen Zahlen. Bestimme alle Funktionen $f: \BR^+ \rightarrow \BR^+$, sodass f�r alle $x,y>0$ gilt
\[f(x+f(y))=f(x+y)+f(y).\]

\end{enumerate}


\pagebreak


\begin{center}
{\huge IMO Selektion 2008} \\
\medskip vierte Pr�fung - 25. Mai 2008
\end{center}
\vspace{8mm}
Zeit: 4.5 Stunden\\
Jede Aufgabe ist 7 Punkte wert.

\vspace{15mm}

\begin{enumerate}
\item[\textbf{10.}] Sei $P(x)=x^4-2x^3+px+q$ ein Polynom mit reellen Koeffizienten, dessen Nullstellen alle reell sind. Zeige, dass die gr�sste dieser Nullstellen im Intervall $[1,2]$ liegt.\\

\bigskip
\bigskip


\item[\textbf{11.}] Sei $A=(a_1,a_2,\ldots,a_n)$ eine Folge ganzer Zahlen. Der \emph{Nachfolger} von $A$ ist die Folge $A'=(a'_1,a'_2,\ldots,a'_n)$ mit 
\[a'_k = |\{i<k \; | \; a_i<a_k\}|-|\{i>k \; | \; a_i>a_k\}|.\]
Sei $A_0$ eine endliche Folge ganzer Zahlen und f�r $k \geq 0$ sei $A_{k+1}=A'_k$ der Nachfolger von $A_k$. Zeige, dass eine nat�rliche Zahl $m$ existiert mit $A_m=A_{m+1}$.\\


\bigskip
\bigskip

\item[\textbf{12.}] Seien $x,y,n$ nat�rliche Zahlen mit $x \geq 3$, $n\geq 2$ und
\[x^2+5=y^n.\]
Zeige, dass jeder Primteiler $p$ von $n$ die Kongruenz $p \equiv 1$ (mod $4$) erf�llt.


\end{enumerate}


\end{document}
