\documentclass[12pt,a4paper]{article}

\usepackage{german}
\usepackage[latin1]{inputenc}
\usepackage{amsfonts}
\usepackage{amsthm}
\usepackage{amssymb}
\usepackage{enumerate}

\usepackage{graphicx}
\usepackage{float}
\usepackage{floatflt}
\usepackage{subfigure}

\oddsidemargin=0cm \headheight=-2.1cm
\textwidth=16.4cm \textheight=26cm

\newcommand{\wi}{\angle}
\newcommand{\bogen}{\widehat}
\newcommand{\grad}{^\circ}
\newcommand{\de}{\triangle}
\newcommand{\ab}{\hspace{5pt}}\parindent0pt

\renewcommand{\figurename}{Abb.}

\newcounter{fort}[subsection]
\newcounter{fort2}[subsection]
\newtheorem{bsp}[fort]{Beispiel}
\newtheorem{satz}[fort2]{Satz}

\begin{document}
\thispagestyle{empty}
\begin{figure}[h]
\includegraphics[height=3.1cm]{logo}
\end{figure}

\vspace{1cm}

\begin{center}
\Huge{\textbf{Vorrundenpr�fung SMO 2008}}\\[1.5cm]
\large{Lausanne, Z�rich - 12. Januar 2008}\\[3.5cm]
\end{center}


Bitte beachtet die folgenden Punkte:

\begin{itemize}
\item Da eure L�sungen kopiert und nach Aufgaben sortiert werden, sollen alle Bl�tter nur auf einer Seite beschrieben werden und f�r jede Aufgabe ist ein neues Blatt anzufangen. Um alles, was uns das Korrigieren erleichtert, sind wir sehr froh, hebt z.B. wichtige Stellen hervor oder markiert �berfl�ssiges.

\item Schreibt ALLES auf, was ihr bei einer Aufgabe herausfindet, auch scheinbar triviale Beobachtungen k�nnen Teilpunkte geben. Und gib ALLE lesbaren Bl�tter ab, selbst f�r falsche oder widerspr�chliche Aussagen werden niemals Punkte abgezogen.

\item Die Aufgaben sind nach Schwierigkeit geordnet, es kommt aber sicherlich vor, dass jemand den Kniff einer einfachen Aufgabe nicht sieht, daf�r eine der schwierigen Probleme perfekt l�st. Lasst euch also trotzdem nicht von der Nummer einer Aufgabe beeindrucken. Es geben alle Aufgaben gleich viele Punkte. Ihr habt gen�gend Zeit, euch mit jeder Aufgabe eine Zeit lang auseinanderzusetzen, wir raten euch unbedingt dies zu tun.

\item Man kann die Pr�fung vorzeitig abgeben und den Raum leise verlassen.

\item Damit uns beim Korrigieren nichts verloren geht, soll am Schluss auf jedem Blatt euer Name (oder mindestens ein K�rzel), die Aufgabennummer, die Seitenzahl und die totale Anzahl Seiten zu dieser Aufgabe stehen. Also z.B. "`C.F.Gauss Nr. 3 S. 2/4"'.

\item Deine korrigierte Pr�fung wird dir in der nachfolgenden Woche mit beiliegendem Couvert zugeschickt. Bitte adressiere es jetzt korrekt, gross und gut leserlich mit deiner Adresse.
\end{itemize}

\end{document}