\documentclass[language=french,style=exam]{smo}
\examtime{4 heures}
\place{Deuxième examen}
\examdate{jour de $\pi$}

\usepackage{tikz}

\title{OSM - Tour final}

\begin{document}

\begin{enumerate}

\item[\textbf{6.}] 
Nous disposons d'un échiquier $8 \times 8$. Une \emph{arête intérieure} est une arête qui sépare deux carrés unité $1 \times 1$. Nous découpons l'échiquier en dominos $1 \times 2$. Pour une arête intérieure $k$, on note $N(k)$ le nombre de découpages de l'échiquier dans lesquels l'arête $k$ est découpée. Déterminer le dernier chiffre de la somme que l'on obtient en additionnant tous les $N(k)$, où $k$ est une arête intérieure.

\bigskip

\item[\textbf{7.}] Soient $a,b,c$ des nombres réels tels que:
\[
\frac{a}{b+c}+\frac{b}{c+a}+\frac{c}{a+b} = 1.
\]
Déterminer toutes les valeurs que peut prendre l'expression:
\[
\frac{a^2}{b+c}+\frac{b^2}{c+a}+\frac{c^2}{a+b}.
\]

\bigskip

\item[\textbf{8.}] Soit $ABCD$ un trapèze, où $AB$ et $CD$ sont parallèles. Soit $P$ un point sur le côté $BC$. Montrer que les parallèles à $AP$ et $PD$ passant par $C$, respectivement $B$, se coupent sur $DA$.

\bigskip

\item[\textbf{9.}] Soit $p$ un nombre premier impair. Déterminer le nombre de $p$-uplets  $(a_1,a_2,\dots,a_p)$ de nombres naturels avec les propriétés suivantes:
\begin{enumerate}[1)]
\item $1\leq a_i\leq p$ pour tout $i=1,\dots,p$.
\item $a_1+a_2+\dots +a_p$ n'est pas divisible par $p$.
\item $a_1a_2 + a_2a_3 +\dots a_{p-1}a_p + a_pa_1$ est divisible par $p$.
\end{enumerate}

\bigskip

\item[\textbf{10.}] Déterminer le plus grand nombre naturel $n$ tel que pour tous nombres réels $a,b,c,d$ :
\[
(n+2) \sqrt{a^2+b^2} + (n+1) \sqrt{a^2+c^2} + (n+1)\sqrt{a^2+d^2} \geq n(a+b+c+d).
\]

\bigskip

\end{enumerate}

\vspace{1cm}

\center{\hspace{1cm} Bonne chance!}
\end{document}
