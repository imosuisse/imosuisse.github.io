\documentclass[language=german,style=exam]{smo}
\examtime{4 Stunden}
\place{1. Prüfung}
\examdate{13. März 2015}

\usepackage{tikz}

\title{SMO - Finalrunde}

\begin{document}

\begin{enumerate}

\item[\textbf{1.}] Sei $ABC$ ein spitzwinkliges Dreieck mit $AB\neq BC$ und Umkreis $k$. Seien $P$ und $Q$ die Schnittpunkte von $k$ mit der Winkelhalbierenden beziehungsweise der Aussenwinkelhalbierenden von $\angle CBA$. Sei $D$ der Schnittpunkt von $AC$ und $PQ$. Bestimme das Verhältnis $AD:DC$.

\bigskip

\item[\textbf{2.}] Bestimme alle Paare $(m,p)$ natürlicher Zahlen, sodass $p$ eine Primzahl und
\[2^mp^2+27 \]
die dritte Potenz einer natürlichen Zahl ist.

\bigskip

\item[\textbf{3.}] Finde alle Funktionen $f: \R \rightarrow \R$, sodass für alle $x,y \in \R$ gilt:
\[
(y+1)f(x) + f(xf(y)+f(x+y))=y.
\]

\bigskip

\item[\textbf{4.}] Gegeben seien ein Kreis $k$ und zwei Punkte $A$ und $B$ ausserhalb des Kreises. Gib an, wie man mit Zirkel und Lineal einen Kreis $\ell$ konstruieren kann, sodass $A$ und $B$ auf $\ell$ liegen und sich $k$ und $\ell$ berühren.

\bigskip

\item[\textbf{5.}] Sei $m$ eine natürliche Zahl. Auf der SMO-Wandtafel steht $2^m$ mal die Zahl $1$. In einem Schritt wählen wir zwei Zahlen $a$ und $b$ auf der Tafel und ersetzen sie beide jeweils durch $a+b$.\\
Zeige, dass nach $m2^{m-1}$ Schritten die Summe der Zahlen mindestens $4^m$ beträgt.

\bigskip

\end{enumerate}

\vspace{1cm}

\center{\hspace{1cm} Viel Glück!}
\end{document}
