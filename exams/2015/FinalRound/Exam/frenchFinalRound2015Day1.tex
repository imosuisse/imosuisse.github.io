\documentclass[language=french,style=exam]{smo}
\examtime{4 heures}
\place{Premier examen}
\examdate{13 mars 2015}

\usepackage{tikz}

\title{OSM - Tour final}

\begin{document}

\begin{enumerate}

\item[\textbf{1.}] Soient $ABC$ un triangle aigu avec $AB\neq BC$ et $k$ son cercle circonscrit. Soient $P$ et $Q$ les points d'intersection de $k$ avec la bissectrice intérieure, respectivement extérieure, de $\angle CBA$. Soit $D$ le point d'intersection de $AC$ et $PQ$. Calculer le rapport $\frac{AD}{DC}$.

\bigskip

\item[\textbf{2.}] Trouver toutes les paires $(m,p)$ de nombres naturels, telles que $p$ est un nombre premier et que
\[2^mp^2+27 \]
est le cube d'un nombre naturel.

\bigskip

\item[\textbf{3.}] Trouver toutes les fonctions $f: \R \rightarrow \R$ telles que pour tout $x,y \in \R$ :
\[
(y+1)f(x) + f(xf(y)+f(x+y))=y.
\]

\bigskip

\item[\textbf{4.}] Soient un cercle $k$ et de deux points $A$ et $B$ à l'extérieur du cercle. Donner une construction à la règle et au compas, en la justifiant, d'un cercle $\ell$ qui passe par $A$ et $B$ et qui est tangent à $k$.

\bigskip

\item[\textbf{5.}] Soit $m$ un nombre naturel. Sur le tableau de l'OSM est écrit $2^m$ fois le nombre $1$. À chaque étape, on choisit deux nombres $a$ et $b$ sur le tableau et on les remplace tous deux par $a+b$.\\
Montrer qu'après $m2^{m-1}$ étapes la somme des nombres vaut au moins $4^m$.

\bigskip

\end{enumerate}

\vspace{1cm}

\center{\hspace{1cm} Bonne chance!}
\end{document}
