\documentclass[language=german,style=exam]{smo}
\examtime{4 Stunden}
\place{2. Prüfung}
\examdate{$\pi$-Tag}

\usepackage{tikz}

\title{SMO - Finalrunde}

\begin{document}

\begin{enumerate}

\item[\textbf{6.}] 
Wir haben ein $8 \times 8$ Brett. Eine \emph{innere Kante} ist eine Kante zwischen zwei $1 \times 1$ Feldern. Wir zerschneiden das Brett in $1 \times 2$ Dominosteine. Für eine innere Kante $k$ bezeichnet $N(k)$ die Anzahl Möglichkeiten, das Brett so zu zerschneiden, dass entlang der Kante $k$ geschnitten wird. Berechne die letzte Ziffer der Summe, die wir erhalten, wenn wir alle $N(k)$ addieren, wobei $k$ eine innere Kante ist.

\bigskip

\item[\textbf{7.}] Seien $a,b,c$ reelle Zahlen, sodass gilt:
\[
\frac{a}{b+c}+\frac{b}{c+a}+\frac{c}{a+b} = 1.
\]
Bestimme alle Werte, welche folgender Ausdruck annehmen kann:
\[
\frac{a^2}{b+c}+\frac{b^2}{c+a}+\frac{c^2}{a+b}.
\]

\bigskip

\item[\textbf{8.}] Sei $ABCD$ ein Trapez, wobei $AB$ und $CD$ parallel sind. $P$ sei ein Punkt auf der Seite $BC$. Zeige, dass sich die Parallelen zu $AP$ und $PD$ durch $C$ respektive $B$ auf $DA$ schneiden.

\bigskip

\item[\textbf{9.}] Sei $p$ eine ungerade Primzahl. Bestimme die Anzahl Tupel $(a_1,a_2,\dots,a_p)$ natürlicher Zahlen mit folgenden Eigenschaften:
\begin{enumerate}[1)]
\item $1\leq a_i\leq p$ für alle $i=1,\dots,p$.
\item $a_1+a_2+\dots +a_p$ ist nicht durch $p$ teilbar.
\item $a_1a_2 + a_2a_3 +\dots a_{p-1}a_p + a_pa_1$ ist durch $p$ teilbar.
\end{enumerate}

\bigskip

\item[\textbf{10.}] Finde die grösste natürliche Zahl $n$, sodass für alle reellen Zahlen $a,b,c,d$ folgendes gilt:
\[
(n+2) \sqrt{a^2+b^2} + (n+1) \sqrt{a^2+c^2} + (n+1)\sqrt{a^2+d^2} \geq n(a+b+c+d).
\]

\bigskip

\end{enumerate}

\vspace{1cm}

\center{\hspace{1cm} Viel Glück!}
\end{document}
