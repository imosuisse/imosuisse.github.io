\documentclass[language=german,style=exam]{smo}
\examtime{4 Stunden}
\place{Wila}
\examdate{12. März 2015}

\title{SMO - Testprüfung}

\begin{document}

\begin{enumerate}

\item[\textbf{1.}] Finde alle Funktionen $f: \R \rightarrow \R$, sodass gilt:
\[
f(xy) \leq f(x+y).
\]

\bigskip

\item[\textbf{2.}] 20 Schüler haben Hunger und warten auf das Essen. Leider ist Annalena bei der Essensausgabe ein wenig langsam und daher losen sie aus, wer als erstes Essen bekommt. Dazu verteilen sie Lose mit Zahlen 1 bis 20. Wer die tiefste hat, bekommt als erstes Essen. Dies wiederholen sie nun mit den Zahlen 1 bis 19 und so weiter, bis alle Essen haben. Interessanterweise hat bei diesem Prozess niemand zweimal die selbe Zahl erhalten. Quirin hatte in der ersten Runde die Nummer 14 gezogen. Finde alle Möglichkeiten, welche Zahl er in der neunten Runde gezogen hat.

\bigskip

\item[\textbf{3.}] Finde alle Funktionen $f: \N \rightarrow \N$, sodass für alle Primzahlen $p$ gilt: 
\[
f(n)^p \equiv n \pmod{f(p)}
\]

\bigskip

\item[\textbf{4.}] Seien $a,b,c$ positive reelle Zahlen. Zeige dass gilt: 
\[
\frac{a^2}{a+b} + \frac{b^2}{b+c} \geq \frac{3a+2b-c}{4}.
\]

\bigskip

\item[\textbf{5.}] Sei $ABC$ ein Dreieck und $D$ ein Punkt im Innern der Strecke $BC$. Sei $X$ ein weiterer Punkt im Innern der Strecke $BC$ verschieden von $D$ und sei $Y$ der Schnittpunkt von $AX$ mit  dem Umkreis von $ABC$. Sei $P$ der zweite Schnittpunkt der Umkreise von $ABC$ und $DXY$. Beweise, dass $P$ unabhängig von der Wahl von $X$ ist.\\

\bigskip

\end{enumerate}

\vspace{1cm}

\center{\hspace{1cm} Viel Glück!}
\end{document}
