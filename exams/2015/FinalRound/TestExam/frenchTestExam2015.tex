\documentclass[language=french,style=exam]{smo}
\examtime{4 heures}
\place{Wila}
\examdate{12 mars 2015}

\title{OSM - Test beige}

\begin{document}

\begin{enumerate}

\item[\textbf{1.}] Trouver toutes les fonctions $f: \R \rightarrow \R$ telles que pour tous $x,y \in \R$ :
\[
f(xy) \leq f(x+y).
\]

\bigskip

\item[\textbf{2.}] 20 enfants ont faim et attendent leur dinner. Malheureusement Louis est un peu lent et il tire au sort l'ordre pour venir chercher à manger. Il distribue donc des tickets avec les numéros de 1 à 20. Celui qui a le 1, reçoit sa nourriture en premier. On redistribue ensuite des tickets avec les numéros de 1 à 19 et on répète ce même processus jusqu'à ce que tout le monde ait à manger. Curieusement, personne n'a reçu deux fois le même numéro. Horace a reçu le numéro 14 lors de la première distribution. Trouver tous les numéros que Horace a pu piocher lors du 9ème tirage.

\bigskip

\item[\textbf{3.}] Trouver toutes les fonctions $f: \N \rightarrow \N$ telles que pour tout nombre premier $p$ et $n\in \N$: 
\[
f(n)^p \equiv n \pmod{f(p)}
\]

\bigskip

\item[\textbf{4.}] Soient $a,b,c$ des nombres réels positifs. Montrer que: 
\[
\frac{a^2}{a+b} + \frac{b^2}{b+c} \geq \frac{3a+2b-c}{4}.
\]

\bigskip

\item[\textbf{5.}] Soit $ABC$ un triangle et $D$ un point sur le segment $BC$. Soit $X$ un point à l'intérieur du segment $BD$ et soit $Y$ le point d'intersection de $AX$ avec le cercle circonscrit de $ABC$. Soit $P$ le deuxième point d'intersection des cercles circonscrits de $ABC$ et de $DXY$. Montrer que $P$ est indépendant du choix de $X$.\\

\bigskip

\end{enumerate}

\vspace{1cm}

\center{\hspace{1cm} Bonne chance!}
\end{document}
