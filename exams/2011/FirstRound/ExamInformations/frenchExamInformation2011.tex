
\documentclass[12pt,a4paper]{article}

\usepackage{german}
\usepackage[latin1]{inputenc}
\usepackage{amsfonts}
\usepackage{amsthm}
\usepackage{amssymb}
\usepackage{enumerate}

\usepackage{graphicx}
\usepackage{float}
\usepackage{floatflt}
\usepackage{subfigure}

\oddsidemargin=0cm \headheight=-2.1cm
\textwidth=16.4cm \textheight=26cm

\newcommand{\wi}{\angle}
\newcommand{\bogen}{\widehat}
\newcommand{\grad}{^\circ}
\newcommand{\de}{\triangle}
\newcommand{\ab}{\hspace{5pt}}\parindent0pt

\renewcommand{\figurename}{Abb.}

\newcounter{fort}[subsection]
\newcounter{fort2}[subsection]
\newtheorem{bsp}[fort]{Beispiel}
\newtheorem{satz}[fort2]{Satz}

\begin{document}
\thispagestyle{empty}
\begin{figure}[h]
\includegraphics[height=3.1cm]{logo_fr}
\end{figure}

\vspace{1cm}

\begin{center}
\Huge{\textbf{Examen pr�liminaire OSM 2011}}\\[1.5cm]
\large{Bellinzona, Lausanne, Zurich - le 8 janvier 2011}\\[3.5cm]
\end{center}


Merci de respecter les points suivants:

\begin{itemize}
\item Vos solutions seront copi�es et tri�es par exercice. N'�crivez
donc que sur un c�t� des feuilles et commencez une nouvelle
feuille pour chaque exercice. Tous les d�tails facilitant les
corrections sont les bienvenus, essayez par exemple de mettre en
�vidence les passages importants et de marquer ce qui est superflu.

\item �crivez TOUT ce que vous trouvez, m�me des observations
triviales peuvent valoir des points. Et rendez TOUTES les feuilles
lisibles, m�me pour des assertions compl�tement fausses on n'enl�ve
jamais de points.

\item Les exercices sont tri�s par difficult� mais il arrive souvent
que quelqu'un ne voit pas la solution d'un exercice simple tout en
r�solvant parfaitement un des probl�mes difficiles. Ne vous laissez
donc pas impressionner par le num�ro d'un exercice. Tous les
exercices valent le m�me nombre de points. Vous avez assez de temps
pour vous occuper de chaque exercice et on vous conseille vivement
de le faire.

\item Il est possible de rendre l'examen avant la fin et de quitter
la salle sans faire de bruit.

\item Pour qu'on ne perde rien en corrigeant mettez � la fin de
l'examen sur chaque page votre nom (ou au moins une abr�viation), le
num�ro de la page et le nombre total de pages pour cet exercice. Par
exemple "`A.Cauchy No 3 p. 2/4"'.

\item On va t'envoyer ta copie corrig�e la semaine suivant l'examen
dans l'enveloppe ci-joint. S'il te pla�t �cris maintenant dessus ton
adresse correcte et bien lisible.

\end{itemize}

\vspace{0.5cm}

\begin{center}
Bonne chance!
\end{center}

\end{document}