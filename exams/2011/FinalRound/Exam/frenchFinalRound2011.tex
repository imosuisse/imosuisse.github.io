\documentclass[11pt,a4paper]{article}

\usepackage{amsfonts}
\usepackage[latin1]{inputenc}
\usepackage[centertags]{amsmath}
\usepackage{german}
\usepackage{amsthm}
\usepackage{amssymb}

\leftmargin=0pt \topmargin=0pt \headheight=0in \headsep=0in \oddsidemargin=0pt \textwidth=6.5in
\textheight=8.5in

\catcode`\� = \active \catcode`\� = \active \catcode`\� = \active \catcode`\� = \active \catcode`\� = \active
\catcode`\� = \active

\def�{"A}
\def�{"a}
\def�{"O}
\def�{"o}
\def�{"U}
\def�{"u}







% Schriftabk�rzungen

\newcommand{\eps}{\varepsilon}
\renewcommand{\phi}{\varphi}
\newcommand{\Sl}{\ell}    % sch�nes l
\newcommand{\ve}{\varepsilon}  %Epsilon

\newcommand{\BA}{{\mathbb{A}}}
\newcommand{\BB}{{\mathbb{B}}}
\newcommand{\BC}{{\mathbb{C}}}
\newcommand{\BD}{{\mathbb{D}}}
\newcommand{\BE}{{\mathbb{E}}}
\newcommand{\BF}{{\mathbb{F}}}
\newcommand{\BG}{{\mathbb{G}}}
\newcommand{\BH}{{\mathbb{H}}}
\newcommand{\BI}{{\mathbb{I}}}
\newcommand{\BJ}{{\mathbb{J}}}
\newcommand{\BK}{{\mathbb{K}}}
\newcommand{\BL}{{\mathbb{L}}}
\newcommand{\BM}{{\mathbb{M}}}
\newcommand{\BN}{{\mathbb{N}}}
\newcommand{\BO}{{\mathbb{O}}}
\newcommand{\BP}{{\mathbb{P}}}
\newcommand{\BQ}{{\mathbb{Q}}}
\newcommand{\BR}{{\mathbb{R}}}
\newcommand{\BS}{{\mathbb{S}}}
\newcommand{\BT}{{\mathbb{T}}}
\newcommand{\BU}{{\mathbb{U}}}
\newcommand{\BV}{{\mathbb{V}}}
\newcommand{\BW}{{\mathbb{W}}}
\newcommand{\BX}{{\mathbb{X}}}
\newcommand{\BY}{{\mathbb{Y}}}
\newcommand{\BZ}{{\mathbb{Z}}}

\newcommand{\Fa}{{\mathfrak{a}}}
\newcommand{\Fb}{{\mathfrak{b}}}
\newcommand{\Fc}{{\mathfrak{c}}}
\newcommand{\Fd}{{\mathfrak{d}}}
\newcommand{\Fe}{{\mathfrak{e}}}
\newcommand{\Ff}{{\mathfrak{f}}}
\newcommand{\Fg}{{\mathfrak{g}}}
\newcommand{\Fh}{{\mathfrak{h}}}
\newcommand{\Fi}{{\mathfrak{i}}}
\newcommand{\Fj}{{\mathfrak{j}}}
\newcommand{\Fk}{{\mathfrak{k}}}
\newcommand{\Fl}{{\mathfrak{l}}}
\newcommand{\Fm}{{\mathfrak{m}}}
\newcommand{\Fn}{{\mathfrak{n}}}
\newcommand{\Fo}{{\mathfrak{o}}}
\newcommand{\Fp}{{\mathfrak{p}}}
\newcommand{\Fq}{{\mathfrak{q}}}
\newcommand{\Fr}{{\mathfrak{r}}}
\newcommand{\Fs}{{\mathfrak{s}}}
\newcommand{\Ft}{{\mathfrak{t}}}
\newcommand{\Fu}{{\mathfrak{u}}}
\newcommand{\Fv}{{\mathfrak{v}}}
\newcommand{\Fw}{{\mathfrak{w}}}
\newcommand{\Fx}{{\mathfrak{x}}}
\newcommand{\Fy}{{\mathfrak{y}}}
\newcommand{\Fz}{{\mathfrak{z}}}

\newcommand{\FA}{{\mathfrak{A}}}
\newcommand{\FB}{{\mathfrak{B}}}
\newcommand{\FC}{{\mathfrak{C}}}
\newcommand{\FD}{{\mathfrak{D}}}
\newcommand{\FE}{{\mathfrak{E}}}
\newcommand{\FF}{{\mathfrak{F}}}
\newcommand{\FG}{{\mathfrak{G}}}
\newcommand{\FH}{{\mathfrak{H}}}
\newcommand{\FI}{{\mathfrak{I}}}
\newcommand{\FJ}{{\mathfrak{J}}}
\newcommand{\FK}{{\mathfrak{K}}}
\newcommand{\FL}{{\mathfrak{L}}}
\newcommand{\FM}{{\mathfrak{M}}}
\newcommand{\FN}{{\mathfrak{N}}}
\newcommand{\FO}{{\mathfrak{O}}}
\newcommand{\FP}{{\mathfrak{P}}}
\newcommand{\FQ}{{\mathfrak{Q}}}
\newcommand{\FR}{{\mathfrak{R}}}
\newcommand{\FS}{{\mathfrak{S}}}
\newcommand{\FT}{{\mathfrak{T}}}
\newcommand{\FU}{{\mathfrak{U}}}
\newcommand{\FV}{{\mathfrak{V}}}
\newcommand{\FW}{{\mathfrak{W}}}
\newcommand{\FX}{{\mathfrak{X}}}
\newcommand{\FY}{{\mathfrak{Y}}}
\newcommand{\FZ}{{\mathfrak{Z}}}

\newcommand{\CA}{{\cal A}}
\newcommand{\CB}{{\cal B}}
\newcommand{\CC}{{\cal C}}
\newcommand{\CD}{{\cal D}}
\newcommand{\CE}{{\cal E}}
\newcommand{\CF}{{\cal F}}
\newcommand{\CG}{{\cal G}}
\newcommand{\CH}{{\cal H}}
\newcommand{\CI}{{\cal I}}
\newcommand{\CJ}{{\cal J}}
\newcommand{\CK}{{\cal K}}
\newcommand{\CL}{{\cal L}}
\newcommand{\CM}{{\cal M}}
\newcommand{\CN}{{\cal N}}
\newcommand{\CO}{{\cal O}}
\newcommand{\CP}{{\cal P}}
\newcommand{\CQ}{{\cal Q}}
\newcommand{\CR}{{\cal R}}
\newcommand{\CS}{{\cal S}}
\newcommand{\CT}{{\cal T}}
\newcommand{\CU}{{\cal U}}
\newcommand{\CV}{{\cal V}}
\newcommand{\CW}{{\cal W}}
\newcommand{\CX}{{\cal X}}
\newcommand{\CY}{{\cal Y}}
\newcommand{\CZ}{{\cal Z}}

% Theorem Stil

\theoremstyle{plain}
\newtheorem{lem}{Lemma}
\newtheorem{Satz}[lem]{Satz}

\theoremstyle{definition}
\newtheorem{defn}{Definition}[section]

\theoremstyle{remark}
\newtheorem{bem}{Bemerkung}    %[section]



\newcommand{\card}{\mathop{\rm card}\nolimits}
\newcommand{\Sets}{((Sets))}
\newcommand{\id}{{\rm id}}
\newcommand{\supp}{\mathop{\rm Supp}\nolimits}

\newcommand{\ord}{\mathop{\rm ord}\nolimits}
\renewcommand{\mod}{\mathop{\rm mod}\nolimits}
\newcommand{\sign}{\mathop{\rm sign}\nolimits}
\newcommand{\ggT}{\mathop{\rm ggT}\nolimits}
\newcommand{\kgV}{\mathop{\rm kgV}\nolimits}
\renewcommand{\div}{\, | \,}
\newcommand{\notdiv}{\mathopen{\mathchoice
             {\not{|}\,}
             {\not{|}\,}
             {\!\not{\:|}}
             {\not{|}}
             }}

\newcommand{\im}{\mathop{{\rm Im}}\nolimits}
\newcommand{\coim}{\mathop{{\rm coim}}\nolimits}
\newcommand{\coker}{\mathop{\rm Coker}\nolimits}
\renewcommand{\ker}{\mathop{\rm Ker}\nolimits}

\newcommand{\pRang}{\mathop{p{\rm -Rang}}\nolimits}
\newcommand{\End}{\mathop{\rm End}\nolimits}
\newcommand{\Hom}{\mathop{\rm Hom}\nolimits}
\newcommand{\Isom}{\mathop{\rm Isom}\nolimits}
\newcommand{\Tor}{\mathop{\rm Tor}\nolimits}
\newcommand{\Aut}{\mathop{\rm Aut}\nolimits}

\newcommand{\adj}{\mathop{\rm adj}\nolimits}

\newcommand{\Norm}{\mathop{\rm Norm}\nolimits}
\newcommand{\Gal}{\mathop{\rm Gal}\nolimits}
\newcommand{\Frob}{{\rm Frob}}

\newcommand{\disc}{\mathop{\rm disc}\nolimits}

\renewcommand{\Re}{\mathop{\rm Re}\nolimits}
\renewcommand{\Im}{\mathop{\rm Im}\nolimits}

\newcommand{\Log}{\mathop{\rm Log}\nolimits}
\newcommand{\Res}{\mathop{\rm Res}\nolimits}
\newcommand{\Bild}{\mathop{\rm Bild}\nolimits}

\renewcommand{\binom}[2]{\left({#1}\atop{#2}\right)}
\newcommand{\eck}[1]{\langle #1 \rangle}
\newcommand{\wi}{\hspace{1pt} < \hspace{-6pt} ) \hspace{2pt}}


\begin{document}

\pagestyle{empty}

\begin{center}
{\huge OSM Tour final 2011} \\
\medskip premier examen - le 11 mars 2011
\end{center}
\vspace{8mm}
Dur�e : 4 heures\\
Chaque exercice vaut 7 points.

\vspace{15mm}

\begin{enumerate}
\item[\textbf{1.}]
Lors d'une f�te, 2011 personnes sont assises � une table ronde avec un verre de sirop � la menthe sucr�e dans la main. Pendant chaque unit� de temps, un nombre quelconque de personnes fait sant� en respectant les r�gles suivantes :

(a) Pendant une unit� de temps, chacun ne peut faire sant� qu'avec une autre personne.

(b) On ne croise pas en faisant sant�.

Combien d'unit�s de temps sont n�cessaires, au minimum, pour que tout le monde aie fait sant� avec tout le monde ?


\bigskip

\item[\textbf{2.}] 
Soit $ABC$ un triangle aigu. Soient $D$, $E$ et $F$ des points sur $BC$, $CA$ et $AB$, tels que :
$$\angle AFE=\angle BFD,\ \ \angle BDF=\angle CDE, \ \ \angle CED=\angle AEF$$
Montrer que $D$, $E$ et $F$ sont les pieds des hauteurs du triangle.

\bigskip

\item[\textbf{3.}] Trouver la plus petite valeur de l'expression 
$$ \left| 2011^m - 45^n \right| $$ pour des nombres naturels $m$ et $n$.
\bigskip

\item[\textbf{4.}] Trouver toutes les fonctions $f: \mathbb{R}^{+} \rightarrow \mathbb{R}^{+}$ telles que pour tout $a, b, c, d > 0$ avec $abcd=1$ on a :
$$(f(a)+f(b))(f(c)+f(d))=(a+b)(c+d)$$

\bigskip

\item[\textbf{5.}] 
Les tangentes en $A$ et $B$ du cercle circonscrit du triangle $ABC$ se coupent en $T$. Le cercle passant par  $A$, $B$ et $T$ recoupe $BC$ et $AC$ en $D$ et $E$. $CT$ coupe $BE$ en $F$. On suppose que $D$ est le milieu de $BC$. Calculer le rapport $BF : FE$.


\end{enumerate}
\bigskip
\begin{center}Bonne chance !\end{center}
\newpage

\begin{center}
{\huge OSM Tour final 2011} \\
\medskip deuxi�me examen - le 12 mars 2011
\end{center}
\vspace{8mm}
Dur�e: 4 heures\\
Chaque exercice vaut 7 points.

\vspace{15mm}

\begin{enumerate}
\item[\textbf{6.}] Soient $a,b,c,d>0$ des nombres r�els positifs avec $a+b+c+d=1$. Montrer que
$$\frac{2}{(a+b)(c+d)}\leq\frac{1}{\sqrt{ab}}+\frac{1}{\sqrt{cd}}.$$

\bigskip

\item[\textbf{7.}] Trouver tous les nombres $z\in \BZ$, tels que 
$$ 2^{z}+2=r^{2} $$ 
o� $r\in \BQ$ est un nombre rationel.

\bigskip

\item[\textbf{8.}] Soit $ABCD$ un parall�logramme et $H$ l'orthocentre du triangle $ABC$. La parall�le � $AB$ passant par $H$ coupe $BC$ en $P$ et $AD$ en $Q$. La parall�le � $BC$ passant par $H$ coupe $AB$ en $R$ et $CD$ en $S$. Montrer que $P$, $Q$, $R$ et $S$ sont sur un cercle. 

\bigskip

\item[\textbf{9.}] Soit $n$ un nombre naturel. Soit $f(n)$ le nombre de diviseurs de $n$ qui se terminent par le chiffre $1$ ou $9$ et soit $g(n)$ le nombre de diviseurs de $n$ qui se terminent avec le chiffre $3$ ou $7$. Montrer que $f(n) \geq g(n)$ pour tout $n\in\mathbb{N}$.

\bigskip

\item[\textbf{10.}] Sur chaque case d'un �chiquier, il y a deux punaises. Chaque punaise se d�place sur une case adjacente. Deux punaises qui se trouvent sur la m�me case se d�placent sur deux cases distinctes. Quel est le nombre maximal de cases qui peuvent �tre vide apr�s le d�placement?


\bigskip

\end{enumerate}
\bigskip
\begin{center}Bonne chance !\end{center}
\newpage
\end{document}
