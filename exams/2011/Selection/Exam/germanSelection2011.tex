\documentclass[11pt,a4paper]{article}

\usepackage{amsfonts}
\usepackage{amsmath}
%\usepackage[centertags]{amsmath}
\usepackage{german}
\usepackage{amsthm}
\usepackage{amssymb}

\leftmargin=0pt \topmargin=0pt \headheight=0in \headsep=0in \oddsidemargin=0pt \textwidth=6.5in
\textheight=8.5in

\catcode`\� = \active \catcode`\� = \active \catcode`\� = \active \catcode`\� = \active \catcode`\� = \active
\catcode`\� = \active

\def�{"A}
\def�{"a}
\def�{"O}
\def�{"o}
\def�{"U}
\def�{"u}







% Schriftabk�rzungen

\newcommand{\eps}{\varepsilon}
\renewcommand{\phi}{\varphi}
\newcommand{\Sl}{\ell}    % sch�nes l
\newcommand{\ve}{\varepsilon}  %Epsilon

\newcommand{\BA}{{\mathbb{A}}}
\newcommand{\BB}{{\mathbb{B}}}
\newcommand{\BC}{{\mathbb{C}}}
\newcommand{\BD}{{\mathbb{D}}}
\newcommand{\BE}{{\mathbb{E}}}
\newcommand{\BF}{{\mathbb{F}}}
\newcommand{\BG}{{\mathbb{G}}}
\newcommand{\BH}{{\mathbb{H}}}
\newcommand{\BI}{{\mathbb{I}}}
\newcommand{\BJ}{{\mathbb{J}}}
\newcommand{\BK}{{\mathbb{K}}}
\newcommand{\BL}{{\mathbb{L}}}
\newcommand{\BM}{{\mathbb{M}}}
\newcommand{\BN}{{\mathbb{N}}}
\newcommand{\BO}{{\mathbb{O}}}
\newcommand{\BP}{{\mathbb{P}}}
\newcommand{\BQ}{{\mathbb{Q}}}
\newcommand{\BR}{{\mathbb{R}}}
\newcommand{\BS}{{\mathbb{S}}}
\newcommand{\BT}{{\mathbb{T}}}
\newcommand{\BU}{{\mathbb{U}}}
\newcommand{\BV}{{\mathbb{V}}}
\newcommand{\BW}{{\mathbb{W}}}
\newcommand{\BX}{{\mathbb{X}}}
\newcommand{\BY}{{\mathbb{Y}}}
\newcommand{\BZ}{{\mathbb{Z}}}

\newcommand{\Fa}{{\mathfrak{a}}}
\newcommand{\Fb}{{\mathfrak{b}}}
\newcommand{\Fc}{{\mathfrak{c}}}
\newcommand{\Fd}{{\mathfrak{d}}}
\newcommand{\Fe}{{\mathfrak{e}}}
\newcommand{\Ff}{{\mathfrak{f}}}
\newcommand{\Fg}{{\mathfrak{g}}}
\newcommand{\Fh}{{\mathfrak{h}}}
\newcommand{\Fi}{{\mathfrak{i}}}
\newcommand{\Fj}{{\mathfrak{j}}}
\newcommand{\Fk}{{\mathfrak{k}}}
\newcommand{\Fl}{{\mathfrak{l}}}
\newcommand{\Fm}{{\mathfrak{m}}}
\newcommand{\Fn}{{\mathfrak{n}}}
\newcommand{\Fo}{{\mathfrak{o}}}
\newcommand{\Fp}{{\mathfrak{p}}}
\newcommand{\Fq}{{\mathfrak{q}}}
\newcommand{\Fr}{{\mathfrak{r}}}
\newcommand{\Fs}{{\mathfrak{s}}}
\newcommand{\Ft}{{\mathfrak{t}}}
\newcommand{\Fu}{{\mathfrak{u}}}
\newcommand{\Fv}{{\mathfrak{v}}}
\newcommand{\Fw}{{\mathfrak{w}}}
\newcommand{\Fx}{{\mathfrak{x}}}
\newcommand{\Fy}{{\mathfrak{y}}}
\newcommand{\Fz}{{\mathfrak{z}}}

\newcommand{\FA}{{\mathfrak{A}}}
\newcommand{\FB}{{\mathfrak{B}}}
\newcommand{\FC}{{\mathfrak{C}}}
\newcommand{\FD}{{\mathfrak{D}}}
\newcommand{\FE}{{\mathfrak{E}}}
\newcommand{\FF}{{\mathfrak{F}}}
\newcommand{\FG}{{\mathfrak{G}}}
\newcommand{\FH}{{\mathfrak{H}}}
\newcommand{\FI}{{\mathfrak{I}}}
\newcommand{\FJ}{{\mathfrak{J}}}
\newcommand{\FK}{{\mathfrak{K}}}
\newcommand{\FL}{{\mathfrak{L}}}
\newcommand{\FM}{{\mathfrak{M}}}
\newcommand{\FN}{{\mathfrak{N}}}
\newcommand{\FO}{{\mathfrak{O}}}
\newcommand{\FP}{{\mathfrak{P}}}
\newcommand{\FQ}{{\mathfrak{Q}}}
\newcommand{\FR}{{\mathfrak{R}}}
\newcommand{\FS}{{\mathfrak{S}}}
\newcommand{\FT}{{\mathfrak{T}}}
\newcommand{\FU}{{\mathfrak{U}}}
\newcommand{\FV}{{\mathfrak{V}}}
\newcommand{\FW}{{\mathfrak{W}}}
\newcommand{\FX}{{\mathfrak{X}}}
\newcommand{\FY}{{\mathfrak{Y}}}
\newcommand{\FZ}{{\mathfrak{Z}}}

\newcommand{\CA}{{\cal A}}
\newcommand{\CB}{{\cal B}}
\newcommand{\CC}{{\cal C}}
\newcommand{\CD}{{\cal D}}
\newcommand{\CE}{{\cal E}}
\newcommand{\CF}{{\cal F}}
\newcommand{\CG}{{\cal G}}
\newcommand{\CH}{{\cal H}}
\newcommand{\CI}{{\cal I}}
\newcommand{\CJ}{{\cal J}}
\newcommand{\CK}{{\cal K}}
\newcommand{\CL}{{\cal L}}
\newcommand{\CM}{{\cal M}}
\newcommand{\CN}{{\cal N}}
\newcommand{\CO}{{\cal O}}
\newcommand{\CP}{{\cal P}}
\newcommand{\CQ}{{\cal Q}}
\newcommand{\CR}{{\cal R}}
\newcommand{\CS}{{\cal S}}
\newcommand{\CT}{{\cal T}}
\newcommand{\CU}{{\cal U}}
\newcommand{\CV}{{\cal V}}
\newcommand{\CW}{{\cal W}}
\newcommand{\CX}{{\cal X}}
\newcommand{\CY}{{\cal Y}}
\newcommand{\CZ}{{\cal Z}}

% Theorem Stil

\theoremstyle{plain}
\newtheorem{lem}{Lemma}
\newtheorem{Satz}[lem]{Satz}

\theoremstyle{definition}
\newtheorem{defn}{Definition}[section]

\theoremstyle{remark}
\newtheorem{bem}{Bemerkung}    %[section]



\newcommand{\card}{\mathop{\rm card}\nolimits}
\newcommand{\Sets}{((Sets))}
\newcommand{\id}{{\rm id}}
\newcommand{\supp}{\mathop{\rm Supp}\nolimits}

\newcommand{\ord}{\mathop{\rm ord}\nolimits}
\renewcommand{\mod}{\mathop{\rm mod}\nolimits}
\newcommand{\sign}{\mathop{\rm sign}\nolimits}
\newcommand{\ggT}{\mathop{\rm ggT}\nolimits}
\newcommand{\kgV}{\mathop{\rm kgV}\nolimits}
\renewcommand{\div}{\, | \,}
\newcommand{\notdiv}{\mathopen{\mathchoice
             {\not{|}\,}
             {\not{|}\,}
             {\!\not{\:|}}
             {\not{|}}
             }}

\newcommand{\im}{\mathop{{\rm Im}}\nolimits}
\newcommand{\coim}{\mathop{{\rm coim}}\nolimits}
\newcommand{\coker}{\mathop{\rm Coker}\nolimits}
\renewcommand{\ker}{\mathop{\rm Ker}\nolimits}

\newcommand{\pRang}{\mathop{p{\rm -Rang}}\nolimits}
\newcommand{\End}{\mathop{\rm End}\nolimits}
\newcommand{\Hom}{\mathop{\rm Hom}\nolimits}
\newcommand{\Isom}{\mathop{\rm Isom}\nolimits}
\newcommand{\Tor}{\mathop{\rm Tor}\nolimits}
\newcommand{\Aut}{\mathop{\rm Aut}\nolimits}

\newcommand{\adj}{\mathop{\rm adj}\nolimits}

\newcommand{\Norm}{\mathop{\rm Norm}\nolimits}
\newcommand{\Gal}{\mathop{\rm Gal}\nolimits}
\newcommand{\Frob}{{\rm Frob}}

\newcommand{\disc}{\mathop{\rm disc}\nolimits}

\renewcommand{\Re}{\mathop{\rm Re}\nolimits}
\renewcommand{\Im}{\mathop{\rm Im}\nolimits}

\newcommand{\Log}{\mathop{\rm Log}\nolimits}
\newcommand{\Res}{\mathop{\rm Res}\nolimits}
\newcommand{\Bild}{\mathop{\rm Bild}\nolimits}

\renewcommand{\binom}[2]{\left({#1}\atop{#2}\right)}
\newcommand{\eck}[1]{\langle #1 \rangle}
\newcommand{\gaussk}[1]{\lfloor #1 \rfloor}
\newcommand{\frack}[1]{\{ #1 \}}
\newcommand{\wi}{\hspace{1pt} < \hspace{-6pt} ) \hspace{2pt}}
\newcommand{\dreieck}{\bigtriangleup}

\parindent0mm

\begin{document}

\pagestyle{empty}

\begin{center}
{\huge IMO Selektion 2011} \\
\medskip erste Pr�fung - 7. Mai 2011
\end{center}
\vspace{8mm}
Zeit: 4.5 Stunden\\
Jede Aufgabe ist 7 Punkte wert.

\vspace{15mm}

\begin{enumerate}

\item[\textbf{1.}] Finde alle Paare von Primzahlen $(p,q)$ mit $3\not|p+1$ so dass

$$ \frac{p^3 + 1}{q}$$

das Quadrat einer nat�rlichen Zahl ist.

\bigskip
\bigskip

\item[\textbf{2.}] Die Gerade $g$ schneide den Kreis $k$ in den Punkten $A$ und $B$. Die Mittelsenkrechte der Strecke $AB$ schneide $k$ noch einmal in $C$ und $D$. Sei nun $P$ ein weiterer Punkt auf $g$, der ausserhalb von $k$ liegt. Die Parallelen zu $CA$ und $CB$ durch $P$ schneiden die Geraden $CB$ und $CA$ in den Punkten $X$ und $Y$. Beweise, dass $XY$ senkrecht auf $PD$ steht.


\bigskip


\item[\textbf{3.}] Betrachte ein Spielbrett mit ungeraden Seitenl�ngen, das in Einheitsquadrate aufgeteilt ist. Das Brett ohne ein Eckfeld wird irgendwie mit Dominos bedeckt. Man kann nun in einem Zug ein Domino in L�ngsrichtung um eins verschieben, sodass das vorher leere Feld bedeckt wird, daf�r ein neues (zwei Felder davon entfernt) frei wird. Beweise, dass das leere Feld mit einer Folge von Z�gen in jede beliebige Ecke des Brettes verschoben werden kann.\\
\textit{Bemerkung:} Ein Domino besteht aus aus zwei Einheitsquadraten mit einer gemeinsamen Seite.\\

\end{enumerate}
\bigskip
\begin{center}
Viel Gl�ck !
\end{center}
\pagebreak


\begin{center}
{\huge IMO Selektion 2011} \\
\medskip zweite Pr�fung - 8. Mai 2011
\end{center}
\vspace{8mm}
Zeit: 4.5 Stunden\\
Jede Aufgabe ist 7 Punkte wert.

\vspace{15mm}

\begin{enumerate}

\item[\textbf{4.}] Sei $n$ ein nat�rliche Zahl. In einem Affenk�fig mit $n$ Affen stehen $n$ Kletterstangen.
Damit die Affen etwas Bewegung bekommen platzieren die W�rter zur F�tterung jeweils eine Banane oben an jeder Stange. Zus�tzlich verbinden sie die Stangen mit einer endlichen Anzahl Seile, sodass zwei verschiedene Seilenden an verschiedenen Punkten festgemacht werden. Wenn ein Affe eine Stange hochklettert und ein Seil findet, kann er nicht widerstehen und wird sich �ber das Seil hangeln bevor er seinen Aufstieg fortsetzt. Jeder Affe startet bei einer anderen Stange. Zeige, dass jeder Affe einen Banane kriegt.

\bigskip
\bigskip

\item[\textbf{5.}] Finde nat�rliche Zahlen $a,b,c$, so dass die Quersumme von $a+b,b+c$ und $c+a$ jeweils kleiner als 5 ist, die Quersumme von $a+b+c$ aber gr�sser als 50.

\bigskip
\bigskip

\item[\textbf{6.}]

Finde alle Funktionen $ \ f \  : \mathbb{Q}^+ \rightarrow \mathbb{Q}^+$ so dass f�r alle positiven rationalen Zahlen $x,y$ gilt

$$f(f(x)^2y) = x^3f(xy). $$
 
\end{enumerate}

\bigskip
\begin{center}
Viel Gl�ck !
\end{center}

\pagebreak


\begin{center}
{\huge IMO Selektion 2011} \\
\medskip dritte Pr�fung - 21. Mai 2011
\end{center}
\vspace{8mm}
Zeit: 4.5 Stunden\\
Jede Aufgabe ist 7 Punkte wert.

\vspace{15mm}

\begin{enumerate}
\item[\textbf{7.}] 

Finde alle Polynome $P\neq 0$ mit reellen Koeffizienten, welche die folgende Bedingung erf�llen:
$$
P(P(k))=P(k)^2 \text{ f�r } k=0,1,2,\dots,(\deg P)^2
$$

\bigskip
\bigskip

\item[\textbf{8.}]  

Zeige, dass es mehr als $10^{13}$ M�glichkeiten gibt, $81$ K�nige so auf einem $18\times 18$ Schachbrett zu platzieren, dass sich keine zwei K�nige attackieren.\\
\textit{Bemerkung: }Zwei K�nige k�nnen sich attackieren, falls die Felder, auf denen sie stehen, eine gemeinsame Seite oder eine gemeinsame Ecke besitzen.

\bigskip
\bigskip

\item[\textbf{9.}] 

In einem Dreieck $ABC$ mit $AB\neq AC$ sei $D$ die Projektion von $A$ auf $BC$. Ferner seien $E,F$ die Mittelpunkte der Strecken $AD$ bzw. $BC$ und $G$ die Projektion von $B$ auf $AF$. Zeige, dass die Gerade $EF$ die Tangente im Punkt $F$ an den Umkreis des Dreiecks $GFC$ ist.

\end{enumerate}

\bigskip
\begin{center}
Viel Gl�ck !
\end{center}

\pagebreak


\begin{center}
{\huge IMO Selektion 2011} \\
\medskip vierte Pr�fung - 22. Mai 2011
\end{center}
\vspace{8mm}
Zeit: 4.5 Stunden\\
Jede Aufgabe ist 7 Punkte wert.

\vspace{15mm}

\begin{enumerate}
\item[\textbf{10.}] 

Sei $ABCD$ ein Quadrat und $M$ ein Punkt im Innern der Strecke $BC$. Die Winkelhalbierende des Winkels $\angle BAM$ schneide die Strecke $BC$ im Punkt $E$. Ferner schneide die Winkelhalbierende des Winkels $\angle MAD$ die Gerade $CD$ im Punkt $F$. Zeige, dass $AM$ und $EF$ senkrecht aufeinander stehen.

\bigskip
\bigskip


\item[\textbf{11.}] 

Seien $x_1,\dots,x_{8}\ge 0$ reelle Zahlen, sodass f�r $ i=1,\dots, 8$ gilt $x_{i}+x_{i+1}+x_{i+2}\le 1$, wobei  $x_{9}=x_{1}$ und $x_{10}=x_2$. Beweise die Ungleichung
$$
\sum_{i=1}^{8}x_{i}x_{i+2}\le 1
$$
und finde alle F�lle in denen Gleichheit herrscht.

\bigskip
\bigskip

\item[\textbf{12.}] Sei $a > 1$ eine nat�rliche Zahl und seien $f$ und $g$ Polynome mit ganzzahligen Koeffizienten. Angenommen es gibt eine nat�rliche Zahl $n_0$, so dass $g(n) >0$ f�r alle $n \geq n_0$ und 
$$f(n)\ |\ a^{g(n)} - 1 \ \ \ \ \ \text{f�r alle} \ n \geq n_0.$$

Zeige, dass dann $f$ konstant sein muss.



\end{enumerate}

\bigskip
\begin{center}
Viel Gl�ck !
\end{center}


\end{document}
