\documentclass[12pt,a4paper]{article}

\usepackage{amsfonts}
\usepackage[centertags]{amsmath}
\usepackage{amsthm}
\usepackage{amssymb}

\usepackage{german}
\usepackage[latin1]{inputenc}

\usepackage{graphicx}

\leftmargin=0pt \topmargin=0pt \headheight=0in \headsep=0in \oddsidemargin=0pt \textwidth=6.5in
\textheight=8.5in


% Schriftabk�rzungen

\newcommand{\eps}{\varepsilon}
\renewcommand{\phi}{\varphi}
\newcommand{\Sl}{\ell}    % sch�nes l
\newcommand{\ve}{\varepsilon}  %Epsilon

\newcommand{\BA}{{\mathbb{A}}}
\newcommand{\BB}{{\mathbb{B}}}
\newcommand{\BC}{{\mathbb{C}}}
\newcommand{\BD}{{\mathbb{D}}}
\newcommand{\BE}{{\mathbb{E}}}
\newcommand{\BF}{{\mathbb{F}}}
\newcommand{\BG}{{\mathbb{G}}}
\newcommand{\BH}{{\mathbb{H}}}
\newcommand{\BI}{{\mathbb{I}}}
\newcommand{\BJ}{{\mathbb{J}}}
\newcommand{\BK}{{\mathbb{K}}}
\newcommand{\BL}{{\mathbb{L}}}
\newcommand{\BM}{{\mathbb{M}}}
\newcommand{\BN}{{\mathbb{N}}}
\newcommand{\BO}{{\mathbb{O}}}
\newcommand{\BP}{{\mathbb{P}}}
\newcommand{\BQ}{{\mathbb{Q}}}
\newcommand{\BR}{{\mathbb{R}}}
\newcommand{\BS}{{\mathbb{S}}}
\newcommand{\BT}{{\mathbb{T}}}
\newcommand{\BU}{{\mathbb{U}}}
\newcommand{\BV}{{\mathbb{V}}}
\newcommand{\BW}{{\mathbb{W}}}
\newcommand{\BX}{{\mathbb{X}}}
\newcommand{\BY}{{\mathbb{Y}}}
\newcommand{\BZ}{{\mathbb{Z}}}

\newcommand{\Fa}{{\mathfrak{a}}}
\newcommand{\Fb}{{\mathfrak{b}}}
\newcommand{\Fc}{{\mathfrak{c}}}
\newcommand{\Fd}{{\mathfrak{d}}}
\newcommand{\Fe}{{\mathfrak{e}}}
\newcommand{\Ff}{{\mathfrak{f}}}
\newcommand{\Fg}{{\mathfrak{g}}}
\newcommand{\Fh}{{\mathfrak{h}}}
\newcommand{\Fi}{{\mathfrak{i}}}
\newcommand{\Fj}{{\mathfrak{j}}}
\newcommand{\Fk}{{\mathfrak{k}}}
\newcommand{\Fl}{{\mathfrak{l}}}
\newcommand{\Fm}{{\mathfrak{m}}}
\newcommand{\Fn}{{\mathfrak{n}}}
\newcommand{\Fo}{{\mathfrak{o}}}
\newcommand{\Fp}{{\mathfrak{p}}}
\newcommand{\Fq}{{\mathfrak{q}}}
\newcommand{\Fr}{{\mathfrak{r}}}
\newcommand{\Fs}{{\mathfrak{s}}}
\newcommand{\Ft}{{\mathfrak{t}}}
\newcommand{\Fu}{{\mathfrak{u}}}
\newcommand{\Fv}{{\mathfrak{v}}}
\newcommand{\Fw}{{\mathfrak{w}}}
\newcommand{\Fx}{{\mathfrak{x}}}
\newcommand{\Fy}{{\mathfrak{y}}}
\newcommand{\Fz}{{\mathfrak{z}}}

\newcommand{\FA}{{\mathfrak{A}}}
\newcommand{\FB}{{\mathfrak{B}}}
\newcommand{\FC}{{\mathfrak{C}}}
\newcommand{\FD}{{\mathfrak{D}}}
\newcommand{\FE}{{\mathfrak{E}}}
\newcommand{\FF}{{\mathfrak{F}}}
\newcommand{\FG}{{\mathfrak{G}}}
\newcommand{\FH}{{\mathfrak{H}}}
\newcommand{\FI}{{\mathfrak{I}}}
\newcommand{\FJ}{{\mathfrak{J}}}
\newcommand{\FK}{{\mathfrak{K}}}
\newcommand{\FL}{{\mathfrak{L}}}
\newcommand{\FM}{{\mathfrak{M}}}
\newcommand{\FN}{{\mathfrak{N}}}
\newcommand{\FO}{{\mathfrak{O}}}
\newcommand{\FP}{{\mathfrak{P}}}
\newcommand{\FQ}{{\mathfrak{Q}}}
\newcommand{\FR}{{\mathfrak{R}}}
\newcommand{\FS}{{\mathfrak{S}}}
\newcommand{\FT}{{\mathfrak{T}}}
\newcommand{\FU}{{\mathfrak{U}}}
\newcommand{\FV}{{\mathfrak{V}}}
\newcommand{\FW}{{\mathfrak{W}}}
\newcommand{\FX}{{\mathfrak{X}}}
\newcommand{\FY}{{\mathfrak{Y}}}
\newcommand{\FZ}{{\mathfrak{Z}}}

\newcommand{\CA}{{\cal A}}
\newcommand{\CB}{{\cal B}}
\newcommand{\CC}{{\cal C}}
\newcommand{\CD}{{\cal D}}
\newcommand{\CE}{{\cal E}}
\newcommand{\CF}{{\cal F}}
\newcommand{\CG}{{\cal G}}
\newcommand{\CH}{{\cal H}}
\newcommand{\CI}{{\cal I}}
\newcommand{\CJ}{{\cal J}}
\newcommand{\CK}{{\cal K}}
\newcommand{\CL}{{\cal L}}
\newcommand{\CM}{{\cal M}}
\newcommand{\CN}{{\cal N}}
\newcommand{\CO}{{\cal O}}
\newcommand{\CP}{{\cal P}}
\newcommand{\CQ}{{\cal Q}}
\newcommand{\CR}{{\cal R}}
\newcommand{\CS}{{\cal S}}
\newcommand{\CT}{{\cal T}}
\newcommand{\CU}{{\cal U}}
\newcommand{\CV}{{\cal V}}
\newcommand{\CW}{{\cal W}}
\newcommand{\CX}{{\cal X}}
\newcommand{\CY}{{\cal Y}}
\newcommand{\CZ}{{\cal Z}}

% Theorem Stil

\theoremstyle{plain}
\newtheorem{lem}{Lemma}
\newtheorem{Satz}[lem]{Satz}

\theoremstyle{definition}
\newtheorem{defn}{Definition}[section]

\theoremstyle{remark}
\newtheorem{bem}{Bemerkung}    %[section]



\newcommand{\card}{\mathop{\rm card}\nolimits}
\newcommand{\Sets}{((Sets))}
\newcommand{\id}{{\rm id}}
\newcommand{\supp}{\mathop{\rm Supp}\nolimits}

\newcommand{\ord}{\mathop{\rm ord}\nolimits}
\renewcommand{\mod}{\mathop{\rm mod}\nolimits}
\newcommand{\sign}{\mathop{\rm sign}\nolimits}
\newcommand{\ggT}{\mathop{\rm ggT}\nolimits}
\newcommand{\kgV}{\mathop{\rm kgV}\nolimits}
\renewcommand{\div}{\, | \,}
\newcommand{\notdiv}{\mathopen{\mathchoice
             {\not{|}\,}
             {\not{|}\,}
             {\!\not{\:|}}
             {\not{|}}
             }}

\newcommand{\im}{\mathop{{\rm Im}}\nolimits}
\newcommand{\coim}{\mathop{{\rm coim}}\nolimits}
\newcommand{\coker}{\mathop{\rm Coker}\nolimits}
\renewcommand{\ker}{\mathop{\rm Ker}\nolimits}

\newcommand{\pRang}{\mathop{p{\rm -Rang}}\nolimits}
\newcommand{\End}{\mathop{\rm End}\nolimits}
\newcommand{\Hom}{\mathop{\rm Hom}\nolimits}
\newcommand{\Isom}{\mathop{\rm Isom}\nolimits}
\newcommand{\Tor}{\mathop{\rm Tor}\nolimits}
\newcommand{\Aut}{\mathop{\rm Aut}\nolimits}

\newcommand{\adj}{\mathop{\rm adj}\nolimits}

\newcommand{\Norm}{\mathop{\rm Norm}\nolimits}
\newcommand{\Gal}{\mathop{\rm Gal}\nolimits}
\newcommand{\Frob}{{\rm Frob}}

\newcommand{\disc}{\mathop{\rm disc}\nolimits}

\renewcommand{\Re}{\mathop{\rm Re}\nolimits}
\renewcommand{\Im}{\mathop{\rm Im}\nolimits}

\newcommand{\Log}{\mathop{\rm Log}\nolimits}
\newcommand{\Res}{\mathop{\rm Res}\nolimits}
\newcommand{\Bild}{\mathop{\rm Bild}\nolimits}

\renewcommand{\binom}[2]{\left({#1}\atop{#2}\right)}
\newcommand{\eck}[1]{\langle #1 \rangle}
\newcommand{\wi}{\hspace{1pt} < \hspace{-6pt} ) \hspace{2pt}}


\begin{document}

\pagestyle{empty}

\begin{center}
{\huge SMO -- Turno finale 2010} \\
\medskip primo esame - 12 marzo 2010
\end{center}
\vspace{8mm}
Durata: 4 ore\\
Ogni esercizio vale 7 punti.

\vspace{15mm}

\begin{enumerate}
\item[\textbf{1.}] Tre monete sono posizionate su dei punti interi sulla retta reale. A ogni tappa del gioco scegli due monete e spostane una di $1$ a destra e l'altra di $1$ a sinistra. Per quali posizioni iniziali � possibile spostare tutte le monete in un solo punto con una sequenza di tappe.

\bigskip

\item[\textbf{2.}] Sia $ABC$ un triangolo, $AB \neq AC$ e $I$ il centro del suo cerchio inscritto. Siano $D$, $E$ e rispettivamente $F$ i punti sui lati $BC$, $CA$ e rispettivamente $AB$ toccati dal cerchio inscritto. Sia $M$ il punto medio di $EF$. Sia $P$ il punto di intersezione (diverso da $D$) della retta $AD$ con il cerchio inscritto. Dimostra che $PMID$ � un quadrilatero inscritto in un cerchio.

%Sei $ABC$ ein Dreieck mit Inkreismittelpunkt $I$. Der Inkreis ber�hre die Seiten $BC$, $CA$ bzw. $AB$ bei $D$, $E$ bzw. $F$. 
%Sei $M$ der Mittelpunkt von $EF$. Die Gerade $AD$ schneide den Inkreis bei $P\neq D$. Beweise, dass $PMID$ ein Sehnenviereck ist.


\bigskip

\item[\textbf{3.}] Sia $n$ un numero naturale. Determina il numero di coppie di numeri naturali $(a,b)$ tali che
 \[(4a-b)(4b-a)=2010^n.\]
 
% Sei $n$ eine nat�rliche Zahl. Bestimme die Anzahl Paare $(a,b)$ nat�rlicher Zahlen, f�r die gilt \[(4a-b)(4b-a)=2010^n.\]


\bigskip

\item[\textbf{4.}] Siano $x,y,z>0$ numeri reali tali che $xyz=1$. Dimostra la disuguaglianza
\[\frac{(x+y-1)^2}{z} + \frac{(y+z-1)^2}{x} + \frac{(z+x-1)^2}{y} \geq 
x+y+z.\]

%Seien $x,y,z>0$ reelle Zahlen mit $xyz=1$. Beweise die Ungleichung \[\frac{(x+y-1)^2}{z} + \frac{(y+z-1)^2}{x} + \frac{(z+x-1)^2}{y} \geq 
%x+y+z.\]


\bigskip

\item[\textbf{5.}] 
Considera gli $n$ vertici di un poligono regolare e uniscili con lati o diagonali del poligono in modo da formare un percorso chiuso passante per ogni vertice esattamente una volta. Una {\it coppia parallela} � un insieme di due segmenti paralleli in questo percorso. Dimostra:
\begin{enumerate}
\item[(a)] Se $n$ � pari, allora esiste sempre almeno una coppia parallela.
\item[(b)] Se $n$ � dispari, allora non esiste mai esattamente una coppia parallela.
\end{enumerate}


%Betrachte die Eckpunkte eines regul�ren $n$-Ecks und 
%verbinde diese mit Seiten oder Diagonalen irgendwie zu einem 
%geschlossenen Streckenzug, der jede Ecke genau einmal durchl�uft. Ein 
%\textit{paralleles Paar} ist eine Menge von zwei verschiedenen 
%parallelen Strecken in diesem Streckenzug. Zeige:
%\begin{enumerate}
%\item[(a)] Ist $n$ gerade, dann gibt es stets mindestens ein paralleles 
%Paar.
%\item[(b)] Ist $n$ ungerade, dann existiert nie genau ein paralleles Paar.
%\end{enumerate}


\end{enumerate}

\pagebreak


\begin{center}
{\huge SMO -- Turno finale 2010} \\
\medskip secondo esame - 13 marzo 2010
\end{center}
\vspace{8mm}
Durata: 4 ore\\
Ogni esercizio vale 7 punti.
\vspace{15mm}

\begin{enumerate}

\item[\textbf{6.}] Determina tutte le funzioni $f\colon\BR \rightarrow \BR$ che soddisfano la seguente uguaglianza per qualsiasi coppia $x,y$ di numeri reali:
\[f(f(x))+f(f(y))=2y+f(x-y).\]

%Finde alle Funktionen $f\colon\BR \rightarrow \BR$, 
%die f�r je zwei reelle Zahlen $x,y$ die folgende Gleichung 
%erf�llen:



\bigskip

\item[\textbf{7.}] Siano $m$ e $n$ dei numeri naturali. Supponi che $m+n+1$ sia un numero primo che divide $2(m^2+n^2)-1$. Dimostra che $m=n$.

%Soient $m$ et $n$ des nombres naturels. On suppose que $m+n+1$ est un nombre premier qui divise $2(m^2+n^2)-1$. Montrer que $m=n$.




\bigskip
\item[\textbf{8.}] In un Comune con almeno un abitante ci sono molteplici associazioni. Ogni abitante del Comune � membro di almeno $k$ associazioni e non ci sono due associazioni che hanno pi� di un membro in comune. 
Mostra che ci sono almeno $k$ associazioni che hanno lo stesso numero di membri. 

%Dans un village ayant au moins un habitant il existe plusieurs associations. Chaque habitant du village est membre d'au moins $k$ associations et deux associations distinctes ont au plus un membre en commun. Montrer qu'il existe au moins $k$ associations qui ont le m�me nombre de membres.

\bigskip

\item[\textbf{9.}] Siano $k$ e $k'$ due cerchi concentrici con centro $O$, e sia $k'$ il cerchio pi� grande. 
Considera una retta passante per $O$ che interseca $k$ in $A$ e $k'$ in $B$, in modo che $O$ si trovi tra $A$ e $B$; e un'altra retta passante per $O$ che interseca $k$ in $E$ e $k'$ in $F$, in modo che $E$ si trovi tra $O$ e $F$.
Dimostra che il cerchio circostcritto a $OAE$, il cerchio con diametro $AB$ e il cerchio con diametro $EF$ si intersecano in un punto.

%Seien $k$ und $k'$ zwei konzentrische Kreise mit 
%Mittelpunkt $O$. Der Kreis $k'$ sei gr�sser als der Kreis $k$. 
%Eine 
%Gerade durch $O$ schneide $k$ in $A$ und $k'$ in $B$, sodass $O$ 
%zwischen $A$ und $B$ liegt. 
%Eine andere Gerade durch $O$ schneidet $k$ 
%in $E$ und $k'$ in $F$, sodass $E$ zwischen $O$ und $F$ liegt. 
%Zeige, 
%dass sich der Umkreis von $OAE$, der Kreis mit Durchmesser $AB$ und der 
%Kreis mit Durchmesser $EF$ in einem Punkt schneiden.



\bigskip

\item[\textbf{10.}] 
Sia $P$ un poligono convesso con $n\geq 3$ vertici. Dimostra che � possibile suddividere $P$ in triangoli per mezzo di $n-3$ diagonali che non si intersecano, in modo che il cerchio circoscritto a ogni triangolo contiene tutto $P$. Quand'� che una tale decomposizione � unica? 

%Soit $n\geq 3$ et soit $P$ un $n$-gone convexe. Montrer que l'on peut d�couper $P$ en triangles le long de $n-3$ diagonales qui ne se coupent pas, de telle sorte que le cercle circonscrit de chaque triangle contienne tout $P$. Quand une telle d�composition est-elle unique?


\end{enumerate}
\end{document}
