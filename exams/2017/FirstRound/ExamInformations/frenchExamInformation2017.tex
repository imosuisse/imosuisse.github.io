\documentclass[12pt,a4paper]{article}

\usepackage{german}
\usepackage[utf8]{inputenc}
\usepackage{enumerate}

\usepackage{graphicx}

\oddsidemargin=0cm \headheight=-2.1cm
\textwidth=16.4cm \textheight=26cm

\begin{document}
\thispagestyle{empty}
\begin{figure}[h]
\includegraphics{fr-smo_logo}
\end{figure}

\vspace{1cm}

\begin{center}
\Huge{\textbf{Examen préliminaire OSM 2017}}\\[1.5cm]
\large{Lugano, Lausanne, Zurich - le 14 janvier 2017}\\[3.5cm]
\end{center}


Merci de respecter les points suivants:

\begin{itemize}
\item Vos solutions seront copiées et triées par exercice. N'écrivez donc que sur un côté des feuilles et commencez une nouvelle feuille pour chaque exercice. Tous les détails facilitant les corrections sont les bienvenus, essayez par exemple de mettre en évidence les passages importants et de marquer ce qui est superflu.

\item Écrivez TOUT ce que vous trouvez, même des observations triviales peuvent valoir des points. Et rendez TOUTES les feuilles lisibles, même pour des assertions complètement fausses on n'enlève jamais de points.

%\item Les exercices sont triés par difficulté mais il arrive souvent que quelqu'un ne voit pas la solution d'un exercice simple tout en résolvant parfaitement un des problèmes difficiles. Ne vous laissez donc pas impressionner par le numéro d'un exercice. Tous les exercices valent le même nombre de points. Vous avez assez de temps pour vous occuper de chaque exercice et on vous conseille vivement de le faire.

\item Il est possible de rendre l'examen avant la fin et de quitter la salle sans faire de bruit.

\item Pour qu'on ne perde rien en corrigeant mettez à la fin de l'examen sur chaque page votre nom (ou au moins une abréviation), le numéro de la page et le nombre total de pages pour cet exercice. Par exemple "`A.Cauchy, G1) p. 2/4"'.

\item On va t'envoyer ta copie corrigée la semaine suivant l'examen dans l'enveloppe ci-joint. S'il te plaît écris maintenant dessus ton adresse correcte et bien lisible.

\end{itemize}

\vfill

\begin{center}
Bonne chance!
\end{center}

\end{document}