%& -job-name=imoselektion_2017_fr-4
% ^^^^^^^ change the last bit to e.g. de-2 for second day, german ^^^^^^^^ 

\documentclass[language=french,style=exam]{smo} %language is irrelevant

\examtime{%
    \de{4.5 Stunden}%
    \fr{4.5 heures}%
    \ita{4.5 ore}%
    \en{4.5 hours}%
}
\examdate{%
    \IfStrEq*{\day}{1}{%
        \de{6. Mai 2017}%
        \fr{6 mai 2017}%
        \ita{??}%
        \en{??}%
    }{}%
   \IfStrEq*{\day}{2}{%
        \de{7. Mai 2017}%
        \fr{7 mai 2017}%
        \ita{??}%
        \en{??}%
    }{}%
       \IfStrEq*{\day}{3}{%
        \de{20. Mai 2017}%
        \fr{20 mai 2017}%
        \ita{??}%
        \en{??}%
    }{}%
       \IfStrEq*{\day}{4}{%
        \de{21. Mai 2017}%
        \fr{21 mai 2017}%
        \ita{??}%
        \en{??}%
    }{}%
}
\title{%
    \de{SMO - Selektion 2017}%
    \fr{OSM - Sélection 2017}%
    \ita{OSM - ???}%
    \en{SMO - ???}%
}
\examplace{%
    \de{\day. Prüfung}%
    \fr{\IfStrEq*{\day}{1}{Premier}{}\IfStrEq*{\day}{2}{Deuxième}{}\IfStrEq*{\day}{3}{Troisième}{}\IfStrEq*{\day}{4}{Quatrième}{} examen}%
    \ita{???}%
    \en{???}%
}

\begin{document}

\bigskip
\bigskip

\begin{enumerate}

\IfStrEq*{\day}{1}{%%%%%%%%%%%%%%%%% Day 1 %%%%%%%%%%%%%%%%%

\item[\textbf{1.}] %% Exercise 1 %%
\de{Finde alle Funktionen $f\colon \Z \to \Z$, sodass gilt:
\begin{enumerate}[(i)] 
\item $f(p) > 0$ für alle Primzahlen $p$,
\item $p \div (f(x) + f(p))^{f(p)} - x$ für alle Primzahlen $p$ und alle $x \in \Z$. 
\end{enumerate}
}%
\fr{Trouver toutes les fonctions $f\colon \Z \to \Z$ telles que:
\begin{enumerate}[(i)] 
\item $f(p) > 0$ pour tout nombre premier $p$,
\item $p \div (f(x) + f(p))^{f(p)} - x$ pour tout nombre premier $p$ et pour tout $x \in \Z$. 
\end{enumerate}
}%
\ita{}%
\en{}%

\bigskip
\bigskip

\item[\textbf{2.}] %% Exercise 2 %%
\de{Sei $n\geq 1$ eine natürliche Zahl und seien $x_1,\ldots,x_n$ strikt positive reelle Zahlen. Zeige, dass man $a_1,\ldots,a_n\in\{-1,1\}$ wählen kann, sodass
\[
\sum_{i=1}^na_ix_i^2\geq\left(\sum_{i=1}^n a_ix_i\right)^2.
\]}%
\fr{Soit $n\geq 1$ un entier positif et soient $x_1,\ldots,x_n$ des nombres réels strictement positifs. Montrer que l'on peut choisir $a_1,\ldots,a_n\in\{-1,1\}$ tels que:
\[
\sum_{i=1}^na_ix_i^2\geq\left(\sum_{i=1}^n a_ix_i\right)^2.
\]}%
\ita{}%
\en{}%

\bigskip
\bigskip

\item[\textbf{3.}] %% Exercise 3 %%
\de{Sei $n\geq3$ eine natürliche Zahl. Wie viele Diagonalen eines regulären $n$-Ecks kann man maximal einzeichnen, sodass falls sich zwei Diagonalen im Innern schneiden, sie senkrecht aufeinander stehen?}%
\fr{Soit $n\geq3$ un entier positif. Quel est le nombre maximal de diagonales d'un $n$-gone régulier que l'on peut tracer, telles que si deux diagonales tracées se coupent à l'intérieur du $n$-gone, alors elles sont perpendiculaires ?}%
\ita{}%
\en{}%

}%END Day 1%


\IfStrEq*{\day}{2}{%%%%%%%%%%%%%%%%% Day 2 %%%%%%%%%%%%%%%%

\item[\textbf{4.}] %% Exercise 4 %%
\de{Sei $k$ ein Kreis und $AB$ eine Sehne von $k$, sodass der Mittelpunkt von $k$ nicht auf $AB$ liegt. Sei $C$ ein von $A$ und $B$ verschiedener Punkt auf $k$. Für jede Wahl von $C$ seien $P_C$ und $Q_C$ die Projektionen von $A$ auf $BC$ respektive $B$ auf $AC$. Weiter sei $O_C$ der Umkreismittelpunkt des Dreiecks $P_{C}Q_{C}C$. Zeige, dass es einen Kreis $\omega$ gibt, sodass $O_C$ für jede Wahl von $C$ auf $\omega$ liegt.}%
\fr{Soit $k$ un cercle et $AB$ une corde de $k$ tel que le centre de $k$ ne se trouve pas sur $AB$. Soit $C$ un point sur $k$ différent de $A$ et de $B$. Pour chaque choix de $C$, soient $P_C$ et $Q_C$ les projections de $A$ sur $BC$ respectivement $B$ sur $AC$. Soit encore $O_C$ le centre du cercle circonscrit au triangle $P_{C}Q_{C}C$. Montrer qu'il existe un cercle $\omega$ tel que $O_C$ se trouve sur $\omega$ pour chaque choix de $C$.}%
\ita{}%
\en{}%

\bigskip
\bigskip

\item[\textbf{5.}] %% Exercise 5 %%
\de{Bestimme die kleinste reelle Konstante $C$, sodass für beliebige, nicht notwendigerweise verschiedene, $a_1,a_2,a_3,a_4,a_5\in \R_{>0}$ immer vier paarweise verschiedene Indizes $i,j,k,l$ existieren, sodass gilt:
\[
\abs*{\frac{a_i}{a_j}-\frac{a_k}{a_l}}\leq C.
\]}%
\fr{Déterminer la plus petite constante réelle $C$ telle que pour tous $a_1,a_2,a_3,a_4,a_5\in \R_{>0}$, pas nécessairement distincts, il existe toujours quatre indices distincts $i,j,k,l$ tels que:
\[
\abs*{\frac{a_i}{a_j}-\frac{a_k}{a_l}}\leq C.
\]}%
\ita{}%
\en{}%

\bigskip
\bigskip

\item[\textbf{6.}] %% Exercise 6 %%
\de{Finde alle Funktionen $f\colon\R_{>0}\to\R_{\geq0}$, sodass für alle  $x,y\in\R_{>0}$ gilt:
\[
f(x)-f(x+y)=f(x^2f(y)+x).
\]}%
\fr{Trouver toutes les fonctions $f\colon\R_{>0}\to\R_{\geq0}$ telles que pour tous  $x,y\in\R_{>0}$:
\[
f(x)-f(x+y)=f(x^2f(y)+x).
\]}%
\ita{}%
\en{}%

}%End Day 2%

\IfStrEq*{\day}{3}{ %%%%%%%%%%%%%% Day 3 %%%%%%%%%%%%%%%%%

\item[\textbf{7.}] %% Exercise 7 %%
\de{Der brasilianische IMO-Leader wählt zwei natürliche Zahlen $n$ und $k$ mit $n>k$, und sagt diese dann seinem Deputy und einem Teilnehmer. Dann flüstert der Leader dem Deputy eine binäre Folge der Länge $n$ ins Ohr. Der Deputy schreibt alle binären Folgen der Länge $n$ auf, die sich genau an $k$ Stellen von der Folge des Leaders unterscheiden. (Beispiel für $n=3$ und $k=1$: Wenn der Leader $101$ wählt, schreibt der Deputy 001, 100, 111 auf.) Der Teilnehmer schaut sich die Folgen an, die der Deputy aufgeschrieben hat. Nun versucht der Teilnehmer, die ursprüngliche Folge vom Leader herauszufinden.
	
Wie viele Male muss er mindestens raten (abhängig von $n$ und $k$), bis er sicher einmal korrekt geraten hat?
	 
Bemerkung: Eine binäre Folge der Länge $n$ ist eine Folge der Länge $n$, die nur aus $0$ und $1$ besteht.}%
\fr{Le leader de l'équipe IMO brésilienne choisit deux nombres naturels $n$ et $k$ avec $n>k$ et les dit à son Deputy Leader ainsi qu'à un participant. Ensuite, le Leader chuchote à l'oreille de son Deputy une suite binaire de longueur $n$. Le Deputy écrit toutes les suites binaires de longueur $n$ qui diffèrent de la suite du Leader en exactement $k$ places. (Par exemple pour $n=3$ et $k=1$: Si le Leader choisit $101$, le Deputy écrit $001, 110, 111$.) Le Participant regarde ensuite les suites que le Deputy a écrites et essaye de trouver la suite choisie par le Leader.

Combien de fois doit-il deviner au minimum (en fonction de $n$ et $k$) pour être sûr d'avoir trouvé la bonne suite?

Remarque: Une suite binaire de longueur $n$ est une suite de longueur $n$ composée uniquement de $0$ et de $1$.}%
\ita{}%
\en{}%

\bigskip
\bigskip

\item[\textbf{8.}] %% Exercise 8 %%

\de{Finde alle monoton steigenden Folgen $a_1, a_2, a_3, \ldots$ natürlicher Zahlen, sodass $i+j$ und $a_i+a_j$ für alle $i,j \in \mathbb{N}$ die gleiche Anzahl Teiler haben.}%
\fr{Trouver toutes les suites croissantes de nombres naturels $a_1, a_2, a_3, \ldots$ telles que pour tous $i,j \in \N$, $i+j$ et $a_i + a_j$ ont le même nombre de diviseurs.}%
\ita{}%
\en{}%

\bigskip
\bigskip

\item[\textbf{9.}] %% Exercise 9 %%
\de{Sei $ABC$ ein Dreieck mit $AB=AC\neq BC$ und $I$ dessen Inkreismittelpunkt. Die Gerade $BI$ schneidet $AC$ in $D$, und die Senkrechte  auf $AC$ durch $D$ schneidet $AI$ in $E$. Zeige: Die Spiegelung von $I$ an $AC$ liegt auf dem Umkreis von Dreieck $BDE$.}%
\fr{Soit $ABC$ un triangle avec $AB=AC\neq BC$ et $I$ le centre de son cercle inscrit. La droite $BI$ coupe $AC$ en $D$, et la perpendiculaire à $AC$ passant par $D$ coupe $AI$ en $E$. Montrer que la réflexion de $I$ par rapport à la droite $AC$ est sur le cercle circonscrit au triangle $BDE$.}%
\ita{}%
\en{}%

}%End Day 3%

\IfStrEq*{\day}{4}{ %%%%%%%%%%%%%%% Day 4 %%%%%%%%%%%%%%%%%%

\item[\textbf{10.}] %% Exercise 10 %%
\de{Finde alle Polynome $P$ mit ganzzahligen Koeffizienten, sodass $P(2017n)$ für alle natürlichen Zahlen $n$ prim ist.}%
\fr{Trouver tous les polynômes $P$ à coefficients entiers tels que $P(2017n)$ est un nombre premier pour tout nombre naturel $n$.}%
\ita{}%
\en{}%

\bigskip
\bigskip

\item[\textbf{11.}] %% Exercise 11 %%

\de{Seien $B=(-1,0)$ und $C=(1,0)$ fixe Punkte in der Ebene. Eine nichtleere, beschränkte Teilmenge $S$ der Ebene heisst \emph{nett}, falls die folgenden Bedingungen gelten:

\begin{enumerate}[(i)]
\item Es gibt einen Punkt $T$ in $S$, sodass für jeden anderen Punkt $Q$ in $S$ die Strecke $TQ$ vollständig in $S$ liegt.
\item Für jedes Dreieck $P_1P_2P_3$ existiert ein eindeutiger Punkt $A$ in $S$ und eine Permutation $\sigma$ von $\{1, 2, 3\}$, sodass die Dreiecke $ABC$ und $P_{\sigma(1)}P_{\sigma(2)}P_{\sigma(3)}$ ähnlich sind.
\end{enumerate}

Zeige, dass es zwei verschiedene nette Teilmengen $S$ und $S'$ der Menge $\{(x,y):x\geq 0,y\geq 0\}$ mit folgender Eigenschaft gibt: Das Produkt $BA\cdot BA'$ ist unabhängig von der Wahl des Dreiecks $P_1P_2P_3$, wobei $A\in S$ und $A'\in S'$ jeweils die eindeutigen Punkte aus (ii) für ein beliebiges Dreieck $P_1P_2P_3$ sind.}%
\fr{Soient $B=(-1,0)$ et $C=(1,0)$ deux points du plan. Un sous-ensemble non-vide et borné $S$ du plan est appelé \emph{incroyable} si les conditions suivantes sont vérifiées:

\begin{enumerate}[(i)]
\item Il existe un point $T$ dans $S$ tel que pour chaque autre point $Q$ dans $S$ le segment $TQ$ est entièrement inclus dans $S$.
\item Pour tout triangle $P_1P_2P_3$, il existe un unique point $A$ dans $S$ et une permation $\sigma$ de $\{1, 2, 3\}$ tels que les triangles $ABC$ et $P_{\sigma(1)}P_{\sigma(2)}P_{\sigma(3)}$ sont semblables.
\end{enumerate}

Montrer qu'il existe deux sous-ensembles incroyables différents $S$ et $S'$ de l'ensemble $\{(x,y):x\geq 0,y\geq 0\}$ avec la propriété suivante: Le produit $BA\cdot BA'$ est indépendant du choix du triangle $P_1P_2P_3$, où $A\in S$ et $A'\in S'$ sont les points donnés par la propriété (ii) pour le triangle $P_1P_2P_3$.}%
\ita{}%
\en{}%

\bigskip
\bigskip

\item[\textbf{12.}] %% Exercise 12 %%
\de{Seien $a,c\in\N$ und sei $b\in\Z$. Zeige, dass es ein $x\in\N$ gibt, sodass
\[
a^x+x\equiv b \mod c.
\]}%
\fr{Soient $a, c\in  \N$ et $b\in \Z$. Prouver qu'il existe $x\in \N$ tel que
\[
a^x+x\equiv b \mod c.
\]}%
\ita{}%
\en{}%

}%End Day 4%

\bigskip

\vfill

\center{\de{Viel Glück!}\fr{Bonne chance!}\ita{Buona fortuna!}\en{Good Luck!}}

\end{enumerate}

\end{document}
