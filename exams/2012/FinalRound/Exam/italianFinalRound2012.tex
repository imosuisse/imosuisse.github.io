\documentclass[11pt,a4paper]{article}

\usepackage{amsfonts}
\usepackage[centertags]{amsmath}
\usepackage{german}
\usepackage{amsthm}
\usepackage{amssymb}

\leftmargin=0pt \topmargin=0pt \headheight=0in \headsep=0in \oddsidemargin=0pt \textwidth=6.5in
\textheight=8.5in

\catcode`\� = \active \catcode`\� = \active \catcode`\� = \active \catcode`\� = \active \catcode`\� = \active
\catcode`\� = \active

\def�{"A}
\def�{"a}
\def�{"O}
\def�{"o}
\def�{"U}
\def�{"u}







% Schriftabk�rzungen

\newcommand{\eps}{\varepsilon}
\renewcommand{\phi}{\varphi}
\newcommand{\Sl}{\ell}    % sch�nes l
\newcommand{\ve}{\varepsilon}  %Epsilon

\newcommand{\BA}{{\mathbb{A}}}
\newcommand{\BB}{{\mathbb{B}}}
\newcommand{\BC}{{\mathbb{C}}}
\newcommand{\BD}{{\mathbb{D}}}
\newcommand{\BE}{{\mathbb{E}}}
\newcommand{\BF}{{\mathbb{F}}}
\newcommand{\BG}{{\mathbb{G}}}
\newcommand{\BH}{{\mathbb{H}}}
\newcommand{\BI}{{\mathbb{I}}}
\newcommand{\BJ}{{\mathbb{J}}}
\newcommand{\BK}{{\mathbb{K}}}
\newcommand{\BL}{{\mathbb{L}}}
\newcommand{\BM}{{\mathbb{M}}}
\newcommand{\BN}{{\mathbb{N}}}
\newcommand{\BO}{{\mathbb{O}}}
\newcommand{\BP}{{\mathbb{P}}}
\newcommand{\BQ}{{\mathbb{Q}}}
\newcommand{\BR}{{\mathbb{R}}}
\newcommand{\BS}{{\mathbb{S}}}
\newcommand{\BT}{{\mathbb{T}}}
\newcommand{\BU}{{\mathbb{U}}}
\newcommand{\BV}{{\mathbb{V}}}
\newcommand{\BW}{{\mathbb{W}}}
\newcommand{\BX}{{\mathbb{X}}}
\newcommand{\BY}{{\mathbb{Y}}}
\newcommand{\BZ}{{\mathbb{Z}}}

\newcommand{\Fa}{{\mathfrak{a}}}
\newcommand{\Fb}{{\mathfrak{b}}}
\newcommand{\Fc}{{\mathfrak{c}}}
\newcommand{\Fd}{{\mathfrak{d}}}
\newcommand{\Fe}{{\mathfrak{e}}}
\newcommand{\Ff}{{\mathfrak{f}}}
\newcommand{\Fg}{{\mathfrak{g}}}
\newcommand{\Fh}{{\mathfrak{h}}}
\newcommand{\Fi}{{\mathfrak{i}}}
\newcommand{\Fj}{{\mathfrak{j}}}
\newcommand{\Fk}{{\mathfrak{k}}}
\newcommand{\Fl}{{\mathfrak{l}}}
\newcommand{\Fm}{{\mathfrak{m}}}
\newcommand{\Fn}{{\mathfrak{n}}}
\newcommand{\Fo}{{\mathfrak{o}}}
\newcommand{\Fp}{{\mathfrak{p}}}
\newcommand{\Fq}{{\mathfrak{q}}}
\newcommand{\Fr}{{\mathfrak{r}}}
\newcommand{\Fs}{{\mathfrak{s}}}
\newcommand{\Ft}{{\mathfrak{t}}}
\newcommand{\Fu}{{\mathfrak{u}}}
\newcommand{\Fv}{{\mathfrak{v}}}
\newcommand{\Fw}{{\mathfrak{w}}}
\newcommand{\Fx}{{\mathfrak{x}}}
\newcommand{\Fy}{{\mathfrak{y}}}
\newcommand{\Fz}{{\mathfrak{z}}}

\newcommand{\FA}{{\mathfrak{A}}}
\newcommand{\FB}{{\mathfrak{B}}}
\newcommand{\FC}{{\mathfrak{C}}}
\newcommand{\FD}{{\mathfrak{D}}}
\newcommand{\FE}{{\mathfrak{E}}}
\newcommand{\FF}{{\mathfrak{F}}}
\newcommand{\FG}{{\mathfrak{G}}}
\newcommand{\FH}{{\mathfrak{H}}}
\newcommand{\FI}{{\mathfrak{I}}}
\newcommand{\FJ}{{\mathfrak{J}}}
\newcommand{\FK}{{\mathfrak{K}}}
\newcommand{\FL}{{\mathfrak{L}}}
\newcommand{\FM}{{\mathfrak{M}}}
\newcommand{\FN}{{\mathfrak{N}}}
\newcommand{\FO}{{\mathfrak{O}}}
\newcommand{\FP}{{\mathfrak{P}}}
\newcommand{\FQ}{{\mathfrak{Q}}}
\newcommand{\FR}{{\mathfrak{R}}}
\newcommand{\FS}{{\mathfrak{S}}}
\newcommand{\FT}{{\mathfrak{T}}}
\newcommand{\FU}{{\mathfrak{U}}}
\newcommand{\FV}{{\mathfrak{V}}}
\newcommand{\FW}{{\mathfrak{W}}}
\newcommand{\FX}{{\mathfrak{X}}}
\newcommand{\FY}{{\mathfrak{Y}}}
\newcommand{\FZ}{{\mathfrak{Z}}}

\newcommand{\CA}{{\cal A}}
\newcommand{\CB}{{\cal B}}
\newcommand{\CC}{{\cal C}}
\newcommand{\CD}{{\cal D}}
\newcommand{\CE}{{\cal E}}
\newcommand{\CF}{{\cal F}}
\newcommand{\CG}{{\cal G}}
\newcommand{\CH}{{\cal H}}
\newcommand{\CI}{{\cal I}}
\newcommand{\CJ}{{\cal J}}
\newcommand{\CK}{{\cal K}}
\newcommand{\CL}{{\cal L}}
\newcommand{\CM}{{\cal M}}
\newcommand{\CN}{{\cal N}}
\newcommand{\CO}{{\cal O}}
\newcommand{\CP}{{\cal P}}
\newcommand{\CQ}{{\cal Q}}
\newcommand{\CR}{{\cal R}}
\newcommand{\CS}{{\cal S}}
\newcommand{\CT}{{\cal T}}
\newcommand{\CU}{{\cal U}}
\newcommand{\CV}{{\cal V}}
\newcommand{\CW}{{\cal W}}
\newcommand{\CX}{{\cal X}}
\newcommand{\CY}{{\cal Y}}
\newcommand{\CZ}{{\cal Z}}

% Theorem Stil

\theoremstyle{plain}
\newtheorem{lem}{Lemma}
\newtheorem{Satz}[lem]{Satz}

\theoremstyle{definition}
\newtheorem{defn}{Definition}[section]

\theoremstyle{remark}
\newtheorem{bem}{Bemerkung}    %[section]



\newcommand{\card}{\mathop{\rm card}\nolimits}
\newcommand{\Sets}{((Sets))}
\newcommand{\id}{{\rm id}}
\newcommand{\supp}{\mathop{\rm Supp}\nolimits}

\newcommand{\ord}{\mathop{\rm ord}\nolimits}
\renewcommand{\mod}{\mathop{\rm mod}\nolimits}
\newcommand{\sign}{\mathop{\rm sign}\nolimits}
\newcommand{\ggT}{\mathop{\rm ggT}\nolimits}
\newcommand{\kgV}{\mathop{\rm kgV}\nolimits}
\renewcommand{\div}{\, | \,}
\newcommand{\notdiv}{\mathopen{\mathchoice
             {\not{|}\,}
             {\not{|}\,}
             {\!\not{\:|}}
             {\not{|}}
             }}

\newcommand{\im}{\mathop{{\rm Im}}\nolimits}
\newcommand{\coim}{\mathop{{\rm coim}}\nolimits}
\newcommand{\coker}{\mathop{\rm Coker}\nolimits}
\renewcommand{\ker}{\mathop{\rm Ker}\nolimits}

\newcommand{\pRang}{\mathop{p{\rm -Rang}}\nolimits}
\newcommand{\End}{\mathop{\rm End}\nolimits}
\newcommand{\Hom}{\mathop{\rm Hom}\nolimits}
\newcommand{\Isom}{\mathop{\rm Isom}\nolimits}
\newcommand{\Tor}{\mathop{\rm Tor}\nolimits}
\newcommand{\Aut}{\mathop{\rm Aut}\nolimits}

\newcommand{\adj}{\mathop{\rm adj}\nolimits}

\newcommand{\Norm}{\mathop{\rm Norm}\nolimits}
\newcommand{\Gal}{\mathop{\rm Gal}\nolimits}
\newcommand{\Frob}{{\rm Frob}}

\newcommand{\disc}{\mathop{\rm disc}\nolimits}

\renewcommand{\Re}{\mathop{\rm Re}\nolimits}
\renewcommand{\Im}{\mathop{\rm Im}\nolimits}

\newcommand{\Log}{\mathop{\rm Log}\nolimits}
\newcommand{\Res}{\mathop{\rm Res}\nolimits}
\newcommand{\Bild}{\mathop{\rm Bild}\nolimits}

\renewcommand{\binom}[2]{\left({#1}\atop{#2}\right)}
\newcommand{\eck}[1]{\langle #1 \rangle}
\newcommand{\wi}{\hspace{1pt} < \hspace{-6pt} ) \hspace{2pt}}


\begin{document}

\pagestyle{empty}

%\begin{center}
%{\huge SMO Finalrunde 2012} \\
%\medskip erste Pr�fung - 9. M�rz 2012
%\end{center}
%\vspace{8mm}
%Zeit: 4 Stunden\\
%Jede Aufgabe ist 7 Punkte wert.
\begin{center}
{\huge SMO -- Turno finale 2012} \\
\medskip Prima prova - 9 marzo 2012
\end{center}
\vspace{8mm}
Durata: 4 ore\\
Ogni esercizio vale 7 punti.

\vspace{15mm}

\begin{enumerate}
\item[\textbf{1.}]
2012 camaleonti siedono a un tavolo rotondo. Inizialmente ogni camaleonte \`e di colore rosso oppure verde. Allo scoccare di ogni minuto ogni camaleonte che siede tra due camaleonti dello stesso colore cambia il suo colore. Tutti gli altri mantengono il loro colore. Dimostra che dopo 2012 minuti ci sono almeno due camaleonti che hanno cambiato colore lo stesso numero di volte.
%Es sitzen $10$ Chameleons an einem runden Tisch. Am Anfang besitzt jedes die Farbe rot oder gr�n. Nach jeder vollen Minute wechselt jedes Chameleon, welches zwei gleichfarbige Nachbarn hat, seine Farbe. Alle anderen behalten ihre Farbe. Zeige, dass es nach $10$ Minuten mindestens $2$ Chameleons gibt, welche gleich oft die Farbe gewechselt haben.


\bigskip

\item[\textbf{2.}] 
Determina tutte le funzioni $f: \BR\rightarrow\BR$ tali che per ogni $x,y\in\BR$ vale
\[
  f\left( f(x) + 2f(y) \right) = f(2x) + 8y + 6.
\]

%Bestimme alle Funktionen $f: \BR\rightarrow\BR$, sodass f�r alle $x,y\in\BR$ gilt
%\[
%  f\left( f(x) + 2f(y) \right) = f(2x) + 8y + 6.
%\]


\bigskip

\item[\textbf{3.}] 
I cerchi $k_1$ e $k_2$ si intersecano nei punti $D$ e $P$. La tangente comune ai due cerchi che giace dalla stessa parte del punto $D$ interseca $k_1$ nel punto $A$ e $k_2$ nel punto $B$. La retta $AD$ interseca $k_2$ una seconda volta nel punto $C$. Sia $M$ il punto medio della corda $BC$. Dimostra che $\angle DPM = \angle BDC$. 
%Die Kreise $k_1$ und $k_2$ schneiden sich in den Punkten $D$ und $P$. Die gemeinsame Tangente an die beiden Kreise auf der Seite von $D$ ber�hrt $k_1$ in $A$ und $k_2$ in $B$. Die Gerade $AD$ schneidet $k_2$ ein zweites Mal in $C$. Sei $M$ der Mittelpunkt der Sehne $BC$. Zeige, dass $\angle DPM = \angle BDC$ gilt.


\bigskip

\item[\textbf{4.}]
Dimostra che non esiste nessuna successione infinita di numeri primi $p_1,p_2,p_3,\dots$ tale per cui per ogni $k$ vale $p_{k+1}=2p_k-1$ oppure $p_{k+1}=2p_k+1$. Nota che non � necessario che valga la stessa formula per tutti i valori di $k$. 
%Zeige, dass es keine unendliche Folge von Primzahlen $p_1,p_2,p_3,\dots$ gibt, welche f�r jedes $k$ $p_{k+1}=2p_k-1$ oder $p_{k+1}=2p_k+1$ erf�llt. Beachte, dass nicht f�r jedes $k$ die gleiche Formel gelten muss.



\bigskip

\item[\textbf{5.}]
Sia $n$ un numero naturale. Siano $A_1,A_2, \ldots, A_k$ $k$ sottoinsiemi distinti di cardinalit\`a 3 dell'insieme $\left\{1, 2, \ldots, n\right\}$, tali che $|A_i\cap A_j| \neq 1$ per ogni $1\leq i,j\leq k$.
Determina tutti i valori di $n$ per i quali esistono $n$ sottoinsiemi di questo tipo.
%Sei $n$ eine nat�rliche Zahl. 
%Seien $A_1,A_2, \ldots, A_k$ verschiedene $3$-elementige Teilmengen von $\left\{1, 2, \ldots, n\right\}$, sodass $|A_i\cap A_j| \neq 1$ f�r alle $1\leq i,j\leq k$. Bestimme alle $n$, f�r die es $n$ solche Teilmengen gibt.

\end{enumerate}
\bigskip
\begin{center}
Buon lavoro!
%Viel Gl�ck !
\end{center}

\newpage

%\begin{center}
%{\huge SMO Finalrunde 2012} \\
%\medskip zweite Pr�fung - 10. M�rz 2012
%\end{center}
%\vspace{8mm}
%Zeit: 4 Stunden\\
%Jede Aufgabe ist 7 Punkte wert.
\begin{center}
{\huge SMO -- Turno finale 2012} \\
\medskip Seconda prova - 10 marzo 2012
\end{center}
\vspace{8mm}
Durata: 4 ore\\
Ogni esercizio vale 7 punti.

\vspace{15mm}

\begin{enumerate}
\item[\textbf{6.}] 
Sia $ABCD$ un parallelogramma con almeno un angolo di ampiezza $\neq 90^{\circ}$ e sia $k$ il cerchio circoscritto al triangolo $ABC$.
Sia $E$ il punto su $k$ diametralmente opposto a $B$.
Dimostra che il cerchio circoscritto a $ADE$ e $k$ hanno il raggio della stessa misura.
%Sei $ABCD$ ein Parallelogramm mit mindestens einem Winkel $\neq 90^{\circ}$ und $k$ der Umkreis des Dreiecks $ABC$. Sei $E$ der auf $k$ diametral gegen�berliegende Punkt von $B$.
%Zeige, dass der Umkreis des Dreiecks $ADE$ und $k$ den gleichen Radius haben.

\bigskip

\item[\textbf{7.}] 
Siano $n$ e $k$ due numeri naturali tali che $n=3k +2$. Dimostra che la somma di tutti i divisori di $n$ \`e divisibile per 3.
%Seien $n$ und $k$ nat�rliche Zahlen sodass $n=3k +2$. Zeige, dass die Summe aller Teiler von $n$ durch $3$ teilbar ist.


\bigskip

\item[\textbf{8.}] 
Considera un cubo e due vertici $A$ e $B$ alle estremit\`a di una diagonale di una sua faccia. 
Un \emph{cammino} \`e una successione di spigoli del cubo che in ogni passaggio connette un vertice del cubo con uno dei suoi tre vertici adiacenti.
Sia $a$ il numero di cammini di lunghezza 2012 che iniziano in $A$ e terminano in $A$, e sia $b$ il numero di cammini di lunghezza 2012 che iniziano in $A$ e terminano in $B$.
Stabilisci quale tra i numeri $a$ e $b$ \`e il pi\`u grande.
%Betrachte einen W�rfel und zwei seiner Ecken $A$ und $B$, welche Endpunkte einer Fl�chendiagonalen sind. Ein \emph{Weg} ist eine Folge von W�rfelecken, wobei in jedem Schritt von einer Ecke l�ngs eine W�rfelkante zu einer der drei benachbarten Ecken gegangen wird. Sei $a$ die Anzahl Wege der L�nge $2012$, die im Punkt $A$ beginnen und in $A$ enden und sei $b$ die Anzahl Wege der L�nge $2012$, die in $A$ beginnen und in $B$ enden. Entscheide, welche der beiden Zahlen $a$ und $b$ die gr�ssere ist. 


\bigskip

\item[\textbf{9.}]
%Seien $a,b,c>0$ reelle Zahlen mit $abc = 1$. Zeige
Siano $a,b,c>0$ tre numeri reali tali che $abc = 1$. Dimostra che
\[
  1 + ab + bc + ca \geq \min\left\{ \frac{(a+b)^2}{ab}, \frac{(b+c)^2}{bc}, \frac{(c+a)^2}{ca} \right\}.
\]
Quando vale l'uguaglianza?

\bigskip

\item[\textbf{10.}] 
Sia $O$ un punto interno a un triangolo acutangolo $ABC$.
Siano $A_1$, $B_1$ e $C_1$ le proiezioni del punto $O$ sui lati $BC$, $AC$ e $AB$.
Sia $P$ il punto d'intersezione della retta perpendicolare a $B_1C_1$ passante per $A$ con la retta perpendicolare a $A_1C_1$ passante per $B$.
Sia $H$ la proiezione del punto $P$ su $AB$.
Dimostra che i punti $A_1,B_1, C_1$ e $H$ giacciono su uno stesso cerchio.
%
%Sei $O$ ein innerer Punkt eines spitzwinkligen Dreiecks $ABC$. Seien $A_1$, $B_1$ und $C_1$ die Projektionen von $O$ auf die Seiten $BC$, $AC$ und $AB$. Sei $P$ der Schnittpunkt der Senkrechten zu $B_1C_1$ respektive $A_1C_1$ durch die Punkte $A$ respektive $B$. Sei $H$ die Projektion von $P$ auf $AB$. Zeige, dass die Punkte $A_1,B_1, C_1$ und $H$ auf einem Kreis liegen.



\bigskip

\end{enumerate}
\bigskip
\begin{center}
Buon lavoro!
%Viel Gl�ck !
\end{center}
\end{document}
