\documentclass[11pt,a4paper]{article}

\usepackage{amsfonts}
\usepackage[centertags]{amsmath}
\usepackage{german}
\usepackage{amsthm}
\usepackage{amssymb}

\leftmargin=0pt \topmargin=0pt \headheight=0in \headsep=0in \oddsidemargin=0pt \textwidth=6.5in
\textheight=8.5in

\catcode`\� = \active \catcode`\� = \active \catcode`\� = \active \catcode`\� = \active \catcode`\� = \active
\catcode`\� = \active

\def�{"A}
\def�{"a}
\def�{"O}
\def�{"o}
\def�{"U}
\def�{"u}







% Schriftabk�rzungen

\newcommand{\eps}{\varepsilon}
\renewcommand{\phi}{\varphi}
\newcommand{\Sl}{\ell}    % sch�nes l
\newcommand{\ve}{\varepsilon}  %Epsilon

\newcommand{\BA}{{\mathbb{A}}}
\newcommand{\BB}{{\mathbb{B}}}
\newcommand{\BC}{{\mathbb{C}}}
\newcommand{\BD}{{\mathbb{D}}}
\newcommand{\BE}{{\mathbb{E}}}
\newcommand{\BF}{{\mathbb{F}}}
\newcommand{\BG}{{\mathbb{G}}}
\newcommand{\BH}{{\mathbb{H}}}
\newcommand{\BI}{{\mathbb{I}}}
\newcommand{\BJ}{{\mathbb{J}}}
\newcommand{\BK}{{\mathbb{K}}}
\newcommand{\BL}{{\mathbb{L}}}
\newcommand{\BM}{{\mathbb{M}}}
\newcommand{\BN}{{\mathbb{N}}}
\newcommand{\BO}{{\mathbb{O}}}
\newcommand{\BP}{{\mathbb{P}}}
\newcommand{\BQ}{{\mathbb{Q}}}
\newcommand{\BR}{{\mathbb{R}}}
\newcommand{\BS}{{\mathbb{S}}}
\newcommand{\BT}{{\mathbb{T}}}
\newcommand{\BU}{{\mathbb{U}}}
\newcommand{\BV}{{\mathbb{V}}}
\newcommand{\BW}{{\mathbb{W}}}
\newcommand{\BX}{{\mathbb{X}}}
\newcommand{\BY}{{\mathbb{Y}}}
\newcommand{\BZ}{{\mathbb{Z}}}

\newcommand{\Fa}{{\mathfrak{a}}}
\newcommand{\Fb}{{\mathfrak{b}}}
\newcommand{\Fc}{{\mathfrak{c}}}
\newcommand{\Fd}{{\mathfrak{d}}}
\newcommand{\Fe}{{\mathfrak{e}}}
\newcommand{\Ff}{{\mathfrak{f}}}
\newcommand{\Fg}{{\mathfrak{g}}}
\newcommand{\Fh}{{\mathfrak{h}}}
\newcommand{\Fi}{{\mathfrak{i}}}
\newcommand{\Fj}{{\mathfrak{j}}}
\newcommand{\Fk}{{\mathfrak{k}}}
\newcommand{\Fl}{{\mathfrak{l}}}
\newcommand{\Fm}{{\mathfrak{m}}}
\newcommand{\Fn}{{\mathfrak{n}}}
\newcommand{\Fo}{{\mathfrak{o}}}
\newcommand{\Fp}{{\mathfrak{p}}}
\newcommand{\Fq}{{\mathfrak{q}}}
\newcommand{\Fr}{{\mathfrak{r}}}
\newcommand{\Fs}{{\mathfrak{s}}}
\newcommand{\Ft}{{\mathfrak{t}}}
\newcommand{\Fu}{{\mathfrak{u}}}
\newcommand{\Fv}{{\mathfrak{v}}}
\newcommand{\Fw}{{\mathfrak{w}}}
\newcommand{\Fx}{{\mathfrak{x}}}
\newcommand{\Fy}{{\mathfrak{y}}}
\newcommand{\Fz}{{\mathfrak{z}}}

\newcommand{\FA}{{\mathfrak{A}}}
\newcommand{\FB}{{\mathfrak{B}}}
\newcommand{\FC}{{\mathfrak{C}}}
\newcommand{\FD}{{\mathfrak{D}}}
\newcommand{\FE}{{\mathfrak{E}}}
\newcommand{\FF}{{\mathfrak{F}}}
\newcommand{\FG}{{\mathfrak{G}}}
\newcommand{\FH}{{\mathfrak{H}}}
\newcommand{\FI}{{\mathfrak{I}}}
\newcommand{\FJ}{{\mathfrak{J}}}
\newcommand{\FK}{{\mathfrak{K}}}
\newcommand{\FL}{{\mathfrak{L}}}
\newcommand{\FM}{{\mathfrak{M}}}
\newcommand{\FN}{{\mathfrak{N}}}
\newcommand{\FO}{{\mathfrak{O}}}
\newcommand{\FP}{{\mathfrak{P}}}
\newcommand{\FQ}{{\mathfrak{Q}}}
\newcommand{\FR}{{\mathfrak{R}}}
\newcommand{\FS}{{\mathfrak{S}}}
\newcommand{\FT}{{\mathfrak{T}}}
\newcommand{\FU}{{\mathfrak{U}}}
\newcommand{\FV}{{\mathfrak{V}}}
\newcommand{\FW}{{\mathfrak{W}}}
\newcommand{\FX}{{\mathfrak{X}}}
\newcommand{\FY}{{\mathfrak{Y}}}
\newcommand{\FZ}{{\mathfrak{Z}}}

\newcommand{\CA}{{\cal A}}
\newcommand{\CB}{{\cal B}}
\newcommand{\CC}{{\cal C}}
\newcommand{\CD}{{\cal D}}
\newcommand{\CE}{{\cal E}}
\newcommand{\CF}{{\cal F}}
\newcommand{\CG}{{\cal G}}
\newcommand{\CH}{{\cal H}}
\newcommand{\CI}{{\cal I}}
\newcommand{\CJ}{{\cal J}}
\newcommand{\CK}{{\cal K}}
\newcommand{\CL}{{\cal L}}
\newcommand{\CM}{{\cal M}}
\newcommand{\CN}{{\cal N}}
\newcommand{\CO}{{\cal O}}
\newcommand{\CP}{{\cal P}}
\newcommand{\CQ}{{\cal Q}}
\newcommand{\CR}{{\cal R}}
\newcommand{\CS}{{\cal S}}
\newcommand{\CT}{{\cal T}}
\newcommand{\CU}{{\cal U}}
\newcommand{\CV}{{\cal V}}
\newcommand{\CW}{{\cal W}}
\newcommand{\CX}{{\cal X}}
\newcommand{\CY}{{\cal Y}}
\newcommand{\CZ}{{\cal Z}}

% Theorem Stil

\theoremstyle{plain}
\newtheorem{lem}{Lemma}
\newtheorem{Satz}[lem]{Satz}

\theoremstyle{definition}
\newtheorem{defn}{Definition}[section]

\theoremstyle{remark}
\newtheorem{bem}{Bemerkung}    %[section]



\newcommand{\card}{\mathop{\rm card}\nolimits}
\newcommand{\Sets}{((Sets))}
\newcommand{\id}{{\rm id}}
\newcommand{\supp}{\mathop{\rm Supp}\nolimits}

\newcommand{\ord}{\mathop{\rm ord}\nolimits}
\renewcommand{\mod}{\mathop{\rm mod}\nolimits}
\newcommand{\sign}{\mathop{\rm sign}\nolimits}
\newcommand{\ggT}{\mathop{\rm ggT}\nolimits}
\newcommand{\kgV}{\mathop{\rm kgV}\nolimits}
\renewcommand{\div}{\, | \,}
\newcommand{\notdiv}{\mathopen{\mathchoice
             {\not{|}\,}
             {\not{|}\,}
             {\!\not{\:|}}
             {\not{|}}
             }}

\newcommand{\im}{\mathop{{\rm Im}}\nolimits}
\newcommand{\coim}{\mathop{{\rm coim}}\nolimits}
\newcommand{\coker}{\mathop{\rm Coker}\nolimits}
\renewcommand{\ker}{\mathop{\rm Ker}\nolimits}

\newcommand{\pRang}{\mathop{p{\rm -Rang}}\nolimits}
\newcommand{\End}{\mathop{\rm End}\nolimits}
\newcommand{\Hom}{\mathop{\rm Hom}\nolimits}
\newcommand{\Isom}{\mathop{\rm Isom}\nolimits}
\newcommand{\Tor}{\mathop{\rm Tor}\nolimits}
\newcommand{\Aut}{\mathop{\rm Aut}\nolimits}

\newcommand{\adj}{\mathop{\rm adj}\nolimits}

\newcommand{\Norm}{\mathop{\rm Norm}\nolimits}
\newcommand{\Gal}{\mathop{\rm Gal}\nolimits}
\newcommand{\Frob}{{\rm Frob}}

\newcommand{\disc}{\mathop{\rm disc}\nolimits}

\renewcommand{\Re}{\mathop{\rm Re}\nolimits}
\renewcommand{\Im}{\mathop{\rm Im}\nolimits}

\newcommand{\Log}{\mathop{\rm Log}\nolimits}
\newcommand{\Res}{\mathop{\rm Res}\nolimits}
\newcommand{\Bild}{\mathop{\rm Bild}\nolimits}

\renewcommand{\binom}[2]{\left({#1}\atop{#2}\right)}
\newcommand{\eck}[1]{\langle #1 \rangle}
\newcommand{\wi}{\hspace{1pt} < \hspace{-6pt} ) \hspace{2pt}}


\begin{document}

\pagestyle{empty}

\begin{center}
{\huge SMO Finalrunde 2012} \\
\medskip erste Pr�fung - 9. M�rz 2012
\end{center}
\vspace{8mm}
Zeit: 4 Stunden\\
Jede Aufgabe ist 7 Punkte wert.

\vspace{15mm}

\begin{enumerate}
\item[\textbf{1.}]
Es sitzen $2012$ Cham�leons an einem runden Tisch. Am Anfang besitzt jedes die Farbe rot oder gr�n. Nach jeder vollen Minute wechselt jedes Cham�leon, welches zwei gleichfarbige Nachbarn hat, seine Farbe von rot zu gr�n respektive von gr�n zu rot. Alle anderen behalten ihre Farbe. Zeige, dass es nach $2012$ Minuten mindestens $2$ Cham�leons gibt, welche gleich oft die Farbe gewechselt haben.


\bigskip

\item[\textbf{2.}] 
Bestimme alle Funktionen $f: \BR\rightarrow\BR$, sodass f�r alle $x,y\in\BR$ gilt
\[
  f\left( f(x) + 2f(y) \right) = f(2x) + 8y + 6.
\]


\bigskip

\item[\textbf{3.}] 
Die Kreise $k_1$ und $k_2$ schneiden sich in den Punkten $D$ und $P$. Die gemeinsame Tangente an die beiden Kreise auf der Seite von $D$ ber�hrt $k_1$ in $A$ und $k_2$ in $B$. Die Gerade $AD$ schneidet $k_2$ ein zweites Mal in $C$. Sei $M$ der Mittelpunkt der Sehne $BC$. Zeige, dass $\angle DPM = \angle BDC$ gilt.


\bigskip

\item[\textbf{4.}]
Zeige, dass es keine unendliche Folge von Primzahlen $p_1,p_2,p_3,\dots$ gibt, welche f�r jedes $k$ $p_{k+1}=2p_k-1$ oder $p_{k+1}=2p_k+1$ erf�llt. Beachte, dass nicht f�r jedes $k$ die gleiche Formel gelten muss.



\bigskip

\item[\textbf{5.}]
Sei $n$ eine nat�rliche Zahl.
Seien $A_1,A_2, \ldots, A_k$ verschiedene $3$-elementige Teilmengen von $\left\{1, 2, \ldots, n\right\}$, sodass $|A_i\cap A_j| \neq 1$ f�r alle $1\leq i,j\leq k$. Bestimme alle $n$, f�r die es $n$ solche Teilmengen gibt.

\end{enumerate}
\bigskip
\begin{center}
Viel Gl�ck !
\end{center}

\newpage

\begin{center}
{\huge SMO Finalrunde 2012} \\
\medskip zweite Pr�fung - 10. M�rz 2012
\end{center}
\vspace{8mm}
Zeit: 4 Stunden\\
Jede Aufgabe ist 7 Punkte wert.

\vspace{15mm}

\begin{enumerate}
\item[\textbf{6.}] 
Sei $ABCD$ ein Parallelogramm mit mindestens einem Winkel ungleich $90^{\circ}$ und $k$ der Umkreis des Dreiecks $ABC$. Sei $E$ der auf $k$ diametral gegen�berliegende Punkt von $B$.
Zeige, dass der Umkreis des Dreiecks $ADE$ und $k$ den gleichen Radius haben.

\bigskip

\item[\textbf{7.}] 
Seien $n$ und $k$ nat�rliche Zahlen sodass $n=3k +2$. Zeige, dass die Summe aller Teiler von $n$ durch $3$ teilbar ist.


\bigskip

\item[\textbf{8.}] 
Betrachte einen W�rfel und zwei seiner Ecken $A$ und $B$, welche die Endpunkte einer Fl�chendiagonalen sind. Ein \emph{Weg} ist eine Folge von W�rfelecken, wobei in jedem Schritt von einer Ecke l�ngs eine W�rfelkante zu einer der drei benachbarten Ecken gegangen wird. Sei $a$ die Anzahl Wege der L�nge $2012$, die im Punkt $A$ beginnen und in $A$ enden und sei $b$ die Anzahl Wege der L�nge $2012$, die in $A$ beginnen und in $B$ enden. Entscheide, welche der beiden Zahlen $a$ und $b$ die gr�ssere ist. 


\bigskip

\item[\textbf{9.}]
Seien $a,b,c>0$ reelle Zahlen mit $abc = 1$. Zeige
\[
  1 + ab + bc + ca \geq \min\left\{ \frac{(a+b)^2}{ab}, \frac{(b+c)^2}{bc}, \frac{(c+a)^2}{ca} \right\}.
\]
Wann gilt Gleichheit?

\bigskip

\item[\textbf{10.}] 
Sei $O$ ein innerer Punkt eines spitzwinkligen Dreiecks $ABC$. Seien $A_1$, $B_1$ und $C_1$ die Projektionen von $O$ auf die Seiten $BC$, $AC$ und $AB$. Sei $P$ der Schnittpunkt der Senkrechten zu $B_1C_1$ respektive $A_1C_1$ durch die Punkte $A$ respektive $B$. Sei $H$ die Projektion von $P$ auf $AB$. Zeige, dass die Punkte $A_1,B_1, C_1$ und $H$ auf einem Kreis liegen.



\bigskip

\end{enumerate}
\bigskip
\begin{center}
Viel Gl�ck !
\end{center}
\end{document}
