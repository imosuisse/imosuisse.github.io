\documentclass[12pt,a4paper]{article}

\usepackage{amsfonts}
\usepackage{amsmath}
%\usepackage[centertags]{amsmath}
\usepackage{german}
\usepackage{amsthm}
\usepackage{amssymb}

\leftmargin=0pt \topmargin=0pt \headheight=0in \headsep=0in \oddsidemargin=0pt \textwidth=6.5in
\textheight=8.5in

\catcode`\� = \active \catcode`\� = \active \catcode`\� = \active \catcode`\� = \active \catcode`\� = \active
\catcode`\� = \active

\def�{"A}
\def�{"a}
\def�{"O}
\def�{"o}
\def�{"U}
\def�{"u}







% Schriftabk�rzungen

\newcommand{\eps}{\varepsilon}
\renewcommand{\phi}{\varphi}
\newcommand{\Sl}{\ell}    % sch�nes l
\newcommand{\ve}{\varepsilon}  %Epsilon

\newcommand{\BA}{{\mathbb{A}}}
\newcommand{\BB}{{\mathbb{B}}}
\newcommand{\BC}{{\mathbb{C}}}
\newcommand{\BD}{{\mathbb{D}}}
\newcommand{\BE}{{\mathbb{E}}}
\newcommand{\BF}{{\mathbb{F}}}
\newcommand{\BG}{{\mathbb{G}}}
\newcommand{\BH}{{\mathbb{H}}}
\newcommand{\BI}{{\mathbb{I}}}
\newcommand{\BJ}{{\mathbb{J}}}
\newcommand{\BK}{{\mathbb{K}}}
\newcommand{\BL}{{\mathbb{L}}}
\newcommand{\BM}{{\mathbb{M}}}
\newcommand{\BN}{{\mathbb{N}}}
\newcommand{\BO}{{\mathbb{O}}}
\newcommand{\BP}{{\mathbb{P}}}
\newcommand{\BQ}{{\mathbb{Q}}}
\newcommand{\BR}{{\mathbb{R}}}
\newcommand{\BS}{{\mathbb{S}}}
\newcommand{\BT}{{\mathbb{T}}}
\newcommand{\BU}{{\mathbb{U}}}
\newcommand{\BV}{{\mathbb{V}}}
\newcommand{\BW}{{\mathbb{W}}}
\newcommand{\BX}{{\mathbb{X}}}
\newcommand{\BY}{{\mathbb{Y}}}
\newcommand{\BZ}{{\mathbb{Z}}}

\newcommand{\Fa}{{\mathfrak{a}}}
\newcommand{\Fb}{{\mathfrak{b}}}
\newcommand{\Fc}{{\mathfrak{c}}}
\newcommand{\Fd}{{\mathfrak{d}}}
\newcommand{\Fe}{{\mathfrak{e}}}
\newcommand{\Ff}{{\mathfrak{f}}}
\newcommand{\Fg}{{\mathfrak{g}}}
\newcommand{\Fh}{{\mathfrak{h}}}
\newcommand{\Fi}{{\mathfrak{i}}}
\newcommand{\Fj}{{\mathfrak{j}}}
\newcommand{\Fk}{{\mathfrak{k}}}
\newcommand{\Fl}{{\mathfrak{l}}}
\newcommand{\Fm}{{\mathfrak{m}}}
\newcommand{\Fn}{{\mathfrak{n}}}
\newcommand{\Fo}{{\mathfrak{o}}}
\newcommand{\Fp}{{\mathfrak{p}}}
\newcommand{\Fq}{{\mathfrak{q}}}
\newcommand{\Fr}{{\mathfrak{r}}}
\newcommand{\Fs}{{\mathfrak{s}}}
\newcommand{\Ft}{{\mathfrak{t}}}
\newcommand{\Fu}{{\mathfrak{u}}}
\newcommand{\Fv}{{\mathfrak{v}}}
\newcommand{\Fw}{{\mathfrak{w}}}
\newcommand{\Fx}{{\mathfrak{x}}}
\newcommand{\Fy}{{\mathfrak{y}}}
\newcommand{\Fz}{{\mathfrak{z}}}

\newcommand{\FA}{{\mathfrak{A}}}
\newcommand{\FB}{{\mathfrak{B}}}
\newcommand{\FC}{{\mathfrak{C}}}
\newcommand{\FD}{{\mathfrak{D}}}
\newcommand{\FE}{{\mathfrak{E}}}
\newcommand{\FF}{{\mathfrak{F}}}
\newcommand{\FG}{{\mathfrak{G}}}
\newcommand{\FH}{{\mathfrak{H}}}
\newcommand{\FI}{{\mathfrak{I}}}
\newcommand{\FJ}{{\mathfrak{J}}}
\newcommand{\FK}{{\mathfrak{K}}}
\newcommand{\FL}{{\mathfrak{L}}}
\newcommand{\FM}{{\mathfrak{M}}}
\newcommand{\FN}{{\mathfrak{N}}}
\newcommand{\FO}{{\mathfrak{O}}}
\newcommand{\FP}{{\mathfrak{P}}}
\newcommand{\FQ}{{\mathfrak{Q}}}
\newcommand{\FR}{{\mathfrak{R}}}
\newcommand{\FS}{{\mathfrak{S}}}
\newcommand{\FT}{{\mathfrak{T}}}
\newcommand{\FU}{{\mathfrak{U}}}
\newcommand{\FV}{{\mathfrak{V}}}
\newcommand{\FW}{{\mathfrak{W}}}
\newcommand{\FX}{{\mathfrak{X}}}
\newcommand{\FY}{{\mathfrak{Y}}}
\newcommand{\FZ}{{\mathfrak{Z}}}

\newcommand{\CA}{{\cal A}}
\newcommand{\CB}{{\cal B}}
\newcommand{\CC}{{\cal C}}
\newcommand{\CD}{{\cal D}}
\newcommand{\CE}{{\cal E}}
\newcommand{\CF}{{\cal F}}
\newcommand{\CG}{{\cal G}}
\newcommand{\CH}{{\cal H}}
\newcommand{\CI}{{\cal I}}
\newcommand{\CJ}{{\cal J}}
\newcommand{\CK}{{\cal K}}
\newcommand{\CL}{{\cal L}}
\newcommand{\CM}{{\cal M}}
\newcommand{\CN}{{\cal N}}
\newcommand{\CO}{{\cal O}}
\newcommand{\CP}{{\cal P}}
\newcommand{\CQ}{{\cal Q}}
\newcommand{\CR}{{\cal R}}
\newcommand{\CS}{{\cal S}}
\newcommand{\CT}{{\cal T}}
\newcommand{\CU}{{\cal U}}
\newcommand{\CV}{{\cal V}}
\newcommand{\CW}{{\cal W}}
\newcommand{\CX}{{\cal X}}
\newcommand{\CY}{{\cal Y}}
\newcommand{\CZ}{{\cal Z}}

% Theorem Stil

\theoremstyle{plain}
\newtheorem{lem}{Lemma}
\newtheorem{Satz}[lem]{Satz}

\theoremstyle{definition}
\newtheorem{defn}{Definition}[section]

\theoremstyle{remark}
\newtheorem{bem}{Bemerkung}    %[section]



\newcommand{\card}{\mathop{\rm card}\nolimits}
\newcommand{\Sets}{((Sets))}
\newcommand{\id}{{\rm id}}
\newcommand{\supp}{\mathop{\rm Supp}\nolimits}

\newcommand{\ord}{\mathop{\rm ord}\nolimits}
\renewcommand{\mod}{\mathop{\rm mod}\nolimits}
\newcommand{\sign}{\mathop{\rm sign}\nolimits}
\newcommand{\ggT}{\mathop{\rm ggT}\nolimits}
\newcommand{\kgV}{\mathop{\rm kgV}\nolimits}
\renewcommand{\div}{\, | \,}
\newcommand{\notdiv}{\mathopen{\mathchoice
             {\not{|}\,}
             {\not{|}\,}
             {\!\not{\:|}}
             {\not{|}}
             }}

\newcommand{\im}{\mathop{{\rm Im}}\nolimits}
\newcommand{\coim}{\mathop{{\rm coim}}\nolimits}
\newcommand{\coker}{\mathop{\rm Coker}\nolimits}
\renewcommand{\ker}{\mathop{\rm Ker}\nolimits}

\newcommand{\pRang}{\mathop{p{\rm -Rang}}\nolimits}
\newcommand{\End}{\mathop{\rm End}\nolimits}
\newcommand{\Hom}{\mathop{\rm Hom}\nolimits}
\newcommand{\Isom}{\mathop{\rm Isom}\nolimits}
\newcommand{\Tor}{\mathop{\rm Tor}\nolimits}
\newcommand{\Aut}{\mathop{\rm Aut}\nolimits}

\newcommand{\adj}{\mathop{\rm adj}\nolimits}

\newcommand{\Norm}{\mathop{\rm Norm}\nolimits}
\newcommand{\Gal}{\mathop{\rm Gal}\nolimits}
\newcommand{\Frob}{{\rm Frob}}

\newcommand{\disc}{\mathop{\rm disc}\nolimits}

\renewcommand{\Re}{\mathop{\rm Re}\nolimits}
\renewcommand{\Im}{\mathop{\rm Im}\nolimits}

\newcommand{\Log}{\mathop{\rm Log}\nolimits}
\newcommand{\Res}{\mathop{\rm Res}\nolimits}
\newcommand{\Bild}{\mathop{\rm Bild}\nolimits}

\renewcommand{\binom}[2]{\left({#1}\atop{#2}\right)}
\newcommand{\eck}[1]{\langle #1 \rangle}
\newcommand{\gaussk}[1]{\lfloor #1 \rfloor}
\newcommand{\frack}[1]{\{ #1 \}}
\newcommand{\wi}{\hspace{1pt} < \hspace{-6pt} ) \hspace{2pt}}
\newcommand{\dreieck}{\bigtriangleup}

\parindent0mm

\begin{document}

\pagestyle{empty}

\begin{center}
{\huge IMO Selektion 2012} \\
\medskip erste Pr�fung - 28. April 2012
\end{center}
\vspace{8mm}
Zeit: 4.5 Stunden\\
Jede Aufgabe ist 7 Punkte wert.

\vspace{15mm}

\begin{enumerate}

\item[\textbf{1.}] 
Sei $n\ge 2$ eine nat�rliche Zahl. Finde in Abh�ngigkeit von $n$ die gr�sste nat�rliche Zahl $d$, sodass eine Permutation $a_1,a_2,\dots,a_n$ der Zahlen $1,2,\dots,n$ existiert mit 
\[|a_i-a_{i+1}| \geq d, \ \ \text{f�r }i=1,2,\dots,n-1.\]
\bigskip


\item[\textbf{2.}] 
Eine ganze Zahl $m$ ist eine \textit{echte Potenz}, falls es positive ganze Zahlen $a$ und $n$ gibt, sodass $n>1$ und $m=a^n$.
\begin{enumerate}
\item[(a)] Zeige, dass es $2012$ verschiedene positive ganze Zahlen gibt, sodass sich keine nichtleere Teilmenge davon zu einer echten Potenz aufsummiert.
\item[(b)] Zeige, dass es $2012$ verschiedene positive ganze Zahlen gibt, sodass sich jede nichtleere Teilmenge davon zu einer echten Potenz aufsummiert.
\end{enumerate}

\bigskip
\bigskip

\item[\textbf{3.}] 
Sei $ABCD$ ein Sehnenviereck mit Umkreis $k$. Sei $S$ der Schnittpunkt von $AB$ und $CD$ und $T$ der Schnittpunkt der Tangenten an $k$ in $A$ und $C$. Zeige, dass $ADTS$ genau dann ein Sehnenviereck ist, wenn $BD$ die Strecke $AC$ halbiert.

\end{enumerate}

\bigskip
\begin{center}
Viel Gl�ck !
\end{center}

\pagebreak



\begin{center}
{\huge IMO Selektion 2012} \\
\medskip zweite Pr�fung - 29. April 2012
\end{center}
\vspace{8mm}
Zeit: 4.5 Stunden\\
Jede Aufgabe ist 7 Punkte wert.

\vspace{15mm}

\begin{enumerate}

\item[\textbf{4.}] 
Im Dreieck $ABC$ sei $\angle BAC = 60^\circ$. Sei $E$ ein Punkt auf der Geraden $AB$, sodass $B$ zwischen $A$ und $E$ liegt und $BE = BC$ gilt. Analog sei $F$ ein Punkt auf $AC$, sodass $C$ zwischen $A$ und $F$ liegt und $CF = BC$ gilt. Der Umkreis von $ACE$ schneide $EF$ in $K$. Zeige, dass $K$ auf der Winkelhalbierenden von $\angle BAC$ liegt. 
\bigskip
\bigskip

\item[\textbf{5.}] Sei $n \geq 6$ eine nat�rliche Zahl. Betrachte eine Menge $S$ von $n$ verschiedenen reellen Zahlen. Beweise, dass es mindestens $n-1$ verschiedene zweielementige Teilmengen von $S$ gibt, sodass das arithmetische Mittel der beiden Elemente in jeder dieser Teilmengen mindestens gleich dem arithmetischen Mittel aller Elemente in $S$ ist.

\bigskip
\bigskip

\item[\textbf{6.}]Finde alle surjektiven Funktionen $f: \mathbb{R} \rightarrow \mathbb{R}$, sodass f�r alle $x,y \in \mathbb{R}$ gilt:

\[f(x+f(x)+2f(y)) = f(2x) + f(2y)\]

 
\end{enumerate}

\bigskip
\begin{center}
Viel Gl�ck !
\end{center}

\pagebreak


\begin{center}
{\huge IMO Selektion 2012} \\
\medskip dritte Pr�fung - 12. Mai 2012
\end{center}
\vspace{8mm}
Zeit: 4.5 Stunden\\
Jede Aufgabe ist 7 Punkte wert.

\vspace{15mm}

\begin{enumerate}
\item[\textbf{7.}] Seien $p,q$ zwei Primzahlen mit 

\[pq \ | \ 2012^{p+q-1} -1.\]

Zeige, dass genau eine der Primzahlen $2011$ ist.


\bigskip
\bigskip

\item[\textbf{8.}] Seien $f,g$ zwei Polynome mit ganzen Koeffizienten und seien $a,b$ ganzzahlige Fixpunkte von $f \circ g$. Beweise, dass ganzzahlige Fixpunkte $c,d$ von $g \circ f$ existieren mit $a+c=b+d$.\\
\textit{Hinweis: F�r zwei Polynome $p,q$ ist $p\circ q$ durch $(p\circ q)(x) = p(q(x))$ definiert.}


\bigskip
\bigskip

\item[\textbf{9.}] 
Bestimme die gr�sste nat�rliche Zahl $k$ mit der folgenden Eigenschaft: Die Menge der nat�rlichen Zahlen kann so in $k$ disjunkte Teilmengen  $A_1,\dots,A_k$ aufgeteilt werden, dass sich jede nat�rliche Zahl $n\ge 15$ f�r jedes $i \in \{1,\dots,k\}$ als Summe zweier verschiedener Elemente aus $A_i$ schreiben l�sst.

\end{enumerate}

\bigskip
\begin{center}
Viel Gl�ck !
\end{center}

\pagebreak


\begin{center}
{\huge IMO Selektion 2012} \\
\medskip vierte Pr�fung - 13. Mai 2012
\end{center}
\vspace{8mm}
Zeit: 4.5 Stunden\\
Jede Aufgabe ist 7 Punkte wert.

\vspace{15mm}

\begin{enumerate}
\item[\textbf{10.}] Seien $a,b,c$ positive reelle Zahlen mit $abc \geq 1$. Beweise die Ungleichung

\[\frac{a^4-1}{ab^3+abc+ac^3} + \frac{b^4-1}{bc^3+abc+ba^3} + \frac{c^4-1}{ca^3+abc+cb^3}\ge 0.\]


\bigskip
\bigskip


\item[\textbf{11.}] Sei $I$ der Inkreismittelpunkt und $AD$ der Durchmesser des Umkreises eines Dreiecks $ABC$. Seien $E$ und $F$ Punkte auf den Strahlen $BA$ und $CA$ mit

\[BE = CF = \frac{AB+BC+CA}{2}.\]

Zeige, dass sich die Geraden $EF$ und $DI$ rechtwinklig schneiden.

\bigskip
\bigskip

\item[\textbf{12.}] Finde alle ganzen Zahlen  $m,n \geq 2$, welche die folgenden zwei Bedingungen erf�llen:

\begin{enumerate}
\item[(i)] $m+1$ ist eine Primzahl von der Form $4k + 3$ f�r eine ganze Zahl $k$. 
\item[(ii)] Es existiert eine Primzahl $p$ und eine nichtnegative ganze Zahl $a$ mit

\[\frac{m^{2^n-1}-1}{m-1} = m^n + p^a.\]
\end{enumerate}
\end{enumerate}
\bigskip
\begin{center}
Viel Gl�ck !
\end{center}

\end{document}
