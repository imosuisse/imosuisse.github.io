\documentclass[12pt]{article}

\usepackage{amsfonts}
\usepackage{german}
\usepackage{amsthm}
\usepackage{amssymb}

\leftmargin=0pt \topmargin=0pt \headheight=0in \headsep=0in \oddsidemargin=0pt \textwidth=6.5in
\textheight=8.5in

\catcode`\� = \active \catcode`\� = \active \catcode`\� = \active \catcode`\� = \active \catcode`\� = \active
\catcode`\� = \active

\def�{"A}
\def�{"a}
\def�{"O}
\def�{"o}
\def�{"U}
\def�{"u}


\renewcommand{\phi}{\varphi}
\newcommand{\eps}{\varepsilon}

\newcommand{\pfeil}{\quad \Longrightarrow \quad}

\newcommand{\wi}{\mathop{\hspace{1pt}<\hspace{-6pt})\hspace{2pt}}\nolimits}


\newcommand{\binom}[2]{\left({#1}\atop{#2}\right)}
\newcommand{\nequiv}{\not \equiv}
\newcommand{\mod}{\mathop{\rm mod}\nolimits}
\newcommand{\Mod}[1]{(\mod\;#1)}
\newcommand{\ggT}{\mathop{\rm ggT}\nolimits}
\newcommand{\kgV}{\mathop{\rm kgV}\nolimits}
\newcommand{\ord}{\mathop{\rm ord}\nolimits}
\newcommand{\sgn}{\mathop{\rm sign}\nolimits}
\renewcommand{\div}{\, | \,}
\newcommand{\notdiv}{\mathopen{\mathchoice
             {\not{|}\,}
             {\not{|}\,}
             {\!\not{\:|}}
             {\not{|}}
             }}


\begin{document}

\begin{center}
{\huge Schweizer IMO - Selektion} \\
\medskip erste Pr�fung - 10. Mai 2003
\end{center}
\vspace{8mm}
Zeit: 4 Stunden\\
Jede Aufgabe ist 7 Punkte wert.

\vspace{15mm}

\begin{enumerate}

\item[\textbf{1.}] F�r die rellen Zahlen $x,y,a$ gelten die folgenden Gleichungen:
\begin{eqnarray*}
x+y &=& a\\
x^3+y^3 &=& a\\
x^5+y^5 &=& a.
\end{eqnarray*}
Bestimme alle m�glichen Werte von $a$.

\bigskip

\item[\textbf{2.}] Sei $ABC$ ein beliebiges spitzwinkliges Dreieck. $E$ und $F$ seien
die Fusspunkte der H�hen durch $B$ und $C$. $G$ ist die Projektion von $B$ auf die Gerade $EF$ und $H$ die
Projektion von $C$ auf $EF$. Zeige, dass gilt
\[|HE|=|FG|.\]

\bigskip

\item[\textbf{3.}] Finde die gr�sste reelle Zahl $C_{1}$ und die kleinste reelle Zahl $C_{2}$, sodass f�r alle positiven
Zahlen $a,b,c,d,e$ gilt
\[C_{1}<\frac{a}{a+b}+\frac{b}{b+c}+\frac{c}{c+d}+\frac{d}{d+e}+\frac{e}{e+a}<C_{2}.\]

\bigskip

\item[\textbf{4.}] Finde die gr�sste nat�rliche Zahl $n$, die
f�r jede ganze Zahl $a$ ein Teiler ist von $a^{25}-a$.

\bigskip

\item[\textbf{5.}] Auf einem Spielbrett mit $5 \times 9$ Feldern liegen $n$ Steine, wobei zu jedem Zeitpunkt auf
jedem Feld h�chstens ein Stein liegen darf. Ein Spielzug besteht darin, jeden Stein in eines der angrenzenden
Felder oben, unten, links oder rechts zu verschieben. Dies geschieht f�r alle Steine gleichzeitig. Wird dabei
ein Stein in einem Zug horizontal bewegt, dann muss er im n�chsten Zug vertikal bewegt werden und umgekehrt.
Bestimme den gr�ssten Wert f�r $n$, sodass es eine Anfangsposition der $n$ Steine und eine Folge von Spielz�gen
gibt, sodass dieses Spiel beliebig lange fortgesetzt werden kann.


\end{enumerate}


\pagebreak


\begin{center}
{\huge Schweizer IMO - Selektion} \\
\medskip zweite Pr�fung - 24. Mai 2003
\end{center}
\vspace{8mm}
Zeit: 4 Stunden\\
Jede Aufgabe ist 7 Punkte wert.

\vspace{15mm}

\begin{enumerate}

\item[\textbf{6.}] F�r die positiven reellen Zahlen $a,b,c$ gelte $a+b+c=2$. Zeige, dass die folgende
Ungleichung erf�llt ist und bestimme alle F�lle, in denen das Gleichheitszeichen steht:
\[\frac{1}{1+ab}+\frac{1}{1+bc}+\frac{1}{1+ca}\geq \frac{27}{13}\]

\bigskip

\item[\textbf{7.}] Finde alle Polynome $Q(x)=ax^2+bx+c$ mit ganzzahligen Koeffizienten, sodass drei
verschiedene Primzahlen $p_{1},p_{2},p_{3}$ existieren mit
\[|Q(p_{1})|=|Q(p_{2})|=|Q(p_{3})|=11.\]

\bigskip

\item[\textbf{8.}] Sei $A_{1}A_{2}A_{3}$ ein Dreieck und $\omega_1$ ein Kreis, der durch $A_1$ und $A_2$ geht.
Nehme an, es existieren Kreise $\omega_2, \ldots, \omega_7$ mit den folgenden Eigenschaften:
\begin{enumerate}
\item $\omega_k$ geht durch die Punkte $A_k$ und $A_{k+1}$ f�r $k=2,3, \ldots, 7$, \hspace{8mm}($A_i=A_{i+3}$)
\item $\omega_k$ und $\omega_{k+1}$ ber�hren sich �usserlich f�r $k=1,2, \ldots, 6$.
\end{enumerate}
Zeige: $\omega_1=\omega_7$.

\bigskip

\item[\textbf{9.}] Gegeben sind ganze Zahlen $0<a_{1}<a_{2}< \ldots <a_{101}<5050$, zeige, dass man daraus immer vier
verschiedene $a_{k},a_{l},a_{m},a_{n}$ ausw�hlen kann mit
\[5050 | (a_{k}+a_{l}-a_{m}-a_{n}).\]

\bigskip

\item[\textbf{10.}] Finde alle streng monotonen Funktionen $f: \mathbb{N}\rightarrow \mathbb{N}$, sodass f�r
alle $n \in \mathbb{N}$ gilt
\[f(f(n))=3n.\]

\end{enumerate}
\end{document}

