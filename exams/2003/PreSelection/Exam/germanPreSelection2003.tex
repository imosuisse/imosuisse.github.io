\documentclass[12pt]{article}

\usepackage{amsfonts}
\usepackage{german}
\usepackage{amsthm}
\usepackage{amssymb}

\leftmargin=0pt \topmargin=0pt \headheight=0in \headsep=0in \oddsidemargin=0pt \textwidth=6.5in
\textheight=8.5in

\catcode`\� = \active \catcode`\� = \active \catcode`\� = \active \catcode`\� = \active \catcode`\� = \active
\catcode`\� = \active

\def�{"A}
\def�{"a}
\def�{"O}
\def�{"o}
\def�{"U}
\def�{"u}

\begin{document}

\pagestyle{empty}
\newcommand{\wi}{\hspace{1pt} < \hspace{-6pt} ) \hspace{2pt}}

\begin{center}
{\huge Schweizer IMO - Vorselektion} \\
\medskip Bern, Z�rich - 5. April 2003
\end{center}
\vspace{8mm}
Zeit: 2 Stunden\\
Jede Aufgabe ist 7 Punkte wert.

\vspace{15mm}

\begin{enumerate}

\item[\textbf{1.}]67 Sch�ler schreiben eine Pr�fung. Die Pr�fung besteht aus 6 multiple-choice Fragen, die alle
mit \textit{ja} oder \textit{nein} beantwortet werden m�ssen. Jeder Sch�ler beantwortet dabei alle 6 Fragen.
Eine richtige Antwort auf die $k$-te Frage gibt $k$ Punkte, eine falsche Antwort $-k$ Punkte.
\begin{enumerate}
\item Zeige, dass mindestens zwei Sch�ler das Pr�fungsblatt gleich ausgef�llt haben.
\item Zeige, dass mindestens vier Sch�ler gleich viele Punkte erzielten.
\end{enumerate}

\bigskip

\item[\textbf{2.}] $ABC$ sei ein spitzwinkliges Dreieck mit Umkreismittelpunkt $O$. Das Lot von $A$ auf $BC$
schneide den Umkreis im Punkt $D \neq A$, und die Gerade $BO$ schneide den Umkreis im Punkt $E \neq B$. Zeige,
dass $ABC$ und $BDCE$ denselben Fl�cheninhalt haben.

\bigskip

\item[\textbf{3.}] Bestimme alle Funktionen $f: \mathbb{R} \rightarrow \mathbb{R}$, sodass f�r alle $x,y \in
\mathbb{R}$ die folgende Gleichung erf�llt ist:
\[f((x-y)^2)=x^2-2yf(x)+(f(y))^2\]

\bigskip

\item[\textbf{4.}] Betrachte eine Tabelle mit $m$ Zeilen und $n$ Spalten. Auf wieviele Arten kann diese Tabelle
mit lauter Nullen und Einsen ausgef�llt werden, sodass in jeder Zeile und jeder Spalte eine gerade Anzahl Einsen
stehen?

\bigskip

\item[\textbf{5.}] Beweise f�r positive reelle Zahlen $x,y,z$ mit $x+y+z=1$ die folgende Ungleichung:
\[\frac{x^2+y^2}{z}+\frac{y^2+z^2}{x}+\frac{z^2+x^2}{y}\geq 2\]


\end{enumerate}

\end{document}

