\documentclass[12pt]{article}

\usepackage{amsfonts}
\usepackage{german}
\usepackage{amsthm}
\usepackage{amssymb}

\leftmargin=0pt \topmargin=0pt \headheight=0in \headsep=0in \oddsidemargin=0pt \textwidth=6.5in
\textheight=8.5in


\begin{document}

\pagestyle{empty}
\newcommand{\wi}{\hspace{1pt} < \hspace{-6pt} ) \hspace{2pt}}

\begin{center}
{\huge OIM Suisse - Pr\'{e}s\'{e}lection} \\
\medskip Berne, Zurich - 5 avril, 2003
\end{center}
\vspace{8mm}
Dur\'{e}e: 2 heures\\
Chaque probl\`{e}me vaut 7 points.\\
\vspace{15mm}

\begin{enumerate}

\item[\textbf{1.}]67 \'{e}l\`{e}ves doivent passer un examen. Celui-ci
comprend 6 questions \`{a} choix multiples qui doivent toutes
\^{e}tre r\'{e}pondues soit par \textit{oui} soit par
\textit{non}. Chaque \'{e}l\`{e}ve r\'{e}pond \`{a} toutes les six
questions. Une r\'{e}ponse correcte \`{a} la $k$-i\`{e}me question
donne $k$ points, une fausse $-k$ points.
\begin{enumerate}
\item D\'{e}montrez qu'au moins deux \'{e}l\`{e}ves doivent forc\'{e}ment donner les m\`{e}mes
r\'{e}ponses.
\item D\'{e}montrez qu'au moins quatre \'{e}l\`{e}ves ont obtenu le m\`{e}me nombre
de points.
\end{enumerate}

\bigskip

\item[\textbf{2.}] Soit $ABC$ un triangle \`{a} angles aigus et soit $O$ le
centre du cercle circonscrit du triangle. La hauteur issue de $A$
sur $BC$ coupe le cercle en un point $D \neq A$, et la droite $BO$
coupe le cercle en un point $E \neq B$. D\'{e}montrez qu'$ABC$ et
$BCDE$ ont la m\`{e}me aire.

\bigskip

\item[\textbf{3.}] Trouvez toutes les fonctions $f: \mathbb{R} \rightarrow \mathbb{R}$
 qui satisfont  l'\'{e}quation suivante pour tout $x,y \in
 \mathbb{R}$:
\[f((x-y)^2)=x^2-2yf(x)+(f(y))^2\]

\bigskip

\item[\textbf{4.}] Etant donn\'{e} un tableau \`{a} $m$ lignes et $n$
colonnes, de combien de fa\c{c}ons diff\'{e}rentes peut-on remplir
ce tableau avec des $0$ et des $1$ pour que dans chaque ligne et
chaque colonne il y ait un nombre pair de $1$?

\bigskip

\item[\textbf{5.}] D\'{e}montrez l'in\'{e}galit\'{e} suivante pour tout les nombres reels
 $x,y,z$ avec $x+y+z=1$:
\[\frac{x^2+y^2}{z}+\frac{y^2+z^2}{x}+\frac{z^2+x^2}{y}\geq 2\]


\end{enumerate}

\end{document}
