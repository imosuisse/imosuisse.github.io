\documentclass[12pt]{article}
\usepackage{amsfonts}
\leftmargin=0pt \topmargin=0pt \headheight=0in \headsep=0in \oddsidemargin=0pt \textwidth=6.5in
\textheight=8.5in

\usepackage{german}

\catcode`\� = \active \catcode`\� = \active \catcode`\� = \active \catcode`\� = \active \catcode`\� = \active
\catcode`\� = \active

\def�{"A}
\def�{"a}
\def�{"O}
\def�{"o}
\def�{"U}
\def�{"u}

\begin{document}

\pagestyle{empty}
\newcommand{\wi}{\hspace{1pt} < \hspace{-6pt} ) \hspace{2pt}}

\begin{center}
{\huge IMO-qualification suisse 2002} \\
\medskip Premier test - 10 mai 2002
\end{center}
\vspace{8mm}
Dur\'{e}e: 3.5 heures\\
Chaque probl\`{e}me vaut 7 points.

\vspace{15mm}

\begin{enumerate}
\item Soient 24 points dans l'espace, dont il y a pas 3 colin\'{e}aires.
Chaque 3 points forment un plan, et c'est connu que ainsi exactement 2002 plans
sont form\'{e}s. A prouver qu'il existe un plan contenant au moins 6 de points.


\item $O$ soit un point \`{a} l'int\'{e}rieur du parallelogramme $ABCD$,
tel que $\wi AOB  + \wi DOC = \pi$.\\
A montrer que:\\
$$\wi CBO = \wi CDO$$


\item $n$ soit un nombre positif entier qui a au moins quatre diviseurs
positifs distincts, dont les quatre les plus petits soient $d_{1}, d_{2}, d_{3}, d_{4}$.
Trouve tous les nombres $n$, tels que
$$d_{1}^{2} + d_{2}^{2} + d_{3}^{2} + d_{4}^{2} = n.$$


\item Consid\`{e}re un champ carr\'{e} divis\'{e} par des lignes horizontales
verticales en $7 \times 7$ carr\'{e}s d'unit\'{e}. Dans ce champ on veut mettre
des Croix de Suisse (consistant d'un carr\'{e} au centre et un carr\'{e} attach\'{e}
aux quatre c\^{o}t\'{e}s du carr\'{e} central) de la mani\`{e}re que les bords des croix
soient allign\'{e}s aux lignes divisant le champ. D\'{e}t\'{e}rmine le nombre minimal
de carr\'{e}s qu'il faut marquer de sorte que chaque Croix, n'importe o\`{u} elle est plac\'{e}e
sur le camp, couvre au moins un carr\'{e} marqu\'{e}.


\item D\'{e}t\'{e}rmine toutes les fonctions $f:\mathbb{R} \rightarrow \mathbb{R}$, pour lesquelles:
\begin{enumerate}
  \item l'ensemble $\{\frac{f(x)}{x} | x \in \mathbb{R}, x\neq0\}$ est fini et
  \item $f(x-1-f(x))=f(x)-1-x$ pour tous les
   $x \in \mathbb{R}$.
\end{enumerate}

\end{enumerate}

\vspace{20mm}
\begin{center}
Bonne chance!
\end{center}


\pagebreak


\begin{center}
{\huge IMO-qualification suisse 2002} \\
\medskip Second test - 25 mai 2002
\end{center}
\vspace{8mm}
Dur\'{e}e: 3.5 heures\\
Chaque probl\`{e}me vaut 7 points.

\vspace{15mm}

\begin{enumerate}
\item  Soit $x_1 , x_2 , x_3 , ...$ une suite des entiers avec les
propri\'{e}t\'{e}s suivantes:
\begin{itemize}
\item $1 = x_1 < x_2 < x_3 < ...$
\item $x_{n+1}\leq 2n$ pour $n = 1, 2, 3, ...$
\end{itemize}
Montre qu'ils existent pour chaque nombre entier positive $k$ deux
indices $i$ et $j$ tels que\\
$k = x_{i} - x_{j}$.


\item Soit $ABC$ un triangle \'{e}quilateral et $P$ un point \`{a}
l'int\'{e}rieur. Soient $X, Y$ et $Z$ les projections de $P$ sur
les c\^{o}t\'{e}s $BC, CA$ et $AB$. Montre que la somme des
surfaces des triangles $BXP, CYP$ et $AZP$ ne d\'{e}pendend pas de
$P$.


\item Dans un groupe de $n$ personnes, il y a chaque week-end
quelqu'un qui organise une f\^{e}te o\`{u} il pr\'{e}sente ceux
qu'il conna\^{i}t entre eux. Apr\`{e}s que chacun des $n$
personnes ait organis\'{e} une f\^{e}te, il y en a toujours deux
qui ne se connaissent pas.\\
Montre que ces deux personnes ne vont jamais faire la connaissance
de l'autre \`{a} une f\^{e}te pareille.\\
(Deux personnes se connaissent mutuellement ou ne se connaissent
pas mutuellement)

\item Montre l'in\'{e}galit\'{e} suivante pour des nombres reels
$a$ et chaque $n\geq1$ entier:
$$a^{n} + \frac{1}{a^{n}} - 2 \geq n^{2}(a + \frac{1}{a} - 2)$$
et d\'{e}termine tous les cas d'\'{e}galit\'{e}.


\item Soi $m$ un nombre naturel. D\'{e}termine, en d\'{e}pendance de $m$
le plus petit nombre naturel $k$ avec: Si $\{m, m+1, ... , k\} = A
\cup B$ est une partition arbitraire en deux ensembles, $A$ ou $B$
eontient trois \'{e}l\'{e}ments $a, b, c$ (pas n\'{e}cessairement
diff\'{e}rents) tels que $a^{b} = c$.


\end{enumerate}
\vspace{20mm}
\begin{center}
Bonne chance!
\end{center}

\end{document}

