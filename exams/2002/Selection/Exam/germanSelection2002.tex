\documentclass[12pt,a4paper]{article}

\usepackage{amsfonts}
\usepackage[centertags]{amsmath}
\usepackage{german}
\usepackage{amsthm}
\usepackage{amssymb}

\leftmargin=0pt \topmargin=0pt \headheight=0in \headsep=0in \oddsidemargin=0pt \textwidth=6.5in
\textheight=8.5in

\catcode`\� = \active \catcode`\� = \active \catcode`\� = \active \catcode`\� = \active \catcode`\� = \active
\catcode`\� = \active

\def�{"A}
\def�{"a}
\def�{"O}
\def�{"o}
\def�{"U}
\def�{"u}







% Schriftabk�rzungen

\newcommand{\eps}{\varepsilon}
\renewcommand{\phi}{\varphi}
\newcommand{\Sl}{\ell}    % sch�nes l
\newcommand{\ve}{\varepsilon}  %Epsilon

\newcommand{\BA}{{\mathbb{A}}}
\newcommand{\BB}{{\mathbb{B}}}
\newcommand{\BC}{{\mathbb{C}}}
\newcommand{\BD}{{\mathbb{D}}}
\newcommand{\BE}{{\mathbb{E}}}
\newcommand{\BF}{{\mathbb{F}}}
\newcommand{\BG}{{\mathbb{G}}}
\newcommand{\BH}{{\mathbb{H}}}
\newcommand{\BI}{{\mathbb{I}}}
\newcommand{\BJ}{{\mathbb{J}}}
\newcommand{\BK}{{\mathbb{K}}}
\newcommand{\BL}{{\mathbb{L}}}
\newcommand{\BM}{{\mathbb{M}}}
\newcommand{\BN}{{\mathbb{N}}}
\newcommand{\BO}{{\mathbb{O}}}
\newcommand{\BP}{{\mathbb{P}}}
\newcommand{\BQ}{{\mathbb{Q}}}
\newcommand{\BR}{{\mathbb{R}}}
\newcommand{\BS}{{\mathbb{S}}}
\newcommand{\BT}{{\mathbb{T}}}
\newcommand{\BU}{{\mathbb{U}}}
\newcommand{\BV}{{\mathbb{V}}}
\newcommand{\BW}{{\mathbb{W}}}
\newcommand{\BX}{{\mathbb{X}}}
\newcommand{\BY}{{\mathbb{Y}}}
\newcommand{\BZ}{{\mathbb{Z}}}

\newcommand{\Fa}{{\mathfrak{a}}}
\newcommand{\Fb}{{\mathfrak{b}}}
\newcommand{\Fc}{{\mathfrak{c}}}
\newcommand{\Fd}{{\mathfrak{d}}}
\newcommand{\Fe}{{\mathfrak{e}}}
\newcommand{\Ff}{{\mathfrak{f}}}
\newcommand{\Fg}{{\mathfrak{g}}}
\newcommand{\Fh}{{\mathfrak{h}}}
\newcommand{\Fi}{{\mathfrak{i}}}
\newcommand{\Fj}{{\mathfrak{j}}}
\newcommand{\Fk}{{\mathfrak{k}}}
\newcommand{\Fl}{{\mathfrak{l}}}
\newcommand{\Fm}{{\mathfrak{m}}}
\newcommand{\Fn}{{\mathfrak{n}}}
\newcommand{\Fo}{{\mathfrak{o}}}
\newcommand{\Fp}{{\mathfrak{p}}}
\newcommand{\Fq}{{\mathfrak{q}}}
\newcommand{\Fr}{{\mathfrak{r}}}
\newcommand{\Fs}{{\mathfrak{s}}}
\newcommand{\Ft}{{\mathfrak{t}}}
\newcommand{\Fu}{{\mathfrak{u}}}
\newcommand{\Fv}{{\mathfrak{v}}}
\newcommand{\Fw}{{\mathfrak{w}}}
\newcommand{\Fx}{{\mathfrak{x}}}
\newcommand{\Fy}{{\mathfrak{y}}}
\newcommand{\Fz}{{\mathfrak{z}}}

\newcommand{\FA}{{\mathfrak{A}}}
\newcommand{\FB}{{\mathfrak{B}}}
\newcommand{\FC}{{\mathfrak{C}}}
\newcommand{\FD}{{\mathfrak{D}}}
\newcommand{\FE}{{\mathfrak{E}}}
\newcommand{\FF}{{\mathfrak{F}}}
\newcommand{\FG}{{\mathfrak{G}}}
\newcommand{\FH}{{\mathfrak{H}}}
\newcommand{\FI}{{\mathfrak{I}}}
\newcommand{\FJ}{{\mathfrak{J}}}
\newcommand{\FK}{{\mathfrak{K}}}
\newcommand{\FL}{{\mathfrak{L}}}
\newcommand{\FM}{{\mathfrak{M}}}
\newcommand{\FN}{{\mathfrak{N}}}
\newcommand{\FO}{{\mathfrak{O}}}
\newcommand{\FP}{{\mathfrak{P}}}
\newcommand{\FQ}{{\mathfrak{Q}}}
\newcommand{\FR}{{\mathfrak{R}}}
\newcommand{\FS}{{\mathfrak{S}}}
\newcommand{\FT}{{\mathfrak{T}}}
\newcommand{\FU}{{\mathfrak{U}}}
\newcommand{\FV}{{\mathfrak{V}}}
\newcommand{\FW}{{\mathfrak{W}}}
\newcommand{\FX}{{\mathfrak{X}}}
\newcommand{\FY}{{\mathfrak{Y}}}
\newcommand{\FZ}{{\mathfrak{Z}}}

\newcommand{\CA}{{\cal A}}
\newcommand{\CB}{{\cal B}}
\newcommand{\CC}{{\cal C}}
\newcommand{\CD}{{\cal D}}
\newcommand{\CE}{{\cal E}}
\newcommand{\CF}{{\cal F}}
\newcommand{\CG}{{\cal G}}
\newcommand{\CH}{{\cal H}}
\newcommand{\CI}{{\cal I}}
\newcommand{\CJ}{{\cal J}}
\newcommand{\CK}{{\cal K}}
\newcommand{\CL}{{\cal L}}
\newcommand{\CM}{{\cal M}}
\newcommand{\CN}{{\cal N}}
\newcommand{\CO}{{\cal O}}
\newcommand{\CP}{{\cal P}}
\newcommand{\CQ}{{\cal Q}}
\newcommand{\CR}{{\cal R}}
\newcommand{\CS}{{\cal S}}
\newcommand{\CT}{{\cal T}}
\newcommand{\CU}{{\cal U}}
\newcommand{\CV}{{\cal V}}
\newcommand{\CW}{{\cal W}}
\newcommand{\CX}{{\cal X}}
\newcommand{\CY}{{\cal Y}}
\newcommand{\CZ}{{\cal Z}}

% Theorem Stil

\theoremstyle{plain}
\newtheorem{lem}{Lemma}
\newtheorem{Satz}[lem]{Satz}

\theoremstyle{definition}
\newtheorem{defn}{Definition}[section]

\theoremstyle{remark}
\newtheorem{bem}{Bemerkung}    %[section]



\newcommand{\card}{\mathop{\rm card}\nolimits}
\newcommand{\Sets}{((Sets))}
\newcommand{\id}{{\rm id}}
\newcommand{\supp}{\mathop{\rm Supp}\nolimits}

\newcommand{\ord}{\mathop{\rm ord}\nolimits}
\renewcommand{\mod}{\mathop{\rm mod}\nolimits}
\newcommand{\sign}{\mathop{\rm sign}\nolimits}
\newcommand{\ggT}{\mathop{\rm ggT}\nolimits}
\newcommand{\kgV}{\mathop{\rm kgV}\nolimits}
\renewcommand{\div}{\, | \,}
\newcommand{\notdiv}{\mathopen{\mathchoice
             {\not{|}\,}
             {\not{|}\,}
             {\!\not{\:|}}
             {\not{|}}
             }}

\newcommand{\im}{\mathop{{\rm Im}}\nolimits}
\newcommand{\coim}{\mathop{{\rm coim}}\nolimits}
\newcommand{\coker}{\mathop{\rm Coker}\nolimits}
\renewcommand{\ker}{\mathop{\rm Ker}\nolimits}

\newcommand{\pRang}{\mathop{p{\rm -Rang}}\nolimits}
\newcommand{\End}{\mathop{\rm End}\nolimits}
\newcommand{\Hom}{\mathop{\rm Hom}\nolimits}
\newcommand{\Isom}{\mathop{\rm Isom}\nolimits}
\newcommand{\Tor}{\mathop{\rm Tor}\nolimits}
\newcommand{\Aut}{\mathop{\rm Aut}\nolimits}

\newcommand{\adj}{\mathop{\rm adj}\nolimits}

\newcommand{\Norm}{\mathop{\rm Norm}\nolimits}
\newcommand{\Gal}{\mathop{\rm Gal}\nolimits}
\newcommand{\Frob}{{\rm Frob}}

\newcommand{\disc}{\mathop{\rm disc}\nolimits}

\renewcommand{\Re}{\mathop{\rm Re}\nolimits}
\renewcommand{\Im}{\mathop{\rm Im}\nolimits}

\newcommand{\Log}{\mathop{\rm Log}\nolimits}
\newcommand{\Res}{\mathop{\rm Res}\nolimits}
\newcommand{\Bild}{\mathop{\rm Bild}\nolimits}

\renewcommand{\binom}[2]{\left({#1}\atop{#2}\right)}
\newcommand{\eck}[1]{\langle #1 \rangle}
\newcommand{\wi}{\hspace{1pt} < \hspace{-6pt} ) \hspace{2pt}}


\begin{document}

\pagestyle{empty}

\begin{center}
{\huge Schweizer IMO - Selektion} \\
\medskip erste Pr�fung - 4. Mai 2002
\end{center}
\vspace{8mm}
Zeit: 3.5 Stunden\\
Jede Aufgabe ist 7 Punkte wert.

\vspace{15mm}

\begin{enumerate}

\item[\textbf{1.}] Gegeben sind 24 Punkte im Raum. Je drei dieser Punkte spannen eine Ebene auf, und es ist
bekannt, dass die 24 Punkte auf diese Weise genau 2002 verschiedene Ebenen aufspannen. Beweise, dass eine dieser
Ebenen mindestens 6 der Punkte enth�lt.

\bigskip

\item[\textbf{2.}] Gegeben sei ein Parallelogramm $ABCD$ und ein Punkt $O$ in dessen Innern, sodass
$\wi AOB+\wi DOC = \pi$. Zeige dass gilt
\[\wi CBO = \wi CDO\]

\bigskip

\item[\textbf{3.}] $n$ sei eine positive ganze Zahl mit mindestens vier verschiedenen positiven Teilern. Die
vier kleinsten unter diesen Teilern seien $d_{1}, d_{2}, d_{3}, d_{4}$. Finde alle solchen Zahlen $n$, f�r die
gilt
\[d_{1}^{2} + d_{2}^{2} + d_{3}^{2} + d_{4}^{2} = n.\]

\bigskip

\item[\textbf{4.}] Betrachte ein $7 \times 7$ Feld, das in 49 Einheitsquadrate unterteilt ist. In dieses Feld
wollen wir Kacheln der Form eines Schweizerkreuzes, bestehend aus 5 Einheitsquadraten, hineinlegen. Dabei sollen
die Kanten der Kreuze auf den Linien des Feldes zu liegen kommen. Bestimme die kleinstm�gliche Anzahl Quadrate,
die auf dem Feld markiert werden m�ssen, damit jedes Kreuz, egal wo es auf das Feld gelegt wird, mindestens ein
markiertes Quadrat bedeckt.

\bigskip

\item[\textbf{5.}] Bestimme alle Funktionen $f:\BR \rightarrow \BR$, f�r die gilt:
\begin{enumerate}
\item $f(x-1-f(x))=f(x)-1-x$ f�r alle $x \in \mathbb{R}$,
\item Die Menge $\{f(x)/x\; |\; x \in \BR,\; x\neq 0\}$ ist endlich.
\end{enumerate}

\end{enumerate}

\pagebreak


\begin{center}
{\huge Schweizer IMO - Selektion} \\
\medskip zweite Pr�fung - 25. Mai 2002
\end{center}
\vspace{8mm}
Zeit: 3.5 Stunden\\
Jede Aufgabe ist 7 Punkte wert.

\vspace{15mm}

\begin{enumerate}

\item[\textbf{6.}]  Sei $x_1 , x_2 , x_3 , \ldots$ eine Folge ganzer Zahlen mit den Eigenschaften
\begin{itemize}
\item $1 = x_1 < x_2 < x_3 < \ldots,$
\item $x_{n+1}\leq 2n \quad$ f�r $\quad n \geq 1.$
\end{itemize}
Zeige, dass es zu jeder positiven ganzen Zahl $k$ zwei Indizes $i$ und $j$ gibt mit $k = x_{i} - x_{j}$.

\bigskip

\item[\textbf{7.}] Sei $ABC$ ein gleichseitiges Dreieck und $P$ ein Punkt in dessen Innern. $X, Y$ und $Z$
seien die Fusspunkte der Lote von $P$ auf die Seiten $BC, CA$ und $AB$. Zeige dass die Summe der Fl�chen der
Dreiecke $BXP$, $CYP$ und $AZP$ nicht von $P$ abh�ngt.

\bigskip

\item[\textbf{8.}] In einer Gruppe von $n$ Leuten veranstaltet jedes Wochenende jemand eine Party,
an der er alle seine Bekannten einander gegenseitig vorstellt. Nachdem jeder der $n$ Leute einmal eine Party
gemacht hat, gibt es immer noch zwei Personen unter ihnen, die sich nicht kennen. Zeige, dass diese zwei sich
auch in Zukunft nie an einer dieser Partys kennen lernen werden. (Zwei Leute kennen sich immer gegenseitig oder
gegenseitig nicht)

\bigskip

\item[\textbf{9.}] Beweise f�r jede positive reelle Zahl $a$ und jedes ganze $n\geq1$ die Ungleichung
\[a^{n} + \frac{1}{a^{n}} - 2 \geq n^{2}\left(a + \frac{1}{a} - 2\right),\]
und bestimme alle F�lle, in denen das Gleichheitszeichen gilt.

\bigskip

\item[\textbf{10.}] $m$ sei eine beliebige nat�rliche Zahl. Bestimme in Abh�ngigkeit von $m$ die
kleinste nat�rliche Zahl $k$, f�r die gilt: Ist $\{m, m+1, \ldots , k\} = A \cup B$ eine beliebige Zerlegung in
zwei Mengen $A$ und $B$, dann enth�lt $A$ oder $B$ drei Elemente $a, b, c$ (die nicht notwendigerweise
verschieden sein m�ssen) mit $a^{b} = c$.

\end{enumerate}

\end{document}
