\en{\en{Let $n > 6$ be a perfect number. Let $p_1^{a_1}\cdot p_2^{a_2}\cdot \ldots\cdot p_k^{a_k}$ be the prime factorisation of $n$ where we assume that $p_1 < p_2<\ldots<p_k$ and $a_i>0$ for all $i=1,\ldots, k$. Prove that $a_1$ is even.

\emph{Remark: An integer $n\geq 2$ is called a \emph{perfect number} if the sum of its positive divisors, excluding $n$ itself, is equal to $n$. For example, $6$ is perfect, as its positive divisors are $\{1, 2, 3, 6\}$ and $1+2+3 = 6$.} 
}

\de{Sei $n > 6$ eine perfekte Zahl. Sei $p_1^{a_1}\cdot p_2^{a_2}\cdot \ldots\cdot p_k^{a_k}$ die Primfaktorzerlegung von $n$, wobei wir annehmen, dass $p_1 < p_2<\ldots<p_k$ und $a_i>0$ für alle $i=1,\ldots, k$. Beweise, dass $a_1$ gerade ist.

\emph{Bemerkung: Eine ganze Zahl $n\geq 2$ heisst \emph{perfekte Zahl}, wenn die Summe ihrer positiven Teiler, ausser $n$ selbst, gleich $n$ ist. Zum Beispiel ist $6$ eine perfekte Zahl, da ihre positiven Teiler $\{1, 2, 3, 6\}$ sind und $1+2+3 = 6$ ergibt.} }

\fr{Soit $n>6$ un nombre parfait. Soit $p_1^{a_1}\cdot p_2^{a_2}\cdot \ldots\cdot p_k^{a_k}$ la décomposition en facteurs premiers de $n$, où l'on suppose que $p_1 < p_2<\ldots<p_k$ et $a_i>0$ pour tout $i=1,\ldots, k$. Montrer que $a_1$ est pair.

\emph{Remarque : Un nombre entier $n\geq 2$ est un \emph{nombre parfait} si la somme de ses diviseurs positifs, excepté $n$ lui-même, vaut $n$. Par exemple, $6$ est un nombre parfait, car ses diviseurs positifs sont $\{1, 2, 3, 6\}$ et $1+2+3=6$.}}

\bigskip
\textbf{Solution:} Since $n$ is perfect, we can write
$$2n = \sum_{1 \leq d\mid n}d = \sum_{0\leq b_i \leq a_i}p_1^{b_1}p_2^{b_2}\cdots p_k^{b_k} = \prod_{i=1}^k(1+p_i+\cdots+p_i^{a_i}).$$
Now assuming $a_1$ is odd, we find that
$$(1+p_1+\cdots+p_1^{a_1}) \equiv 1-1+\cdots-1 \equiv 0\pmod {p_1+1}$$
and therefore $p_1+1\mid 2n$. 

If $p_1 > 2$, then $p_1+1\mid n$, but since $p_1$ is the smallest prime divisor of $n$, no prime divisor of $p_1+1$ can divide $n$, leading to a contradiction. 

We conclude that $p_1 = 2\mid n$. Since $p_1+1 = 3\mid 2n$, we also get $3\mid n$. But now note that since $n > 6$, the integers $1, n/2, n/3, n/6$ are distinct, proper divisors of $n$ which sum to $n+1 > n$, contradicting the fact that $n$ is perfect. We conclude that $a_1$ must be even.

\textbf{Note:} The case where $2\mid n$ can also be solved as follows. The Euler-Euclid Theorem says that if $n$ is an even perfect number, then there exists a prime $p$ such that $n=2^{p-1}(2^p-1)$. Since $n>6$, then $p>2$ is odd and so $a_1=p-1$ is even.

\textbf{Marking Scheme:}
\begin{itemize}
    \item Any attempt to sort the divisors of $n$ according to powers of $p_1$ or writing down the formula $2n=\prod_{i=1}^k (1+p_i+\cdots+p_i^{a_i})$ \dotfill (1 point)
    \item Stating that $1+p_1+\ldots+p_1^{a_1}$ divides $2n$ \dotfill (1 point)
    \item Proving that if $a_1$ is odd, then $p_1 +1\mid 2n$ \dotfill (2 points)
    \item In the case that $n$ is even: proving that $a_1$ is even directly, or getting a contradiction assuming that $a_1$ is odd \dotfill (2 points)
    \item In the case that $n$ is odd: proving that $a_1$ is even directly, or getting a contradiction assuming that $a_1$ is odd \dotfill (1 point)
\end{itemize}
}