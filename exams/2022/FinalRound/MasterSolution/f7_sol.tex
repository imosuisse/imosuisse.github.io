\en{\en {
Let $m\geq n$ be positive integers. Frieder is given $mn$ posters of Linus with different integer dimensions $k \times l$ with $1 \le k \le m$ and $1 \le l \le n$. He must put them all up one by one on his bedroom wall without rotating them. Every time he puts up a poster, he can either put it on an empty spot on the wall, or on a spot where it entirely covers a single visible poster and does not overlap any other visible poster. Determine the minimal area of the wall that will be covered by posters.
%\\
%\emph{Remark: a wall is a structure often made of bricks and concrete (or cardboard if you live in Winterthur) commonly found in houses.}
}

\de {
Seien $m\geq n$ natürliche Zahlen. Frieder werden $mn$ Poster von Linus mit unterschiedlichen ganzzahligen Dimensionen $k \times l$ mit $1 \le k \le m$ und $1 \le l \le n$ überreicht. Eines nach dem anderen muss er alle Poster an seiner Schlafzimmerwand aufhängen ohne sie zu rotieren. Jedes Poster das er aufhängt kann entweder auf einen freien Platz an der Wand gehängt werden, oder an einen Platz wo es ein einziges sichtbares Poster komplett überdeckt und kein anders sichtbares Poster überlappt. Bestimme die minimale Fläche der Wand die am Ende mit Poster bedeckt ist.
}

\fr {Soient $m\geq n$ des entiers strictement positifs. On confie à Frieder $mn$ posters de Linus, tous avec des dimensions entières différentes $k\times l$ où $1\leq k\leq m$ et $1\leq l\leq n$. Il doit tous les accrocher, un par un, sur le mur de sa chambre sans les pivoter. Chaque fois qu'il accroche un nouveau poster, il peut soit le placer sur un emplacement libre du mur, ou de manière à ce qu'il recouvre complètement un poster déjà accroché sans pour autant chevaucher un autre poster déjà accroché. Déterminer l'aire minimale du mur qui sera recouverte par des posters.

}

\ita {

}

\bigskip
\textbf{Solution:} Since $n$ is perfect, we can write
$$2n = \sum_{1 \leq d\mid n}d = \sum_{0\leq b_i \leq a_i}p_1^{b_1}p_2^{b_2}\cdots p_k^{b_k} = \prod_{i=1}^k(1+p_i+\cdots+p_i^{a_i}).$$
Now assuming $a_1$ is odd, we find that
$$(1+p_1+\cdots+p_1^{a_1}) \equiv 1-1+\cdots-1 \equiv 0\pmod {p_1+1}$$
and therefore $p_1+1\mid 2n$. 

If $p_1 > 2$, then $p_1+1\mid n$, but since $p_1$ is the smallest prime divisor of $n$, no prime divisor of $p_1+1$ can divide $n$, leading to a contradiction. 

We conclude that $p_1 = 2\mid n$. Since $p_1+1 = 3\mid 2n$, we also get $3\mid n$. But now note that since $n > 6$, the integers $1, n/2, n/3, n/6$ are distinct, proper divisors of $n$ which sum to $n+1 > n$, contradicting the fact that $n$ is perfect. We conclude that $a_1$ must be even.

\textbf{Note:} The case where $2\mid n$ can also be solved as follows. The Euler-Euclid Theorem says that if $n$ is an even perfect number, then there exists a prime $p$ such that $n=2^{p-1}(2^p-1)$. Since $n>6$, then $p>2$ is odd and so $a_1=p-1$ is even.

\textbf{Marking Scheme:}
\begin{itemize}
    \item Any attempt to sort the divisors of $n$ according to powers of $p_1$ or writing down the formula $2n=\prod_{i=1}^k (1+p_i+\cdots+p_i^{a_i})$ \dotfill (1 point)
    \item Stating that $1+p_1+\ldots+p_1^{a_1}$ divides $2n$ \dotfill (1 point)
    \item Proving that if $a_1$ is odd, then $p_1 +1\mid 2n$ \dotfill (2 points)
    \item In the case that $n$ is even: proving that $a_1$ is even directly, or getting a contradiction assuming that $a_1$ is odd \dotfill (2 points)
    \item In the case that $n$ is odd: proving that $a_1$ is even directly, or getting a contradiction assuming that $a_1$ is odd \dotfill (1 point)
\end{itemize}
}