\en{\de{Eine Gruppe von Kindern sitzt im Kreis. Am Anfang hat jedes Kind eine gerade Anzahl Bonbons. In jedem Schritt muss jedes Kind die Hälfte seiner Bonbons dem Kind zu seiner Rechten abgeben. Sollte ein Kind nach einem Schritt eine ungerade Anzahl Bonbons haben, bekommt es vom Kindergärtner ein zusätzliches Bonbon geschenkt. Zeige, dass nach einer endlichen Anzahl Schritten alle Kinder gleich viele Bonbons haben.}
\fr{Un groupe d'enfants est assis en cercle. Au début, chaque enfant possède un nombre pair de bonbons. À chaque tour, chaque enfant donne la moitié de ses bonbons à l'enfant assis à sa droite. Si, après un tour, un enfant possède un nombre impair de bonbons, le professeur lui donne un bonbon supplémentaire. Montrer qu'après un nombre fini de tours tous les enfants auront le même nombre de bonbons.}
\ita{}
\en{A group of children sits in a circle. At the beginning, each child has an even number of sweets. In each step, each child has to give half of its sweets to the child sitting on its right. If after a step a child has an odd number of sweets, it gets one additional sweet from the teacher. Show that after a finite number of steps all children have the same amount of sweets. }

\bigskip

\textbf{Solution:} Let $p_i=a_i/\delta (a_i)$ be the smallest divisor of $a_i$ not equal to $1$. We then have 
\[
a_{i+1} =a_i + \frac{a_i}{p_i}= (p_i+1)\frac{a_i}{p_i}.
\]
So, if $p_1$ is odd, then $a_{i+1}$ will be even. Note that the sequence $a_n$ fulfills the condition exactly when there exists a $k$, such that $v_3(a_k) \ge 2022$. 
%If $a_i$ is odd, then $a_{i+1} = \frac{p_i+1}{p_i}a_i$ is even. If $a_i$ is even, then $a_{i+1} = \frac{3}{2}a_i$. It follows that $v_3(a_{i+1}) = v_3(a_i)+1$. In other words if $a_i$ is even, then $v_3$ increases and $v_2$ decreases. So at some point we get an odd number divisible by $3$. So when does $v_3(a_i)$ decrease? Well looking at the note above we see that it only decreases if $p_i = 3$ (since $p_i$ must be prime); So in other words if $a_i$ is odd and divisible by 3. In this case $a_{i+1} = \frac{4}{3} a_i$ is divisble by 4. We find that $a_{i+2} = \frac{3}{2}\frac{4}{3}a_i= 2a_i$ and $a_{i+3} = \frac{3}{2}a_{i+1} = 3a_i$. Now most importantly the number $a_{i+3} = 3a_i$ is also odd and divisble by 3. So by induction we have that $v_3(a_{i+3k}) = v_3(a_i) + k$. Also for $l\ge i+3k$ we have $v_3(a_{l}) \ge v_3(a_i)+k-1$.\\ Finally summarising together all our findings, we have that $v_3(a_i)$ always somewhat increases and if it goes down by 1 it actually goes up by 2 again directly after. So after the first $a_k$ with $v_3(a_k)\ge 2023$, all the $a_i, i\ge k$ are divisible by $3^{2022}$.

We differentiate between the three cases:
\begin{itemize}
    \item \textbf{($a_i$ even)}: In this case we have that $a_{i+1} = \frac{3}{2}a_i$. So here $v_3(a_{i+1}) = v_3(a_i)+1$
    \item \textbf{($a_i$ is odd and not divisible by 3)}: In this case $v_3(a_{i+1}) = v_3((p_i+1)\frac{a_i}{p_i}) \geq  v_3(a_i)$ and then by the above case $v_3(a_{i+2}) \ge v_3(a_i)+1$.
    \item \textbf{($a_i$ is odd and divisble by 3)}: In this case we have $a_{i+1}=\frac{4}{3}a_i$ and so $a_{i+1}$ is even. Thus, $a_{i+2} = \frac{3}{2}a_{i+1} = 2a_i$ and $a_{i+3} = \frac{3}{2}a_{i+2} = 3a_i$. So, here we get $v_3(a_{i+3}) \geq  v_3(a_i)+1$.
\end{itemize}
So, if we start with the number $a_1$, we have a subsequence $(a_{i_l})_{l\geq 1}$, with $v_3(a_{i_{l+1}}) \geq v_3(a_{i_{l}})+1$. By that we have $$3^{2022} \div a_{i_{2022}}.$$ 

\bigskip

\textbf{Marking scheme:}

The following additive partial points can be obtained:
\begin{itemize}
\item The idea of comparing $v_3(a_{i+1})$ with $v_3(a_i)$ \dotfill ($1$ points)
\item Note that for the smallest prime divisor $p_i$ of $a_i$, we have $a_{i+1} = p_{i+1}\frac{a_i}{p_i}$. \dotfill ($1$ points)
\item Look at the $a_i$ even case \dotfill ($1$ points)
\item Look at case $a_i$ odd and not divisible by 3 \dotfill ($1$ points)
\item Look at case $a_i$ odd and divisible by 3 \dotfill ($2$ points)
\item Conclude \dotfill ($1$ points)
\end{itemize}
}


\de{Für eine ganze Zahl $a\geq 2$, bezeichne mit $\delta(a)$ den zweitgrössten Teiler von $a$. Sei $(a_n)_{n\geq 1}$ eine Folge von positiven ganzen Zahlen, so dass $a_1\geq 2$ und 
\begin{align*}
a_{n+1} = a_n + \delta(a_n)
\end{align*}
für alle $n\geq 1$. Beweise, dass es eine positive ganze Zahl $k$ gibt, so dass $a_k$ durch $3^{2022}$ teilbar ist.}