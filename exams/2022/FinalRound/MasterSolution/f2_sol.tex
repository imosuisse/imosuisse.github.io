\en{

\de{Sei $\mathbb{P}$ die Menge aller Primzahlen und $M$ eine Teilmenge von $\mathbb{P}$ mit mindestens drei Elementen, sodass folgende Eigenschaft gilt: Für jede natürliche Zahl $k$ und für jede Teilmenge $A=\{p_1, p_2, \ldots, p_k\}$ von $M$ mit $A\neq M$ sind alle Primfaktoren von $p_1\cdot p_2 \cdot \ldots \cdot p_k - 1$ in $M$. Zeige, dass $M = \mathbb{P}$ gilt.}
\fr{Soit $\mathbb{P}$ l'ensemble de tous les nombres premiers et $M$ un sous-ensemble de $\mathbb{P}$ ayant au moins trois éléments. On suppose que pour tout entier $k\geq 1$ et pour tout sous-ensemble $A=\{p_1, p_2, \ldots, p_k\}$ de $M$ tel que $A\neq M$, tous les facteurs premiers du nombre $p_1\cdot p_2 \cdot \ldots \cdot p_k - 1$ se trouvent dans $M$. Montrer que $M = \mathbb{P}$.}
%\textit{Remarque: un sous-ensemble $Y$ d'un ensemble $X$ est appelé \emph{non-trivial} si $Y$ est non-vide et $Y\neq X$.}}
\ita{}
\en{Let $ \mathbb{P}$ be the set of all primes and let $M$ be a subset of $\mathbb{P}$, having at least three elements, and such that the following property is satisfied: For any positive integer $k$ and for any subset $A=\{p_1, p_2, \ldots ,p_k\}$ of $M$ with $A\neq M$, all of the prime factors of the number $p_1\cdot p_2 \cdot \ldots \cdot p_k - 1$ are contained in $M$. Prove that $M= \mathbb{P}$.}

\textbf{Solution (Johann):}
Let's prove the statement by induction. The base case $n=1$ is trivial. Let's prove the inductive step. Given that $1^1,\ 3^3,\ 5^5,\ \dots,\ (2^n-1)^{2^n-1}$ have different residues $\pmod 2^n$, we want to show that $1^1,\ 3^3,\ 5^5,\ \dots,\ (2^{n+1}-1)^{2^{n+1}-1}$ give different results mod $2^{n+1}$. Let's split the $2^n$ elements in two sets $A$ and $B$, where $A = \{ 1^1,\ 3^3,\ 5^5,\ \dots,\ (2^n-1)^{2^n-1}\}$ and $B = \{(2^n+1)^{2^n+1},\ (2^n+3)^{2^n+3},\ (2^n+5)^{2^n+5}, \ \dots,\ (2^n+(2^n-1))^{2^n+(2^n-1)}\} $. Let's closely examine the elements in $B$ mod $2^{n+1}$. By Euler-Fermat (we can use it in this case because $\gcd(2^n+k, 2^{n+1}) = 1$) and the binomial expansion, we have:
$$(2^n+k)^{2^n+k} \equiv (2^n+k)^k \equiv \sum_{s=0}^{k}{k \choose s} \cdot 2^{ns} \cdot k^{k-s} \equiv 2^n \cdot k^{k}+  k^k \equiv 2^n +k^k \pmod {2^{n+1}}$$
In other words, $B \equiv \{1^1+2^n,\ 3^3+2^n,\ 5^5+2^n,\ \dots,\ (2^n-1)^{2^n-1}+2^n\} \pmod {2^{n+1}}$. Now, by the inductive hypothesis, the elements in $A$ are distinct $\pmod {2^n}$ and so are the elements in $B$. Finally, since the elements in $B \pmod {2^{n+1}}$ are simply obtained by adding $2^n$ to the elements in $A$, the $2^n$ numbers give different remainders$\pmod {2^{n+1}}$, concluding the inductive step. 
Hence, the initial statement is true for all $n \in \N$.

\bigskip

\textbf{Marking scheme:}

\textbf{Partial solutions} \dotfill ($\leq 4$ points)

The following additive partial points can be obtained:
\begin{enumerate}
\item Any attempt to use induction or to pair up the numbers (even in a way which might not work, eg. $\{k^k, (2^{n+1}-k)^{2^{n+1}-k}\}$ ) \dotfill (1 point)
\item Any attempt of considering pairs of the form $\{k^k, (2^n+k)^{2^n+k}\}$ or splitting the numbers into the two sets $A$ and $B$ \dotfill ($1$ point)
\item Observing $(2^n+k)^{2^n} \equiv 1 \pmod{2^{n+1}}$ \dotfill ($1$ point)
\item Proving that $(2^n+k)^{2^n+k} \equiv 2^n + k^k \pmod {2^{n+1}}$ \dotfill ($1$ point)
\end{enumerate}

\textbf{Complete solutions} \dotfill ($\geq 6$ points)

At most one point can be deducted if one has not explicitly stated one of the following elements:
\begin{itemize}
\item Forgetting the base case \dotfill ($-1$ point)
\item Not mentioning $\gcd(2^n+k, 2^{n+1}) = 1$ for Euler-Fermat \dotfill ($-1$ point)
\end{itemize}
}