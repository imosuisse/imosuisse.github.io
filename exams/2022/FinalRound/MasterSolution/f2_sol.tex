\en{\de{
Sei $ABC$ ein spitzwinkliges Dreieck. Seien $M_A$, $M_B$ und $M_C$ die Mittelpunkte der Seiten $BC$, $CA$, respektive $AB$. Seien $M_A'$, $M_B'$ und $M_C'$ die Mittelpunkte der Kreisbögen über den jeweiligen Seiten $BC$, $CA$ und $AB$ auf dem Umkreis von $ABC$. Sei $P_A$ der Schnittpunkt der Gerade $M_BM_C$ und der Senkrechten zu $M_B'M_C'$ durch $A$. Definiere $P_B$ und $P_C$ analog. Zeige, dass die Geraden $M_AP_A$, $M_BP_B$ und $M_CP_C$ sich in einem Punkt schneiden.}
\fr {Soit $ABC$ un triangle aigu. Soient $M_A$, $M_B$ et $M_C$ les milieux respectifs des côtés $BC$, $CA$ et $AB$. Soient $M_A',M_B'$ et $M_C'$ les milieux respectifs des arcs mineurs $BC$, $CA$ et $AB$ sur le cercle circonscrit au triangle $ABC$. Soit $P_A$ l'intersection de la droite $M_BM_C$ et de la perpendiculaire à $M_B'M_C'$ par $A$. Les points $P_B$ et $P_C$ sont définis de manière analogue. Montrer que les droites $M_AP_A$, $M_BP_B$ et $M_CP_C$ se coupent en un point.}
\ita{}
\en{
Let $ABC$ be an acute triangle. Denote by $M_A$, $M_B$ and $M_C$ the midpoints of sides $BC$, $CA$ and $AB$, respectively. Let $M_A'$, $M_B'$ and $M_C'$ be respectively the midpoints of the minor arcs $BC$, $CA$ and $AB$ on the circumcircle of $ABC$. Let $P_A$ be the intersection of the lines $M_BM_C$ and the perpendicular to $M_B'M_C'$ containing $A$. Let $P_B$ and $P_C$ be defined analogously. Prove that the lines $M_AP_A$, $M_BP_B$ and $M_CP_C$ meet at a point.}

\textbf{Solution (Johann):}
Let's prove the statement by induction. The base case $n=1$ is trivial. Let's prove the inductive step. Given that $1^1,\ 3^3,\ 5^5,\ \dots,\ (2^n-1)^{2^n-1}$ have different residues $\pmod 2^n$, we want to show that $1^1,\ 3^3,\ 5^5,\ \dots,\ (2^{n+1}-1)^{2^{n+1}-1}$ give different results mod $2^{n+1}$. Let's split the $2^n$ elements in two sets $A$ and $B$, where $A = \{ 1^1,\ 3^3,\ 5^5,\ \dots,\ (2^n-1)^{2^n-1}\}$ and $B = \{(2^n+1)^{2^n+1},\ (2^n+3)^{2^n+3},\ (2^n+5)^{2^n+5}, \ \dots,\ (2^n+(2^n-1))^{2^n+(2^n-1)}\} $. Let's closely examine the elements in $B$ mod $2^{n+1}$. By Euler-Fermat (we can use it in this case because $\gcd(2^n+k, 2^{n+1}) = 1$) and the binomial expansion, we have:
$$(2^n+k)^{2^n+k} \equiv (2^n+k)^k \equiv \sum_{s=0}^{k}{k \choose s} \cdot 2^{ns} \cdot k^{k-s} \equiv 2^n \cdot k^{k}+  k^k \equiv 2^n +k^k \pmod {2^{n+1}}$$
In other words, $B \equiv \{1^1+2^n,\ 3^3+2^n,\ 5^5+2^n,\ \dots,\ (2^n-1)^{2^n-1}+2^n\} \pmod {2^{n+1}}$. Now, by the inductive hypothesis, the elements in $A$ are distinct $\pmod {2^n}$ and so are the elements in $B$. Finally, since the elements in $B \pmod {2^{n+1}}$ are simply obtained by adding $2^n$ to the elements in $A$, the $2^n$ numbers give different remainders$\pmod {2^{n+1}}$, concluding the inductive step. 
Hence, the initial statement is true for all $n \in \N$.

\bigskip

\textbf{Marking scheme:}

\textbf{Partial solutions} \dotfill ($\leq 4$ points)

The following additive partial points can be obtained:
\begin{enumerate}
\item Any attempt to use induction or to pair up the numbers (even in a way which might not work, eg. $\{k^k, (2^{n+1}-k)^{2^{n+1}-k}\}$ ) \dotfill (1 point)
\item Any attempt of considering pairs of the form $\{k^k, (2^n+k)^{2^n+k}\}$ or splitting the numbers into the two sets $A$ and $B$ \dotfill ($1$ point)
\item Observing $(2^n+k)^{2^n} \equiv 1 \pmod{2^{n+1}}$ \dotfill ($1$ point)
\item Proving that $(2^n+k)^{2^n+k} \equiv 2^n + k^k \pmod {2^{n+1}}$ \dotfill ($1$ point)
\end{enumerate}

\textbf{Complete solutions} \dotfill ($\geq 6$ points)

At most one point can be deducted if one has not explicitly stated one of the following elements:
\begin{itemize}
\item Forgetting the base case \dotfill ($-1$ point)
\item Not mentioning $\gcd(2^n+k, 2^{n+1}) = 1$ for Euler-Fermat \dotfill ($-1$ point)
\end{itemize}
}