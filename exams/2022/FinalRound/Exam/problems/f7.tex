\en{Let $n > 6$ be a perfect number. Let $p_1^{a_1}\cdot p_2^{a_2}\cdot \ldots\cdot p_k^{a_k}$ be the prime factorisation of $n$ where we assume that $p_1 < p_2<\ldots<p_k$ and $a_i>0$ for all $i=1,\ldots, k$. Prove that $a_1$ is even.

\emph{Remark: An integer $n\geq 2$ is called a \emph{perfect number} if the sum of its positive divisors, excluding $n$ itself, is equal to $n$. For example, $6$ is perfect, as its positive divisors are $\{1, 2, 3, 6\}$ and $1+2+3 = 6$.} 
}

\de{Sei $n > 6$ eine perfekte Zahl. Sei $p_1^{a_1}\cdot p_2^{a_2}\cdot \ldots\cdot p_k^{a_k}$ die Primfaktorzerlegung von $n$, wobei wir annehmen, dass $p_1 < p_2<\ldots<p_k$ und $a_i>0$ für alle $i=1,\ldots, k$. Beweise, dass $a_1$ gerade ist.

\emph{Bemerkung: Eine ganze Zahl $n\geq 2$ heisst \emph{perfekte Zahl}, wenn die Summe ihrer positiven Teiler, ausser $n$ selbst, gleich $n$ ist. Zum Beispiel ist $6$ eine perfekte Zahl, da ihre positiven Teiler $\{1, 2, 3, 6\}$ sind und $1+2+3 = 6$ ergibt.} }

\fr{Soit $n>6$ un nombre parfait. Soit $p_1^{a_1}\cdot p_2^{a_2}\cdot \ldots\cdot p_k^{a_k}$ la décomposition en facteurs premiers de $n$, où l'on suppose que $p_1 < p_2<\ldots<p_k$ et $a_i>0$ pour tout $i=1,\ldots, k$. Montrer que $a_1$ est pair.

\emph{Remarque : Un nombre entier $n\geq 2$ est un \emph{nombre parfait} si la somme de ses diviseurs positifs, excepté $n$ lui-même, vaut $n$. Par exemple, $6$ est un nombre parfait, car ses diviseurs positifs sont $\{1, 2, 3, 6\}$ et $1+2+3=6$.}}