\documentclass[11pt,a4paper]{article}

\usepackage[utf8]{inputenc}
\usepackage[ngerman]{babel}%ngerman, french, italian
\usepackage{amsmath, amsthm, amssymb}
\usepackage{mathtools}%DeclarePairedDelimiter
\usepackage{enumerate}

\usepackage{graphicx}

\leftmargin=0pt \topmargin=0pt \headheight=0in \headsep=0in \oddsidemargin=0pt \textwidth=6.5in
\textheight=8.5in


\theoremstyle{plain}
\newtheorem{thm}{Theorem}[section]
\newtheorem{satz}[thm]{Satz}
\newtheorem{lem}[thm]{Lemma}
\newtheorem{cor}[thm]{Korollar}
\newtheorem{alg}[thm]{Algorithmus}
\newtheorem{bsp}{Beispiel}

\theoremstyle{definition}
\newtheorem{defn}{Definition}[section]

\renewcommand{\phi}{\varphi} %kleines Phi
\newcommand{\nequiv}{\not \equiv}

\DeclareMathOperator{\ggT}{ggT}
\DeclareMathOperator{\kgV}{kgV}
\DeclareMathOperator{\ord}{ord}

\DeclarePairedDelimiter\abs{\lvert}{\rvert}
\DeclarePairedDelimiter\floor{\lfloor}{\rfloor}
\DeclarePairedDelimiter\frack{\{}{\}}

\renewcommand{\div}{\, | \,}
\newcommand{\ndiv}{\mathopen{\mathchoice{\not{|}\,}{\not{|}\,}{\!\not{\:|}}{\not{|}}}}
\newcommand{\R}{\mathbb{R}}
\newcommand{\N}{\mathbb{N}}
\newcommand{\Z}{\mathbb{Z}}

\begin{document}

\pagestyle{empty}

\begin{center}
{\huge SMO Finalrunde 2014} \\
\medskip erste Prüfung - 14. März 2014
\end{center}
\vspace{8mm}
Zeit: 4 Stunden\\
Jede Aufgabe ist 7 Punkte wert.

\vspace{15mm}

\begin{enumerate}
\item[\textbf{1.}]
Die Punkte $A, B, C$ und $D$ liegen in dieser Reihenfolge auf dem Kreis $k$. Sei $t$ die Tangente an $k$ durch $C$ und $s$ die Spiegelung von $AB$ an $AC$. Sei $G$ der Schnittpunkt der Geraden $AC$ und $BD$ und $H$ der Schnittpunkt der Geraden $s$ und $CD$. Zeige, dass $GH$ parallel zu $t$ ist.

\bigskip

\item[\textbf{2.}]
Seien $a,b$ natürliche Zahlen, für die gilt:
\[ab(a-b)\div a^3+b^3+ab\]
Zeige, dass $\kgV(a,b)$ eine Quadratzahl ist.

\bigskip

\item[\textbf{3.}] 
Finde alle Funktionen $f:\R \rightarrow \R$, sodass für alle $x,y \in \R$ folgende Gleichung gilt:
\[
f(x^2) + f(xy)=f(x)f(y)+yf(x)+xf(x+y)
\]

\bigskip

\item[\textbf{4.}]
Gegeben sei die karierte Ebene (unendlich grosses Häuschenpapier). Für welche Paare $(a,b)$ kann man jedes der Felder mit einer von $a\cdot b$ Farben färben, sodass jedes Rechteck der Grösse $a \times b$ oder $b \times a$, welches passend in die karierte Ebene gelegt wird, stets ein Einheitsquadrat jeder Farbe enthält?

\bigskip

\item[\textbf{5.}]
Sei $a_1, a_2, \ldots$ eine Folge ganzer Zahlen, sodass für jedes $n \in \N$ gilt:
 \[
\sum_{d\div n}a_d = 2^n
\]
Zeige für jedes $n \in \N$, dass $n$ ein Teiler von $a_n$ ist.

\emph{Bemerkung: Für $n=6$ lautet die Gleichung $a_1+a_2+a_3+a_6=2^6$.}
\end{enumerate}
\bigskip
\begin{center}
Viel Glück !
\end{center}

\newpage

\begin{center}
{\huge SMO Finalrunde 2014} \\
\medskip zweite Prüfung - 15. März 2014
\end{center}
\vspace{8mm}
Zeit: 4 Stunden\\
Jede Aufgabe ist 7 Punkte wert.

\vspace{15mm}

\begin{enumerate}

\item[\textbf{6.}]
Seien $a,b,c \in \R_{\geq 0}$ mit $a+b+c=1$. Zeige, dass gilt:
\[
\frac{3-b}{a+1}+\frac{a+1}{b+1}+\frac{b+1}{c+1}\geq 4
\]

\bigskip

\item[\textbf{7.}]

An einem runden See liegen $n\geq 4$ Städte, zwischen denen $n-4$ Personenfähren und eine grüne Autofähre verkehren. Jede Fähre verbindet zwei nicht benachbarte Städte, wobei sich keine zwei Fährenrouten überkreuzen, damit Kollisionen vermieden werden können.\\
Um die Transportrouten besser den Bedürfnissen der Passagiere anzupassen, kann folgende Änderung vorgenommen werden: Einer beliebigen Fähre kann eine neue Route zugeordnet werden. Dabei dürfen die Routen der restlichen Fähren nicht überkreuzt und auch nicht gleichzeitig verändert werden.\\
Seien Santa Marta und Kapstadt zwei nicht benachbarte Städte. Zeige, dass man endlich viele Routenänderungen vornehmen kann, sodass die grüne Autofähre nach diesen Änderungen zwischen Santa Marta und Kapstadt verkehrt.

\emph{Bemerkung: Zu keinem Zeitpunkt dürfen zwei Fähren zwischen denselben Städten oder eine Fähre zwischen zwei benachbarten Städten verkehren.}

\bigskip

\item[\textbf{8.}]

Im spitzwinkligen Dreieck $ABC$ sei $M$ der Mittelpunkt der Höhe $h_b$ durch $B$ und $N$ der Mittelpunkt der Höhe $h_c$ durch $C$. Weiter sei $P$ der Schnittpunkt von $AM$ und $h_c$ und $Q$ der Schnittpunkt von $AN$ und $h_b$. Zeige, dass $M, N, P$ und $Q$ auf einem Kreis liegen.

\bigskip

\item[\textbf{9.}]
Die Folge $a_1, a_2, \ldots $ ganzer Zahlen sei wie folgt definiert:
\[
a_n = 
\begin{cases}
0 \text{\,,\, falls $n$ eine gerade Anzahl an Teilern grösser als 2014 besitzt}\\
1 \text{\,,\, falls $n$ eine ungerade Anzahl an Teilern grösser als 2014 besitzt}
\end{cases}
\] 
Zeige, dass die Folge $a_n$ nie periodisch wird.

\bigskip

\item[\textbf{10.}] 
Sei $k$ ein Kreis mit Durchmesser $AB$. Sei $C$ ein Punkt auf der Geraden $AB$, sodass $B$ zwischen $A$ und $C$ liegt. Sei $T$ ein Punkt auf $k$, sodass $CT$ eine Tangente an $k$ ist. Sei $l$ die Parallele zu $CT$ durch $A$ und $D$ der Schnittpunkt von $l$ und der Senkrechten zu $AB$ durch $T$.\\
Zeige, dass die Gerade $DB$ die Strecke $CT$ halbiert.

\end{enumerate}
\bigskip
\begin{center}
Viel Glück !
\end{center}
\end{document}
