\documentclass[11pt,a4paper]{article}

\usepackage[german]{babel}
\usepackage[utf8]{inputenc} 

\usepackage{amsfonts}
\usepackage{amsmath}
\usepackage{amsthm}
\usepackage{amssymb}

\oddsidemargin=10pt
\textwidth=15cm

\begin{document}

\pagestyle{empty}

\begin{center}
{\huge SMO - Vorrunde} \\
\medskip Zürich, Lausanne, Lugano - 11. Januar 2014
\end{center}
\vspace{8mm}
Zeit: 3 Stunden\\
Jede Aufgabe ist 7 Punkte wert.

\vspace{15mm}

\begin{enumerate}

\item[\textbf{1.}] 
Bestimme alle natürlichen Zahlen $n>1$, für die $n-d$ ein Teiler von $n$ ist, wobei $d$ der kleinste positive Teiler von $n$ ist, der grösser als 1 ist.

\bigskip

\item[\textbf{2.}] 
Zwei Kreise $k_1,k_2$ mit Mittelpunkten $M_1$ bzw. $M_2$ schneiden sich in den Punkten $A$ und $B$. Die Tangente an $k_1$ durch $A$ schneidet $k_2$ ein weiteres Mal im Punkt $P$, während die Gerade $M_1B$ $k_2$ ein weiteres Mal im Punkt $Q$ schneidet. Nehme an, $Q$ liege ausserhalb von $k_1$ und es gelte $P\neq Q$. Zeige, dass $PQ$ parallel zu $M_1M_2$ ist.

\bigskip

\item[\textbf{3.}]
Wie viele achtstellige natürliche Zahlen gibt es, für die jede Ziffer entweder strikt grösser als alle Ziffern links davon oder strikt kleiner als alle Ziffern links davon ist?

\bigskip

\item[\textbf{4.}]
Jeder Eckpunkt eines regelmässigen 11-Ecks ist mit genau 4 anderen Eckpunkten durch gerade Strecken verbunden. Zeige, dass man eine Strecke hinzufügen kann, so dass es mindestens ein gleichschenkliges Dreieck gibt. 

\bigskip

\item[\textbf{5.}] 
Bestimme alle Primzahlen $p$, so dass natürliche Zahlen $n,m$ existieren mit
\[
\frac{p^2+1}{p+1} = \frac{n^2}{m^2}.
\]

\bigskip

\end{enumerate}

\vspace{1.5cm}

\center{Viel Glück!}
\end{document}
