\RequirePackage[l2tabu]{nag}

\newif\ifgerman \newif\iffrench \newif\ifitalian
% v v v Select language
\germanfalse
\frenchtrue
\italianfalse
% ^ ^ ^ Select language
\newif \iftitlepage
\titlepagefalse

\documentclass[12pt,a4paper]{article}

\usepackage{parskip}
\usepackage[left=24mm, right=24mm, top=30mm, bottom=20mm]{geometry}
\usepackage[utf8]{inputenc}
\usepackage[T1]{fontenc}
\ifgerman \usepackage[ngerman]{babel}\fi
\iffrench \usepackage[french]{babel}\fi
\ifitalian \usepackage[italian]{babel}\fi
\usepackage{amsmath, amsthm, amssymb}
\usepackage{mathtools}%DeclarePairedDelimiter
\usepackage{graphicx}

\begingroup %parskip and amsthm fix
\makeatletter
   \@for\theoremstyle:=definition,remark,plain\do{%
     \expandafter\g@addto@macro\csname th@\theoremstyle\endcsname{%
        \addtolength\thm@preskip\parskip
     }%
   }
\endgroup

\theoremstyle{plain}
\ifgerman \newtheorem{thm}{Theorem}[section] \fi
\iffrench \newtheorem{thm}{Théorèm}[section] \fi
\ifitalian \newtheorem{thm}{Teorema}[section] \fi
\ifgerman \newtheorem{satz}[thm]{Satz} \fi
\iffrench \newtheorem{satz}[thm]{Théorèm} \fi
\ifitalian \newtheorem{satz}[thm]{Teorema} \fi
\ifgerman \newtheorem{lem}[thm]{Lemma} \fi
\iffrench \newtheorem{lem}[thm]{Lemme} \fi
\ifitalian \newtheorem{lem}[thm]{Lemma} \fi
\ifgerman \newtheorem{cor}[thm]{Korollar} \fi
\iffrench \newtheorem{cor}[thm]{Corollaire} \fi
\ifitalian \newtheorem{cor}[thm]{Corollario} \fi
\ifgerman \newtheorem{alg}[thm]{Algorithmus} \fi
\iffrench \newtheorem{alg}[thm]{Algorithme} \fi
\ifitalian \newtheorem{alg}[thm]{Algoritmo} \fi
\ifgerman \newtheorem{bsp}{Beispiel} \fi
\iffrench \newtheorem{bsp}{Exemple} \fi
\ifitalian \newtheorem{bsp}{Esempio} \fi
\theoremstyle{definition}
\ifgerman \newtheorem{defn}{Definition}[section] \fi
\iffrench \newtheorem{defn}{Définition}[section] \fi
\ifitalian \newtheorem{defn}{Definizione}[section] \fi

\DeclareMathOperator{\ggT}{ggT}
\DeclareMathOperator{\kgV}{kgV}
\DeclareMathOperator{\ord}{ord}
\DeclarePairedDelimiter\abs{\lvert}{\rvert}
\DeclarePairedDelimiter\floor{\lfloor}{\rfloor}
\DeclarePairedDelimiter\frack{\{}{\}}

\renewcommand{\phi}{\varphi} %kleines Phi
\newcommand{\nequiv}{\not \equiv}
\renewcommand{\div}{\, | \,}

\newcommand{\ndiv}{\mathopen{\mathchoice{\not{|}\,}{\not{|}\,}{\!\not{\:|}}{\not{|}}}}
\newcommand{\R}{\mathbb{R}}
\newcommand{\N}{\mathbb{N}}
\newcommand{\Z}{\mathbb{Z}}

\begin{document}

\pagestyle{empty}

\begin{center}
{\huge Sélection IMO 2014} \\
\medskip Premier examen - 3 mai 2014
\end{center}
\vspace{8mm}
Zeit: 4.5 heures\\
Chaque exercice vaut 7 points.

\vspace{15mm}

\begin{enumerate}

\item[\textbf{1.}] Trouver toutes les fonctions $f: \mathbb{N} \to \mathbb{N}$, telles que pour tous $m,n\in \mathbb{N}$, on ait
\[m^2+f(n)|mf(m)+n.\]


\bigskip
\bigskip

\item[\textbf{2.}] Soient $2n$ jetons côte à côte, en ligne. Un coup consiste à échanger deux jetons voisins. Combien doit-on effectuer de coups au minimum pour que chaque jeton ait été au moins une fois au début et à la fin de la ligne ?

\bigskip
\bigskip

\item[\textbf{3.}] Soient $4$ points dans le plan disposés de sorte que les $4$ triangles qu'ils forment possèdent tous le même rayon du cercle inscrit. Montrer que les $4$ triangles sont égaux.

\end{enumerate}

\vspace{15mm}
\begin{center}
Bonne chance !
\end{center}

\pagebreak




\begin{center}
{\huge Sélection IMO 2014} \\
\medskip Deuxième examen - 4 mai 2014
\end{center}
\vspace{8mm}
Zeit: 4.5 heures\\
Chaque exercice vaut 7 points.

\vspace{15mm}

\begin{enumerate}

\item[\textbf{4.}] Déterminer tous les polynômes $P$ à coefficients réels tels que pour tout $x \in\mathbb{R}$, on ait
\[(x+2014)P(x)=xP(x+1).\]
\bigskip
\bigskip
\item[\textbf{5.}] Soit $ABC$ un triangle dans lequel $\alpha=\angle BAC$ est le plus petit angle (strictement). Soit $P$ un point sur le côté $BC$ et $D$ un point de la droite $AB$ tel que $B$ se situe entre $A$ et $D$ et $\angle BPD=\alpha$. De même, soit $E$ un point de la droite $AC$ tel que $C$ se situe entre $A$ et $E$ et $\angle EPC=\alpha$. Montrer que les droites $AP$, $BE$ et $CD$ sont concourantes si et seulement si $AP$ est perpendiculaire à $BC$.

\bigskip
\bigskip

\item[\textbf{6.}] Montrer qu'il n'existe pas deux nombres entiers naturels distincts tels que leur moyenne harmonique, géométrique, arithmétique et quadratique soient toutes des nombres entiers naturels.


 
\end{enumerate}

\vspace{15mm}
\begin{center}
Bonne chance !
\end{center}

\pagebreak


\begin{center}
{\huge Sélection IMO 2014} \\
\medskip troisième examen - 17 mai 2014
\end{center}
\vspace{8mm}
Temps : 4 heures 30\\
Chaque exercice vaut 7 points.

\vspace{15mm}

\begin{enumerate}
\item[\textbf{7.}] Deux cercles $\omega_1$ et $\omega_2$ se touchent tangentiellement en un point $A$ et se trouvent à l'intérieur d'un cercle $\Omega$. De plus, $\omega_1$ touche tangentiellement $\Omega$ en un point $B$ et $\omega_2$ touche tangentiellement $\Omega$ en un point $C$. La droite $AC$ coupe $\omega_1$ une deuxième fois en un point $D$.\\
Montrer que le triangle $DBC$ est rectangle si $A$, $B$ et $C$ ne sont pas alignés.

\bigskip
\bigskip

\item[\textbf{8.}] Trouver toutes les fonctions $f:\mathbb{R} \rightarrow \mathbb{R}$ telles que pour tous $x,y \in \mathbb{R}$ :
\[f(f(x)-y^2)= f(x^2) +y^2f(y) -2f(xy)\]

\bigskip
\bigskip

\item[\textbf{9.}] Soit $n$ un nombre naturel et $A=\{P_1,P_2,\dots,P_n\}$ un ensemble de $n$ points dans le plan tel que trois de ces points ne sont jamais alignés. Un \textit{chemin à travers $A$} consiste en $n-1$ segments $P_{\sigma(i)}P_{\sigma(i+1)}$ avec $i=1,\dots,n-1$, où $\sigma$ est une permutation de $\{1,2,\dots,n\}$, tel qu'aucun segment ne coupe un autre.\\
Montrer que le nombre de chemins distincts à travers $A$ est minimal si et seulement si les points de $A$ forment un $n$-gone convexe.



%Sei $n$ eine natürliche Zahl. Der Präsident von IMO-Land möchte die $n$ Städte seines Landes mit $n-1$ Strassen verbinden, sodass folgendes gilt:
%\begin{enumerate}
%\item[(i)] Man kann von jeder Stadt in jede andere Stadt entlang der $n-1$ Strassen fahren.
%\item[(ii)] Keine Stadt liegt an mehr als zwei Strassen.
%\item[(iii)] Keine zwei Strassen überkreuzen sich.
%\end{enumerate}
%Zeige, dass die Anzahl Möglichkeiten, die der Präsident hat um die Strassen zu bauen, genau dann minimal ist, wenn die $n$ Städte ein konvexes $n$-Eck bilden.




\end{enumerate}




\bigskip
\bigskip
\begin{center}
Puisse le sort vous être favorable.
\end{center}

\pagebreak


\begin{center}
{\huge Sélection IMO 2014} \\
\medskip quatrième examen - 18 mai 2014
\end{center}
\vspace{8mm}
Temps : 4 heures 30\\
Chaque exercice vaut 7 points.

\vspace{15mm}

\begin{enumerate}
\item[\textbf{10.}] Un carré $7\times 7$ est divisé en $49$ petits carrés $1\times 1$. Deux scarabées marchent le long des côtés des petits carrés, de manière à ce que chaque scarabée parcourt son propre chemin fermé et qu'il passe exactement une fois par chacun des $64$ sommets des petits carrés. \\
Quel est le nombre minimal de côtés des petits carrés qui ont été parcourus par les deux scarabées ?


\bigskip
\bigskip


\item[\textbf{11.}] Déterminer tous les entiers naturels $n$ avec la propriété suivante : \\
Pour tout premier $p<n$, $n-\lfloor \frac{n}{p}\rfloor p$ n'est pas divisible par le carré d'un nombre naturel plus grand que $1$.\\
\textit{Remarque : Pour $x \in \mathbb{R}$, $\lfloor x\rfloor$ est défini comme le plus grand entier plus petit ou égal à $x$.}
\bigskip
\bigskip

\item[\textbf{12.}] Soient $n$ un entier naturel et $a_1,a_2, \dots,a_n$ des entiers naturels. On prolonge périodiquement la suite de sorte que $a_{n+i} = a_i$ pour tout $i\geq 1$. Supposons que les deux conditions suivantes sont vérifiées :
\begin{enumerate}
\item[(i)] 	$a_1\leq a_2\leq\dots \leq a_n\leq a_1 +n$.
\item[(ii)] $a_{a_i} \leq n+i-1$ pour $i=1,2\dots,n$.
\end{enumerate}
Montrer que :
\[a_1+a_2+\dots +a_n \leq n^2\]

\end{enumerate}
\bigskip
\begin{center}
Bon chance !
\end{center}

\end{document}