\de{Quirin wurde von Banditen an die Wand eines speziellen Raums gefesselt. Die Wände dieses Raums bestehen aus Spiegeln und formen ein spitzwinkliges Dreieck. Um sich von seinen Fesseln zu befreien, schiesst Quirin mit seinem Laserblick einen Laserstrahl los. Der Laserstrahl trifft die beiden anderen Wände genau einmal und kommt dann zu Quirin zurück. Wäre Quirin nicht im Weg gestanden, hätte der Strahl danach noch einmal die gleiche Bahn verfolgt, wie direkt nach Quirins Schuss. Bestimme alle Positionen, an denen sich Quirin befinden könnte.

\textit{Bemerkung: Wir interpretieren Quirin als einzelnen Punkt auf dem Rand des Dreiecks und nehmen an, dass der Laserstrahl nicht die Ecken des Raums trifft.}}

\fr{Maurice se trouve collé à un des murs d'une salle qui a la forme d'un triangle aigu et dont les murs sont des miroirs. Maurice a un pistolet laser dont les rayons sont réfléchis sur les murs. Il tire un coup avec son pistolet. Le rayon est réfléchi exactement une fois par les deux autres murs et revient percuter Maurice. Maurice constate par ailleurs que s'il s'en était allé avant que le rayon ne l'atteigne, alors le rayon aurait été réfléchi par le mur contre lequel il se trouvait et aurait percuté le mur suivant au même endroit que précédemment. Déterminer toutes les positions où Maurice peut se trouver.

\textit{Note: on assimile Maurice a un point sur le bord de la salle et on suppose que le rayon ne percute jamais un des trois coins de la salle.}}

\en{Quirin finds himself tied to a wall of a peculiar room. The walls of that room are mirrors and form an acute triangle. To free himself, Quirin shoots a laser beam from his eyes. The beam hits each of the two other walls exactly once and returns to Quirin. If he had not been in the way, the laser beam would have moved along the same path as directly after his shot. Determine all possible positions of Quirin.

\textit{Remark: We interpret Quirin as a single point on the edge of the triangle and assume that the laser beam doesn't hit the vertices}}