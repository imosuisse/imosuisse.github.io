\de{Zeige, dass keine Funktion $f:\Z\to\Z$ existiert, sodass für alle ganzen Zahlen $m,n$ gilt:
\[
    f(m+f(n))=f(m)-n.
\]

\textbf{Lösung:} (Paul) Wir nehmen an, dass die Funktionalgleichung eine Lösung hat und führen dies zu einem Widerspruch. Setzen wir $m=0$ ein, dann erhalten wir die Gleichung
\[
f(f(n)) = f(0)-n.
\]
Daraus folgt sofort, dass $f$ eine surjektive Funktion ist ($f$ ist sogar bijektiv, das brauchen wir aber nicht für die Lösung). Also können wir ein $a\in\Z$ finden sodass $f(a)=0$. Nun setzen wir $m=n=a$ ein, dies ergibt die Gleichung $0=a$, woraus wir $f(0)=0$ schliessen. 

Die Gleichung oben wird dann zu $f(f(n))=-n$. Jetzt ersetzen wir in der ursprünglichen Gleichung $n$ durch $f(n)$ und $m$ durch $m+n$. Dies ergibt
\[
f(m) = f(m+n)-f(n)
\]
für alle $m,n\in\N$. Diese Funktionalgleichung hat die Lösungen $f(n)=f(1)n$ (siehe Beispiel 11). Dann gilt jedoch
\[
-1 = f(f(1)) = f(1)^2,
\]
der gewünschte Widerspruch.

\textbf{Alternative:} (Arnaud) On commence par poser $m=0$ et on obtient de même que $f$ est bijective. En posant à présent $n=0$, on obtient $f(m+f(0))=f(m)$ et donc, en utilisant l'injectivité pour simplifier par $f$, on obtient $f(0)=0$.

Observer que l'égalité $f(m+f(0))=f(m)$ implique que $f$ est périodique de période $f(0)$. Comme $f$ est bijective, elle ne peut pas être périodique, ce qui implique également $f(0)=0$.

Il également possible de montrer directement que $f(f(n))=-n$ (sans passer par $f(0)=0$). Si on remplace $n$ par $n+f(m)$, on obtient 
\[
-n=f(m)-(n+f(m))=f(m+f(n+f(m)))=f(m+f(n)-m)=f(f(n)).
\]

\textbf{Marking scheme:}
\begin{enumerate}
\item +3P: $f(f(n))=-n$ with maximum 2 \textbf{non-additive} partial points
\begin{itemize}
    \item 1P: $f$ is surjevctive (enough to get $f^{-1}(\{0\})\neq\emptyset$) or $f$ is injective or $f(0)\neq 0\Rightarrow f$ periodic.
    \item 2P: $f(0)=0$
\end{itemize}
\item +3P: $f(n)=nf(1)$ with maximum 2 \textbf{non-additive} partial points
\begin{itemize}
    \item 1P: a first inductive step to get $f(n)=nf(1)$ (eg. $f(2m)=2f(m)$)
    \item 2P: $f(m+n)=f(m)+f(n)$
\end{itemize}

\item +1P: conclude
\end{enumerate}


}