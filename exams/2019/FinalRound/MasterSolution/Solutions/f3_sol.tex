\de{Déterminer toutes les suites périodiques $x_1, x_2, x_3, \ldots$ de nombres réels strictement positifs telles que pour tout $n\geq 1$
\[
    x_{n+2}=\frac{1}{2}\left(\frac{1}{x_{n+1}}+x_{n}\right).
\]

\textbf{Réponse:} Les seules solutions sont les suites de période au plus 2. Plus précisément ce sont toutes les suites avec $x_1 = a$ et $x_2 = \frac{1}{a}$, où $a$ désigne n'importe quel nombre réel positif.

\textbf{Première solution:} (Arnaud, by David) Comme proposé en cours, on commence par malaxer l'expression qui définit la suite. Si l'on supprime la fraction, on obtient
\[
2x_{n+2}x_{n+1}=x_{n+1}x_n+1.
\]
Il paraît à présent naturel d'introduire la nouvelle suite $y_n:=x_{n+1}x_n$ pour $n\geq 1$. Si la suite $\{x_n\}$ est périodique, alors la suite $\{y_n\}$ l'est aussi. On a la relation $2y_{n+1}=y_n+1$. Au feeling, il semble qu'une telle suite ne puisse pas vraiment être périodique en général.\newline
Une obstruction classique à la périodicité est la \textbf{monotonie}. Que peut-on dire de la monotonie de la suite $\{y_n\}$ ? On remarque que
\[
y_{n+1}<y_n\Longleftrightarrow y_n>1.
\] 
De plus, si $y_n>1$, alors clairement $y_{n+1}=1/2(y_n+1)>1$. Autrement dit, la monotonie de la suite $\{y_n\}$ dépend uniquement du signe de $y_1-1$. En effet, on a
\begin{itemize}
    \item si $y_1>1$, alors $y_n>1,\, \forall n$ et $\{y_n\}$ est une suite strictement décroissante,
    \item si $y_1<1$, alors $y_n<1,\, \forall n$ et $\{y_n\}$ est une suite strictement croissante,
    \item $y_1=1$, alors $y_n= 1,\, \forall n$ est une suite constante.
\end{itemize}
Dans les deux premiers cas, la périodicité contredit la monotonie stricte. On conclut ainsi que $y_n=1$ pour tout $n$. En revenant à la suite $\{x_n\}$, on a alors $x_{n+1}x_n=1$ pour tout $n$. Autrement dit, $x_{n+1}=1/x_n$ ou encore 
\[
x_n=\left\{\begin{array}{cc}
    x_2 & \mbox{si $n$ est pair,} \\
    x_1 & \mbox{si $n$ est impair}
\end{array}
\right.
,\quad x_1x_2=1.
\]
On vérifie à présent que toutes ces suites satisfont les hypothèses de départ. Une telle suite est périodique avec période 2 et on a bien $2=1+1$.

\textbf{Deuxième solution:} (Arnaud) Plutôt qu'introduire la nouvelle suite $y_n=x_{n+1}x_n$, on introduit $z_n:=x_{n+1}x_n-1=y_n-1$ pour $n\geq 1$. Si la suite $\{x_n\}$ est périodique, alors la nouvelle suite $\{z_n\}$ est évidement périodique. La relation devient alors,
\[
z_{n+1}=\frac{1}{2}z_n.
\]
Autrement dit, la suite $\{z_n\}$ est une suite géométrique de raison $1/2$. Si $z_1\neq 0$, alors $\{z_n\}$ est une suite strictement monotone (et convergente) et donc ne peur pas être périodique. On doit donc avoir $z_1=0$ et ainsi $z_n=0$ pour tout $n$. On conclut de la même manière que précédemment.


\textbf{Dritte Lösung (Sketch):} (Paul) Es geht auch ganz ohne Substitution, aber mit vielen Fallunterscheidungen. Um Platz zu sparen, schreiben wir symmetrische Fälle nicht auf.

\begin{enumerate}
\item \textbf{Lemma:} $x_0<1<x_1\Rightarrow x_{2n}<1<x_{2n+1}$ für alle $n$. 
\item $x_0=1, x_1 > 1$: Dann ist $x_2<1$, dank (a) kann die Folge danach die Zahl $x_0=1$ nie wieder erreichen.
\item $x_0, x_1 > 1$: Wegen (a) und (b) muss dann $x_i>1$ für alle $i$ gelten. Dann kann man für alle $n\in\N$ zeigen, dass $x_{n+2} < x_n$ gilt, ein Widerspruch zur Periodizität.
\item $\tfrac{1}{x_1} < x_0 < 1 < x_1$: Dann kann man $x_{2n} <x_0 < 1 < x_1 < x_{2n+1}$ zeigen, wieder ein Widerspruch zur Periodizität.
\item $\tfrac{1}{x_1} = x_0 \leq 1 \leq x_1$: Dann erhält man alle Lösungen.
\end{enumerate}

\textbf{Marking scheme:}

\begin{enumerate}
    \item 1P: explicitely proving that all sequences of the form $x_1x_2=1$, $x_{2n+1}=x_1$ and $x_{2n}=x_2$ satisfy the conditions of the problem
    \item 6P: proving that any sequence satisfying the conditions of the problem indeed is of this form.
    \begin{itemize}
        \item First solution ($y_n=x_{n+1}x_n$):
    \begin{itemize}
        \item substituition $y_n=x_{n+1}x_n$ and rewriting the inductive relation: 2P.
        \item $\{x_n\}$ periodic $\Rightarrow$ $\{y_n\}$ periodic: 1P.
        \item prove that if $y_1\neq 1$, then the sequence is strictly monotone (or any other property that contradicts periodicity) : 2P. 
        
        Note: only one of the cases ($y_1>1$ or $y_1<1$) gives one point.
        \item conclusion: 1P.
    \end{itemize}
    \item Second solution ($z_n=x_{n+1}x_n-1$):
    \begin{itemize}
        \item substituition $z_n=x_{n+1}x_n-1$ and rewriting the inductive relation: 3P.
        \item $\{x_n\}$ periodic $\Rightarrow$ $\{z_n\}$ periodic: 1P.
        \item prove that if $z_1\neq 0$, then the sequence is strictly monotone (or any other property that contradicts periodicity) : 1P. 
        \item conclusion: 1P.
    \end{itemize}
    \item Third solution (blind cases distinction):
    \begin{itemize}
        \item part (a) of the solution, i.e. Lemma: 1P.
        \item part (b) of the solution, i.e. case $x_1=1,x_2\neq 1$ and $x_1\neq 1,x_2=1$: 1P.
        \item part (c) of the solution, i.e. case $x_1,x_2>1$ and $x_1,x_2<1$: 2P.
        \item part (d) of the solution, i.e. case $x_1<1<x_2$ and $x_2<1<x_1$ with $x_1x_2\neq 1$: 2P.
        \item Note: the point (e) in the solution is basically checking solutions and thus is marked according to the first part of the marking scheme (i.e. part (a) of the marking scheme)
    \end{itemize}
    \end{itemize}
\end{enumerate}
}