

\de{Sei $\mathbb{P}$ die Menge aller Primzahlen und $M$ eine Teilmenge von $\mathbb{P}$ mit mindestens drei Elementen, sodass folgende Eigenschaft gilt: Für jede natürliche Zahl $k$ und für jede Teilmenge $A=\{p_1, p_2, \ldots, p_k\}$ von $M$ mit $A\neq M$ sind alle Primfaktoren von $p_1\cdot p_2 \cdot \ldots \cdot p_k - 1$ in $M$. Zeige, dass $M = \mathbb{P}$ gilt.}
\fr{Soit $\mathbb{P}$ l'ensemble de tous les nombres premiers et $M$ un sous-ensemble de $\mathbb{P}$ ayant au moins trois éléments. On suppose que pour tout entier $k\geq 1$ et pour tout sous-ensemble $A=\{p_1, p_2, \ldots, p_k\}$ de $M$ tel que $A\neq M$, tous les facteurs premiers du nombre $p_1\cdot p_2 \cdot \ldots \cdot p_k - 1$ se trouvent dans $M$. Montrer que $M = \mathbb{P}$.}
%\textit{Remarque: un sous-ensemble $Y$ d'un ensemble $X$ est appelé \emph{non-trivial} si $Y$ est non-vide et $Y\neq X$.}}
\ita{}
\en{Let $ \mathbb{P}$ be the set of all primes and let $M$ be a subset of $\mathbb{P}$, having at least three elements, and such that the following property is satisfied: For any positive integer $k$ and for any subset $A=\{p_1, p_2, \ldots ,p_k\}$ of $M$ with $A\neq M$, all of the prime factors of the number $p_1\cdot p_2 \cdot \ldots \cdot p_k - 1$ are contained in $M$. Prove that $M= \mathbb{P}$.}