\de{Sei $A$ ein Punkt und sei $k$ ein Kreis durch $A$. Seien $B$ und $C$ zwei weitere Punkte auf $k$. Sei nun $X$ der Schnittpunkt der Winkelhalbierenden von $\angle ABC$ mit $k$. Sei $Y$ die Spiegelung von $A$ am Punkt $X$, und $D$ der Schnittpunkt der Geraden $YC$ mit $k$. Zeige, dass der Punkt $D$ nicht von der Wahl von $B$ und $C$ auf dem Kreis $k$ abhängt.}
\fr{Soit $A$ un point et $k$ un cercle passant par $A$. Soient $B$ et $C$ deux autres points sur $k$. Soit $X$ l'intersection de la bissectrice de $\angle ABC$ avec $k$ et soit $Y$ l'image de $A$ par la symétrie de centre $X$. Finalement, soit $D$ l'intersection de la droite $YC$ avec $k$. Montrer que le point $D$ ne dépend pas du choix des points $B$ et $C$ sur le cercle $k$.}
\ita{}
\en{Let $A$ be a point and $k$ a circle through $A$. Let $B$ and $C$ be two other points on $k$. Let now $X$ be the intersection of the angle bisector of $\angle ABC$ with $k$. Further, let $Y$ be the reflection of $A$ in point $X$, and $D$ be the intersection of the line $YC$ and $k$. Show that the point $D$ does not depend on the choice of the points $B$ and $C$ on the circle $k$.}