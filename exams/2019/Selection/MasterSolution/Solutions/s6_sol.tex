Let $(a,b)$ be a pair of positive integers. Henning and Paul are playing a game: Initially, there are two piles of $a$ and $b$ stones, respectively, on a table. The pair $(a,b)$ is called the \textit{initial configuration} of the game. The players proceed as follows:

\begin{itemize}
    \item The players alternate and Henning begins.
    \item In each turn, a player either removes a positive number of stones from one of the two piles or the same positive number of stones from both piles.
    \item The player who removes the last stone from the table wins the game.
\end{itemize} 

Let $A$ be the set of all positive integers $a$ for which there exists a positive integer $b<a$ such that Paul has a winning strategy for the initial configuration $(a,b)$. Order the elements of $A$ increasingly as $a_1<a_2<\ldots$.

\begin{enumerate}
    \item[(a)] Prove that the set $A$ is infinite.
    \item[(b)] Prove that the sequence $(m_k)_{k \geq 1}$ defined by $m_k := a_{k+1} - a_k$ for all $k\geq 1$ is not eventually periodic.
\end{enumerate}

\textit{Remark}: A sequence $(x_k)_{k \geq 1}$ is \textit{eventually periodic} if there exists an integer $k_0\geq 0$ such that the sequence $(x_{k+k_0})_{k\geq 1}$ is periodic.

\textbf{Erste Lösung:} (David, by Patrick)
\begin{enumerate}
    \item[(a)] Wir nennen ein allgemeines Paar von Startwerten $(a,b)$ \textit{gut}, falls Paul dafür eine Gewinnstrategie besitzt. Hier erlauben wir der Vollständigkeit halber auch $a=0$ oder $b=0$ und nennen $(0,0)$ auch gut. Analog nennen wir alle Paare, die nicht gut sind, \textit{schlecht} und bemerken, dass für diese Startwerte Henning offenbar eine Gewinnstrategie besitzt. Wir benutzen das folgende allgemeine Prinzip:
    \newline \newline
    Wenn $(a,b)$ gut ist, dann sind $(a+n,b), \, (a,b+n), \, (a+n,b+n)$ für jedes $n \in \N$ schlecht, da Henning in seinem ersten Zug einfach die Werte $(a,b)$ erreichen könnte und dann Paul ein neues Spiel beginnen muss mit Startwerten, die nun für Henning gut sind. Also gibt es für jedes $a$ höchstens ein $b$, sodass $(a,b)$ gut ist.
    \newline 
    Zudem gilt: Falls Henning für die Startwerte $(a,b)$ in seinem ersten Zug nur schlechte Werte erreichen kann, ist $(a,b)$ gut, weil Paul dann unabhängig von Hennings Zug nach obiger Bemerkung eine Gewinnstrategie hat.
    \newline
    \newline
    Nehme nun an, dass $A$ nur endlich viele Elemente hat. Dann gibt es auch nur endlich viele gute Startwerte (denn für jedes gute Feld gibt es nach obigem Prinzip und wegen $(a,b) \, \text{gut} \iff (b,a) \, \text{gut}$ genau ein Element aus $A$, nämlich der kleinere der beiden Startwerte). Insbesondere gibt es eine Zahl $C$ die strikt grösser ist als alle $a, b$ mit $(a,b)$ gut. Dann gilt allerdings für $(C,2C)$, dass die Paare $(C-n,2C) \, (C-n, 2C-n)$ für jedes $n \in \{1,2,\dots,C\}$ und $(C,2C-m)$ für jedes $m \in \{1,2,\dots, 2C \}$ schlecht sind, nach Konstruktion von $C$, also ist $(C,2C)$ gut. Widerspruch!
    
    \item[(b)] Wir wollen nun erst einmal zeigen, dass jede natürliche Zahl entweder in $A$ oder in $B = \{b_k  \mid \, (a_k,b_k) \text{ gut} \}$ vorkommt. Was wir in Teilaufgabe (a) herausgefunden haben impliziert, dass eine natürliche Zahl nicht in beiden Mengen vorkommen kann. Nehme nun an, es gäbe ein $n \in \N$, sodass $n$ weder in $A$ noch in $B$ auftritt und wähle $n$ minimal. Mit anderen Worten ist für alle Zahlen $m \in \N$ das Paar $(m,n)$ schlecht. Das bedeutet aber auch, dass für jedes $m$ eines der Paare $(m,n-1), (m,n-2), \dots, (m,0), (m-1,n-1), (m-2,n-2) \dots (m - \text{min}\{m,n\}, n-\text{min}\{m,n\}) $ gut ist. Betrachten wir beispielsweise $m=n+1, \, 2n+1, \, 3n+1, \, \dots (n+1)n+1$, so sind alle diese Tupel paarweise verschieden. Wir erhalten also mit unserer Annahme $n+1$ verschiedene gute Paare, die alle von der Form $(x,y)$ sind mit $y \in \{0,1, \dots ,n-1\}$. Dann müssten allerdings nach Schubfachprinzip zwei $y$ den gleichen Wert annehmen, was unserem Zwischenresultat aus (a) widerspricht. Somit wissen wir also $A \sqcup B = \N$
    \newline
    \newline
    Der nächste Schritt besteht darin zu zeigen, dass $a_k - b_k = k$ gilt für alle $k \in \N$ und dass jeweils $b_k = \text{min}(\N \backslash \{a_1, \dots a_{k-1}, b_1, \dots, b_{k-1} \} ) $ gilt. Das findet man leicht durch Ausprobieren und lässt sich mit Induktion beweisen.
    \begin{itemize}
        \item Verankerung: Offensichtlich sind $a_0=0$, $b_0=0$ und $a_1=2$, $b_1=1$ die ersten guten Paare und erfüllen die Aussage.
        \item Nehme nun an, die Zahlen $a_1, \dots, a_n$ und $b_1, \dots, b_n$ erfüllen die Aussage. Nun setzen wir $b_{n+1} = \text{min}(\N \backslash \{a_1, \dots a_n, b_1, \dots, b_n \} )$. Weil nach Induktionsannahme $b_k < b_{n+1}$ gilt für alle $k \leq n$, gilt für $b_{n+1} + n+1$ auch $ b_{n+1} + n + 1 > b_k + k = a_k$ für alle $k \leq n$. Insbesondere können wir einmal $a_{n+1} = b_{n+1} + n+1$ setzen, denn dieses ist noch nicht in der Menge $\{a_1, \dots , a_n, b_1, \dots , b_n \}$ enthalten. Es bleibt zu zeigen, dass das Paar $(a_{n+1},b_{n+1})$ tatsächlich gut ist. Dafür unterscheiden wir die folgenden Möglichkeiten für Hennings ersten Zug mit Startwerten $(a_{n+1},b_{n+1})$:
        \newline
        Wenn Henning nur den Stapel mit $b_{n+1}$ Münzen verkleinert, erreicht er nach Konstruktion eine Anzahl Münzen, die bereits in der Menge $\{ a_0, a_1, \dots , a_n, b_0, b_1, \dots, b_n \}$ liegt. In jedem Fall kann Paul nun durch Verkleinern des Stapels mit $a_{n+1}$ Münzen ein gutes Paar erreichen, denn es gilt $a_{n+1} > a_k \geq b_k$ für alle $k \in \N \cup \{0\}$.
        \newline
        Wenn Henning beide Stapel verkleinert, ist der Stapel mit ursprünglich $b_{n+1}$ Münzen auf eine Anzahl in $\{ a_0, a_1, \dots , a_n, b_0, b_1, \dots, b_n \}$ geschrumpft. Falls es ein $a_k$ ist, kann Paul den anderen Stapel natürlich auf $b_k$ Münzen reduzieren wegen $a_{n+1}-x > b_{n+1}-x = a_k \geq b_k$.
        Falls es ein $b_k$ ist, kann Paul den anderen Stapel auf $a_k$ Münzen reduzieren wegen $a_k = b_k + k < b_k +n+1 = (b_{n+1}-x) + n+1 = a_{n+1} - x$.
        \newline
        Verkleinert Henning nur den Stapel mit $a_{n+1}$ auf $x$ Münzen, gibt es wieder einige Möglichkeiten:
        \newline 
        Falls $x \geq b_{n+1}$ gilt, hat Henning die Differenz der beiden Stapel verkleinert. Wegen $b_{n+1} > b_k$ für alle $k \in \{0,1, \dots, n \}$, erreicht Paul durch Entfernen einer geeigneten Anzahl Münzen von beiden Stapeln das Paar mit der neuen Differenz.
        \newline
        Falls $x < b_{n+1}$ gilt, dann gilt entweder $x=a_k$ oder $x=b_k$ für ein $k \leq n$. Falls $x=a_k$ gilt, dann gilt $b_k \leq a_k < b_{n+1}$, also kann Paul durch Verkleinern des Stapels mit $b_{n+1}$ Münzen das gute Paar $(a_k,b_k)$ erreichen. Falls $x=b_k$ gilt, dann unterscheiden wir noch einmal zwei Fälle: Entweder es gilt $a_k = b_k + k < b_{n+1}$; in diesem Fall verkleinert Paul den Stapel mit $b_{n+1}$ Münzen, um das gute Paar $(a_k,b_k)$ zu erreichen, oder aber es gilt $a_k = b_k + k > b_{n+1}$ (beachte, dass $a_k = b_{n+1}$ nach Konstruktion von $b_{n+1}$ nicht auftreten kann). In diesem Fall gilt $b_{n+1} - x < a_k - x = a_k - b_k = k$. Also ist die Differenz von $x$ und $b_{n+1}$ kleiner als $k$ und Paul kann durch Entfernen einer geeigneten Anzahl Münzen das gute Paar mit dieser Differenz erreichen.
        \newline
        \newline
        Das sind alle möglichen Fälle. Somit haben wir gezeigt, dass auch $(a_{n+1},b_{n+1})$ gut ist.
    \end{itemize}
    
    Nun verwenden wir alle diese Erkenntnisse, um zu zeigen, dass die Folge $(m_k)_{k\geq 1}$ niemals periodisch werden kann:
    \newline
    Nehme an, die Folge $(m_k)_{k\geq K}$ sei periodisch mit Periode $c \geq 1$. Wir definieren uns $d = a_{K+c}-a_K$ und bemerken (induktiv), dass $a_{k+nc}=a_k + nd$ gilt für alle $k \geq K$ und $n \in \N$. Zudem gilt $d > c$, weil sonst aufgrund der Periodizität jede natürliche Zahl grösser als $K$ in der Menge $A$ enthalten wäre und somit die Menge $B$ endlich wäre, was offensichtlich absurd ist. Also gibt es ein $n \in \N$, das $n(d-c) \geq K$ erfüllt. Für dieses $n$ gilt jedoch:
    \[
    b_{nd} = a_{nd} - nd = a_{n(d-c)}
    \]
    Also ein Widerspruch zu $A$ und $B$ disjunkt! Somit sind wir fertig.
\end{enumerate} 

\textbf{Zweite Lösung:} (Linus)
Wie in der ersten Lösung zeigen wir, dass $A^{(0)}:=\{a\in\N_0\ |\ \exists 0\leq b\leq a:\ (a,b)\text{ gut}\}=\{a_0,a_1,…\}$ (wobei $a_0<a_1<…$) unendlich ist. Darüber hinaus bemerken wir, dass falls $(a,b)$ und $(a’,b’)$ gut sind und zusätzlich $a=a’$ oder $b=b’$ oder $a-b=a’-b’$, dann gilt $(a,b)=(a’,b’)$. Das führt dazu, dass für jedes $a_k\in A^{(0)}$ genau ein $b_k\in\N_0$ existiert sodass $(a_k,b_k)$ gut ist; per Definition von $A^{(0)}$ gilt zusätzlich $a_k\geq b_k$. Darüber hinaus folgt, dass wenn wir $B^{(0)}:=\{b_0,b_1,…\}$ setzen, wir $A^{(0)}\cap B^{(0)}=\{0\}$ haben, da falls $a_k=b_l$ mit $k,l>0$ die Paare $(a_l,a_k)$ und $(b_k,a_k)$ gut sind, aber auch $b_k<a_k=b_l<a_l$, Widerspruch. Zeigen wir nun per starker Induktion, dass für alle $k\geq 1$ folgendes gilt:
\begin{enumerate}
\item[(i)] $\{0,1,…,b_k\}\subset\{a_0,…,a_k\}\cup\{b_0,…,b_k\}$
\item[(ii)] $a_k-b_k=k$
\item[(iii)] $$
(a_k,b_k)=
\begin{cases}
(a_{k-1},b_{k-1})+(2,1) & \text{falls } b_{k-1}+1\notin\{a_0,…,a_{k-1}\}\\
(a_{k-1},b_{k-1})+(3,2) & \text{ansonsten.}
\end{cases}
$$
\end{enumerate}
Die Aussage gilt für $k=1$ und $k=2$, nehmen wir also an, sie gilt für alle $i\leq k$ wobei $k\geq 2$. Nehmen wir an, es gäbe ein gutes Paar $(a_k+1,b)$, dann muss nach (i) der Induktionshypothese $b>b_k$, da alle Werte bis und mit $b_k$ schon in einem guten Paar mit kleinerem anderen Stapel auftauchen. Dann haben wir aber $a_k+1-n\leq k$, was ein Widerspruch ist, da diese Differenz schon von einem kleineren guten Paar besetzt ist, nach (ii) der Induktionshypothese. Wir machen nun eine Fallunterscheidung.\\
Fall 1: $b_k+1\notin\{a_0,…,a_k\}$\\
Wir zeigen, dass $(a_k+2,b_k+1)$ gut ist, und somit $(a_{k+1},b_{k+1})=(a_k,b_k)+(2,1)$. Tatsächlich sind alle $(a_k+2,b_k+1-m)$ schlecht, da nach (i) $b_k+1-m$ schon in einem guten Paar mit kleinerem anderem Stapel auftaucht, $(a_k+2-m,b_k+1-m)$ ist schlecht da $m=1$ wegen dem obigen Argument nicht geht und da falls für $m\geq 2$ das Paar $(a_k+2-m,b_k+1-m)$ gut wäre, wir zwingend $(a_k+2-m,b_k+1-m)=(a_l,b_l)$ für ein $l\leq k$ hätten, was wegen $k+1=(a_k+2-m)-(b_k+1-m)=a_l-b_l=l$ nicht geht. Schliesslich ist auch $(a_k+2-m,b_k+1)$ schlecht, da erneut $m=1$ nicht geht, und falls $(a_k+2-m,b_k+1)$ gut ist und $m\geq 2$, dann kann $b_k+1$ nicht der kleinere Stapel sein da er sonst höchstens $b_k$ Steine enthalten könnte. Weil aber $b_k+1\leq b_k+k=a_k$ muss somit $b_k+1\in\{a_0,…,a_k\}$, was unserer Annahme widerspricht.\\
Fall 2: $b_k+1\in\{a_0,…,a_k\}$\\
Wir zeigen zuerst, dass $a_{k+1}\geq a_k+3$. Dazu nehmen wir an, es gäbe ein gutes Paar $(a_k+2,b)$. Da alle Werte bis und mit $b_k$ schon in einem guten Paar vorkommen wobei der andere Stapel strikt weniger als $a_k+2$ Steine hat, muss $b>b_k$ gelten. Da nach Annahme $b_k+1\in\{a_0,…,a_k\}$ ist $b=b_k+1$ auch nicht möglich. Somit gilt aber $a_k+2-b\leq a_k+2-(b_k+2)=k$, was erneut ein Widerspruch ist, da diese Differenz schon von einem kleineren guten Paar besetzt ist. Zeigen wir nun, dass $(a_k+3,b_k+2)$ gut ist und somit $(a_{k+1},b_{k+1})=(a_k,b_k)+(3,2)$. Tatsächlich ist $(a_k+3,b_k+2-m)$ schlecht da alle Werte bis und mit $b_k+1$ schon in einem kleineren guten Paar vorkommen, $(a_k+3-m,b_k+2-m)$ ist schlecht da wir $m=1$ resp. $m=2$ schon zu Beginn dieses Falls resp. zu Beginn der Induktion ausgeschlossen haben, und falls $(a_k+3-m,b_k+2-m)$ gut ist mit $m\geq 3$, so gilt zwingend $(a_k+3-m,b_k+2-m)=(a_l,b_l)$ für ein $l\leq k$, was wegen $k+1=(a_k+3-m)-(b_k+2-m)=a_l-b_l=l$ widersprüchlich ist. Schliesslich ist auch $(a_k+3-m,b_k+2)$ schlecht, da wir $m=1$ und $m=2$ schon ausgeschlossen haben, und falls $(a_k+3-m,b_k+2)$ mit $m\geq 3$ gut ist, so muss $b_k+2$ der grössere der beiden Stapel sein, da er sonst höchstens $b_k$ Steine enthalten könnte. Somit gilt, da $b_k+2\leq b_k+k\leq a_k$, dass $b_k+2\in\{a_0,…,a_k\}$, was unserer Annahme $b_k+1\in\{a_0,…,a_k\}$ widerspricht, da nach (iii) der Induktionshypothese die Differenz zweier Werte aus $\{a_0,…,a_k\}$ nicht gleich $1$ sein kann.
Somit haben wir (iii) für $k+1$ bewiesen, woraus (i) und (ii) direkt folgen. Zusätzlich bemerken wir, dass aus (i) $\N_0=A^{(0)}\cup B^{(0)}$ folgt.
Berechnet man die ersten paar Werte der Folgen $(a_k)$ und $(b_k)$, so kommt man zur Vermutung, dass die folgende, etwas anschaulichere Rekursionsformel gilt:
$$
(a_k,b_k)=
\begin{cases}
(a_{k-1},b_{k-1})+(2,1) & \text{falls } k-1\in\{a_0,…,a_{k-1}\}\\
(a_{k-1},b_{k-1})+(3,2) & \text{ansonsten.}
\end{cases}
$$
Um diese zu zeigen brauchen wir folgendes Zwischenresultat, welches wir per Induktion für alle $k\geq 1$ zeigen: anscheinend gilt $b_{b_k}+1=a_k$ und $b_{b_k+1}=a_k+1$. Die beiden Gleichungen gelten für $k=1$, nehmen wir daher an, sie gelten für ein $k\geq 1$. Wir machen die gleiche Fallunterscheidung wie vorher.\\
Fall 1: $b_k+1\notin\{a_0,…,a_k\}$\\
Wir haben also $b_{k+1}=b_k+1$ und somit $b_{b_{k+1}}+1=b_{b_k+1}+1=a_k+1+1=a_{k+1}$ nach unserer Induktionshypothese. Da aber somit insbesondere $b_{b_k+1}+1\in\{a_0,…,a_{b_k+1}\}$, gilt dank unserer schon bewiesenen Rekursionsformel, dass $b_{b_{k+1}+1}=b_{b_k+2}=b_{b_k+1}+2=a_k+3=a_{k+1}+1$.\\
Fall 2: $b_k+1\in\{a_0,…,a_k\}$\\
In diesem Fall gilt $b_{k+1}=b_k+2$ und $a_{k+1}=a_k+3$. Nach Induktionshypothese haben wir $b_{b_k+1}+1=a_k+2<a_{k+1}$ und somit $b_{b_k+1}+1\notin\{a_0,…,a_{b_k+1}\}$. Daraus folgt $b_{b_{k+1}}+1=b_{b_k+2}+1=b_{b_k+1}+1+1=a_k+1+1+1=a_{k+1}$. Da somit insbesondere auch $b_{b_k+2}+1\in\{a_0,…,a_{b_k+2}\}$ folgt, erhalten wir auch $b_{b_{k+1}+1}=b_{b_k+3}=b_{b_k+2}+2=a_{k+1}+1$. Somit sind die beiden Gleichungen bewiesen.\\
Nun können wir folgende Äquivalenz zeigen: $b_k+1\in\{a_0,…,a_k\}\iff k\notin\{a_0,…,a_k\}$. Denn falls $b_k+1=a_l$ dann folgt $l\geq 1$ und somit $b_k+1=a_l=b_{b_l}+1$ und somit $b_k=b_{b_l}\implies k=b_l\implies k\notin\{a_0,…,a_k\}$. Und falls $k\notin\{a_0,…,a_k\}$, so gibt es $l\geq 1$ mit $k=b_l$ und daher $b_k+1=b_{b_l}+1=a_l\in\{a_0,…,a_k\}$. Diese Äquivalenz führt direkt zur anschaulicheren Rekursionsformel. Definiert man nun $s_k=|\{l\geq 0\ |\ a_l<k\}|$ so folgt daraus wiederum $a_k=3(k-s_k)+2s_k=3k-s_k$. Kommen wir nun endlich zum eigentlichen Beweis.\\
Nehmen wir an, es gibt ein $K$ sodass $(m_k)_{k\geq K}$ periodisch ist mit Periode $c\geq 1$. Setzt man $A:=a_{K+c}-a_K$, so folgt aus der Periodizität, dass $a_{k+c}-a_k=A$ für alle $k\geq K$. Wir bemerken ebenfalls, dass falls $k\geq a_K$, dann enthält $\{l\geq 0\ |\ k\leq a_l<k+A\}$ genau $c$ Elemente, woraus $s_{k+A}=s_k+c$ für alle $k\geq a_K$ folgt (am besten überlegt man sich das mit einer Zeichnung). Sei nun $k\geq\max\{K,a_K\}$, dann erhalten wir
$$
a_k+A^2=a_{k+cA}=3(k+cA)-s_{k+cA}=3(k+cA)-s_k-c^2
$$
und somit
$$
A^2-3cA+c^2=3k-a_k-s_k=0.
$$
Daraus erhalten wir jedoch schliesslich, dass $\frac{A}{c}\in\{\varphi^{-2},\varphi^2\}$, wobei $\varphi$ der goldene Schnitt ist. Das ist ein Widerspruch, da $\varphi^{-2}$ und $\varphi^2$ irrational sind.

\newpage
\textbf{Marking scheme:}

\begin{enumerate}
    \item Erste Lösung:
\begin{itemize}
    \item +1P: $A$ is infinite
    \item +1P: Geometric: In every row and coloumn is exactly. Algebraic: $A \bigsqcup B = \N$
    \item +1P: $a_k - b_k = k$
    \item +1P: $(a_{k+1},b_{k+1}) = (a_k +2, b_k +1)$ or $(a_k+3,b_k+2)$
    \item +2P: Idea to mirror the minimal period length
    \item +1P: finish
\end{itemize}
\end{enumerate}