Let $S$ be a nonempty set of positive integers. Prove that at least one of the following two assertions holds:
\begin{enumerate}[(i)]
    \item There exist distinct finite nonempty subsets $F$ and $G$ of $S$ such that 
    \[\sum_{x\in F}\frac{1}{x}=\sum_{x\in G}\frac{1}{x}.\]
    \item There exists a positive rational number $r<1$ such that for all finite nonempty subsets $F$ of $S$ it holds that
    \[\sum_{x\in F} \frac{1}{x}\neq r.\]
\end{enumerate}

\textbf{Première solution:} (Arnaud) On doit montrer qu'au moins une des deux conditions est vérifiées, on va donc supposer que les deux sont fausses et essayer d'aboutir à une contradiction. 

Remarquez tout d'abord que la condition (i) se lit comme une condition de "non-injectivité" (si l'on considérait la fonction définie sur l'ensemble des sous-ensembles finis de $S$ vers les nombres rationnels positifs). La condition (ii) s'apparente à une condition de non-surjectivité vers $(0,1)\cap \Q$. Autrement dit, avec notre hypothèse absurde, on supposer que pour tout rationnel $r\in (0,1)$, il existe un unique sous-ensemble fini $F_r\subset S$ tel que $r$ est la somme des inverses des éléments de $F_r$.

En particulier, s'il on a $r,s\in (0,1)\cap \Q$ tels qu'il existe $x\in S$ avec $r=s+1/x$, alors soit $x\notin F_s$ et forcément $F_r=F_s\cup\{x\}$ (par unicité), ou simplement $x$ se trouve déjà dans $F_s$, et alors $x\notin F_r$, sinon $F_s$ s'obtiendrait par $F_s=F_r\setminus \{x\}$. Autrement dit,
\[
x\notin F_s \Leftrightarrow x\in F_r=F_{s+1/x}.
\]

Cette relation fait pressentir un certain phénomène d'alternance. Plus précisément, prenons $r\in (0,1)\cap\Q$ et $x\in S$. Soit $n$ le plus grand entier tel que $r-n/x>0$ (autrement dit $n=\lfloor rx\rfloor$). Certainement, $r-n/x<1/x$, par définition de $n$. Donc $x\notin F_{r-n/x}$. L'observation ci-dessus nous donne la relation d'alternance suivante:
\[
x\notin F_{r-n/x}\Rightarrow x\in F_{r-(n-1)/x} \Rightarrow x\notin F_{r-(n-2)/x}\Rightarrow \ldots
\]
Au final, en remontant jusqu'à $r$, on observe que $x\in F_r$ si et seulement si $n$ est impair.

La dernière étape de l'argument est la plus subtile. Jusqu'ici, il s'agissait essentiellement d'observer les caractéristiques principales du problèmes. Considérons $F_{2/3}$. L'argument précédent nous donne, si $x\in F_{2/3}$, alors $\lfloor 2x/3\rfloor$ est un nombre impair. En particulier, et là est la subtile observation de la preuve, le nombre $2x/3$ n'est pas entier ! (sinon il serait pair). En particulier, on peut trouver $q_x\in (0,1)\cap\Q$ tel que $\lfloor 2x/3\rfloor$=$\lfloor x(2/3-q_x)\rfloor$. Comme $F_{2/3}$ est fini, on peut prendre $\varepsilon =\min_{x\in F_{2/3}}q_x$ et on obtient que $\lfloor 2x/3\rfloor$=$\lfloor x(2/3-\varepsilon)\rfloor$ pour tout $x\in F_{2/3}$. 

La contradiction suit de l'observation suivante:
\[
x\in F_{2/3} \Rightarrow \lfloor 2x/3\rfloor \text{ odd } \Rightarrow \lfloor x(2/3-\varepsilon)\rfloor \text{ odd } \Rightarrow x\in F_{2/3-\varepsilon},
\]
donc $F_{2/3}\subset F_{2/3-\varepsilon}$. Contradiction.

\textbf{Zweite Lösung:} (David)
Wie in der ersten Lösung nehmen wir an, dass eine Menge $S$ existiert, die weder (i) noch (ii) erfüllt und halten fest, dass somit für jedes $r \in  (0,1) \cap \mathbb{Q}$ eine eindeutige endliche Teilmenge $F_r \subset S$ existiert, die 
\[\sum_{x\in F_r}\frac{1}{x}=r \]
erfüllt. Wir bemerken zudem die folgenden beiden grundlegenden Punkte:
\begin{itemize}
    \item Für alle $r \in (0,1) \cap \mathbb{Q}$ gilt $1 \notin F_r$. Deshalb dürfen wir oBdA $1 \notin S$ annehmen.
    \item Die Menge S muss unendlich sein. Sonst gäbe es nur endlich viele Möglichkeiten, eine endliche Teilmenge auszuwählen und somit auch nur endlich viele Zahlen, die wir darstellen können.
\end{itemize}
Wir schreiben also $S=\{s_1 < s_2 < s_3 < \dots \}$. Betrachte für ein beliebiges $s_k \in S$ die Mengen $A_k = (0,\frac{1}{s_k}) \cap \Q$ und $B_k = (\frac{1}{s_k}, \frac{2}{s_k}) \cap \mathbb{Q}$. Dann wissen wir, dass für jedes $a \in A_k$ gilt $s_k \notin F_a$. Analog wie in der ersten Lösung schliessen wir daraus mithilfe der Eindeutigkeit von $F_b$, dass für jedes $b \in B_k$ gilt $s_k \in F_b$. \\ \\ 
Nun gilt: Falls $B_k \cap B_{k+1}$ nichtleer ist, gibt es ein $r \in (0,1) \cap \Q$ mit $s_k \in F_r, s_{k+1} \in F_r$. \\ Wegen $\frac{1}{s_k} + \frac{1}{s_{k+1}} > \frac{2}{s_{k+1}} > r$ führt das zu einem Widerspruch. Also ist $B_k \cap B_{k+1}$ leer, oder mit anderen Worten gilt für jedes $k$ die Ungleichung $\frac{2}{s_{k+1}} \leq \frac{1}{s_k} \; \Longleftrightarrow \; s_{k+1} \geq 2s_k$ und somit induktiv $s_k \geq 2^{k-1}s_1$. Nun können wir folgende Abschätzung machen: 
\[
\sum_{x\in S}\frac{1}{x} = \sum_{k=1}^{\infty}\frac{1}{s_k} \leq \sum_{k=1}^{\infty} \frac{1}{2^{k-1}s_1} \leq \sum_{k=1}^{\infty}\frac{1}{2^k} = 1.
\]
Wenn wir also in irgendeinem Schritt strikte Ungleichheit hätten, gäbe es wegen \[
\sum_{x\in F}\frac{1}{x} < \sum_{x\in S}\frac{1}{x} < 1
\]
einige rationale Zahlen sehr nahe bei $1$, für die es keine Menge $F$ gibt, Widerspruch! \newline
Wenn jedoch in jedem Schritt Gleichheit gilt, dann wissen wir auch, dass $a_1 = 2$ und $a_{k+1} = 2a_k$ gilt für jedes $k \in \N$. Also gilt $S=\{2,4,8,16, \dots \}$. Nun können wir aber offensichtlich keine endliche Teilmenge von $S$ finden, die $1/3$ erzeugt, da die Binärdarstellung von $1/3$ unendlich viele Einsen enthält. Auch ein Widerspruch! Dies beendet den Beweis.

\textbf{Marking scheme:}

\begin{enumerate}

    \item Première solution (Arnaud):
    \begin{itemize}
    \item reformulate the problem by looking for contradiction assuming "bijectivity": 0P.
    \item $x\notin F_s \Leftrightarrow x\in F_{s+1/x}$: 1P.
    \item relation d'alternance: 1P.
    \item $x\in F_r$ si et seulement si $\lfloor rx\rfloor$ est impair: 2P.
    
    \item concluding the proof (analytical argument): 3P.
    \end{itemize}

    \item Zweite Lösung (David):
    \begin{itemize}
    \item Für alle $b\in B_k$ gilt $s_k\in F_b$ : 2P.
    \item Die Idee, ein $r\in B_k\cap B_{k+1}\cap \Q$ zu betrachten: 1P.
    \item $s_{k+1}\geq 2s_k$: 1P.
    \item $s_{k+1} \geq 2^k s_1$: 1P.
    \item $s_{k+1} = 2s_k$: 1P.
    \item Widerspruch herbeiführen: 1P.
    \end{itemize}
\end{enumerate}