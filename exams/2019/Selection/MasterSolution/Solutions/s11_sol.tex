Soit $n$ un nombre entier strictement positif. Déterminer s'il existe un nombre réel $\varepsilon >0$ (dépendant de $n$)
tel que, pour tous nombres réels strictement positifs $x_1,x_2,\ldots,x_n$, on ait
\[
 \sqrt[n]{x_1x_2\cdots x_n} \le (1-\epsilon) \cdot \frac{x_1+x_2+\dots + x_n}{n}+\epsilon \cdot \frac{n}{\frac{1}{x_1}+\frac{1}{x_2}+\dots+\frac{1}{x_n}}.
\]

\textbf{Answer:} There is such an $\epsilon$ for every $n\in\N$.

\textbf{First solution:} (David, Paul)

If there is such an $\epsilon$ for every $n$, we expect it to depend on $n$. Keeping the chain of inequalities AM $\geq$ GM $\geq$ HM in mind, we see that the right hand side becomes larger if $\epsilon$ becomes smaller. So an educated guess could be $\epsilon = \frac{1}{n}$, and this will actually work. \par
Since 
\[
1-\epsilon = \frac{n-1}{n},
\]
we can write the right hand side as a sum of $n-1$ arithmetic means and $1$ harmonic mean. We apply AM-GM with $n$ variables to this (where we multiply both the numerator and the denominator of the harmonic mean by $x_1\cdots x_n$ beforehand):
\[
(n-1)\frac{x_1+x_2+\dots + x_n}{n^2} + \frac{x_1\cdots x_n}{\sum_{j=1}^n \prod_{i\neq j} x_i} \geq n\cdot \sqrt[n]{\left(\frac{x_1+x_2+\dots + x_n}{n^2}\right)^{n-1}\frac{x_1\cdots x_n}{\sum_{j=1}^n \prod_{i\neq j} x_i}.   }
\]
We want to show that the term on the right hand side is larger than $\sqrt[n]{x_1\cdots x_n}$. A computation shows that this is equivalent to
\[
\frac{x_1+\cdots + x_n}{n} \geq \sqrt[n-1]{\frac{\sum_{j=1}^n \prod_{i\neq j} x_i}{n}}.
\]
This is just McLaurin's inequality.\par
Alternatively (actually this is just a proof of McLaurin), one can bring the inequality into the form
\[
n^2(x_1+\cdots +x_n)^{n-1} \geq n^n\left(\sum_{j=1}^n \prod_{i\neq j} x_i\right).
\]
Note that this is completely symmetric, and the right hand side is a multiple (in the Muirhead notation from class) of $[1,\ldots , 1, 0]$. If we multiply out the $(n-1)$th power on the left side, we get a lot of symmetric sums. Since we only raise it to the $(n-1)$th power, and not the $n$th power, no term of the form $x_1\cdots x_n$ will appear. Said differently, the symmetric sums will be of the type $[\alpha_1,\ldots,\alpha_{n-1},0]$ where the $\alpha_i$ are integers such that $\sum \alpha_i=n-1$ and $\alpha_1\geq \alpha_2\geq\ldots\geq \alpha_{n-1}\geq 0$. All these sums majorize $[1,\ldots,1,0]$. Since both sides have the same number of summands, namely $n^{n+1}$, we are done using Muirhead's inequality.

\textbf{Remark:} One observation that can be made at the beginning is that if the inequality holds for $\varepsilon_1$, then it holds for any $0<\varepsilon_2\leq \varepsilon_1$ as 
\[
(1-\varepsilon_1)\text{AM}+\varepsilon_1 \text{HM}\leq (1-\varepsilon_2)\text{AM}+\varepsilon_2 \text{HM},
\]
by AM-HM.

\newpage 
\textbf{Marking Scheme:}

Partial solution ($\leq 4P$):
\begin{itemize}
\item +1P: claiming $\epsilon(n)= 1/n$ works
\item small cases (non-additive):
\begin{itemize}
    \item +0P: case $n=1$
    \item +1P: one case $n\geq 2$
    \item +2P: two cases $n\geq 2$
\end{itemize}
\item +2P: applying AM-GM by considering AM $(n-1)$ times and AM once \textbf{or} applying MacLaurin on the term $\sum_{j=1}^n \prod_{i\neq j} x_i$ in the inital RHS
\end{itemize}

Full solution ($\geq 5P$): 7P