\de{Sei $n$ eine positive ganze Zahl. Bestimme, ob es eine reelle Zahl $\epsilon>0$ (abhängig von $n$) gibt, sodass für alle positiven reellen Zahlen $x_1,x_2,\ldots,x_n$ gilt:
\[
 \sqrt[n]{x_1x_2\cdots x_n} \le (1-\epsilon) \cdot \frac{x_1+x_2+\dots + x_n}{n}+\epsilon \cdot \frac{n}{\frac{1}{x_1}+\frac{1}{x_2}+\dots+\frac{1}{x_n}}.
\]}
\fr{Soit $n$ un nombre entier strictement positif. Déterminer s'il existe un nombre réel $\varepsilon >0$ (dépendant de $n$)
tel que, pour tous nombres réels strictement positifs $x_1,x_2,\ldots,x_n$, on ait
\[
 \sqrt[n]{x_1x_2\cdots x_n} \le (1-\epsilon) \cdot \frac{x_1+x_2+\dots + x_n}{n}+\epsilon \cdot \frac{n}{\frac{1}{x_1}+\frac{1}{x_2}+\dots+\frac{1}{x_n}}.
\]}
\ita{} 
\en{Let $n$ be a positive integer. Determine whether there exists a real number $\epsilon>0$ (depending on $n$) such that, for all positive real numbers $x_1,x_2,\ldots,x_n$, we have
\[
 \sqrt[n]{x_1x_2\cdots x_n} \le (1-\epsilon) \cdot \frac{x_1+x_2+\dots + x_n}{n}+\epsilon \cdot \frac{n}{\frac{1}{x_1}+\frac{1}{x_2}+\dots+\frac{1}{x_n}}.
\]
}