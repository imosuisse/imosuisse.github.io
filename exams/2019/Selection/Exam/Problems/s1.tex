\de{Sei $ABC$ ein Dreieck und seien $D$, $E$ und $F$ die Höhenfusspunkte der Höhen von $A$, $B$ respektive $C$. Sei $H$ der Höhenschnittpunkt von Dreieck $ABC$. Die Strecken $EF$ und $AD$ schneiden sich in $G$. Sei $K$ der Punkt auf dem Umkreis von Dreieck $ABC$, der diametral gegenüber von $A$ liegt. Die Gerade $AK$ schneide $BC$ in $M$. Zeige, dass die Geraden $GM$ und $HK$ parallel sind.}
\fr{Soit $ABC$ un triangle et $D$, $E$, $F$ les pieds des hauteurs issues de $A$, $B$ et $C$, respectivement. Soit $H$ l'orthocentre du triangle $ABC$. Les segments $EF$ et $AD$ se coupent en $G$. Soit $K$ le point diamétralement opposé à $A$ sur le cercle circonscrit au triangle $ABC$. La droite $AK$ coupe $BC$ en $M$. Montrer que les droites $GM$ et $HK$ sont parallèles.}
\ita{}
\en{Let $ABC$ be a triangle and $D$, $E$, $F$ be the feet of the altitudes from $A$, $B$, $C$, respectively. Let $H$ be the orthocenter of triangle $ABC$. The segments $EF$ and $AD$ intersect at $G$. Let $K$ be the point on the circumcircle of $ABC$ such that $AK$ is a diameter of this circle. The line $AK$ intersects $BC$ at $M$. Prove that the lines $GM$ and $HK$ are parallel.
}