\de{Sei $(a,b)$ ein Paar natürlicher Zahlen. Henning und Paul spielen ein Spiel: Zu Beginn befinden sich zwei Stapel mit $a$ beziehungsweise $b$ Münzen auf dem Tisch. Wir nennen das Paar $(a,b)$ die \textit{Startwerte} des Spiels. Henning und Paul spielen nach folgenden Regeln:

\begin{itemize}
    \item Die Spieler ziehen abwechslungsweise, wobei Henning beginnt.
    \item In jedem Zug entfernt ein Spieler entweder eine positive Anzahl Münzen von einem der beiden Stapel oder die gleiche positive Anzahl Münzen von beiden Stapeln.
    \item Der Spieler, der die letzte Münze vom Tisch entfernt, gewinnt das Spiel.
\end{itemize} 

Sei $A$ die Menge aller natürlicher Zahlen $a$, für die es eine natürliche Zahl $b<a$ gibt, sodass Paul eine Gewinnstrategie für die Startwerte $(a,b)$ hat. Ordne die Elemente von $A$ aufsteigend als $a_1<a_2<\ldots$.

\begin{enumerate}
    \item[(a)] Zeige, dass $A$ unendlich viele Elemente enthält.
    \item[(b)] Zeige, dass die Folge $(m_k)_{k \geq 1}$, gegeben durch $m_k := a_{k+1} - a_k$ für alle $k\geq 1$, niemals periodisch wird.
\end{enumerate}

\textit{Bemerkung}: Wir sagen, dass eine Folge $(x_k)_{k \geq 1}$  \textit{periodisch wird}, falls es eine ganze Zahl $k_0\geq 0$ gibt, sodass die Folge $(x_{k+k_0})_{k\geq 1}$ periodisch ist.}

\fr{Soit $(a,b)$ une paire de nombres entiers strictement positifs. Henning et Paul jouent au jeu suivant: initialement, sur une table, se trouvent deux piles composées de respectivement $a$ et $b$ pièces. On nomme la paire $(a, b)$ la \emph{valeur de départ} du jeu. Henning et Paul jouent d'après les règles suivantes:

\begin{itemize}
    \item Les joueurs jouent à tour de rôle et Henning commence.
    \item À chaque coup, un joueur enlève soit un nombre strictement positif de pièces d'une des deux piles, soit le même nombre strictement positif de pièces des deux piles.
    \item Le joueur qui enlève la dernière pièce de la table gagne.
\end{itemize} 

Soit $A$ l'ensemble des nombres entiers strictement positifs $a$ pour lesquels il existe un nombre entier strictement positif $b < a$ tel que Paul a une stratégie gagnante pour la valeur de départ $(a, b)$. On ordonne les éléments de $A$ par ordre croissant: $a_1<a_2<\ldots$.

\begin{enumerate}
    \item[(a)] Montrer que $A$ contient une infinité d'éléments.
    \item[(b)] Montrer que la suite $(m_k)_{k \geq 1}$ définie par $m_k := a_{k+1} - a_k$, pour tous $k\geq 1$, ne devient jamais périodique
\end{enumerate}

\textit{Remarque}: On dit qu'une suite $(x_k)_{k \geq 1}$  \emph{devient périodique} si il existe un nombre entier $k_0\geq 0$ tel que la suite $(x_{k+k_0})_{k\geq 1}$ est périodique.} 
\ita{}
\en{Let $(a,b)$ be a pair of positive integers. Henning and Paul are playing a game: Initially, there are two piles of $a$ and $b$ stones, respectively, on a table. The pair $(a,b)$ is called the \textit{initial configuration} of the game. The players proceed as follows:

\begin{itemize}
    \item The players alternate and Henning begins.
    \item In each turn, a player either removes a positive number of stones from one of the two piles or the same positive number of stones from both piles.
    \item The player who removes the last stone from the table wins the game.
\end{itemize} 

%Let $A:=\{a_1 < a_2 < \ldots < a_k < \ldots \}$ be the set of all positive integers $a_k$ for which there exists a positive integer $b_k < a_k$ such that Paul has a winning strategy for the initial configuration $(a_k,b_k)$. 

Let $A$ be the set of all positive integers $a$ for which there exists a positive integer $b<a$ such that Paul has a winning strategy for the initial configuration $(a,b)$. Order the elements of $A$ increasingly as $a_1<a_2<\ldots$.

\begin{enumerate}
    \item[(a)] Prove that the set $A$ is infinite.
    \item[(b)] Prove that the sequence $(m_k)_{k \geq 1}$ defined by $m_k := a_{k+1} - a_k$ for all $k\geq 1$ is not eventually periodic.
\end{enumerate}

\textit{Remark}: A sequence $(x_k)_{k \geq 1}$ is \textit{eventually periodic} if there exists an integer $k_0\geq 0$ such that the sequence $(x_{k+k_0})_{k\geq 1}$ is periodic.
}
