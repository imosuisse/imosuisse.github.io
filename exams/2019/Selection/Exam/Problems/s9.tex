\de{
Sei $ABC$ ein spitzwinkliges Dreieck mit $AB<AC$. Seien $E$ und $F$ die Höhenfusspunkte der Höhen von $B$ respektive $C$, sowie $M$ der Mittelpunkt der Strecke $BC$. Die Tangente an den Umkreis von $ABC$ im Punkt $A$ schneide die Gerade $BC$ in $P$. Die Parallele zu $BC$ durch $A$ schneide die Gerade $EF$ in $Q$. Zeige, dass die Geraden $PQ$ und $AM$ senkrecht aufeinander stehen.
}
\fr{Soit $ABC$ un triangle aigu tel que $AB<AC$. Soient $E$ et $F$ les pieds des hauteurs issues de $B$ et $C$, respectivement, et soit $M$ le milieu de $BC$. La droite tangente au cercle circonscrit à $ABC$ en $A$ coupe la droite $BC$ en $P$. La droite passant par $A$ parallèle à la droite $BC$ coupe la droite $EF$ en $Q$. Montrer que la droite $PQ$ est perpendiculaire à la droite $AM$.}
\ita{}
\en{Let $ABC$ be an acute triangle with $AB<AC$. Let $E$ and $F$ be the feet of the altitudes from $B$ and $C$, respectively, and $M$ the midpoint of the segment $BC$. The tangent line to the circumcircle of $ABC$ at $A$ intersects the line $BC$ at $P$. The line parallel to $BC$ through $A$ intersects the line $EF$ at $Q$. Prove that the line $PQ$ is perpendicular to the line $AM$.}