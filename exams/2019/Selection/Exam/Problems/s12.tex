 \de{Definiere die Folge $(a_n)_{n\geq 0}$ ganzer Zahlen mit $a_n:=2^n+2^{\floor{n/2}}$. Zeige, dass es in dieser Folge unendlich viele Terme gibt, welche als Summe von zwei oder mehr verschiedenen Termen der Folge ausgedrückt werden können, und auch, dass es unendlich viele Terme in der Folge gibt, die nicht so ausgedrückt werden können.}
\fr{On définit la suite $(a_n)_{n\geq 0}$ de nombres entiers par $a_n:=2^n+2^{\floor{n/2}}$. Montrer qu'il existe une infinité de termes de la suite pouvant être exprimés comme la somme d'au moins deux termes distincts de la suite, de même qu'il existe une infinité de termes de la suite ne pouvant pas s'écrire de la sorte.}
\ita{}
\en{Define the sequence $(a_n)_{n\geq 0}$ of integers by $a_n:=2^n+2^{\floor{n/2}}$. Prove that there are infinitely many terms of the sequence which can be expressed as a sum of two or more distinct terms of the sequence, as well as infinitely many terms which cannot be expressed in such a way.}