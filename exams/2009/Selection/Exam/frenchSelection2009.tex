\documentclass[12pt,a4paper]{article}

\usepackage{amsfonts}
\usepackage[centertags]{amsmath}
\usepackage[french]{babel}
\usepackage[latin2]{inputenc}
\usepackage{amsthm}
\usepackage{amssymb}

\leftmargin=0pt \topmargin=0pt \headheight=0in \headsep=0in \oddsidemargin=0pt \textwidth=6.5in
\textheight=8.5in

% \catcode`\� = \active \catcode`\� = \active \catcode`\� = \active \catcode`\� = \active \catcode`\� = \active
% \catcode`\� = \active 
% 
% \def�{"A}
% \def�{"a}
% \def�{"O}
% \def�{"o}
% \def�{"U}
% \def�{"u}








% Schriftabk�rzungen

\newcommand{\eps}{\varepsilon}
\renewcommand{\phi}{\varphi}
\newcommand{\Sl}{\ell}    % sch�nes l
\newcommand{\ve}{\varepsilon}  %Epsilon

\newcommand{\BA}{{\mathbb{A}}}
\newcommand{\BB}{{\mathbb{B}}}
\newcommand{\BC}{{\mathbb{C}}}
\newcommand{\BD}{{\mathbb{D}}}
\newcommand{\BE}{{\mathbb{E}}}
\newcommand{\BF}{{\mathbb{F}}}
\newcommand{\BG}{{\mathbb{G}}}
\newcommand{\BH}{{\mathbb{H}}}
\newcommand{\BI}{{\mathbb{I}}}
\newcommand{\BJ}{{\mathbb{J}}}
\newcommand{\BK}{{\mathbb{K}}}
\newcommand{\BL}{{\mathbb{L}}}
\newcommand{\BM}{{\mathbb{M}}}
\newcommand{\BN}{{\mathbb{N}}}
\newcommand{\BO}{{\mathbb{O}}}
\newcommand{\BP}{{\mathbb{P}}}
\newcommand{\BQ}{{\mathbb{Q}}}
\newcommand{\BR}{{\mathbb{R}}}
\newcommand{\BS}{{\mathbb{S}}}
\newcommand{\BT}{{\mathbb{T}}}
\newcommand{\BU}{{\mathbb{U}}}
\newcommand{\BV}{{\mathbb{V}}}
\newcommand{\BW}{{\mathbb{W}}}
\newcommand{\BX}{{\mathbb{X}}}
\newcommand{\BY}{{\mathbb{Y}}}
\newcommand{\BZ}{{\mathbb{Z}}}

\newcommand{\Fa}{{\mathfrak{a}}}
\newcommand{\Fb}{{\mathfrak{b}}}
\newcommand{\Fc}{{\mathfrak{c}}}
\newcommand{\Fd}{{\mathfrak{d}}}
\newcommand{\Fe}{{\mathfrak{e}}}
\newcommand{\Ff}{{\mathfrak{f}}}
\newcommand{\Fg}{{\mathfrak{g}}}
\newcommand{\Fh}{{\mathfrak{h}}}
\newcommand{\Fi}{{\mathfrak{i}}}
\newcommand{\Fj}{{\mathfrak{j}}}
\newcommand{\Fk}{{\mathfrak{k}}}
\newcommand{\Fl}{{\mathfrak{l}}}
\newcommand{\Fm}{{\mathfrak{m}}}
\newcommand{\Fn}{{\mathfrak{n}}}
\newcommand{\Fo}{{\mathfrak{o}}}
\newcommand{\Fp}{{\mathfrak{p}}}
\newcommand{\Fq}{{\mathfrak{q}}}
\newcommand{\Fr}{{\mathfrak{r}}}
\newcommand{\Fs}{{\mathfrak{s}}}
\newcommand{\Ft}{{\mathfrak{t}}}
\newcommand{\Fu}{{\mathfrak{u}}}
\newcommand{\Fv}{{\mathfrak{v}}}
\newcommand{\Fw}{{\mathfrak{w}}}
\newcommand{\Fx}{{\mathfrak{x}}}
\newcommand{\Fy}{{\mathfrak{y}}}
\newcommand{\Fz}{{\mathfrak{z}}}

\newcommand{\FA}{{\mathfrak{A}}}
\newcommand{\FB}{{\mathfrak{B}}}
\newcommand{\FC}{{\mathfrak{C}}}
\newcommand{\FD}{{\mathfrak{D}}}
\newcommand{\FE}{{\mathfrak{E}}}
\newcommand{\FF}{{\mathfrak{F}}}
\newcommand{\FG}{{\mathfrak{G}}}
\newcommand{\FH}{{\mathfrak{H}}}
\newcommand{\FI}{{\mathfrak{I}}}
\newcommand{\FJ}{{\mathfrak{J}}}
\newcommand{\FK}{{\mathfrak{K}}}
\newcommand{\FL}{{\mathfrak{L}}}
\newcommand{\FM}{{\mathfrak{M}}}
\newcommand{\FN}{{\mathfrak{N}}}
\newcommand{\FO}{{\mathfrak{O}}}
\newcommand{\FP}{{\mathfrak{P}}}
\newcommand{\FQ}{{\mathfrak{Q}}}
\newcommand{\FR}{{\mathfrak{R}}}
\newcommand{\FS}{{\mathfrak{S}}}
\newcommand{\FT}{{\mathfrak{T}}}
\newcommand{\FU}{{\mathfrak{U}}}
\newcommand{\FV}{{\mathfrak{V}}}
\newcommand{\FW}{{\mathfrak{W}}}
\newcommand{\FX}{{\mathfrak{X}}}
\newcommand{\FY}{{\mathfrak{Y}}}
\newcommand{\FZ}{{\mathfrak{Z}}}

\newcommand{\CA}{{\cal A}}
\newcommand{\CB}{{\cal B}}
\newcommand{\CC}{{\cal C}}
\newcommand{\CD}{{\cal D}}
\newcommand{\CE}{{\cal E}}
\newcommand{\CF}{{\cal F}}
\newcommand{\CG}{{\cal G}}
\newcommand{\CH}{{\cal H}}
\newcommand{\CI}{{\cal I}}
\newcommand{\CJ}{{\cal J}}
\newcommand{\CK}{{\cal K}}
\newcommand{\CL}{{\cal L}}
\newcommand{\CM}{{\cal M}}
\newcommand{\CN}{{\cal N}}
\newcommand{\CO}{{\cal O}}
\newcommand{\CP}{{\cal P}}
\newcommand{\CQ}{{\cal Q}}
\newcommand{\CR}{{\cal R}}
\newcommand{\CS}{{\cal S}}
\newcommand{\CT}{{\cal T}}
\newcommand{\CU}{{\cal U}}
\newcommand{\CV}{{\cal V}}
\newcommand{\CW}{{\cal W}}
\newcommand{\CX}{{\cal X}}
\newcommand{\CY}{{\cal Y}}
\newcommand{\CZ}{{\cal Z}}

% Theorem Stil

\theoremstyle{plain}
\newtheorem{lem}{Lemma}
\newtheorem{Satz}[lem]{Satz}

\theoremstyle{definition}
\newtheorem{defn}{Definition}[section]

\theoremstyle{remark}
\newtheorem{bem}{Bemerkung}    %[section]



\newcommand{\card}{\mathop{\rm card}\nolimits}
\newcommand{\Sets}{((Sets))}
\newcommand{\id}{{\rm id}}
\newcommand{\supp}{\mathop{\rm Supp}\nolimits}

\newcommand{\ord}{\mathop{\rm ord}\nolimits}
\renewcommand{\mod}{\mathop{\rm mod}\nolimits}
\newcommand{\sign}{\mathop{\rm sign}\nolimits}
\newcommand{\ggT}{\mathop{\rm ggT}\nolimits}
\newcommand{\kgV}{\mathop{\rm kgV}\nolimits}
\renewcommand{\div}{\, | \,}
\newcommand{\notdiv}{\mathopen{\mathchoice
             {\not{|}\,}
             {\not{|}\,}
             {\!\not{\:|}}
             {\not{|}}
             }}

\newcommand{\im}{\mathop{{\rm Im}}\nolimits}
\newcommand{\coim}{\mathop{{\rm coim}}\nolimits}
\newcommand{\coker}{\mathop{\rm Coker}\nolimits}
\renewcommand{\ker}{\mathop{\rm Ker}\nolimits}

\newcommand{\pRang}{\mathop{p{\rm -Rang}}\nolimits}
\newcommand{\End}{\mathop{\rm End}\nolimits}
\newcommand{\Hom}{\mathop{\rm Hom}\nolimits}
\newcommand{\Isom}{\mathop{\rm Isom}\nolimits}
\newcommand{\Tor}{\mathop{\rm Tor}\nolimits}
\newcommand{\Aut}{\mathop{\rm Aut}\nolimits}

\newcommand{\adj}{\mathop{\rm adj}\nolimits}

\newcommand{\Norm}{\mathop{\rm Norm}\nolimits}
\newcommand{\Gal}{\mathop{\rm Gal}\nolimits}
\newcommand{\Frob}{{\rm Frob}}

\newcommand{\disc}{\mathop{\rm disc}\nolimits}

\renewcommand{\Re}{\mathop{\rm Re}\nolimits}
\renewcommand{\Im}{\mathop{\rm Im}\nolimits}

\newcommand{\Log}{\mathop{\rm Log}\nolimits}
\newcommand{\Res}{\mathop{\rm Res}\nolimits}
\newcommand{\Bild}{\mathop{\rm Bild}\nolimits}

\renewcommand{\binom}[2]{\left({#1}\atop{#2}\right)}
\newcommand{\eck}[1]{\langle #1 \rangle}
\newcommand{\gaussk}[1]{\lfloor #1 \rfloor}
\newcommand{\frack}[1]{\{ #1 \}}
\newcommand{\wi}{\hspace{1pt} < \hspace{-6pt} ) \hspace{2pt}}
\newcommand{\dreieck}{\bigtriangleup}

\parindent0mm

\begin{document}

\pagestyle{empty}

\begin{center}
{\huge S\'election OIM 2009} \\
\medskip Premier examen - le 16 mai 2009
\end{center}
\vspace{8mm}
Dur\'ee : 4.5 heures\\
Chaque exercice vaut 7 points.

\vspace{15mm}

\begin{enumerate}

\item[\textbf{1.}] Soit $GERMANYISHOT$ un dod\'ecagone r\'egulier et soit $P$ le point d'intersection de $GN$ et $MI$. Montrer que
\begin{enumerate}
\item[(a)] le cercle circonscrit du triangle $GIP$ a la m\^eme taille que le cercle circonscrit de $GERMANYISHOT$.
\item[(b)] le segment $PA$ a la m\^eme longueur qu'un c\^ot\'e de $GERMANYISHOT$.
\end{enumerate}


\bigskip
\bigskip

\item[\textbf{2.}] Trouver toutes les paires $(m,n)$ de nombres naturels impairs tels que
\[m \div n^2+8, \qquad n \div m^2+8.\]


\bigskip
\bigskip

\item[\textbf{3.}] Soit $n$ un nombre naturel. Trouver le nombre de permutations $(a_1,\ldots,a_n)$ 
de l'ensemble $\{1,2,\ldots, n\}$ qui ont la propri\'et\'e suivante:
\[2(a_1+\ldots +a_k) \quad \mbox{est divisible par $k$} \quad \forall k\in \{1,2,\ldots ,n\}.\]


\end{enumerate}

\pagebreak


\begin{center}
{\huge S\'election OIM 2009} \\
\medskip Deuxi\`eme examen - le 17 mai 2009
\end{center}
\vspace{8mm}
Dur\'ee : 4.5 heures\\
Chaque exercice vaut 7 points.

\vspace{15mm}

\begin{enumerate}

\item[\textbf{4.}] Pour quels nombres naturels $n$ existe-t-il un polyn\^ome $P(x)$ \`a coefficients entiers tel que
pour tout diviseur positif $d$ de $n$ on a $P(d)=(n/d)^2$?\\

\bigskip
\bigskip

\item[\textbf{5.}] Soit $ABCD$ un quadrilat\`ere convexe et soient $P$ et $Q$ des points \`a l'int\'erieur de $ABCD$ tels que $PQDA$ et $QPBC$ sont des quadrilat\`eres inscrits. Supposons qu'il existe un point $E$ sur le segment $PQ$ tel que $\angle PAE = \angle QDE$ et $\angle PBE = \angle QCE$. Montrer que $ABCD$ est un quadrilat\`ere inscrit.

\bigskip
\bigskip

\item[\textbf{6.}] Soit $P$ l'ensemble des premiers 2009 nombres premiers et soit $X$ l'ensemble de tous les nombres naturels qui  poss\`edent uniquement des diviseurs premiers qui sont dans $P$. Trouver tous les nombres naturels $k$ pour lesquels il existe une fonction $f: X\rightarrow X$ qui satisfait l'\'equation suivante pour tout $m,n\in X$:
\[f(mf(n))=f(m)n^k.\]
 
\end{enumerate}


\pagebreak


\begin{center}
{\huge S\'election OIM 2009} \\
\medskip Troisi\`eme examen - le 23 mai 2009
\end{center}
\vspace{8mm}
Dur\'ee : 4.5 heures\\
Chaque exercice vaut 7 points.

\vspace{15mm}

\begin{enumerate}
\item[\textbf{7.}] Consid\'erer un ensemble $A$ de 2009 points dans le plan parmi lesquels il n'y a pas trois qui sont colin\'eaires. Un triangle dont tous les sommets appartiennent \`a $A$ s'appelle \emph{triangle int\'erieur}. Montrer que chaque point de $A$ est contenu dans un nombre pair de triangles int\'erieurs.\\


\bigskip
\bigskip


\item[\textbf{8.}] Trouver toutes les fonctions $f:\BR \rightarrow \BR$ telles que pour tous les nombres r\'eels $x,y$ l'\'equation suivante est satisfaite:
\[f(f(x)-f(y))=(x-y)^2f(x+y).\]



\bigskip
\bigskip

\item[\textbf{9.}] Soient $BE$ et $CF$ les hauteurs dans un triangle $ABC$ \`a angles aigus.
Deux cercles passant par $A$ et $F$ touchent la droite $BC$ en $P$ et $Q$ de telle mani\`ere que $B$ se trouve entre $C$ et $Q$. Montrer que le point d'intersection de $PE$ et $QF$ est sur le cercle circonscrit de $AEF$.


\end{enumerate}


\pagebreak


\begin{center}
{\huge S\'election OIM 2009} \\
\medskip Quatri\`eme examen - le 24 mai 2009
\end{center}
\vspace{8mm}
Dur\'ee : 4.5 heures\\
Chaque exercice vaut 7 points.

\vspace{15mm}

\begin{enumerate}
\item[\textbf{10.}] Soit $n$ un nombre naturel et soient $a,b$ deux nombres entiers distincts avec la propri\'et\'e suivante: Pour tout nombre naturel $m$, $a^m-b^m$ est divisible par $n^m$. Montrer que les deux nombres $a$ et $b$ sont divisibles par $n$.\\


\bigskip
\bigskip


\item[\textbf{11.}] On consid\`ere $n$ points colin\'eaires $P_1, \ldots, P_n$ et tous les cercles avec diam\`etre $P_iP_j$ pour $1 \leq i<j \leq n$. Chacun de ces cercles est color\'e avec une de $k$ couleurs. Un tel ensemble de cercles color\'es s'appelle un $(n,k)$-\emph{fouillis}. Un \emph{huit unicolore} consiste en deux cercles de m\^eme couleur qui sont tangents de l'ext\'erieur. Montrer que tout $(n,k)$-fouillis contient un huit unicolore si et seulement si on a $n>2^k$.\\



\bigskip
\bigskip

\item[\textbf{12.}]Soient $x,y,z$ des nombres r\'eels satisfaisant l'\'equation $x+y+z=xy+yz+zx$. 
Montrer que 
\[\frac{x}{x^2+1}+\frac{y}{y^2+1}+\frac{z}{z^2+1} \geq -\frac{1}{2}.\]


\end{enumerate}


\end{document}
