\documentclass[12pt,a4paper]{article}

\usepackage{amsfonts}
\usepackage[centertags]{amsmath}
\usepackage{german}
\usepackage{amsthm}
\usepackage{amssymb}

\leftmargin=0pt \topmargin=0pt \headheight=0in \headsep=0in \oddsidemargin=0pt \textwidth=6.5in
\textheight=8.5in

\catcode`\� = \active \catcode`\� = \active \catcode`\� = \active \catcode`\� = \active \catcode`\� = \active
\catcode`\� = \active

\def�{"A}
\def�{"a}
\def�{"O}
\def�{"o}
\def�{"U}
\def�{"u}







% Schriftabk�rzungen

\newcommand{\eps}{\varepsilon}
\renewcommand{\phi}{\varphi}
\newcommand{\Sl}{\ell}    % sch�nes l
\newcommand{\ve}{\varepsilon}  %Epsilon

\newcommand{\BA}{{\mathbb{A}}}
\newcommand{\BB}{{\mathbb{B}}}
\newcommand{\BC}{{\mathbb{C}}}
\newcommand{\BD}{{\mathbb{D}}}
\newcommand{\BE}{{\mathbb{E}}}
\newcommand{\BF}{{\mathbb{F}}}
\newcommand{\BG}{{\mathbb{G}}}
\newcommand{\BH}{{\mathbb{H}}}
\newcommand{\BI}{{\mathbb{I}}}
\newcommand{\BJ}{{\mathbb{J}}}
\newcommand{\BK}{{\mathbb{K}}}
\newcommand{\BL}{{\mathbb{L}}}
\newcommand{\BM}{{\mathbb{M}}}
\newcommand{\BN}{{\mathbb{N}}}
\newcommand{\BO}{{\mathbb{O}}}
\newcommand{\BP}{{\mathbb{P}}}
\newcommand{\BQ}{{\mathbb{Q}}}
\newcommand{\BR}{{\mathbb{R}}}
\newcommand{\BS}{{\mathbb{S}}}
\newcommand{\BT}{{\mathbb{T}}}
\newcommand{\BU}{{\mathbb{U}}}
\newcommand{\BV}{{\mathbb{V}}}
\newcommand{\BW}{{\mathbb{W}}}
\newcommand{\BX}{{\mathbb{X}}}
\newcommand{\BY}{{\mathbb{Y}}}
\newcommand{\BZ}{{\mathbb{Z}}}

\newcommand{\Fa}{{\mathfrak{a}}}
\newcommand{\Fb}{{\mathfrak{b}}}
\newcommand{\Fc}{{\mathfrak{c}}}
\newcommand{\Fd}{{\mathfrak{d}}}
\newcommand{\Fe}{{\mathfrak{e}}}
\newcommand{\Ff}{{\mathfrak{f}}}
\newcommand{\Fg}{{\mathfrak{g}}}
\newcommand{\Fh}{{\mathfrak{h}}}
\newcommand{\Fi}{{\mathfrak{i}}}
\newcommand{\Fj}{{\mathfrak{j}}}
\newcommand{\Fk}{{\mathfrak{k}}}
\newcommand{\Fl}{{\mathfrak{l}}}
\newcommand{\Fm}{{\mathfrak{m}}}
\newcommand{\Fn}{{\mathfrak{n}}}
\newcommand{\Fo}{{\mathfrak{o}}}
\newcommand{\Fp}{{\mathfrak{p}}}
\newcommand{\Fq}{{\mathfrak{q}}}
\newcommand{\Fr}{{\mathfrak{r}}}
\newcommand{\Fs}{{\mathfrak{s}}}
\newcommand{\Ft}{{\mathfrak{t}}}
\newcommand{\Fu}{{\mathfrak{u}}}
\newcommand{\Fv}{{\mathfrak{v}}}
\newcommand{\Fw}{{\mathfrak{w}}}
\newcommand{\Fx}{{\mathfrak{x}}}
\newcommand{\Fy}{{\mathfrak{y}}}
\newcommand{\Fz}{{\mathfrak{z}}}

\newcommand{\FA}{{\mathfrak{A}}}
\newcommand{\FB}{{\mathfrak{B}}}
\newcommand{\FC}{{\mathfrak{C}}}
\newcommand{\FD}{{\mathfrak{D}}}
\newcommand{\FE}{{\mathfrak{E}}}
\newcommand{\FF}{{\mathfrak{F}}}
\newcommand{\FG}{{\mathfrak{G}}}
\newcommand{\FH}{{\mathfrak{H}}}
\newcommand{\FI}{{\mathfrak{I}}}
\newcommand{\FJ}{{\mathfrak{J}}}
\newcommand{\FK}{{\mathfrak{K}}}
\newcommand{\FL}{{\mathfrak{L}}}
\newcommand{\FM}{{\mathfrak{M}}}
\newcommand{\FN}{{\mathfrak{N}}}
\newcommand{\FO}{{\mathfrak{O}}}
\newcommand{\FP}{{\mathfrak{P}}}
\newcommand{\FQ}{{\mathfrak{Q}}}
\newcommand{\FR}{{\mathfrak{R}}}
\newcommand{\FS}{{\mathfrak{S}}}
\newcommand{\FT}{{\mathfrak{T}}}
\newcommand{\FU}{{\mathfrak{U}}}
\newcommand{\FV}{{\mathfrak{V}}}
\newcommand{\FW}{{\mathfrak{W}}}
\newcommand{\FX}{{\mathfrak{X}}}
\newcommand{\FY}{{\mathfrak{Y}}}
\newcommand{\FZ}{{\mathfrak{Z}}}

\newcommand{\CA}{{\cal A}}
\newcommand{\CB}{{\cal B}}
\newcommand{\CC}{{\cal C}}
\newcommand{\CD}{{\cal D}}
\newcommand{\CE}{{\cal E}}
\newcommand{\CF}{{\cal F}}
\newcommand{\CG}{{\cal G}}
\newcommand{\CH}{{\cal H}}
\newcommand{\CI}{{\cal I}}
\newcommand{\CJ}{{\cal J}}
\newcommand{\CK}{{\cal K}}
\newcommand{\CL}{{\cal L}}
\newcommand{\CM}{{\cal M}}
\newcommand{\CN}{{\cal N}}
\newcommand{\CO}{{\cal O}}
\newcommand{\CP}{{\cal P}}
\newcommand{\CQ}{{\cal Q}}
\newcommand{\CR}{{\cal R}}
\newcommand{\CS}{{\cal S}}
\newcommand{\CT}{{\cal T}}
\newcommand{\CU}{{\cal U}}
\newcommand{\CV}{{\cal V}}
\newcommand{\CW}{{\cal W}}
\newcommand{\CX}{{\cal X}}
\newcommand{\CY}{{\cal Y}}
\newcommand{\CZ}{{\cal Z}}

% Theorem Stil

\theoremstyle{plain}
\newtheorem{lem}{Lemma}
\newtheorem{Satz}[lem]{Satz}

\theoremstyle{definition}
\newtheorem{defn}{Definition}[section]

\theoremstyle{remark}
\newtheorem{bem}{Bemerkung}    %[section]



\newcommand{\card}{\mathop{\rm card}\nolimits}
\newcommand{\Sets}{((Sets))}
\newcommand{\id}{{\rm id}}
\newcommand{\supp}{\mathop{\rm Supp}\nolimits}

\newcommand{\ord}{\mathop{\rm ord}\nolimits}
\renewcommand{\mod}{\mathop{\rm mod}\nolimits}
\newcommand{\sign}{\mathop{\rm sign}\nolimits}
\newcommand{\ggT}{\mathop{\rm ggT}\nolimits}
\newcommand{\kgV}{\mathop{\rm kgV}\nolimits}
\renewcommand{\div}{\, | \,}
\newcommand{\notdiv}{\mathopen{\mathchoice
             {\not{|}\,}
             {\not{|}\,}
             {\!\not{\:|}}
             {\not{|}}
             }}

\newcommand{\im}{\mathop{{\rm Im}}\nolimits}
\newcommand{\coim}{\mathop{{\rm coim}}\nolimits}
\newcommand{\coker}{\mathop{\rm Coker}\nolimits}
\renewcommand{\ker}{\mathop{\rm Ker}\nolimits}

\newcommand{\pRang}{\mathop{p{\rm -Rang}}\nolimits}
\newcommand{\End}{\mathop{\rm End}\nolimits}
\newcommand{\Hom}{\mathop{\rm Hom}\nolimits}
\newcommand{\Isom}{\mathop{\rm Isom}\nolimits}
\newcommand{\Tor}{\mathop{\rm Tor}\nolimits}
\newcommand{\Aut}{\mathop{\rm Aut}\nolimits}

\newcommand{\adj}{\mathop{\rm adj}\nolimits}

\newcommand{\Norm}{\mathop{\rm Norm}\nolimits}
\newcommand{\Gal}{\mathop{\rm Gal}\nolimits}
\newcommand{\Frob}{{\rm Frob}}

\newcommand{\disc}{\mathop{\rm disc}\nolimits}

\renewcommand{\Re}{\mathop{\rm Re}\nolimits}
\renewcommand{\Im}{\mathop{\rm Im}\nolimits}

\newcommand{\Log}{\mathop{\rm Log}\nolimits}
\newcommand{\Res}{\mathop{\rm Res}\nolimits}
\newcommand{\Bild}{\mathop{\rm Bild}\nolimits}

\renewcommand{\binom}[2]{\left({#1}\atop{#2}\right)}
\newcommand{\eck}[1]{\langle #1 \rangle}
\newcommand{\gaussk}[1]{\lfloor #1 \rfloor}
\newcommand{\frack}[1]{\{ #1 \}}
\newcommand{\wi}{\hspace{1pt} < \hspace{-6pt} ) \hspace{2pt}}
\newcommand{\dreieck}{\bigtriangleup}

\parindent0mm

\begin{document}

\pagestyle{empty}

\begin{center}
{\huge IMO Selektion 2009} \\
\medskip erste Pr�fung - 16. Mai 2009
\end{center}
\vspace{8mm}
Zeit: 4.5 Stunden\\
Jede Aufgabe ist 7 Punkte wert.

\vspace{15mm}

\begin{enumerate}

\item[\textbf{1.}] Sei $GERMANYISHOT$ ein regul�res Zw�lfeck. Die Geraden $GN$ und $MI$ schneiden
sich im Punkt $P$. Beweise, dass
\begin{enumerate}
\item[(a)] der Umkreis von Dreieck $GIP$ gleich gross ist wie der Umkreis von\\
$GERMANYISHOT$.
\item[(b)] die Strecke $PA$ gleich lang ist wie eine Seite von $GERMANYISHOT$.\\
\end{enumerate}

\bigskip
\bigskip

\item[\textbf{2.}] Finde alle Paare $(m,n)$ ungerader nat�rlicher Zahlen mit $m,n \leq 2009$ und 
\[m \div n^2+8, \qquad n \div m^2+8.\]

\bigskip
\bigskip

\item[\textbf{3.}] Sei $n$ eine nat�rliche Zahl. Bestimme die Anzahl Permutationen $(a_1,\ldots,a_n)$ der Menge $\{1,2,\ldots, n\}$ mit der folgenden Eigenschaft:
\[2(a_1+\ldots +a_k) \quad \mbox{ist durch $k$ teilbar} \quad \forall k\in \{1,2,\ldots ,n\}.\]


\end{enumerate}

\pagebreak


\begin{center}
{\huge IMO Selektion 2009} \\
\medskip zweite Pr�fung - 17. Mai 2009
\end{center}
\vspace{8mm}
Zeit: 4.5 Stunden\\
Jede Aufgabe ist 7 Punkte wert.

\vspace{15mm}

\begin{enumerate}

\item[\textbf{4.}] F�r welche nat�rlichen Zahlen $n$ existiert ein Polynom $P(x)$ mit ganzen Koeffizienten, sodass $P(d)=(n/d)^2$ gilt f�r alle positiven Teiler $d$ von $n$?\\

\bigskip
\bigskip

\item[\textbf{5.}] Sei $ABCD$ ein konvexes Viereck und seien $P$ und $Q$ Punkte innerhalb des Vierecks $ABCD$, so dass $PQDA$ und $QPBC$ Sehnenvierecke sind. Nehme an, dass ein Punkt $E$ auf der Strecke $PQ$ existiert, so dass $\angle PAE=\angle QDE$ und $\angle PBE=\angle QCE$. Zeige, dass $ABCD$ ein Sehnenviereck ist.\\

\bigskip
\bigskip

\item[\textbf{6.}] Sei $P$ die Menge der ersten 2009 Primzahlen und sei $X$ die Menge aller nat�rlichen Zahlen, welche nur Primfaktoren aus $P$ besitzen. Bestimme alle nat�rlichen Zahlen $k$, f�r die eine Funktion $f: X \rightarrow X$ existiert, welche f�r alle $m,n \in X$ die folgende Gleichung erf�llt:
\[f(mf(n))=f(m)n^k.\]
 
\end{enumerate}


\pagebreak


\begin{center}
{\huge IMO Selektion 2009} \\
\medskip dritte Pr�fung - 23. Mai 2009
\end{center}
\vspace{8mm}
Zeit: 4.5 Stunden\\
Jede Aufgabe ist 7 Punkte wert.

\vspace{15mm}

\begin{enumerate}
\item[\textbf{7.}] Betrachte eine Menge $A$ von 2009 Punkte in der Ebene, von denen keine drei auf einer Geraden liegen. Ein Dreieck, dessen Eckpunkte alle in $A$ liegen, heisst \emph{internes Dreieck}. Beweise, dass jeder Punkt aus $A$ im Innern einer geraden Anzahl interner Dreiecke enthalten ist.\\

\bigskip
\bigskip

\item[\textbf{8.}] Finde alle Funktionen $f:\BR \rightarrow \BR$, sodass f�r alle reellen $x,y$ die folgende Gleichung erf�lllt ist:
\[f(f(x)-f(y))=(x-y)^2f(x+y).\]

\bigskip
\bigskip

\item[\textbf{9.}] In einem spitzwinkligen Dreieck $ABC$ seien $BE$ und $CF$ H�hen. Zwei Kreise durch $A$ und $F$ ber�hren die Gerade $BC$ bei $P$ und $Q$, so dass $B$ zwischen $C$ und $Q$ liegt. Beweise, dass sich die Geraden $PE$ und $QF$ auf dem Umkreis von $AEF$ schneiden.

\end{enumerate}


\pagebreak


\begin{center}
{\huge IMO Selektion 2009} \\
\medskip vierte Pr�fung - 24. Mai 2009
\end{center}
\vspace{8mm}
Zeit: 4.5 Stunden\\
Jede Aufgabe ist 7 Punkte wert.

\vspace{15mm}

\begin{enumerate}
\item[\textbf{10.}] Sei $n$ eine nat�rliche Zahl und seien $a,b$ zwei verschiedene ganze Zahlen mit folgender Eigenschaft: F�r jede nat�rliche Zahl $m$ ist $a^m-b^m$ durch $n^m$ teilbar. Zeige, dass $a,b$ beide durch $n$ teilbar sind.\\

\bigskip
\bigskip


\item[\textbf{11.}] Betrachte $n$ kollineare Punkte $P_1, \ldots, P_n$ und alle Kreise mit Durchmesser $P_iP_j$ f�r $1 \leq i < j \leq n$. Jeder dieser Kreise wird mit einer von $k$ Farben gef�rbt. Eine solche Menge von gef�rbten Kreisen heisst ein $(n,k)$-\emph{Gewusel}. Eine \emph{einfarbige Acht} sind zwei Kreise derselben Farbe, die sich �usserlich tangential ber�hren. Zeige, dass genau dann jedes $(n,k)$-Gewusel eine einfarbige Acht enth�lt, wenn $n>2^k$ gilt.\\

\bigskip
\bigskip

\item[\textbf{12.}] Seien $x,y,z$ reelle Zahlen, welche die Gleichung $x+y+z=xy+yz+zx$ erf�llen. Beweise die Ungleichung
\[\frac{x}{x^2+1}+\frac{y}{y^2+1}+\frac{z}{z^2+1} \geq -\frac{1}{2}.\]


\end{enumerate}


\end{document}
