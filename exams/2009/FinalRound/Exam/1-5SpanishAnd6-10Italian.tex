\documentclass[12pt,a4paper]{article}

\usepackage{amsfonts}
\usepackage[centertags]{amsmath}
\usepackage{german}
\usepackage{amsthm}
\usepackage{amssymb}

\leftmargin=0pt \topmargin=0pt \headheight=0in \headsep=0in \oddsidemargin=0pt \textwidth=6.5in
\textheight=8.5in

\catcode`\� = \active \catcode`\� = \active \catcode`\� = \active \catcode`\� = \active \catcode`\� = \active
\catcode`\� = \active

\catcode`\� = \active
\catcode`\� = \active
\catcode`\� = \active
\catcode`\� = \active
\catcode`\� = \active

\def�{"A}
\def�{"a}
\def�{"O}
\def�{"o}
\def�{"U}
\def�{"u}

\def�{\'{a}}
\def�{\'{e}}
\def�{\'{i}}
\def�{\'{o}}
\def�{\'{u}}





% Schriftabk�rzungen

\newcommand{\eps}{\varepsilon}
\renewcommand{\phi}{\varphi}
\newcommand{\Sl}{\ell}    % sch�nes l
\newcommand{\ve}{\varepsilon}  %Epsilon

\newcommand{\BA}{{\mathbb{A}}}
\newcommand{\BB}{{\mathbb{B}}}
\newcommand{\BC}{{\mathbb{C}}}
\newcommand{\BD}{{\mathbb{D}}}
\newcommand{\BE}{{\mathbb{E}}}
\newcommand{\BF}{{\mathbb{F}}}
\newcommand{\BG}{{\mathbb{G}}}
\newcommand{\BH}{{\mathbb{H}}}
\newcommand{\BI}{{\mathbb{I}}}
\newcommand{\BJ}{{\mathbb{J}}}
\newcommand{\BK}{{\mathbb{K}}}
\newcommand{\BL}{{\mathbb{L}}}
\newcommand{\BM}{{\mathbb{M}}}
\newcommand{\BN}{{\mathbb{N}}}
\newcommand{\BO}{{\mathbb{O}}}
\newcommand{\BP}{{\mathbb{P}}}
\newcommand{\BQ}{{\mathbb{Q}}}
\newcommand{\BR}{{\mathbb{R}}}
\newcommand{\BS}{{\mathbb{S}}}
\newcommand{\BT}{{\mathbb{T}}}
\newcommand{\BU}{{\mathbb{U}}}
\newcommand{\BV}{{\mathbb{V}}}
\newcommand{\BW}{{\mathbb{W}}}
\newcommand{\BX}{{\mathbb{X}}}
\newcommand{\BY}{{\mathbb{Y}}}
\newcommand{\BZ}{{\mathbb{Z}}}

\newcommand{\Fa}{{\mathfrak{a}}}
\newcommand{\Fb}{{\mathfrak{b}}}
\newcommand{\Fc}{{\mathfrak{c}}}
\newcommand{\Fd}{{\mathfrak{d}}}
\newcommand{\Fe}{{\mathfrak{e}}}
\newcommand{\Ff}{{\mathfrak{f}}}
\newcommand{\Fg}{{\mathfrak{g}}}
\newcommand{\Fh}{{\mathfrak{h}}}
\newcommand{\Fi}{{\mathfrak{i}}}
\newcommand{\Fj}{{\mathfrak{j}}}
\newcommand{\Fk}{{\mathfrak{k}}}
\newcommand{\Fl}{{\mathfrak{l}}}
\newcommand{\Fm}{{\mathfrak{m}}}
\newcommand{\Fn}{{\mathfrak{n}}}
\newcommand{\Fo}{{\mathfrak{o}}}
\newcommand{\Fp}{{\mathfrak{p}}}
\newcommand{\Fq}{{\mathfrak{q}}}
\newcommand{\Fr}{{\mathfrak{r}}}
\newcommand{\Fs}{{\mathfrak{s}}}
\newcommand{\Ft}{{\mathfrak{t}}}
\newcommand{\Fu}{{\mathfrak{u}}}
\newcommand{\Fv}{{\mathfrak{v}}}
\newcommand{\Fw}{{\mathfrak{w}}}
\newcommand{\Fx}{{\mathfrak{x}}}
\newcommand{\Fy}{{\mathfrak{y}}}
\newcommand{\Fz}{{\mathfrak{z}}}

\newcommand{\FA}{{\mathfrak{A}}}
\newcommand{\FB}{{\mathfrak{B}}}
\newcommand{\FC}{{\mathfrak{C}}}
\newcommand{\FD}{{\mathfrak{D}}}
\newcommand{\FE}{{\mathfrak{E}}}
\newcommand{\FF}{{\mathfrak{F}}}
\newcommand{\FG}{{\mathfrak{G}}}
\newcommand{\FH}{{\mathfrak{H}}}
\newcommand{\FI}{{\mathfrak{I}}}
\newcommand{\FJ}{{\mathfrak{J}}}
\newcommand{\FK}{{\mathfrak{K}}}
\newcommand{\FL}{{\mathfrak{L}}}
\newcommand{\FM}{{\mathfrak{M}}}
\newcommand{\FN}{{\mathfrak{N}}}
\newcommand{\FO}{{\mathfrak{O}}}
\newcommand{\FP}{{\mathfrak{P}}}
\newcommand{\FQ}{{\mathfrak{Q}}}
\newcommand{\FR}{{\mathfrak{R}}}
\newcommand{\FS}{{\mathfrak{S}}}
\newcommand{\FT}{{\mathfrak{T}}}
\newcommand{\FU}{{\mathfrak{U}}}
\newcommand{\FV}{{\mathfrak{V}}}
\newcommand{\FW}{{\mathfrak{W}}}
\newcommand{\FX}{{\mathfrak{X}}}
\newcommand{\FY}{{\mathfrak{Y}}}
\newcommand{\FZ}{{\mathfrak{Z}}}

\newcommand{\CA}{{\cal A}}
\newcommand{\CB}{{\cal B}}
\newcommand{\CC}{{\cal C}}
\newcommand{\CD}{{\cal D}}
\newcommand{\CE}{{\cal E}}
\newcommand{\CF}{{\cal F}}
\newcommand{\CG}{{\cal G}}
\newcommand{\CH}{{\cal H}}
\newcommand{\CI}{{\cal I}}
\newcommand{\CJ}{{\cal J}}
\newcommand{\CK}{{\cal K}}
\newcommand{\CL}{{\cal L}}
\newcommand{\CM}{{\cal M}}
\newcommand{\CN}{{\cal N}}
\newcommand{\CO}{{\cal O}}
\newcommand{\CP}{{\cal P}}
\newcommand{\CQ}{{\cal Q}}
\newcommand{\CR}{{\cal R}}
\newcommand{\CS}{{\cal S}}
\newcommand{\CT}{{\cal T}}
\newcommand{\CU}{{\cal U}}
\newcommand{\CV}{{\cal V}}
\newcommand{\CW}{{\cal W}}
\newcommand{\CX}{{\cal X}}
\newcommand{\CY}{{\cal Y}}
\newcommand{\CZ}{{\cal Z}}

% Theorem Stil

\theoremstyle{plain}
\newtheorem{lem}{Lemma}
\newtheorem{Satz}[lem]{Satz}

\theoremstyle{definition}
\newtheorem{defn}{Definition}[section]

\theoremstyle{remark}
\newtheorem{bem}{Bemerkung}    %[section]



\newcommand{\card}{\mathop{\rm card}\nolimits}
\newcommand{\Sets}{((Sets))}
\newcommand{\id}{{\rm id}}
\newcommand{\supp}{\mathop{\rm Supp}\nolimits}

\newcommand{\ord}{\mathop{\rm ord}\nolimits}
\renewcommand{\mod}{\mathop{\rm mod}\nolimits}
\newcommand{\sign}{\mathop{\rm sign}\nolimits}
\newcommand{\ggT}{\mathop{\rm ggT}\nolimits}
\newcommand{\kgV}{\mathop{\rm kgV}\nolimits}
\renewcommand{\div}{\, | \,}
\newcommand{\notdiv}{\mathopen{\mathchoice
             {\not{|}\,}
             {\not{|}\,}
             {\!\not{\:|}}
             {\not{|}}
             }}

\newcommand{\im}{\mathop{{\rm Im}}\nolimits}
\newcommand{\coim}{\mathop{{\rm coim}}\nolimits}
\newcommand{\coker}{\mathop{\rm Coker}\nolimits}
\renewcommand{\ker}{\mathop{\rm Ker}\nolimits}

\newcommand{\pRang}{\mathop{p{\rm -Rang}}\nolimits}
\newcommand{\End}{\mathop{\rm End}\nolimits}
\newcommand{\Hom}{\mathop{\rm Hom}\nolimits}
\newcommand{\Isom}{\mathop{\rm Isom}\nolimits}
\newcommand{\Tor}{\mathop{\rm Tor}\nolimits}
\newcommand{\Aut}{\mathop{\rm Aut}\nolimits}

\newcommand{\adj}{\mathop{\rm adj}\nolimits}

\newcommand{\Norm}{\mathop{\rm Norm}\nolimits}
\newcommand{\Gal}{\mathop{\rm Gal}\nolimits}
\newcommand{\Frob}{{\rm Frob}}

\newcommand{\disc}{\mathop{\rm disc}\nolimits}

\renewcommand{\Re}{\mathop{\rm Re}\nolimits}
\renewcommand{\Im}{\mathop{\rm Im}\nolimits}

\newcommand{\Log}{\mathop{\rm Log}\nolimits}
\newcommand{\Res}{\mathop{\rm Res}\nolimits}
\newcommand{\Bild}{\mathop{\rm Bild}\nolimits}

\renewcommand{\binom}[2]{\left({#1}\atop{#2}\right)}
\newcommand{\eck}[1]{\langle #1 \rangle}
\newcommand{\wi}{\hspace{1pt} < \hspace{-6pt} ) \hspace{2pt}}


\begin{document}

\pagestyle{empty}

\begin{center}
{\huge OMS ronda final 2009} \\
\medskip primero examen - 13 de marzo, 2009
\end{center}
\vspace{8mm}
Tiempo: 4 horas\\
Cada problema vale 7 puntos.

\vspace{15mm}

\begin{enumerate}
\item[\textbf{1.}] Se considera un hex�gono regular $P$. Por und punto $A$ sean  $d_1 \leq d_2 \leq \ldots \leq d_6$ las distancias entre $A$ y las seis esquinas de $P$, en orden de sus valores. Encontrar el lugar geom�trico de todos los puntos $A$ al interior o en el borde de $P$, donde
\begin{enumerate}
\item[(a)] $d_3$ toma el valor m�s peque�o possible.
\item[(b)] $d_4$ toma el valor m�s grande possible.\\
\end{enumerate}

\bigskip

\item[\textbf{2.}] Un \emph{pal�ndromo} es un n�mero natural, de forma que en el sistema decimal su valor es lo mismo, independiente de que si es leido de delante � de atras. (por ejemplo $1129211$ o $7337$). Determinar todas las parejas $(m,n)$ de num�ros naturales tales que
\[(\underbrace{11\ldots 11}_{m})\cdot (\underbrace{11\ldots 11}_{n})\]
es und pal�ndromo.\\

\bigskip

\item[\textbf{3.}] Sean $a,b,c,d$ n�meros reales positivos. Demonstrar la desigualdad siguiente y determinar los casos de igualdad:
\[\frac{a-b}{b+c}+\frac{b-c}{c+d}+\frac{c-d}{d+a}+\frac{d-a}{a+b} \geq 0.\]

\bigskip

\item[\textbf{4.}] Sea $n$ un n�mero natural. Cada una de las casillas de un quadrado de $n\times n$ contiene uno de $n$ s�mbolos diferentes, de forma que cada s�mbolo est� en precisamente $n$ casillas. Demonstrar que existe una l�nea o una columna que contiene al menos $\sqrt{n}$ s�mbolos diferentes.\\

\bigskip

\item[\textbf{5.}] Sea $ABC$ un tri�ngulo con $AB \neq AC$ y incentro $I$. 
La circunferencia inscrita toca $BC$ en $D$. Sea $M$ el punto medio de $BC$. Demonstrar que la recta $IM$ biseca el trazo $AD$.

\end{enumerate}

\pagebreak


\begin{center}
{\huge OSM Tour final 2009} \\
\medskip secondo esame - 14 marzo 2009
\end{center}
\vspace{8mm}
Durata: 4 ore\\ 
Ogni esercizio vale 7 punti.
\vspace{15mm}

\begin{enumerate}

\item[\textbf{6.}] Trovare tutte le funzioni $f: \BR_{>0}\rightarrow \BR_{>0}$ che per ogni $x>y>z>0$ soddisfano la condizione seguente:
\[f(x-y+z)=f(x)+f(y)+f(z)-xy-yz+xz.\]

\bigskip

\item[\textbf{7.}] I punti $A$, $M_1$, $M_2$ et $C$ si trovano in questo ordine su una retta. Sia $k_1$ il cerchio con centro $M_1$ passante per $A$, et $k_2$ il cerchio con centro $M_2$ passante per $C$. I due cerchi si intersecano ai punti $E$ e $F$. Una tangente comune di $k_1$ e $k_2$ � tangente a $k_1$ in $B$ e a $k_2$ in $D$. Mostra che les rette $AB$, $CD$ e $EF$ si intersecano in un solo punto.\\


\bigskip

\item[\textbf{8.}] Considera un terreno qualsiasi composto da $n$ quadrati unit�. Alberto et Berta vorrebbero coprire questo terreno con delle piastrelle che hanno sia la forma di un domino $1\times 2$, sia di un  T-Tetromino. Alberto ha a disposizione delle piastrelle di un solo colore mentre Berta ha dei domino di due colori e dei T-Tetromino di quattro colori. Alberto pu� coprire il terreno in $a$ modi differenti, Berta in $b$ modi. Supponendo che $a\neq 0$, trova il rapporto $b/a$.\\

\bigskip

\item[\textbf{9.}] Trova tutte le funzioni iniettive $f:\BN \rightarrow \BN$ tali che per ogni numero naturale $n$ si ha che 
\[f(f(n)) \leq \frac{f(n)+n}{2}.\]

\bigskip

\item[\textbf{10.}] Sia $n>3$ un numero naturale. Mostra che $4^n+1$ ammette un divisore primo $>20$. 
\end{enumerate}
\end{document}
