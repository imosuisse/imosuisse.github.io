\en{Let $O$ be the centre of the circumcircle of an acute triangle $ABC$. The line $AC$ intersects the circumcircle of the triangle $ABO$ a second time at $S$. Prove that the line $OS$ is perpendicular to the line $BC$.
}

\de{Sei $ABC$ ein spitzwinkliges Dreieck mit Umkreismittelpunkt $O$. Die Gerade $AC$ schneidet den Umkreis des Dreiecks $ABO$ ein zweites Mal in $S$. Beweise, dass die Geraden $OS$ und $BC$ senkrecht aufeinander stehen.
}

\ita{Sia $O$ il centro del cerchio circoscritto ad un triangolo acutangolo $ABC$. La retta $AC$ interseca il cerchio circoscritto del triangolo $ABO$ una seconda volta in $S$. Dimostra che la retta $OS$ \`e perpendicolare alla retta $BC$.}

\fr{Soit $O$ le centre du cercle circonscrit d'un triangle aigu $ABC$. La droite $AC$ coupe le cercle circonscrit du triangle $ABO$ une deuxième fois en $S$. Montrer que la droite $OS$ est perpendiculaire à la droite $BC$.}