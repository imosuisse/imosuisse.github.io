\en {Find all finite sets $S$ of positive integers with at least two elements, such that if $m>n$ are two elements of $S$, then $$\frac{n^2}{m-n}$$
is also an element of $S$.

\newline
\newline

\textbf{Answer}: $S=\{s,2s\}$, for a positive integer $s$.

\textbf{Solution}(Arnaud):

Trying to apply number theoretical methods to deduce something from the fact that $m-n$ divides $n^2$ does not seem to lead anywhere. Instead, we will try to find the extreme values that the quotient $n^2/(m-n)$ can achieve. This will give us some interesting bounds on the elements of $S$.

\begin{enumerate}
    \item First, since $S$ contains at least two elements, we take $l>s$ to be the largest and smallest elements of $S$. Since $s^2/(l-s)$ also belongs to $S$ we must have
    \[
    \frac{s^2}{l-s}\geq s\quad \Longrightarrow \quad 2s\geq l.
    \]
    \item Now, consider $k<l$ the second largest element of $S$. The number $k^2/(l-k)$ belongs to $S$. We claim that $k^2/(l-k)\neq l$, which implies
    \[
    \frac{k^2}{l-k}\leq k\quad \Longrightarrow \quad 2k\leq l.
    \]
\end{enumerate}

We prove the previous claim in the following lemma.

\begin{lem}
The equation $k^2/(l-k)= l$ has no positive integer solutions $l>k$.
\end{lem}
\begin{proof}
Assume such a solution exists, then it also solves
    \[
    l^2=k^2+kl.
    \]
    Since the equation is homogeneous (of degree 2), we can assume that $\gcd (k,l)=1$. Since we must have $l\div k^2$, we deduce that $l=1$. This is a contradiction since $k$ is a positive integer with $k<l$.
\end{proof}

Combining the two relations above, and using that $s\leq k$, we get
\[
l\leq 2s\leq 2k\leq l.
\]
Therefore $k=s$ and $S$ contains exactly two elements. Moreover, the inequalities above also implies $l=2s$. So $S=\{s,2s\}$ as desired.

Any such set satisfies the desired property because
\[
\frac{s^2}{2s-s}=s\in S.
\]

\newpage

\textbf{Marking scheme}

Case $\vert S\vert =2$ (3P)
\begin{itemize}
    \item 1P: Stating the correct answer and checking that it indeed works.
    \item 2P: Statement and proof of the lemma.
\end{itemize}

Reduction to $\vert S\vert =2$ (4P)
\begin{itemize}
    \item 1P: Considering the smallest and largest elements of $S$.
    \item 1P: $l\leq 2k_1$ for some $k_1\neq l$ in $S$ (e.g.\ $k_1=s$).
    \item 2P: Conclude.
\end{itemize}
-1P for logical flaws.
}
