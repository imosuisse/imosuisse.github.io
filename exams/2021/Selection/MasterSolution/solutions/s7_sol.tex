 Let $n$ be a positive integer. Call a sequence of positive integers $a_1, a_2, \ldots , a_n$ \emph{tame} if it satisfies
\[
1 \cdot a_1 \leq 2\cdot a_2\leq  \ldots \leq n \cdot a_n.
\]
Determine the number of tame permutations of $1,2, \ldots, n$.

\bigskip

\textbf{Answer:} The answer is the $n$th Fibonacci number.

\bigskip

\textbf{Solution 1, induction (Tanish):} We prove that the number of possibilities is the $n^{th}$ Fibonacci number $F_n$ by induction ($F_0 = 1, F_1 = 1$). For the base case, observe that there is $1$ way to do it for $n = 1$ and $2$ ways for $n = 2$ (all the arrangements work in both cases). \\
Now suppose this holds true up to $n-1$ and now consider the arrangements for $n$. Where can we place $n$ in the sequence? Clearly, it can always go in the final position and placing the rest of the numbers is doable in $F_{n-1}$ ways. If it goes in the penultimate position, then the last number must be $n-1$ (since no other number k would satisfy $n(n-1) \leq kn$) and placing the rest is doable in $F_{n-2}$ ways. We now prove $n$ cannot go anywhere else. \\
Let us place $n$ as $a_k$, $k < n-1$. Now consider the smallest element of $a_{k+1}, \dots a_n$. This element is at most $k$ and so must go in the last position (as it will not satisfy the inequality if it is placed any earlier.) However, we now cannot place anyone in the penultimate position, as the smallest number that can go there is $k+1$ but $(k+1)(n-1) > kn$. You can also obtain a contradiction by trying to place $n-1$, as the earliest it can appear is as $a_{k+1}$.

\bigskip

\textbf{Solution 2, bijection (Tanish):} We prove that the problem is strictly equivalent to writing $n$ as a sum of $1$s and $2$s, which is a well known recurrence problem left to the reader :) \\
Consider a permutation that works that is not the identity, and let $i$ be the first index such that $a_i \neq i$. It follows that $a_j = i$ for some $j > i$. Now, where can we place $j$? If $j$ appears before $a_j$, it has to be $a_i$, as otherwise the inequality is not satisfied. But if $j \neq i+1$ then we are not able to take any value for $a_{j-1}$, as the smallest number that can go there is $i+1$ but $(i+1)(j-1) > ij$. So either $j = i+1$ and we have swapped around two consecutive terms, or $j$ appears after $a_j$. In the latter case, just take the largest value in $a_i, a_{i+1}, \dots a_{j-1}$. This value is at least $j+1$ but this immediately contradicts the inequality again. In the former case, the placement of $j$ and $j+1$ do not actually affect our ability to place anything afterwards, so we can look at the next index not mapped to itself and concluded it has been swapped with the one immediately following it by the same reasoning, and continue onwards in this manner. Therefore the only changes we can make from the identity permutation are swapping two consecutive values, and this is equivalent to writing $n$ as a sum of $1$s and $2$s as desired; a 1 is an element mapped to itself and a 2 is a swap of two consecutive elements. 

\bigskip 

\textbf{Marking Scheme:} \\
This first section is additive.
\begin{enumerate}
\item 1P: Stating the answer is $F_n$ or that it satisfies the recurrence relation $f(n) = f(n-1) + f(n-2)$.
\item 1P: Assuming we are not dealing with the identity, in order to either consider the placement of $n$ or the smallest/largest index not mapped to itself 
\item 2P: Showing that this index has to have been swapped (if you write the permutation as a product of disjoint cycles, it must belong to a 2-cycle). 
\item 1P: Showing that if it was swapped with an index that is more than $2$ away, we have a contradiction.
\item 2P: Concluding (recursive formula if they considered $n$; looking at next index not mapped to itself $\to$ equivalence to a problem whose solution is Fibonacci, etc.) If the student does a solution by induction and forgets to note the base cases ($f(1) = 1, f(2) = 2$) they will not get these points, \textbf{even} if they say that the solution is the Fibonacci numbers. 
\end{enumerate}
The second item should \emph{not} be awarded if the student is considering the placement of $1$. \\
If the student obtains $\le$ 1P from the first additive section, they are eligible to receive \textbf{one} of the following items:
\begin{enumerate}
\item 1P: Considering the placement of $1$.
\item 2P: Showing $1$ has to have been swapped with $2$ or sent to itself.
\item 1P: Stating the set of plausible sequences is the identity and all variations of the identity where an index is moved by at most $1$ place. 
\item 1P: Attempting to show that when the permutation is written as a product of disjoint cycles, all the cycles are of length 2.
\end{enumerate}
Finally, the following points may be awarded \emph{instead} of the points previously received \textbf{if} they have fewer than this many points. 
\begin{enumerate}
\item 4P: Any actual proof that there are no cycles of length >2
\item 5P: Any actual proof that there are no cycles of length >2 and stating the answer is $F_n$ or satisfies the recurrence relation $f(n) = f(n-1) + f(n-2)$.
\end{enumerate}
In other words, the student receives max(best item from section 3, section 1 + \{best item from section 2 if section 1 $\le 1$ \}).  
