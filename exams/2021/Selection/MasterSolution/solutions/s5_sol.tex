\en{
Let $n$ be a positive integer. Some of the squares of a $3n \times 3n$ board are marked. For any marked square $T$, we denote by $\ell(T)$ the number of marked squares in the same row to the left of $T$ and by $d(T)$ the number of marked squares in the same column below $T$. Determine the maximal number of marked squares given that $\ell(T)+d(T)$ is even for every marked square $T$.

\bigskip

\textbf{Answer:} The maximal number of marked squares is $6n^2$.

\bigskip

\textbf{Solution (Valentin):} Suppose that $l(T) + d(T)$ is even for all marked squares $T$. We first proof an essential lemma.

\bigskip

\textbf{Lemma.}

There are at most $k$ rows with an odd number of marked squares within the left-most $k$ columns.

\begin{proof}
We proceed by induction. For $k = 1$, assume that there is more than $1$ marked squares in the left-most column and denote the second lowest one by $T$. We have $l(T)+d(T) = 1$ is odd, a contradiction. It follows that the lemma holds for $k = 1$.

For $k > 1$ assume the lemma holds for $k-1$, which implies that there are at most $k-1$ rows that have an odd number of marked squares in the first $k-1$ columns. Denote by $a$ the number of these rows that also have a marked square in column $k$. Since the parity of $l(T)$ alternates among the marked squares in each column, it follows that there are at most $a+1$ marked squares in the $k$-th column with $l(T)$ even. The number of rows that have an odd number of marked squares in the first $k$ columns is therefore bounded above by $(k-1-a) + (a+1) = k$. This concludes the inductive step.
\end{proof}

\bigskip

From the lemma, and the fact that the parity of $l(T)$ alternates among the marked squares in a given column, it follows immediately that there are at most $2k-1$ marked squares in the $k$-th column from the left. Similarly there are at most $2k-1$ marked squares in the $k$-th row from the bottom by symmetry of the argument.

Applying these bounds to the left-most $n$ columns and the bottom-most $n$ rows, we find that the total number of marked squares is bounded from above by
$$2\cdot\Big(1+3+5+\cdots+(2n-1)\Big) + (2n)^2 = 6n^2.$$

\newpage

This is indeed the maximal number of marked squared and can be obtained via the following construction illustrated here for $n = 3$:

\bigskip

\renewcommand{\mark}{$\bigstar$}

\begin{center}
\begin{tikzpicture}
\draw[step=1.0,black,thin] (0,0) grid (9,9);
\draw[very thick] (3,0) -- (3,9);
\draw[very thick] (0,3) -- (9,3);

\foreach \x in {0,...,2} {
    \foreach \y in {-\x,...,\x} {
        \node at (0.5 + \x, 6.5 + \y) {\mark};
        \node at (6.5 + \y, 0.5 + \x) {\mark};
    }
}

\foreach \x in {0,...,5} {
    \foreach \y in {0,...,5} {
        \node at (3.5 + \x, 3.5 + \y) {\mark};
    }
}
\end{tikzpicture}
\end{center}

\bigskip

\textbf{Marking Scheme (additive):}

\begin{enumerate}
\item 1P: Claiming that in the $k$-th column/row there are at most $2k-1$ marked squares
\item 3P: Proving that in the $k$-th column/row there are at most $2k-1$ marked squares
\begin{enumerate}
\item[(b1)] 1P: Using or stating that the parities of $l(T)$ and $d(T)$ alternate among the marked squares in a given line
\end{enumerate}
\item 2P: Giving a general construction with $6n^2$ marked squares
\item 1P: Finishing assuming (b)
\end{enumerate}
}