\en{We call a positive integer \emph{silly} if the sum of its positive divisors is a square. Prove that there are infinitely many silly numbers.

\bigskip

\textbf{Solution (Louis):} Let $\sigma(n)$ denote the sum of all positive divisors of the integer $n$, and we will order the prime numbers $2=p_1 < p_2 < p_3 < \ldots$. An important observation is that $\sigma(n)$ is multiplicative, in the sense that if $n_1, n_2$ are coprime, then $\sigma(n_1n_2) = \sigma(n_1)\sigma(n_2)$. More generally, we have the identity
\[
\sigma(p_1^{\alpha_1}\cdot\ldots\cdot p_k^{\alpha_k}) = \prod_{i=1}^k{(1 + p_i + \ldots + p_i^{\alpha_i})}.
\]

We could first try to solve the exercise in the following way: For any integer $k > 2$, consider $S = \{p_1, \ldots, p_k\}$ the set of the $k$ smallest prime numbers. We notice that all prime factors of the numbers $\sigma(p_i)$ are smaller that $p_k$ (since either $\sigma(p_i) < p_k$ or $\sigma(p_k) = p_k + 1$ is not prime). Therefore there are $2^k$ subsets $A$ of $S$, but if we consider the product of all elements in $A$, the sum of its divisors will only be divisible by primes among $p_1, \ldots, p_{k-1}$ so if we consider the parity of the exponent of each of these primes there are only $2^{k-1}$ results possible, so by the pigeonhole principle there must exist two distinct subsets $A$ and $A'$ such that all the exponents have the same parity. Therefore the product of all the elements in the symmetric difference $A\Delta A'$ will be a silly number (the symmetric difference of two sets is defined as the set of all elements that belong to exactly one of the sets).

This contains most of the ideas for the solution, but the problem is that if we found sets $A, A'$ for $k$, we will also find these same sets for $k+1$ so this argument is not enough to guarantee the existence of infinitely many silly numbers and we need a more subtle argument.

Assume that we already have $l$ silly numbers $n_1, \ldots, n_l$. We will prove that there exist a silly number distinct of all these numbers. Let $p_{\alpha}$ be the largest prime factor of one of these $l$ silly numbers and let $\gamma$ be the largest exponent in the prime factor decompositions of these numbers. Finally let $p_{\beta}$ be a prime number larger than $\sigma(p_{\alpha}^{\gamma+1})$ and consider the set 
\[
    S = \{p_1^{\gamma+1}, \ldots, p_{\alpha}^{\gamma+1}, p_{\alpha+1}, \ldots, p_{\beta}\}.
\]

Now by construction we can prove again that for each $a\in S$ the prime factors of $\sigma(a)$ are smaller that $p_{\beta}$ so we can make the same argument as before, but now the element we will construct will be distinct from all the $n_1, \ldots, n_l$, since either it will be divisible by a prime number which does not divide any of these numbers, or one of the prime numbers will appear with a higher multiplicity than in any of the $l$ silly numbers. 

\bigskip

\textbf{Marking Scheme (additive):}

\begin{enumerate}
    \item 1P: Notice that $\sigma(p_1), \ldots, \sigma(p_k)$ are only divisible by $p_1, \ldots, p_{k-1}$.
    \item 2P: Use pigeonhole to construct a silly number as the product of the elements of a certain subset.
    \item 4P: Finish the solution.
\end{enumerate}
}