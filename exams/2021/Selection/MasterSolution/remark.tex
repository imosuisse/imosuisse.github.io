\de{Eine vollständige Lösung ist 7 Punkte wert. Bei jeder Aufgabe kann es bis zu 2 Punkte Abzug geben für kleine Fehler bei einer sonst korrekten Lösung. Teilpunkte werden gemäss dem Punkteschema vergeben.

Im Anschluss befinden sich die Lösungen mit Vorrundentheorie, die den Korrektoren bekannt sind. Am Ende jedes Problems werden noch alternative Lösungen präsentiert, die auch andere Theorie verwenden können. Während des Trainings zu Hause werden die Teilnehmenden dazu ermutigt, alle ihnen bekannten Methoden zu verwenden. An der Prüfung hingegen ist es nicht empfohlen mit Methoden, welche sie unter Prüfungskonditionen nicht genügend beherrschen, nach alternativen Lösungen zu suchen. Damit wird riskiert, dass wertvolle Zeit verloren geht.}

\fr{\textbf{Remarque liminaire:} Une solution complète rapporte 7 points. Pour chaque problème, jusqu'à 2 points pourront être déduits d'une solution correcte en cas de lacunes mineures. Les solutions partielles sont évaluées selon le barème suivant.

Ci-dessous vous trouverez les solutions élémentaires connues des correcteurs. Des solutions alternatives sont présentées à la fin de chaque problème. Les étudiants sont naturellement encouragés à essayer toutes les méthodes à leur disposition lors de l'entrainement, mais sont également avisés de ne pas chercher de solutions alternatives qui emploient des méthodes qu'ils ne maîtrisent pas en condition d'examen, au risque de perdre un temps précieux.}

\en{\textbf{Preliminary remark:} A complete solution is worth 7 points. For every problem, up to 2 points can be deducted from a correct solution for (minor) flaws. Partial marks are attributed according to the marking schemes.

Below you will find the elementary solutions known to correctors. Alternative solutions are presented in a complementary section at the end of each problem. Students are encouraged to use any methods at their disposal when training at home, but should be wary of attempting to find alternative solutions using methods they do not feel comfortable with under exam conditions as they risk losing valuable time.}