\documentclass[12pt,a4paper]{article}

\usepackage{amsfonts}
\usepackage[centertags]{amsmath}
\usepackage{german}
\usepackage{amsthm}
\usepackage{amssymb}

\leftmargin=0pt \topmargin=-5mm \headheight=0in \headsep=0in \oddsidemargin=0pt \textwidth=6.5in
\textheight=9.3in

\catcode`\� = \active \catcode`\� = \active \catcode`\� = \active \catcode`\� = \active \catcode`\� = \active
\catcode`\� = \active

\def�{"A}
\def�{"a}
\def�{"O}
\def�{"o}
\def�{"U}
\def�{"u}





% Schriftabk�rzungen

\newcommand{\eps}{\varepsilon}
\renewcommand{\phi}{\varphi}
\newcommand{\Sl}{\ell}    % sch�nes l
\newcommand{\ve}{\varepsilon}  %Epsilon

\newcommand{\BA}{{\mathbb{A}}}
\newcommand{\BB}{{\mathbb{B}}}
\newcommand{\BC}{{\mathbb{C}}}
\newcommand{\BD}{{\mathbb{D}}}
\newcommand{\BE}{{\mathbb{E}}}
\newcommand{\BF}{{\mathbb{F}}}
\newcommand{\BG}{{\mathbb{G}}}
\newcommand{\BH}{{\mathbb{H}}}
\newcommand{\BI}{{\mathbb{I}}}
\newcommand{\BJ}{{\mathbb{J}}}
\newcommand{\BK}{{\mathbb{K}}}
\newcommand{\BL}{{\mathbb{L}}}
\newcommand{\BM}{{\mathbb{M}}}
\newcommand{\BN}{{\mathbb{N}}}
\newcommand{\BO}{{\mathbb{O}}}
\newcommand{\BP}{{\mathbb{P}}}
\newcommand{\BQ}{{\mathbb{Q}}}
\newcommand{\BR}{{\mathbb{R}}}
\newcommand{\BS}{{\mathbb{S}}}
\newcommand{\BT}{{\mathbb{T}}}
\newcommand{\BU}{{\mathbb{U}}}
\newcommand{\BV}{{\mathbb{V}}}
\newcommand{\BW}{{\mathbb{W}}}
\newcommand{\BX}{{\mathbb{X}}}
\newcommand{\BY}{{\mathbb{Y}}}
\newcommand{\BZ}{{\mathbb{Z}}}

\newcommand{\Fa}{{\mathfrak{a}}}
\newcommand{\Fb}{{\mathfrak{b}}}
\newcommand{\Fc}{{\mathfrak{c}}}
\newcommand{\Fd}{{\mathfrak{d}}}
\newcommand{\Fe}{{\mathfrak{e}}}
\newcommand{\Ff}{{\mathfrak{f}}}
\newcommand{\Fg}{{\mathfrak{g}}}
\newcommand{\Fh}{{\mathfrak{h}}}
\newcommand{\Fi}{{\mathfrak{i}}}
\newcommand{\Fj}{{\mathfrak{j}}}
\newcommand{\Fk}{{\mathfrak{k}}}
\newcommand{\Fl}{{\mathfrak{l}}}
\newcommand{\Fm}{{\mathfrak{m}}}
\newcommand{\Fn}{{\mathfrak{n}}}
\newcommand{\Fo}{{\mathfrak{o}}}
\newcommand{\Fp}{{\mathfrak{p}}}
\newcommand{\Fq}{{\mathfrak{q}}}
\newcommand{\Fr}{{\mathfrak{r}}}
\newcommand{\Fs}{{\mathfrak{s}}}
\newcommand{\Ft}{{\mathfrak{t}}}
\newcommand{\Fu}{{\mathfrak{u}}}
\newcommand{\Fv}{{\mathfrak{v}}}
\newcommand{\Fw}{{\mathfrak{w}}}
\newcommand{\Fx}{{\mathfrak{x}}}
\newcommand{\Fy}{{\mathfrak{y}}}
\newcommand{\Fz}{{\mathfrak{z}}}

\newcommand{\FA}{{\mathfrak{A}}}
\newcommand{\FB}{{\mathfrak{B}}}
\newcommand{\FC}{{\mathfrak{C}}}
\newcommand{\FD}{{\mathfrak{D}}}
\newcommand{\FE}{{\mathfrak{E}}}
\newcommand{\FF}{{\mathfrak{F}}}
\newcommand{\FG}{{\mathfrak{G}}}
\newcommand{\FH}{{\mathfrak{H}}}
\newcommand{\FI}{{\mathfrak{I}}}
\newcommand{\FJ}{{\mathfrak{J}}}
\newcommand{\FK}{{\mathfrak{K}}}
\newcommand{\FL}{{\mathfrak{L}}}
\newcommand{\FM}{{\mathfrak{M}}}
\newcommand{\FN}{{\mathfrak{N}}}
\newcommand{\FO}{{\mathfrak{O}}}
\newcommand{\FP}{{\mathfrak{P}}}
\newcommand{\FQ}{{\mathfrak{Q}}}
\newcommand{\FR}{{\mathfrak{R}}}
\newcommand{\FS}{{\mathfrak{S}}}
\newcommand{\FT}{{\mathfrak{T}}}
\newcommand{\FU}{{\mathfrak{U}}}
\newcommand{\FV}{{\mathfrak{V}}}
\newcommand{\FW}{{\mathfrak{W}}}
\newcommand{\FX}{{\mathfrak{X}}}
\newcommand{\FY}{{\mathfrak{Y}}}
\newcommand{\FZ}{{\mathfrak{Z}}}

\newcommand{\CA}{{\cal A}}
\newcommand{\CB}{{\cal B}}
\newcommand{\CC}{{\cal C}}
\newcommand{\CD}{{\cal D}}
\newcommand{\CE}{{\cal E}}
\newcommand{\CF}{{\cal F}}
\newcommand{\CG}{{\cal G}}
\newcommand{\CH}{{\cal H}}
\newcommand{\CI}{{\cal I}}
\newcommand{\CJ}{{\cal J}}
\newcommand{\CK}{{\cal K}}
\newcommand{\CL}{{\cal L}}
\newcommand{\CM}{{\cal M}}
\newcommand{\CN}{{\cal N}}
\newcommand{\CO}{{\cal O}}
\newcommand{\CP}{{\cal P}}
\newcommand{\CQ}{{\cal Q}}
\newcommand{\CR}{{\cal R}}
\newcommand{\CS}{{\cal S}}
\newcommand{\CT}{{\cal T}}
\newcommand{\CU}{{\cal U}}
\newcommand{\CV}{{\cal V}}
\newcommand{\CW}{{\cal W}}
\newcommand{\CX}{{\cal X}}
\newcommand{\CY}{{\cal Y}}
\newcommand{\CZ}{{\cal Z}}

% Theorem Stil

\theoremstyle{plain}
\newtheorem{lem}{Lemma}
\newtheorem{Satz}[lem]{Satz}

\theoremstyle{definition}
\newtheorem{defn}{Definition}[section]

\theoremstyle{remark}
\newtheorem{bem}{Bemerkung}    %[section]


\newcommand{\ord}{\mathop{\rm ord}\nolimits}
\renewcommand{\mod}{\mathop{\rm mod}\nolimits}
\newcommand{\sign}{\mathop{\rm sign}\nolimits}
\newcommand{\ggT}{\mathop{\rm ggT}\nolimits}
\newcommand{\kgV}{\mathop{\rm kgV}\nolimits}
\renewcommand{\div}{\, | \,}
\newcommand{\notdiv}{\mathopen{\mathchoice
             {\not{|}\,}
             {\not{|}\,}
             {\!\not{\:|}}
             {\not{|}}
             }}



\renewcommand{\Re}{\mathop{\rm Re}\nolimits}
\renewcommand{\Im}{\mathop{\rm Im}\nolimits}
\newcommand{\eck}[1]{\langle #1 \rangle}
\newcommand{\wi}{\angle}

\renewcommand{\arraystretch}{1.5}

\begin{document}

\pagestyle{empty}

\begin{center}
{\huge SMO - Turno finale 2007} \\
\medskip primo esame - 23 marzo 2007
\end{center}
\vspace{8mm}
Durata: 4 ore\\
Ogni esercizio vale 7 punti.

\vspace{15mm}

\begin{enumerate}
\item[\textbf{1.}] Determina tutte le soluzioni reali positive del seguente sistema di equazioni:
\vspace{-0.2cm}
\begin{center}
\begin{tabular}{cp{1cm}cp{1cm}c}
$a=\max\{\frac{1}{b},\frac{1}{c}\}$ & & $b=\max\{\frac{1}{c},\frac{1}{d}\}$ & & $c=\max\{\frac{1}{d},\frac{1}{e}\}$\\
$d=\max\{\frac{1}{e},\frac{1}{f}\}$ & & $e=\max\{\frac{1}{f},\frac{1}{a}\}$ & & $f=\max\{\frac{1}{a},\frac{1}{b}\}$\\
\end{tabular}
\end{center}

\bigskip

\item[\textbf{2.}] Siano $a,b,c$ tre numeri interi tali che $a+b+c$ \`e divisibile per 13. Dimostra che anche 
\[a^{2007}+b^{2007}+c^{2007}+2\cdot 2007 abc\]
\`e divisibile per 13.\\

\smallskip

\item[\textbf{3.}] Considera il piano diviso in quadrati unitari. Ogni quadrato deve essere colorato con uno tra $n$ colori in modo che se quattro quadrati possono essere coperti con un L-Tetromino, allora questi quadrati devono avere quattro colori diversi (il L-Tetromino pu\`o essere ruotato e riflesso). Determina il pi\`u piccolo valore di $n$ per cui questo \`e possibile.\\

\smallskip

\item[\textbf{4.}] Sia $ABC$ un triangolo acuto con $AB>AC$ e sia $H$ il punto di intersezione delle altezze.  Sia $D$ il punto di intersezione dell'altezza passante per $A$ con il lato $BC$. Si ottenga $E$ specchiando $C$ rispetto a $D$. Sia $S$ il punto di intersezione delle rette $AE$ e $BH$. Sia $N$ il punto medio di $AE$, e $M$ quello di $BH$. Dimostra che $MN$ e $DS$ sono perpendicolari.\\

\smallskip

\item[\textbf{5.}] Determina tutte le funzioni $f:\BR_{\geq 0} \rightarrow \BR_{\geq 0}$ con le seguenti caratteristiche:
\begin{enumerate}
\item[(a)] $f(1)=0$,
\item[(b)] $f(x)>0$ per ogni $x>1$,
\item[(c)] Per ogni $x,y \geq 0$ tali che $x+y>0$, vale
\[f(xf(y))f(y)=f\left(\frac{xy}{x+y}\right).\]
\end{enumerate}

\end{enumerate}

\vspace{1.5cm}

\begin{center}
 Buon lavoro!
\end{center}

\pagebreak


\begin{center}
{\huge SMO - Turno finale 2007} \\
\medskip secondo esame - 24 marzo 2007
\end{center}
\vspace{8mm}
Durata: 4 ore \\
Ogni esercizio vale 7 punti.

\vspace{15mm}

\begin{enumerate}
\item[\textbf{6.}] Tre cerchi $k_1,k_2,k_3$ della stessa grandezza si incrociano (non tangenzialmente) in un punto $P$. Siano $A$ e $B$ i centri dei cerchi $k_1$ e $k_2$. Sia $D$ il punto di intersezione diverso da $P$ di $k_3$ con $k_1$, e $C$ quello di $k_3$ con $k_2$. Dimostra che $ABCD$ \`e un parallelogrammo.\\

\medskip

\item[\textbf{7.}] Siano $a,b,c$ numeri reali non negativi con media aritmetica $m=\frac{a+b+c}{3}$. Dimostra che vale
\[\sqrt{a+\sqrt{b+\sqrt{c}}}+\sqrt{b+\sqrt{c+\sqrt{a}}}+\sqrt{c+\sqrt{a+\sqrt{b}}}\leq 3\;\sqrt{m+\sqrt{m+\sqrt{m}}}.\]

\medskip

\item[\textbf{8.}] Sia $M \subset \{1,2,3,\ldots,2007\}$ un insieme con la seguente caratteristica: tra qualsiasi tre numeri di $M$, se ne possono sempre scegliere due tali che il primo \`e divisibile per il secondo. Quanti numeri pu\`o contenere $M$ al massimo?\\

\medskip

\item[\textbf{9.}] Determina tutte le coppie $(a,b)$ di numeri naturali tali che 
\[\frac{a^3+1}{2ab^2+1}\]
\`e un numero intero.\\

\medskip

\item[\textbf{10.}] Suddividi il piano in triangoli equilateri di lato 1. Considera un triangolo equilatero di lato $n$ i cui lati giacciono sul reticolo determinato dalla suddivisione in triangoli unitari. Su ogni punto di intersezione del reticolo sui lati e all'interno di questo triangolo \`e posta una pedina. Considera il gioco in cui in ogni mossa si sceglie un triangolo unitario che abbia esattamente due angoli coperti da una pedina. Queste due pedine vengono rimosse, e sul terzo angolo viene posta una nuova pedina. Per quali $n$ \`e possibile che dopo un numero finito di mosse resta solo una pedina?

\end{enumerate}
\vspace{1.5cm}

\center{Buon lavoro!}
\end{document}
