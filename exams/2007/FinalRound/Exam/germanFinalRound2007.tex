\documentclass[12pt,a4paper]{article}

\usepackage{amsfonts}
\usepackage[centertags]{amsmath}
\usepackage{german}
\usepackage{amsthm}
\usepackage{amssymb}

\leftmargin=0pt \topmargin=0pt \headheight=0in \headsep=0in \oddsidemargin=0pt \textwidth=6.5in
\textheight=9in

\catcode`\� = \active \catcode`\� = \active \catcode`\� = \active \catcode`\� = \active \catcode`\� = \active
\catcode`\� = \active

\def�{"A}
\def�{"a}
\def�{"O}
\def�{"o}
\def�{"U}
\def�{"u}





% Schriftabk�rzungen

\newcommand{\eps}{\varepsilon}
\renewcommand{\phi}{\varphi}
\newcommand{\Sl}{\ell}    % sch�nes l
\newcommand{\ve}{\varepsilon}  %Epsilon

\newcommand{\BA}{{\mathbb{A}}}
\newcommand{\BB}{{\mathbb{B}}}
\newcommand{\BC}{{\mathbb{C}}}
\newcommand{\BD}{{\mathbb{D}}}
\newcommand{\BE}{{\mathbb{E}}}
\newcommand{\BF}{{\mathbb{F}}}
\newcommand{\BG}{{\mathbb{G}}}
\newcommand{\BH}{{\mathbb{H}}}
\newcommand{\BI}{{\mathbb{I}}}
\newcommand{\BJ}{{\mathbb{J}}}
\newcommand{\BK}{{\mathbb{K}}}
\newcommand{\BL}{{\mathbb{L}}}
\newcommand{\BM}{{\mathbb{M}}}
\newcommand{\BN}{{\mathbb{N}}}
\newcommand{\BO}{{\mathbb{O}}}
\newcommand{\BP}{{\mathbb{P}}}
\newcommand{\BQ}{{\mathbb{Q}}}
\newcommand{\BR}{{\mathbb{R}}}
\newcommand{\BS}{{\mathbb{S}}}
\newcommand{\BT}{{\mathbb{T}}}
\newcommand{\BU}{{\mathbb{U}}}
\newcommand{\BV}{{\mathbb{V}}}
\newcommand{\BW}{{\mathbb{W}}}
\newcommand{\BX}{{\mathbb{X}}}
\newcommand{\BY}{{\mathbb{Y}}}
\newcommand{\BZ}{{\mathbb{Z}}}

\newcommand{\Fa}{{\mathfrak{a}}}
\newcommand{\Fb}{{\mathfrak{b}}}
\newcommand{\Fc}{{\mathfrak{c}}}
\newcommand{\Fd}{{\mathfrak{d}}}
\newcommand{\Fe}{{\mathfrak{e}}}
\newcommand{\Ff}{{\mathfrak{f}}}
\newcommand{\Fg}{{\mathfrak{g}}}
\newcommand{\Fh}{{\mathfrak{h}}}
\newcommand{\Fi}{{\mathfrak{i}}}
\newcommand{\Fj}{{\mathfrak{j}}}
\newcommand{\Fk}{{\mathfrak{k}}}
\newcommand{\Fl}{{\mathfrak{l}}}
\newcommand{\Fm}{{\mathfrak{m}}}
\newcommand{\Fn}{{\mathfrak{n}}}
\newcommand{\Fo}{{\mathfrak{o}}}
\newcommand{\Fp}{{\mathfrak{p}}}
\newcommand{\Fq}{{\mathfrak{q}}}
\newcommand{\Fr}{{\mathfrak{r}}}
\newcommand{\Fs}{{\mathfrak{s}}}
\newcommand{\Ft}{{\mathfrak{t}}}
\newcommand{\Fu}{{\mathfrak{u}}}
\newcommand{\Fv}{{\mathfrak{v}}}
\newcommand{\Fw}{{\mathfrak{w}}}
\newcommand{\Fx}{{\mathfrak{x}}}
\newcommand{\Fy}{{\mathfrak{y}}}
\newcommand{\Fz}{{\mathfrak{z}}}

\newcommand{\FA}{{\mathfrak{A}}}
\newcommand{\FB}{{\mathfrak{B}}}
\newcommand{\FC}{{\mathfrak{C}}}
\newcommand{\FD}{{\mathfrak{D}}}
\newcommand{\FE}{{\mathfrak{E}}}
\newcommand{\FF}{{\mathfrak{F}}}
\newcommand{\FG}{{\mathfrak{G}}}
\newcommand{\FH}{{\mathfrak{H}}}
\newcommand{\FI}{{\mathfrak{I}}}
\newcommand{\FJ}{{\mathfrak{J}}}
\newcommand{\FK}{{\mathfrak{K}}}
\newcommand{\FL}{{\mathfrak{L}}}
\newcommand{\FM}{{\mathfrak{M}}}
\newcommand{\FN}{{\mathfrak{N}}}
\newcommand{\FO}{{\mathfrak{O}}}
\newcommand{\FP}{{\mathfrak{P}}}
\newcommand{\FQ}{{\mathfrak{Q}}}
\newcommand{\FR}{{\mathfrak{R}}}
\newcommand{\FS}{{\mathfrak{S}}}
\newcommand{\FT}{{\mathfrak{T}}}
\newcommand{\FU}{{\mathfrak{U}}}
\newcommand{\FV}{{\mathfrak{V}}}
\newcommand{\FW}{{\mathfrak{W}}}
\newcommand{\FX}{{\mathfrak{X}}}
\newcommand{\FY}{{\mathfrak{Y}}}
\newcommand{\FZ}{{\mathfrak{Z}}}

\newcommand{\CA}{{\cal A}}
\newcommand{\CB}{{\cal B}}
\newcommand{\CC}{{\cal C}}
\newcommand{\CD}{{\cal D}}
\newcommand{\CE}{{\cal E}}
\newcommand{\CF}{{\cal F}}
\newcommand{\CG}{{\cal G}}
\newcommand{\CH}{{\cal H}}
\newcommand{\CI}{{\cal I}}
\newcommand{\CJ}{{\cal J}}
\newcommand{\CK}{{\cal K}}
\newcommand{\CL}{{\cal L}}
\newcommand{\CM}{{\cal M}}
\newcommand{\CN}{{\cal N}}
\newcommand{\CO}{{\cal O}}
\newcommand{\CP}{{\cal P}}
\newcommand{\CQ}{{\cal Q}}
\newcommand{\CR}{{\cal R}}
\newcommand{\CS}{{\cal S}}
\newcommand{\CT}{{\cal T}}
\newcommand{\CU}{{\cal U}}
\newcommand{\CV}{{\cal V}}
\newcommand{\CW}{{\cal W}}
\newcommand{\CX}{{\cal X}}
\newcommand{\CY}{{\cal Y}}
\newcommand{\CZ}{{\cal Z}}

% Theorem Stil

\theoremstyle{plain}
\newtheorem{lem}{Lemma}
\newtheorem{Satz}[lem]{Satz}

\theoremstyle{definition}
\newtheorem{defn}{Definition}[section]

\theoremstyle{remark}
\newtheorem{bem}{Bemerkung}    %[section]


\newcommand{\ord}{\mathop{\rm ord}\nolimits}
\renewcommand{\mod}{\mathop{\rm mod}\nolimits}
\newcommand{\sign}{\mathop{\rm sign}\nolimits}
\newcommand{\ggT}{\mathop{\rm ggT}\nolimits}
\newcommand{\kgV}{\mathop{\rm kgV}\nolimits}
\renewcommand{\div}{\, | \,}
\newcommand{\notdiv}{\mathopen{\mathchoice
             {\not{|}\,}
             {\not{|}\,}
             {\!\not{\:|}}
             {\not{|}}
             }}



\renewcommand{\Re}{\mathop{\rm Re}\nolimits}
\renewcommand{\Im}{\mathop{\rm Im}\nolimits}
\newcommand{\eck}[1]{\langle #1 \rangle}
\newcommand{\wi}{\angle}

\renewcommand{\arraystretch}{1.5}

\begin{document}

\pagestyle{empty}

\begin{center}
{\huge SMO Finalrunde 2007} \\
\medskip erste Pr�fung - 23. M�rz 2007
\end{center}
\vspace{8mm}
Zeit: 4 Stunden\\
Jede Aufgabe ist 7 Punkte wert.

\vspace{15mm}

\begin{enumerate}
\item[\textbf{1.}] Bestimme alle positiven reellen L�sungen des folgenden Gleichungssystems:
\vspace{-0.2cm}
\begin{center}
\begin{tabular}{cp{1cm}cp{1cm}c}
$a=\max\{\frac{1}{b},\frac{1}{c}\}$ & & $b=\max\{\frac{1}{c},\frac{1}{d}\}$ & & $c=\max\{\frac{1}{d},\frac{1}{e}\}$\\
$d=\max\{\frac{1}{e},\frac{1}{f}\}$ & & $e=\max\{\frac{1}{f},\frac{1}{a}\}$ & & $f=\max\{\frac{1}{a},\frac{1}{b}\}$\\
\end{tabular}
\end{center}

\bigskip

\item[\textbf{2.}] Seien $a,b,c$ drei ganze Zahlen, sodass $a+b+c$ durch 13 teilbar ist. Zeige, dass auch 
\[a^{2007}+b^{2007}+c^{2007}+2\cdot 2007 abc\]
durch 13 teilbar ist.\\

\smallskip

\item[\textbf{3.}] Die Ebene wird in Einheitsquadrate unterteilt. Jedes Feld soll mit einer von $n$ Farben gef�rbt werden, sodass gilt: K�nnen vier Felder mit einem L-Tetromino bedeckt werden, dann haben diese Felder vier verschiedene Farben (das L-Tetromino darf gedreht und gespiegelt werden). Bestimme den kleinsten Wert von $n$, f�r den das m�glich ist.\\

\smallskip

\item[\textbf{4.}] Sei $ABC$ ein spitzwinkliges Dreieck mit $AB>AC$ und H�henschnittpunkt $H$. Sei $D$ der H�henfusspunkt von $A$ auf $BC$. Sei $E$ die Spiegelung von $C$ an $D$. Die Geraden $AE$ und $BH$ schneiden sich im Punkt $S$. Sei $N$ der Mittelpunkt von $AE$ und sei $M$ der Mittelpunkt von $BH$. Beweise, dass $MN$ senkrecht auf $DS$ steht.\\

\smallskip

\item[\textbf{5.}] Bestimme alle Funktionen $f:\BR_{\geq 0} \rightarrow \BR_{\geq 0}$ mit folgenden Eigenschaften:
\begin{enumerate}
\item[(a)] $f(1)=0$,
\item[(b)] $f(x)>0$ f�r alle $x>1$,
\item[(c)] F�r alle $x,y \geq 0$ mit $x+y>0$ gilt
\[f(xf(y))f(y)=f\left(\frac{xy}{x+y}\right).\]
\end{enumerate}

\end{enumerate}

\vspace{1.5cm}

\begin{center}
 Viel Gl�ck!
\end{center}

\pagebreak


\begin{center}
{\huge SMO Finalrunde 2007} \\
\medskip zweite Pr�fung - 24. M�rz 2007
\end{center}
\vspace{8mm}
Zeit: 4 Stunden \\
Jede Aufgabe ist 7 Punkte wert.

\vspace{15mm}

\begin{enumerate}
\item[\textbf{6.}] Drei gleich grosse Kreise $k_1,k_2,k_3$ schneiden sich nichttangential in einem Punkt $P$. Seien $A$ und $B$ die Mittelpunkte der Kreise $k_1$ und $k_2$. Sei $D$ bzw. $C$ der von $P$ verschiedene Schnittpunkt von $k_3$ mit $k_1$ bzw. $k_2$. Zeige, dass $ABCD$ ein Parallelogramm ist.\\

\medskip

\item[\textbf{7.}] Seien $a,b,c$ nichtnegative reelle Zahlen mit arithmetischem Mittel $m=\frac{a+b+c}{3}$. Beweise, dass gilt
\[\sqrt{a+\sqrt{b+\sqrt{c}}}+\sqrt{b+\sqrt{c+\sqrt{a}}}+\sqrt{c+\sqrt{a+\sqrt{b}}}\leq 3\;\sqrt{m+\sqrt{m+\sqrt{m}}}.\]

\medskip

\item[\textbf{8.}] Sei $M \subset \{1,2,3,\ldots,2007\}$ eine Menge mit folgender Eigenschaft: Unter je drei Zahlen aus $M$ kann man stets zwei ausw�hlen, sodass die eine durch die andere teilbar ist. Wieviele Zahlen kann $M$ h�chstens enthalten?\\

\medskip

\item[\textbf{9.}] Finde alle Paare $(a,b)$ nat�rlicher Zahlen, sodass 
\[\frac{a^3+1}{2ab^2+1}\]
eine ganze Zahl ist.\\

\medskip

\item[\textbf{10.}] Die Ebene wird in gleichseitige Dreiecke der Seitenl�nge 1 unterteilt. Betrachte ein gleichseitiges Dreieck der Seitenl�nge $n$, dessen Seiten auf den Gitterlinien liegen. Auf jedem Gitterpunkt auf dem Rand und im Innern dieses Dreiecks liegt ein Stein. In einem Spielzug wird ein Einheitsdreieck ausgew�hlt, welches auf genau 2 Ecken mit einem Stein belegt ist. Die beiden Steine werden entfernt, und auf die dritte Ecke wird ein neuer Stein gelegt. F�r welche $n$ ist es m�glich, dass nach endlich vielen Spielz�gen nur noch ein Stein �brig bleibt?


\end{enumerate}
\vspace{1.5cm}

\center{Viel Gl�ck!}
\end{document}
