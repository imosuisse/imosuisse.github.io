\documentclass[12pt,a4paper]{article}

\usepackage{amsfonts}
\usepackage[centertags]{amsmath}
\usepackage{german}
\usepackage{amsthm}
\usepackage{amssymb}

\leftmargin=0pt \topmargin=0pt \headheight=0in \headsep=0in \oddsidemargin=0pt \textwidth=6.5in
\textheight=8.5in

\catcode`\� = \active \catcode`\� = \active \catcode`\� = \active \catcode`\� = \active \catcode`\� = \active
\catcode`\� = \active

\def�{"A}
\def�{"a}
\def�{"O}
\def�{"o}
\def�{"U}
\def�{"u}







% Schriftabk�rzungen

\newcommand{\eps}{\varepsilon}
\renewcommand{\phi}{\varphi}
\newcommand{\Sl}{\ell}    % sch�nes l
\newcommand{\ve}{\varepsilon}  %Epsilon

\newcommand{\BA}{{\mathbb{A}}}
\newcommand{\BB}{{\mathbb{B}}}
\newcommand{\BC}{{\mathbb{C}}}
\newcommand{\BD}{{\mathbb{D}}}
\newcommand{\BE}{{\mathbb{E}}}
\newcommand{\BF}{{\mathbb{F}}}
\newcommand{\BG}{{\mathbb{G}}}
\newcommand{\BH}{{\mathbb{H}}}
\newcommand{\BI}{{\mathbb{I}}}
\newcommand{\BJ}{{\mathbb{J}}}
\newcommand{\BK}{{\mathbb{K}}}
\newcommand{\BL}{{\mathbb{L}}}
\newcommand{\BM}{{\mathbb{M}}}
\newcommand{\BN}{{\mathbb{N}}}
\newcommand{\BO}{{\mathbb{O}}}
\newcommand{\BP}{{\mathbb{P}}}
\newcommand{\BQ}{{\mathbb{Q}}}
\newcommand{\BR}{{\mathbb{R}}}
\newcommand{\BS}{{\mathbb{S}}}
\newcommand{\BT}{{\mathbb{T}}}
\newcommand{\BU}{{\mathbb{U}}}
\newcommand{\BV}{{\mathbb{V}}}
\newcommand{\BW}{{\mathbb{W}}}
\newcommand{\BX}{{\mathbb{X}}}
\newcommand{\BY}{{\mathbb{Y}}}
\newcommand{\BZ}{{\mathbb{Z}}}

\newcommand{\Fa}{{\mathfrak{a}}}
\newcommand{\Fb}{{\mathfrak{b}}}
\newcommand{\Fc}{{\mathfrak{c}}}
\newcommand{\Fd}{{\mathfrak{d}}}
\newcommand{\Fe}{{\mathfrak{e}}}
\newcommand{\Ff}{{\mathfrak{f}}}
\newcommand{\Fg}{{\mathfrak{g}}}
\newcommand{\Fh}{{\mathfrak{h}}}
\newcommand{\Fi}{{\mathfrak{i}}}
\newcommand{\Fj}{{\mathfrak{j}}}
\newcommand{\Fk}{{\mathfrak{k}}}
\newcommand{\Fl}{{\mathfrak{l}}}
\newcommand{\Fm}{{\mathfrak{m}}}
\newcommand{\Fn}{{\mathfrak{n}}}
\newcommand{\Fo}{{\mathfrak{o}}}
\newcommand{\Fp}{{\mathfrak{p}}}
\newcommand{\Fq}{{\mathfrak{q}}}
\newcommand{\Fr}{{\mathfrak{r}}}
\newcommand{\Fs}{{\mathfrak{s}}}
\newcommand{\Ft}{{\mathfrak{t}}}
\newcommand{\Fu}{{\mathfrak{u}}}
\newcommand{\Fv}{{\mathfrak{v}}}
\newcommand{\Fw}{{\mathfrak{w}}}
\newcommand{\Fx}{{\mathfrak{x}}}
\newcommand{\Fy}{{\mathfrak{y}}}
\newcommand{\Fz}{{\mathfrak{z}}}

\newcommand{\FA}{{\mathfrak{A}}}
\newcommand{\FB}{{\mathfrak{B}}}
\newcommand{\FC}{{\mathfrak{C}}}
\newcommand{\FD}{{\mathfrak{D}}}
\newcommand{\FE}{{\mathfrak{E}}}
\newcommand{\FF}{{\mathfrak{F}}}
\newcommand{\FG}{{\mathfrak{G}}}
\newcommand{\FH}{{\mathfrak{H}}}
\newcommand{\FI}{{\mathfrak{I}}}
\newcommand{\FJ}{{\mathfrak{J}}}
\newcommand{\FK}{{\mathfrak{K}}}
\newcommand{\FL}{{\mathfrak{L}}}
\newcommand{\FM}{{\mathfrak{M}}}
\newcommand{\FN}{{\mathfrak{N}}}
\newcommand{\FO}{{\mathfrak{O}}}
\newcommand{\FP}{{\mathfrak{P}}}
\newcommand{\FQ}{{\mathfrak{Q}}}
\newcommand{\FR}{{\mathfrak{R}}}
\newcommand{\FS}{{\mathfrak{S}}}
\newcommand{\FT}{{\mathfrak{T}}}
\newcommand{\FU}{{\mathfrak{U}}}
\newcommand{\FV}{{\mathfrak{V}}}
\newcommand{\FW}{{\mathfrak{W}}}
\newcommand{\FX}{{\mathfrak{X}}}
\newcommand{\FY}{{\mathfrak{Y}}}
\newcommand{\FZ}{{\mathfrak{Z}}}

\newcommand{\CA}{{\cal A}}
\newcommand{\CB}{{\cal B}}
\newcommand{\CC}{{\cal C}}
\newcommand{\CD}{{\cal D}}
\newcommand{\CE}{{\cal E}}
\newcommand{\CF}{{\cal F}}
\newcommand{\CG}{{\cal G}}
\newcommand{\CH}{{\cal H}}
\newcommand{\CI}{{\cal I}}
\newcommand{\CJ}{{\cal J}}
\newcommand{\CK}{{\cal K}}
\newcommand{\CL}{{\cal L}}
\newcommand{\CM}{{\cal M}}
\newcommand{\CN}{{\cal N}}
\newcommand{\CO}{{\cal O}}
\newcommand{\CP}{{\cal P}}
\newcommand{\CQ}{{\cal Q}}
\newcommand{\CR}{{\cal R}}
\newcommand{\CS}{{\cal S}}
\newcommand{\CT}{{\cal T}}
\newcommand{\CU}{{\cal U}}
\newcommand{\CV}{{\cal V}}
\newcommand{\CW}{{\cal W}}
\newcommand{\CX}{{\cal X}}
\newcommand{\CY}{{\cal Y}}
\newcommand{\CZ}{{\cal Z}}

% Theorem Stil

\theoremstyle{plain}
\newtheorem{lem}{Lemma}
\newtheorem{Satz}[lem]{Satz}

\theoremstyle{definition}
\newtheorem{defn}{Definition}[section]

\theoremstyle{remark}
\newtheorem{bem}{Bemerkung}    %[section]



\newcommand{\card}{\mathop{\rm card}\nolimits}
\newcommand{\Sets}{((Sets))}
\newcommand{\id}{{\rm id}}
\newcommand{\supp}{\mathop{\rm Supp}\nolimits}

\newcommand{\ord}{\mathop{\rm ord}\nolimits}
\renewcommand{\mod}{\mathop{\rm mod}\nolimits}
\newcommand{\sign}{\mathop{\rm sign}\nolimits}
\newcommand{\ggT}{\mathop{\rm ggT}\nolimits}
\newcommand{\kgV}{\mathop{\rm kgV}\nolimits}
\renewcommand{\div}{\, | \,}
\newcommand{\notdiv}{\mathopen{\mathchoice
             {\not{|}\,}
             {\not{|}\,}
             {\!\not{\:|}}
             {\not{|}}
             }}

\newcommand{\im}{\mathop{{\rm Im}}\nolimits}
\newcommand{\coim}{\mathop{{\rm coim}}\nolimits}
\newcommand{\coker}{\mathop{\rm Coker}\nolimits}
\renewcommand{\ker}{\mathop{\rm Ker}\nolimits}

\newcommand{\pRang}{\mathop{p{\rm -Rang}}\nolimits}
\newcommand{\End}{\mathop{\rm End}\nolimits}
\newcommand{\Hom}{\mathop{\rm Hom}\nolimits}
\newcommand{\Isom}{\mathop{\rm Isom}\nolimits}
\newcommand{\Tor}{\mathop{\rm Tor}\nolimits}
\newcommand{\Aut}{\mathop{\rm Aut}\nolimits}

\newcommand{\adj}{\mathop{\rm adj}\nolimits}

\newcommand{\Norm}{\mathop{\rm Norm}\nolimits}
\newcommand{\Gal}{\mathop{\rm Gal}\nolimits}
\newcommand{\Frob}{{\rm Frob}}

\newcommand{\disc}{\mathop{\rm disc}\nolimits}

\renewcommand{\Re}{\mathop{\rm Re}\nolimits}
\renewcommand{\Im}{\mathop{\rm Im}\nolimits}

\newcommand{\Log}{\mathop{\rm Log}\nolimits}
\newcommand{\Res}{\mathop{\rm Res}\nolimits}
\newcommand{\Bild}{\mathop{\rm Bild}\nolimits}

\renewcommand{\binom}[2]{\left({#1}\atop{#2}\right)}
\newcommand{\eck}[1]{\langle #1 \rangle}
\newcommand{\gaussk}[1]{\lfloor #1 \rfloor}
\newcommand{\frack}[1]{\{ #1 \}}
\newcommand{\wi}{\hspace{1pt} < \hspace{-6pt} ) \hspace{2pt}}
\newcommand{\dreieck}{\bigtriangleup}

\parindent0mm

\begin{document}

\pagestyle{empty}

\begin{center}
{\huge IMO Selektion 2007} \\
\medskip erste Pr�fung - 5. Mai 2007
\end{center}
\vspace{8mm}
Zeit: 4.5 Stunden\\
Jede Aufgabe ist 7 Punkte wert.

\vspace{15mm}

\begin{enumerate}
\item[\textbf{1.}] Sei $ABCD$ ein Trapez mit $AB \parallel CD$ und $AB > CD$. Die Punkte $K$ und $L$ liegen auf den Seiten $AB$ bzw. $CD$ mit $AK/KB = DL/LC$. Die Punkte $P$ und $Q$ liegen so auf der Strecke $KL$, dass gilt
\[\angle APB = \angle BCD \qquad \mbox{und} \qquad \angle CQD = \angle ABC.\]
Zeige, dass die Punkte $P,Q,B$ und $C$ auf einem Kreis liegen.\\

\bigskip
\bigskip

\item[\textbf{2.}] Bestimme die beiden kleinsten nat�rlichen Zahlen, die sich in der Form $7m^2-11n^2$ mit nat�rlichen Zahlen $m$ und $n$ schreiben lassen.\\

\bigskip
\bigskip

\item[\textbf{3.}] Wir nennen zwei Personen ein \emph{befreundetes Paar}, wenn sie sich kennen, und wir nennen sie ein \emph{nichtbefreundetes Paar}, wenn sie sich nicht kennen (befreundet sein oder nicht befreundet sein ist dabei immer gegenseitig). Seien $m,n$ nat�rliche Zahlen. Finde die kleinste nat�rliche Zahl $k$, sodass Folgendes gilt: In jeder Gruppe von $k$ Leuten gibt es stets $2m$ Leute, die $m$ disjunkte befreundete Paare bilden, oder es gibt $2n$ Leute, die $n$ disjunkte nichtbefreundete Paare bilden.

\end{enumerate}

\pagebreak


\begin{center}
{\huge IMO Selektion 2007} \\
\medskip zweite Pr�fung - 6. Mai 2007
\end{center}
\vspace{8mm}
Zeit: 4.5 Stunden\\
Jede Aufgabe ist 7 Punkte wert.

\vspace{15mm}

\begin{enumerate}

\item[\textbf{4.}] Ein Paar $(r,s)$ nat�rlicher Zahlen heisst \emph{gut}, falls ein Polynom $P$ mit ganzen Koeffizienten und paarweise verschiedene ganze Zahlen $a_1,\ldots,a_r$ und $b_1,\ldots,b_s$ existieren, sodass gilt
\[P(a_1)=P(a_2)=\ldots =P(a_r)=2 \qquad \mbox{und} \qquad P(b_1)=P(b_2)=\ldots =P(b_s)=5.\]
\begin{enumerate}
\item[(a)] Zeige, dass f�r jedes gute Paar $(r,s)$ nat�rlicher Zahlen $r,s \leq 3$ gilt.
\item[(b)] Bestimme alle guten Paare.
\end{enumerate}

\bigskip
\bigskip

\item[\textbf{5.}] Seien $n>1$ und $m$ nat�rliche Zahlen. Ein Parlament besteht aus $mn$ Abgeordneten, die $2n$ Kommissionen gebildet haben, sodass gilt:
\begin{enumerate}
\item[(i)] Jede Kommission besteht aus $m$ Abgeordneten.
\item[(ii)] Jeder Abgeordnete ist Mitglied in genau 2 Kommissionen.
\item[(iii)] Je zwei Kommissionen haben h�chstens ein gemeinsames Mitglied.
\end{enumerate}
Bestimme in Abh�ngigkeit von $n$ den gr�sstm�glichen Wert von $m$, sodass dies m�glich ist.\\

\bigskip
\bigskip

\item[\textbf{6.}] Seien $a,b,c$ positive reelle Zahlen mit $a+b+c \geq abc$. Beweise, dass
von den folgenden drei Ungleichungen mindestens zwei richtig sind:
\[\frac{2}{a}+\frac{3}{b}+\frac{6}{c} \geq 6, \qquad \frac{2}{b}+\frac{3}{c}+\frac{6}{a} \geq 6,
\qquad \frac{2}{c}+\frac{3}{a}+\frac{6}{b} \geq 6.\]

\end{enumerate}


\pagebreak


\begin{center}
{\huge IMO Selektion 2007} \\
\medskip dritte Pr�fung - 19. Mai 2007
\end{center}
\vspace{8mm}
Zeit: 4.5 Stunden\\
Jede Aufgabe ist 7 Punkte wert.

\vspace{15mm}

\begin{enumerate}
\item[\textbf{7.}] Sei $a_1,a_2,\ldots, a_{2007}$ eine Folge, die jede der Zahlen $1,2,\ldots,2007$ genau einmal enth�lt. Es wird nun wiederholt folgende Operation ausgef�hrt: Ist das erste Folgeglied gleich $n$, dann wird die Reihenfolge der ersten $n$ Folgeglieder umgekehrt. Zeige, dass die Folge nach endlich vielen solchen Operation mit der Zahl $1$ beginnt.\\

\bigskip
\bigskip

\item[\textbf{8.}] Sei $ABCDE$ ein konvexes F�nfeck mit
\[\angle BAC = \angle CAD = \angle DAE \qquad \mbox{und} \qquad \angle ABC = \angle ACD = \angle ADE.\]
Die Diagonalen $BD$ und $CE$ treffen sich in $P$. Zeige, dass die Gerade $AP$ die Seite $CD$ in deren Mittelpunkt schneidet.\\

\bigskip
\bigskip

\item[\textbf{9.}] Bestimme alle nat�rlichen Zahlen $n$, f�r die genau eine ganze Zahl $a$ mit $0<a<n!$ existiert, sodass gilt
\[n! \div a^n+1.\]

\end{enumerate}


\pagebreak


\begin{center}
{\huge IMO Selektion 2007} \\
\medskip vierte Pr�fung - 20. Mai 2007
\end{center}
\vspace{8mm}
Zeit: 4.5 Stunden\\
Jede Aufgabe ist 7 Punkte wert.

\vspace{15mm}

\begin{enumerate}
\item[\textbf{10.}] F�r eine nat�rliche Zahl $n$ sei
\[f(n)=\frac{1}{n} \sum_{k=1}^n \left\lfloor \frac{n}{k} \right\rfloor.\]
Beweise, dass es unendlich viele nat�rliche Zahlen $m$ gibt, f�r die die Ungleichung $f(m)<f(m+1)$ gilt, und dass es unendlich viele nat�rlichen Zahlen $m$ gibt, f�r die die Ungleichung $f(m)>f(m+1)$ gilt.\\

\bigskip
\bigskip

\item[\textbf{11.}] Finde alle Funktionen $f:\BR^+ \rightarrow \BR^+$, sodass f�r alle $x,y>0$ gilt
\[f(x^y)=f(x)^{f(y)}.\]

\bigskip
\bigskip

\item[\textbf{12.}] Im Dreieck $ABC$ sei $J$ der Mittelpunkt des Ankreises, welcher die Seite $BC$ in $A_1$ und die Verl�ngerungen der Seiten $AC$ und $AB$ in $B_1$ bzw. $C_1$ ber�hrt. Die Gerade $A_1B_1$ schneide die Gerade $AB$ rechtwinklig in $D$. Sei $E$ die Projektion von $C_1$ auf die Gerade $DJ$. Bestimme die Gr�sse der Winkel $\angle BEA_1$ und $\angle AEB_1$.


\end{enumerate}


\end{document}
