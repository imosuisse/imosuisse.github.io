\de{Sei $k$ ein Kreis mit Mittelpunkt $O$. Seien $A$, $B$, $C$ und $D$ vier unterschiedliche Punkte auf dem Kreis $k$ in dieser Reihenfolge, sodass $AB$ ein Durchmesser von $k$ ist. Der Umkreis des Dreiecks $COD$ schneide $AC$ zum zweiten mal in $P$. Zeige, dass $OP$ und $BD$ parallel sind.

\textbf{Lösung 1 (Louis), Winkeljagd:} 
Wir zeigen, dass $\angle ODB=\angle POD$. Dies beweist, dass die Linien parallel sind.

Da $O$ der Mittelpunkt des Kreises ist, haben wir $OD=OB$ und somit ist das Dreieck $ODB$ gleichschenklig an $O$. Somit ist $\angle ODB = \angle OBD$. Weil $O$ auf der Strecke $AB$ liegt, wissen wir $\angle OBD=\angle ABD$. Die Punkte $A,B,C,D$ sind auf einem Kreis, also $\angle ABD=\angle ACD=\angle PCD$. Zum Schluss sind die Punkte $C,O,D,P$ alle auf einem Kreis und $\angle PCD=\angle POD$. Also haben wir
\[
\angle ODB=\angle POD.
\]
und wir sind fertig.

\textbf{Marking Scheme}

Für unvollständige Lösungen sind folgende Punkte möglich. ($\leq$ 4P):
\begin{enumerate}
    \item Working Forward Punkte:
    
    \textbf{Bemerkung:} Für die folgenden Winkelgleichungen sind Markierungen auf der Skizze \textbf{nicht} ausreichend. Die Beziehungen sollten anderswo explizit genannt werden.
    \begin{itemize}
        \item +1P: Eine Winkelbeziehung im Sehnenviereck $ABCD$ \textbf{oder} $CPDO$ finden (zB. $\angle PCD=\angle POD$ oder $\angle DBA=\angle ACD$)
        \item +1P: Eine Winkelbeziehung zwischen den beiden Sehnenvierecken $ABCD$ \textbf{und} $CPDO$ finden (zB. $\angle PCD=\angle DBA$)
        \item +1P: Eine Winkelbeziehung finden, welche benutzt, dass $O$ der Mittelpunkt des Kreises um $ABCD$ ist (zB. $\angle ODB=\angle OBD$)
    \end{itemize}

    \item Working Backward Punkte:
    \begin{itemize}
        \item +1P: Die gewünschte Bedingung mit Winkeln ausdrücken  (zB. $\angle POD=\angle ODB$ oder $\angle DBA=\angle POA$)
    \end{itemize}
\end{enumerate}
}

\fr{Soit $k$ un cercle de centre $O$. Soient $A$, $B$, $C$ et $D$ quatre points distincts sur $k$, dans cet ordre, tels que $AB$ est un diamètre de $k$. Le cercle circonscrit au triangle $COD$ intersecte $AC$ une deuxième fois en $P$. Montrer que $OP$ et $BD$ sont parallèles.

\textbf{Solution 1 (Louis), Chasse aux angles:}

On va montrer que $\angle ODB=\angle POD$. Cela prouve que les droites sont parallèles.

Comme $O$ est le centre du cercle, on a $OD=OB$ et donc le triangle $ODB$ est isocèle en $O$. Donc $\angle ODB = \angle OBD$. Comme $O$
se trouve sur le segment $AB$, $\angle OBD=\angle ABD$. Les points $A,B,C,D$ étant sur un même cercle, on a $\angle ABD=180^\circ-\angle ACD=\angle PCD$. Finalement, les points $C,O,D,P$ se trouvent sur un même cercle et donc $\angle PCD=\angle POD$. On a donc bien
\[
\angle ODB=\angle POD.
\]

\textbf{Marking Scheme}

Pour les solutions partielles, on attribuera les points additifs suivants ($\leq$ 4P):
\begin{enumerate}
    \item Forward points:
    
    \textbf{Remarque:} Pour les relations d'angles qui suivent, des angles bien marqués sur un dessin propre \textbf{ne sont pas} suffisants. Les relations doivent se trouver explicitement ailleurs.
    \begin{itemize}
        \item +1P: établir une relation d'angles dans les quadrilatères inscrits $ABCD$ \textbf{ou} $CPDO$ (eg. $\angle PCD=\angle POD$ ou $\angle DBA=\angle ACD$)
        \item +1P: établir une relation d'angles en combinant les quadrilatères inscrits $ABCD$ \textbf{et} $CPDO$ (eg. $\angle PCD=\angle DBA$)
        \item +1P: établir une relation d'angle en utilisant que $O$ est le centre du cercle $ABCD$ (eg. $\angle ODB=\angle OBD$)
    \end{itemize}

    \item Backward points:
    \begin{itemize}
        \item +1P: reformuler la conclusion en termes d'angles (eg. $\angle POD=\angle ODB$ ou $\angle DBA=\angle POA$)
    \end{itemize}
\end{enumerate}
}

\ita{}

\en{
Let $k$ be a circle with center $O$. Let $A$, $B$, $C$ and $D$ be four distinct points on $k$ in this order such that $AB$ is a diameter of $k$. The circumcircle of the triangle $COD$ intersects $AC$ again in $P$. Show that $OP$ and $BD$ are parallel.

\textbf{Solution 1 (Louis), Angle chasing:} 
We will show that $\angle ODB=\angle POD$, which proves that the lines are parallel.

As $O$ is the centre of the circle, we have that $OD=OB$ and so the triangle $ODB$ is isoceles in $O$. So $\angle ODB = \angle OBD$. As $O$ lies on the segment $AB$, $\angle OBD=\angle ABD$. The points $A,B,C,D$ are concyclic so we also have $\angle ABD=\angle ACD=\angle PCD$. Finally, the points $C,O,D,P$ are all on a circle and so $\angle PCD=\angle POD$. Therefore, we have
\[
\angle ODB=\angle POD.
\]
and we are done.

\textbf{Marking Scheme}

For partial solutions, the following points are available ($\leq$ 4P):
\begin{enumerate}
    \item Forward points:
    
    \textbf{Remark:} For the following angle equalities, angles marked on a drawing \textbf{are not} sufficient. The relation should be explicitly stated elsewhere.
    \begin{itemize}
        \item +1P: establish an angle equality in the cyclic quadrilaterals $ABCD$ \textbf{or} $CPDO$ (eg. $\angle PCD=\angle POD$ or $\angle DBA=\angle ACD$)
        \item +1P: establish an angle equality combining the two quadrilaterals $ABCD$ \textbf{and} $CPDO$ (eg. $\angle PCD=\angle DBA$)
        \item +1P: establish an angle qualitiy using that $O$ is the centre of the circle $ABCD$ (eg. $\angle ODB=\angle OBD$)
    \end{itemize}

    \item Backward points:
    \begin{itemize}
        \item +1P: reformulate the conclusion in terms of angles (eg. $\angle POD=\angle ODB$ or $\angle DBA=\angle POA$)
    \end{itemize}
\end{enumerate}
}


\newpage

\textbf{Alternative solutions}

\textbf{Solution 2 (Tanish), Nine-point circle:} 

Let $X$ be $AC \cap BD$ and $Q$ be the second intersection of $BD$ and $(COD)$. We will prove that $(COD)$ is the nine-point circle of $AXB$. But this is clear - $O$ is the midpoint of $AB$, $D$ is the base of the altitude dropped from $B$ (Thales) and $C$ is the base of the altitude dropped from $A$ (Thales). It follows that $P$ and $Q$ are the midpoints of $AX$ and $BX$ respectively, and $XPOQ$ is a parallelogram.

\textbf{Solution 3 (Tanish), Inversion:}

Let us invert about $k$. $AC$ is sent to the circumcircle of $OAC$; the circumcircle of $OAD$ is sent to $DC$. This means that $P'$ is the intersection of these two. $OP$ is sent to the line $OP'$ and $BC$ is sent to $BC$, so it suffices to prove that $OP' \parallel BD$. This is equivalent to $\angle AOP' = \angle ABD$. Since $\angle ABD = 2\angle AOD$, we simply have to prove $OP'$ bisects $\angle AOD$, or that (as $AO = OD$), $OP'$ bisects $\angle AP'D.$ However this last statement is clearly true when you consider the circumcircle of $AODP'$ as $\angle OP'A$ subtends the arc $\overset{\frown}{OA}$ and $\angle OP'D$ subtends the arc $\overset{\frown}{OC}$, which are of equal length as $OA = OC$. \newline \newline
\emph{Note}: $OP$ is the same line as $OP'$ and therefore you could equally prove that $OP$ bisects $AOD$ or $APD$, but this is not as trivial.