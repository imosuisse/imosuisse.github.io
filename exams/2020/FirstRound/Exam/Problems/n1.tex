\de{
Falls $p\geq 5$ eine Primzahl ist, sei $q$ die kleinste Primzahl sodass $q>p$ und sei $n$ die Anzahl der positiven Teiler von $p+q$ ($1$ und $p+q$ inklusive).
\begin{enumerate}[a)]
    \item Zeige, dass egal welche Primzahl $p$ gewählt wurde, die Zahl $n$ grösser oder gleich $4$ ist.
    \item Finde den kleinstmöglichen Wert $m$, den $n$ annehmen kann unter allen möglichen Wahlen von $p$. Das heisst:
    \begin{itemize}
        \item Gib ein Beispiel für eine Primzahl $p$ an, sodass der Wert $m$ erreicht wird.
        \item Zeige, dass es keine Primzahl $p$ gibt für die der Wert von $n$ kleiner als $m$ ist.
    \end{itemize}
\end{enumerate}
}

\fr{
Si $p\geq 5$ est un nombre premier, soit $q$ le plus petit nombre premier tel que $q>p$ et soit $n$ le nombre de diviseurs positifs de $p+q$ ($1$ et $p+q$ inclus).
\begin{enumerate}[a)]
    \item Montrer que quel que soit le choix de $p$, le nombre $n$ est toujours plus grand ou égal à $4$.
    \item Trouver la véritable valeur minimale $m$ que peut prendre $n$ parmi tous les choix possibles pour $p$. C'est-à-dire:
    \begin{itemize}
        \item Donner un exemple d'un nombre premier $p$ pour lequel la valeur $m$ est atteinte.
        \item Montrer qu'il n'existe pas de nombre premier $p$ tel que $n$ soit strictement inférieur a $m$.
    \end{itemize}
\end{enumerate}
}

\ita{Se $p\geq 5$ \`e un numero primo, sia $q$ il piu piccolo primo tale che $q>p$, sia inoltre $n$ il numero di divisori positivi di $p+q$ (inclusi i divisori $1$ e $q+p$).

\begin{enumerate}[a)]
    \item Mostrare che, indipendentemente da quale $p$ si sceglie, vale $n\geq 4$.
    \item Trovare il piu piccolo valore $m$ che $n$ pu\`o assumere tra tutte le possibile scelte di $p$. Cio\`e:
    \begin{itemize}
        \item Trovare un $p$ per il quale il valore di $m$ \`e raggiunto.
        \item Mostrare che non esistono primi $p$ per cui $n$  \`e strettamente minore di $m$.
    \end{itemize}
\end{enumerate}

}

\en{If $p\geq 5$ is a prime number, let $q$ denote the smallest prime number such that $q>p$ and let $n$ be the number of positive divisors of $p+q$ ($1$ and $p+q$ included).
\begin{enumerate}[a)]
    \item Prove that no matter the choice of $p$, the number $n$ is always at least $4$.
    \item Find the actual minimal value $m$ that $n$ can reach among all possible choices for $p$. That is:
    \begin{itemize}
        \item Give an example of a prime number $p$ for which the value $m$ is reached.
        \item Prove that there is no prime number $p$ for which $n$ is smaller than $m$.
    \end{itemize}
\end{enumerate}
}