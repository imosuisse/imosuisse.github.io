\en{Let $ABC$ be a triangle with circumcircle $k$. Let $A_1,B_1$ and $C_1$ be points on the interior of the sides $BC,CA$ and $AB$ respectively. Let $X$ be a point on $k$ and denote by $Y$ the second intersection of the circumcircles of $BC_1X$ and $CB_1X$. Define the points $P$ and $Q$ to be the intersections of $BY$ with $B_1A_1$ and $CY$ with $C_1A_1$, respectively. Prove that $A$ lies on the line $PQ$.}

\de{Sei $ABC$ eine Dreieck mit Umkreis $k$. Seien $A_1,B_1$ und $C_1$ Punkte auf den jeweiligen Seiten $BC,CA$ und $AB$. Sei $X$ ein Punkt auf $k$ und sei $Y$ der zweite Schnittpunkt der Umkreise von $BC_1X$ und $CB_1X$. Definiere $P$ und $Q$ als die Schnittpunkte von $BY$ mit $B_1A_1$, beziehungsweise von $CY$ mit $C_1A_1$. Zeige, dass $A$ auf der Geraden $PQ$ liegt.}

\fr{Soit $ABC$ un triangle et soit $k$ son cercle circonscrit. Soient $A_1, B_1$ et $C_1$ des points à l'intérieur des côtés $BC, CA$ et $AB$ respectivement. Soit $X$ un point sur $k$ et soit $Y$ la seconde intersection des cercles circonscrits à $BC_1X$ et $CB_1X$. Les points $P$ et $Q$ sont définis comme l'intersection de $BY$ avec $B_1A_1$ et de $CY$ avec $C_1A_1$ respectivement. Montrer que la droite $PQ$ passe par $A$.}