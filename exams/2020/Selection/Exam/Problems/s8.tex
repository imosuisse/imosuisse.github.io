\en{Let $I$ be the incenter of a non-isosceles triangle $ABC$. Let $F$ be the intersection of the perpendicular to $AI$ through $I$ with $BC$. Let $M$ be the point on the circumcircle of $ABC$ such that $MB=MC$ and such that $M$ is on the same side of the line $BC$ as $A$. Let $N$ be the second intersection of the line $MI$ with the circumcircle of $BIC$. Prove that $FN$ is tangent to the circumcircle of $BIC$.}

\de{Sei $I$ der Inkreismittelpunkt eines nicht gleichschenkligen Dreiecks $ABC$. Sei $F$ der Schnittpunkt der Senkrechten auf $AI$ durch $I$ mit der Gerade $BC$. Sei $M$ der Punkt auf dem Umkreis von $ABC$, sodass $MB=MC$ gilt und sich $M$ auf der gleichen Seite der Gerade $BC$ befindet wie $A$. Sei $N$ der zweite Schnittpunkt der Geraden $MI$ mit dem Umkreis des Dreiecks $BIC$. Zeige, dass $FN$ eine Tangente an den Umkreis von $BIC$ ist.}

\fr{Soit $I$ le centre du cercle inscrit d'un triangle non-isocèle $ABC$. Soit $F$ l'intersection de la perpendiculaire à $AI$ passant par $I$ avec $BC$. Soit $M$ le point sur le cercle circonscrit à $ABC$ tel que $MB = MC$ et tel que $M$ est du même côté de la droite $BC$ que $A$. Soit $N$ la seconde intersection de la droite $MI$ avec le cercle circonscrit à $BIC$. Montrer que la droite $FN$ est tangente au cercle circonscrit au triangle $BIC$.}
%Soit $ABC$ un triangle non-isocèle et soit $I$ le centre de son cercle inscrit...