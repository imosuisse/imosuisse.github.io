\en{Let $n\geq 2$ be an integer. Consider an $n\times n$ chessboard with the usual chessboard colouring. A move consists of choosing a $1\times 1$ square and switching the colour of all squares in its row and column (including the chosen square itself). For which $n$ is it possible to get a monochrome chessboard after a finite sequence of moves?
}
\de{Sei $n \geq 2$ eine ganze Zahl. Betrachte ein $n \times n$ Schachbrett mir der normalen Schachbrettfärbung. Ein Zug besteht darin, ein $1\times 1$ Feld auszuwählen und die Farbe von allen Feldern in derselben Reihe und derselben Spalte zu wechseln (auch die Farbe des ausgwählten Feldes). Für welche $n$ ist es möglich, ein einfarbiges Schachbrett nach einer endlichen Anzahl Züge zu erhalten? }

\fr{Soit $n \geq 2$ un nombre entier. Sur un échiquier $n \times n$ avec la coloration standard, on considère l'opération suivante: on choisit tout d'abord une case et on intervertit ensuite la couleur de toutes les cases dans sa ligne et sa colonne (la case choisie incluse). Pour quels nombres entiers $n$ est-il possible d'obtenir un échiquier monochrome après un nombre fini d'opérations?}