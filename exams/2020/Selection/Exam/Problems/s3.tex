\de{Sei $k$ ein Kreis mit Mittelpunkt $O$. Sei $AB$ eine Sehne dieses Kreises mit Mittelpunkt $M \neq O$. Die Tangenten von $k$ an $A$ und $B$ schneiden sich in $T$. Die Gerade $l$ geht durch $T$ und schneidet $k$ in $C$ und $D$, mit $CT < DT$ und $BC = BM$.

Beweise, dass der Umkreismittelpunkt des Dreiecks $ADM$ die Spiegelung von $O$ an der Geraden $AD$ ist.}

\en{Let $k$ be a circle with centre $O$. Let $AB$ be a chord of this circle with midpoint $M \neq O$. The tangents of $k$ at the points $A$ and $B$ intersect at $T$. A line goes through $T$ and intersects $k$ in $C$ and $D$ with $CT < DT$ and $BC = BM$. Prove that the circumcentre of the triangle $ADM$ is the reflection of $O$ across the line $AD$.}

\fr{Soit $k$ un cercle de centre $O$. Soit $AB$ une corde de ce cercle dont le milieu est $M \neq O$. Les tangentes à $k$ aux points $A$ et $B$ se coupent en $T$. Une droite passant par $T$ intersecte $k$ en $C$ et $D$ de telle manière que $CT < DT$ et $BC=BM$. Montrer que le centre du cercle circonscrit au triangle $ADM$ est la réflection de $O$ par rapport à la droite $AD$.} 
