\fr{Soit $ABC$ un triangle tel que $\angle CAB=90^{\circ}$. Soient $D,E$ des points sur $AC$ et $AB$, respectivement, tels que $BCDE$ est un quadrilatère inscrit. Soient $\Omega_1,\Omega_2$ les cercles qui passent par $A$ et qui sont centrés en $E,D$, respectivement. Soit $P$ le deuxième point d'intersection de $\Omega_1$ et $\Omega_2$. Montrer que la droite $AP$ coupe $BC$ en deux.}
\en{
Let $ABC$ be a triangle with $\angle CAB = 90^{\circ}$. Let $D,E$ be points on $AC,AB$ respectively such that $BCDE$ is cyclic. Let $\Omega_1,\Omega_2$ be the circles through $A$ with centres $E,D$ respectively. Denote by $P$ the second intersection of $\Omega_1$ and $\Omega_2$. Prove that the line $AP$ bisects the side $BC$.
}

\de{
Sei $ABC$ ein Dreieck mit $\angle CAB = 90^{\circ}$. Seien $D,E$ zwei Punkte jeweils auf $AC, AB$ sodass $BCDE$ ein Sehnenviereck ist. Seien $\Omega_1,\Omega_2$ die Kreise durch $A$ mit jeweiligen Mittelpunkten $E,D$. Sei $P$ der zweite Schnittpunkt von $\Omega_1$ und $\Omega_2$. Beweise, dass die gerade $AP$ die Strecke $BC$ halbiert.
}