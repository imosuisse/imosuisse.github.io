Let $\varphi$ denote the Euler phi-function. Prove that for every positive integer $n$
$$2^{n(n+1)}\mid 32\cdot\varphi\left(2^{2^n}-1\right).$$
\textbf{Solution:} (Valentin)
We induct on $n$. The cases $n = 1,2,3$ can easily be checked by hand:
\begin{enumerate}[label=$\bullet$]
\item $n = 1$: $2^2\ \mid 32\cdot2$.
\item $n = 2$: $2^6\ \mid 2^5\cdot\varphi(15) = 2^5\cdot2\cdot2^3$
\item $n = 3$: $2^{12}\mid 2^5\cdot\varphi(255) = 2^5\cdot2\cdot2^2\cdot2^4=2^{12}$
\end{enumerate}

For $n \geq 4$ assume we the statement is true for all $1\leq k < n$ and note that
$$\varphi\left(2^{2^n}-1\right)=\varphi\bigg(\left(2^{2^{n-1}}-1\right)\left(2^{2^{n-1}}+1\right)\bigg) = \varphi\left(2^{2^{n-1}}-1\right)\cdot\varphi\left(2^{2^{n-1}}+1\right)$$
since $\gcd(2^{2^{n-1}}-1,2^{2^{n-1}}+1) = 1$. But from our inductive assumption we know that $$2^{(n-1)n}\mid32\cdot\varphi\left(2^{2^{n-1}}-1\right)$$
so all that is left to prove is that
$$2^{2n}\mid\varphi\left(2^{2^{n-1}}+1\right).$$
Take now any prime $p$ that divides $2^{2^{n-1}}+1$ and let $d$ be the order of $2$ mod $p$. We know that
$$2^{2^{n-1}} \equiv -1 \text{ mod }p, \text{ squaring gives } 2^{2^n} \equiv 1 \text{ mod }p.$$
By the properties of the order we therefore have
$$d\mid 2^n \text{ but } d\nmid 2^{n-1}.$$
This implies $d = 2^n$ and since we also have $d\mid p-1$ we get
$$2^n\mid p-1 \text{ and therefore } p \equiv 1 \text{ mod } 2^n.$$
If we are also able to prove that $2^{2^{n-1}}+1$ contains at least two different prime factors we would be done. This is because if $p,q$ are two different such primes we can write
$$2^{2^{n-1}}+1 = p^x\cdot q^y\cdot N$$
with $N$ a positive integer and $p,q\nmid N$. Then
$$\varphi\left(2^{2^{n-1}}+1\right)=(p-1)(q-1)\cdot p^{x-1}q^{x-1}\varphi(N) \equiv 0 \text{ mod } 2^{2n}.$$

Assume now that $2^{2^{n-1}}+1$ is instead a prime power, say $p^x$. It follows that
$$(p-1)\left(p^{x-1}+p^{x-2}\ldots+p+1\right) = p^x-1 = 2^{2^{n-1}}$$
It follows that $x$ is odd since squares are $\equiv 0$ or $1$ mod $4$ and using the fact that $p\equiv 1$ mod $2^n$ we find
$$p^{x-1}+p^{x-2}\ldots+p+1 \equiv x \text{ mod } 2^n$$
implying $x = 1$. But then $2^{2^{n-1}}+1$ is a prime and
$$\varphi\left(2^{2^{n-1}}+1\right) = 2^{2^{n-1}}$$
and since $n\geq 4$ we have $2^{n-1}\geq 2n$ and we are done in this case as well.\qed

\pagebreak

\textbf{Remarks:}
The fact that $p\mid 2^{^{n-1}}+1 \Longrightarrow p \equiv 1 \text{ mod } 2^n$ for $n\geq 2$ can also be shown using induction or just quoted as a lemma. The contradiction in the case where $2^{2^{n-1}}+1$ is a prime power also follows from Catalan.

\textbf{Marking Scheme:}
\begin{itemize}
    \item 1P: Using induction, reduction or factorisation to reduce the problem to something like $2^{2n}\mid\varphi\left(2^{2^{n-1}}+1\right)$ for large enough $n$. This includes the proof that $\gcd\left(2^{2^i}+1,2^{2^j}+1\right)=1$ for $i\neq j$.
    \item 1P: Considering the order of $2$ mod any factor of $2^{2^{n-1}}+1$.
    \item 1P: Proving that $2^n\mid p-1$.
    \item 1P: Having the idea of distinguishing the case where $2^{2^{n-1}}+1$ is a prime power and the case where it is divisible by two different primes.
    \item 1P: Finishing the case where $2^{2^{n-1}}+1$ is divisible by two different primes.
    \item 1P: Proving that $2^{2^{n-1}}+1$ is not a a prime power for large enough $n$.
    \item 1P: Treating the case where $2^{2^{n-1}}+1$ is prime including the small edge cases and concluding the Proof.
    \item -1P: If the small edge cases for $n$ were missed.
    \item -1P: If any edge cases of stated theorems were missed.
\end{itemize}