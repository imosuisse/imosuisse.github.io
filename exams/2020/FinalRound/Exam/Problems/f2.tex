\de{
Sei $ABC$ ein spitzwinkliges Dreieck. Seien $M_A$, $M_B$ und $M_C$ die Mittelpunkte der Seiten $BC$, $CA$, respektive $AB$. Seien $M_A'$, $M_B'$ und $M_C'$ die Mittelpunkte der Kreisbögen über den jeweiligen Seiten $BC$, $CA$ und $AB$ auf dem Umkreis von $ABC$. Sei $P_A$ der Schnittpunkt der Gerade $M_BM_C$ und der Senkrechten zu $M_B'M_C'$ durch $A$. Definiere $P_B$ und $P_C$ analog. Zeige, dass die Geraden $M_AP_A$, $M_BP_B$ und $M_CP_C$ sich in einem Punkt schneiden.}
\fr {Soit $ABC$ un triangle aigu. Soient $M_A$, $M_B$ et $M_C$ les milieux respectifs des côtés $BC$, $CA$ et $AB$. Soient $M_A',M_B'$ et $M_C'$ les milieux respectifs des arcs mineurs $BC$, $CA$ et $AB$ sur le cercle circonscrit au triangle $ABC$. Soit $P_A$ l'intersection de la droite $M_BM_C$ et de la perpendiculaire à $M_B'M_C'$ par $A$. Les points $P_B$ et $P_C$ sont définis de manière analogue. Montrer que les droites $M_AP_A$, $M_BP_B$ et $M_CP_C$ se coupent en un point.}
\ita{}
\en{
Let $ABC$ be an acute triangle. Denote by $M_A$, $M_B$ and $M_C$ the midpoints of sides $BC$, $CA$ and $AB$, respectively. Let $M_A'$, $M_B'$ and $M_C'$ be respectively the midpoints of the minor arcs $BC$, $CA$ and $AB$ on the circumcircle of $ABC$. Let $P_A$ be the intersection of the lines $M_BM_C$ and the perpendicular to $M_B'M_C'$ containing $A$. Let $P_B$ and $P_C$ be defined analogously. Prove that the lines $M_AP_A$, $M_BP_B$ and $M_CP_C$ meet at a point.}