
\documentclass[12pt,a4paper]{article}

\usepackage{amsfonts}
\usepackage[centertags]{amsmath}
\usepackage{german}
\usepackage{amsthm}
\usepackage{amssymb}

\leftmargin=0pt \topmargin=0pt \headheight=0in \headsep=0in \oddsidemargin=0pt \textwidth=6.5in
\textheight=8.5in

\catcode`\� = \active \catcode`\� = \active \catcode`\� = \active \catcode`\� = \active \catcode`\� = \active
\catcode`\� = \active

\def�{"A}
\def�{"a}
\def�{"O}
\def�{"o}
\def�{"U}
\def�{"u}







% Schriftabk�rzungen

\newcommand{\eps}{\varepsilon}
\renewcommand{\phi}{\varphi}
\newcommand{\Sl}{\ell}    % sch�nes l
\newcommand{\ve}{\varepsilon}  %Epsilon

\newcommand{\BA}{{\mathbb{A}}}
\newcommand{\BB}{{\mathbb{B}}}
\newcommand{\BC}{{\mathbb{C}}}
\newcommand{\BD}{{\mathbb{D}}}
\newcommand{\BE}{{\mathbb{E}}}
\newcommand{\BF}{{\mathbb{F}}}
\newcommand{\BG}{{\mathbb{G}}}
\newcommand{\BH}{{\mathbb{H}}}
\newcommand{\BI}{{\mathbb{I}}}
\newcommand{\BJ}{{\mathbb{J}}}
\newcommand{\BK}{{\mathbb{K}}}
\newcommand{\BL}{{\mathbb{L}}}
\newcommand{\BM}{{\mathbb{M}}}
\newcommand{\BN}{{\mathbb{N}}}
\newcommand{\BO}{{\mathbb{O}}}
\newcommand{\BP}{{\mathbb{P}}}
\newcommand{\BQ}{{\mathbb{Q}}}
\newcommand{\BR}{{\mathbb{R}}}
\newcommand{\BS}{{\mathbb{S}}}
\newcommand{\BT}{{\mathbb{T}}}
\newcommand{\BU}{{\mathbb{U}}}
\newcommand{\BV}{{\mathbb{V}}}
\newcommand{\BW}{{\mathbb{W}}}
\newcommand{\BX}{{\mathbb{X}}}
\newcommand{\BY}{{\mathbb{Y}}}
\newcommand{\BZ}{{\mathbb{Z}}}

\newcommand{\Fa}{{\mathfrak{a}}}
\newcommand{\Fb}{{\mathfrak{b}}}
\newcommand{\Fc}{{\mathfrak{c}}}
\newcommand{\Fd}{{\mathfrak{d}}}
\newcommand{\Fe}{{\mathfrak{e}}}
\newcommand{\Ff}{{\mathfrak{f}}}
\newcommand{\Fg}{{\mathfrak{g}}}
\newcommand{\Fh}{{\mathfrak{h}}}
\newcommand{\Fi}{{\mathfrak{i}}}
\newcommand{\Fj}{{\mathfrak{j}}}
\newcommand{\Fk}{{\mathfrak{k}}}
\newcommand{\Fl}{{\mathfrak{l}}}
\newcommand{\Fm}{{\mathfrak{m}}}
\newcommand{\Fn}{{\mathfrak{n}}}
\newcommand{\Fo}{{\mathfrak{o}}}
\newcommand{\Fp}{{\mathfrak{p}}}
\newcommand{\Fq}{{\mathfrak{q}}}
\newcommand{\Fr}{{\mathfrak{r}}}
\newcommand{\Fs}{{\mathfrak{s}}}
\newcommand{\Ft}{{\mathfrak{t}}}
\newcommand{\Fu}{{\mathfrak{u}}}
\newcommand{\Fv}{{\mathfrak{v}}}
\newcommand{\Fw}{{\mathfrak{w}}}
\newcommand{\Fx}{{\mathfrak{x}}}
\newcommand{\Fy}{{\mathfrak{y}}}
\newcommand{\Fz}{{\mathfrak{z}}}

\newcommand{\FA}{{\mathfrak{A}}}
\newcommand{\FB}{{\mathfrak{B}}}
\newcommand{\FC}{{\mathfrak{C}}}
\newcommand{\FD}{{\mathfrak{D}}}
\newcommand{\FE}{{\mathfrak{E}}}
\newcommand{\FF}{{\mathfrak{F}}}
\newcommand{\FG}{{\mathfrak{G}}}
\newcommand{\FH}{{\mathfrak{H}}}
\newcommand{\FI}{{\mathfrak{I}}}
\newcommand{\FJ}{{\mathfrak{J}}}
\newcommand{\FK}{{\mathfrak{K}}}
\newcommand{\FL}{{\mathfrak{L}}}
\newcommand{\FM}{{\mathfrak{M}}}
\newcommand{\FN}{{\mathfrak{N}}}
\newcommand{\FO}{{\mathfrak{O}}}
\newcommand{\FP}{{\mathfrak{P}}}
\newcommand{\FQ}{{\mathfrak{Q}}}
\newcommand{\FR}{{\mathfrak{R}}}
\newcommand{\FS}{{\mathfrak{S}}}
\newcommand{\FT}{{\mathfrak{T}}}
\newcommand{\FU}{{\mathfrak{U}}}
\newcommand{\FV}{{\mathfrak{V}}}
\newcommand{\FW}{{\mathfrak{W}}}
\newcommand{\FX}{{\mathfrak{X}}}
\newcommand{\FY}{{\mathfrak{Y}}}
\newcommand{\FZ}{{\mathfrak{Z}}}

\newcommand{\CA}{{\cal A}}
\newcommand{\CB}{{\cal B}}
\newcommand{\CC}{{\cal C}}
\newcommand{\CD}{{\cal D}}
\newcommand{\CE}{{\cal E}}
\newcommand{\CF}{{\cal F}}
\newcommand{\CG}{{\cal G}}
\newcommand{\CH}{{\cal H}}
\newcommand{\CI}{{\cal I}}
\newcommand{\CJ}{{\cal J}}
\newcommand{\CK}{{\cal K}}
\newcommand{\CL}{{\cal L}}
\newcommand{\CM}{{\cal M}}
\newcommand{\CN}{{\cal N}}
\newcommand{\CO}{{\cal O}}
\newcommand{\CP}{{\cal P}}
\newcommand{\CQ}{{\cal Q}}
\newcommand{\CR}{{\cal R}}
\newcommand{\CS}{{\cal S}}
\newcommand{\CT}{{\cal T}}
\newcommand{\CU}{{\cal U}}
\newcommand{\CV}{{\cal V}}
\newcommand{\CW}{{\cal W}}
\newcommand{\CX}{{\cal X}}
\newcommand{\CY}{{\cal Y}}
\newcommand{\CZ}{{\cal Z}}

% Theorem Stil

\theoremstyle{plain}
\newtheorem{lem}{Lemma}
\newtheorem{Satz}[lem]{Satz}

\theoremstyle{definition}
\newtheorem{defn}{Definition}[section]

\theoremstyle{remark}
\newtheorem{bem}{Bemerkung}    %[section]



\newcommand{\card}{\mathop{\rm card}\nolimits}
\newcommand{\Sets}{((Sets))}
\newcommand{\id}{{\rm id}}
\newcommand{\supp}{\mathop{\rm Supp}\nolimits}

\newcommand{\ord}{\mathop{\rm ord}\nolimits}
\renewcommand{\mod}{\mathop{\rm mod}\nolimits}
\newcommand{\sign}{\mathop{\rm sign}\nolimits}
\newcommand{\ggT}{\mathop{\rm ggT}\nolimits}
\newcommand{\kgV}{\mathop{\rm kgV}\nolimits}
\renewcommand{\div}{\, | \,}
\newcommand{\notdiv}{\mathopen{\mathchoice
             {\not{|}\,}
             {\not{|}\,}
             {\!\not{\:|}}
             {\not{|}}
             }}

\newcommand{\im}{\mathop{{\rm Im}}\nolimits}
\newcommand{\coim}{\mathop{{\rm coim}}\nolimits}
\newcommand{\coker}{\mathop{\rm Coker}\nolimits}
\renewcommand{\ker}{\mathop{\rm Ker}\nolimits}

\newcommand{\pRang}{\mathop{p{\rm -Rang}}\nolimits}
\newcommand{\End}{\mathop{\rm End}\nolimits}
\newcommand{\Hom}{\mathop{\rm Hom}\nolimits}
\newcommand{\Isom}{\mathop{\rm Isom}\nolimits}
\newcommand{\Tor}{\mathop{\rm Tor}\nolimits}
\newcommand{\Aut}{\mathop{\rm Aut}\nolimits}

\newcommand{\adj}{\mathop{\rm adj}\nolimits}

\newcommand{\Norm}{\mathop{\rm Norm}\nolimits}
\newcommand{\Gal}{\mathop{\rm Gal}\nolimits}
\newcommand{\Frob}{{\rm Frob}}

\newcommand{\disc}{\mathop{\rm disc}\nolimits}

\renewcommand{\Re}{\mathop{\rm Re}\nolimits}
\renewcommand{\Im}{\mathop{\rm Im}\nolimits}

\newcommand{\Log}{\mathop{\rm Log}\nolimits}
\newcommand{\Res}{\mathop{\rm Res}\nolimits}
\newcommand{\Bild}{\mathop{\rm Bild}\nolimits}

\renewcommand{\binom}[2]{\left({#1}\atop{#2}\right)}
\newcommand{\eck}[1]{\langle #1 \rangle}
\newcommand{\wi}{\hspace{1pt} < \hspace{-6pt} ) \hspace{2pt}}


\begin{document}

\pagestyle{empty}

\begin{center}
{\huge SMO Finalrunde 2005} \\
\medskip erste Pr�fung - 24. M�rz 2005
\end{center}
\vspace{8mm}
Zeit: 4 Stunden\\
Jede Aufgabe ist 7 Punkte wert.

\vspace{15mm}

\begin{enumerate}
\item[\textbf{1.}] Sei $ABC$ ein Dreieck und seien $D, E, F$ die Seitenmitten von $BC, CA, AB$. Die Schwerlinien $AD$, $BE$ und $CF$ schneiden sich im
Schwerpunkt $S$. Mindestens zwei der Vierecke
\[AFSE,\quad BDSF,\quad CESD\]
seien Sehnenvierecke. Zeige, dass das Dreieck $ABC$ gleichseitig ist.

\bigskip

\item[\textbf{2.}] Von $4n$ Punkten in einer Reihe sind $2n$ weiss und $2n$ schwarz gef�rbt. Zeige, dass es $2n$
aufeinanderfolgende Punkte gibt, von denen genau $n$ weiss und $n$ schwarz sind.

\bigskip

\item[\textbf{3.}] Beweise f�r alle $a_1,\ldots,a_n>0$ die folgende Ungleichung und bestimme alle F�lle, in
denen das Gleichheitszeichen steht:
\[\sum_{k=1}^n ka_k \leq \binom{n}{2} + \sum_{k=1}^n a_k^k.\]

\bigskip

\item[\textbf{4.}] Bestimme alle Mengen $M$ nat�rlicher Zahlen, sodass f�r je zwei (nicht notwendigweise verschiedene) Elemente $a,b$ aus $M$ auch
\[\frac{a+b}{\ggT(a,b)}\]
in $M$ liegt.

\bigskip

\item[\textbf{5.}] Ein konvexes $n$-Eck zu \textit{zwacken} bedeutet Folgendes: Man w�hlt zwei benachbarte Seiten $AB$ und $BC$ aus und ersetzt diese durch den Streckenzug $AM, MN, NC$, wobei $M \in AB$ und $N \in BC$ beliebige Punkte im Innern dieser Strecken sind. Mit anderen Worten, man schneidet eine Ecke ab und erh�lt ein $(n\!+\!1)$-Eck.\\
Ausgehend von einem regul�ren Sechseck $\CP_6$ mit Fl�cheninhalt 1 wird durch fortlaufendes Zwacken eine Folge $\CP_6, \CP_7, \CP_8,\ldots$ konvexer Polygone erzeugt. Zeige, dass der Fl�cheninhalt von $\CP_n$ f�r alle $n\geq 6$ gr�sser als $\frac{1}{2}$ ist, unabh�ngig davon wie gezwackt wird.

\end{enumerate}

\pagebreak


\begin{center}
{\huge SMO Finalrunde 2005} \\
\medskip zweite Pr�fung - 25. M�rz 2005
\end{center}
\vspace{8mm}
Zeit: 4 Stunden\\
Jede Aufgabe ist 7 Punkte wert.

\vspace{15mm}

\begin{enumerate}
\item[\textbf{6.}] Seien $a,b,c$ positive reelle Zahlen mit $abc=1$. Bestimme alle m�glichen Werte, die der
Ausdruck
\[\frac{1+a}{1+a+ab}+\frac{1+b}{1+b+bc}+\frac{1+c}{1+c+ca}\]
annehmen kann.

\bigskip

\item[\textbf{7.}] Sei $n \geq 1$ eine nat�rliche Zahl. Bestimme alle positiven ganzzahligen L�sungen der
Gleichung
\[7 \cdot 4^n=a^2+b^2+c^2+d^2.\]

\bigskip

\item[\textbf{8.}] Sei $ABC$ ein spitzwinkliges Dreieck. $M$ und $N$ seien zwei beliebige Punkte auf den
Seiten $AB$ respektive $AC$. Die Kreise mit den Durchmessern $BN$ und $CM$ schneiden sich in den Punkten $P$ und $Q$.
Zeige, dass die Punkte $P$, $Q$ und der H�henschnittpunkt des Dreiecks $ABC$ auf einer Geraden liegen.

\bigskip

\item[\textbf{9.}] Finde alle Funktionen $f:\BR^+ \rightarrow \BR^+$, sodass f�r alle $x,y>0$ gilt
\[f(yf(x))(x+y)=x^2(f(x)+f(y)).\]

\bigskip

\item[\textbf{10.}] An einem Fussballturnier nehmen $n>10$ Mannschaften teil. Dabei spielt jede Mannschaft
genau einmal gegen jede andere. Ein Sieg gibt zwei Punkte, ein Unentschieden einen Punkt, und eine Niederlage
keinen Punkt. Nach dem Turnier stellt sich heraus, dass jede Mannschaft genau die H�lfte ihrer Punkte in den
Spielen gegen die 10 schlechtesten Mannschaften gewonnen hat (insbesondere hat jede dieser 10 Mannschaften die
H�lfte ihrer Punke gegen die 9 �brigen gemacht). Bestimme alle m�glichen Werte von $n$, und gib
f�r diese Werte ein Beispiel eines solchen Turniers an.


\end{enumerate}

\end{document}
