\documentclass[12pt,a4paper]{article}

\usepackage{amsfonts}
\usepackage[centertags]{amsmath}
\usepackage{german}
\usepackage{amsthm}
\usepackage{amssymb}

\leftmargin=0pt \topmargin=0pt \headheight=0in \headsep=0in \oddsidemargin=0pt \textwidth=6.5in
\textheight=8.5in

\catcode`\� = \active \catcode`\� = \active \catcode`\� = \active \catcode`\� = \active \catcode`\� = \active
\catcode`\� = \active

\def�{"A}
\def�{"a}
\def�{"O}
\def�{"o}
\def�{"U}
\def�{"u}







% Schriftabk�rzungen

\newcommand{\eps}{\varepsilon}
\renewcommand{\phi}{\varphi}
\newcommand{\Sl}{\ell}    % sch�nes l
\newcommand{\ve}{\varepsilon}  %Epsilon

\newcommand{\BA}{{\mathbb{A}}}
\newcommand{\BB}{{\mathbb{B}}}
\newcommand{\BC}{{\mathbb{C}}}
\newcommand{\BD}{{\mathbb{D}}}
\newcommand{\BE}{{\mathbb{E}}}
\newcommand{\BF}{{\mathbb{F}}}
\newcommand{\BG}{{\mathbb{G}}}
\newcommand{\BH}{{\mathbb{H}}}
\newcommand{\BI}{{\mathbb{I}}}
\newcommand{\BJ}{{\mathbb{J}}}
\newcommand{\BK}{{\mathbb{K}}}
\newcommand{\BL}{{\mathbb{L}}}
\newcommand{\BM}{{\mathbb{M}}}
\newcommand{\BN}{{\mathbb{N}}}
\newcommand{\BO}{{\mathbb{O}}}
\newcommand{\BP}{{\mathbb{P}}}
\newcommand{\BQ}{{\mathbb{Q}}}
\newcommand{\BR}{{\mathbb{R}}}
\newcommand{\BS}{{\mathbb{S}}}
\newcommand{\BT}{{\mathbb{T}}}
\newcommand{\BU}{{\mathbb{U}}}
\newcommand{\BV}{{\mathbb{V}}}
\newcommand{\BW}{{\mathbb{W}}}
\newcommand{\BX}{{\mathbb{X}}}
\newcommand{\BY}{{\mathbb{Y}}}
\newcommand{\BZ}{{\mathbb{Z}}}

\newcommand{\Fa}{{\mathfrak{a}}}
\newcommand{\Fb}{{\mathfrak{b}}}
\newcommand{\Fc}{{\mathfrak{c}}}
\newcommand{\Fd}{{\mathfrak{d}}}
\newcommand{\Fe}{{\mathfrak{e}}}
\newcommand{\Ff}{{\mathfrak{f}}}
\newcommand{\Fg}{{\mathfrak{g}}}
\newcommand{\Fh}{{\mathfrak{h}}}
\newcommand{\Fi}{{\mathfrak{i}}}
\newcommand{\Fj}{{\mathfrak{j}}}
\newcommand{\Fk}{{\mathfrak{k}}}
\newcommand{\Fl}{{\mathfrak{l}}}
\newcommand{\Fm}{{\mathfrak{m}}}
\newcommand{\Fn}{{\mathfrak{n}}}
\newcommand{\Fo}{{\mathfrak{o}}}
\newcommand{\Fp}{{\mathfrak{p}}}
\newcommand{\Fq}{{\mathfrak{q}}}
\newcommand{\Fr}{{\mathfrak{r}}}
\newcommand{\Fs}{{\mathfrak{s}}}
\newcommand{\Ft}{{\mathfrak{t}}}
\newcommand{\Fu}{{\mathfrak{u}}}
\newcommand{\Fv}{{\mathfrak{v}}}
\newcommand{\Fw}{{\mathfrak{w}}}
\newcommand{\Fx}{{\mathfrak{x}}}
\newcommand{\Fy}{{\mathfrak{y}}}
\newcommand{\Fz}{{\mathfrak{z}}}

\newcommand{\FA}{{\mathfrak{A}}}
\newcommand{\FB}{{\mathfrak{B}}}
\newcommand{\FC}{{\mathfrak{C}}}
\newcommand{\FD}{{\mathfrak{D}}}
\newcommand{\FE}{{\mathfrak{E}}}
\newcommand{\FF}{{\mathfrak{F}}}
\newcommand{\FG}{{\mathfrak{G}}}
\newcommand{\FH}{{\mathfrak{H}}}
\newcommand{\FI}{{\mathfrak{I}}}
\newcommand{\FJ}{{\mathfrak{J}}}
\newcommand{\FK}{{\mathfrak{K}}}
\newcommand{\FL}{{\mathfrak{L}}}
\newcommand{\FM}{{\mathfrak{M}}}
\newcommand{\FN}{{\mathfrak{N}}}
\newcommand{\FO}{{\mathfrak{O}}}
\newcommand{\FP}{{\mathfrak{P}}}
\newcommand{\FQ}{{\mathfrak{Q}}}
\newcommand{\FR}{{\mathfrak{R}}}
\newcommand{\FS}{{\mathfrak{S}}}
\newcommand{\FT}{{\mathfrak{T}}}
\newcommand{\FU}{{\mathfrak{U}}}
\newcommand{\FV}{{\mathfrak{V}}}
\newcommand{\FW}{{\mathfrak{W}}}
\newcommand{\FX}{{\mathfrak{X}}}
\newcommand{\FY}{{\mathfrak{Y}}}
\newcommand{\FZ}{{\mathfrak{Z}}}

\newcommand{\CA}{{\cal A}}
\newcommand{\CB}{{\cal B}}
\newcommand{\CC}{{\cal C}}
\newcommand{\CD}{{\cal D}}
\newcommand{\CE}{{\cal E}}
\newcommand{\CF}{{\cal F}}
\newcommand{\CG}{{\cal G}}
\newcommand{\CH}{{\cal H}}
\newcommand{\CI}{{\cal I}}
\newcommand{\CJ}{{\cal J}}
\newcommand{\CK}{{\cal K}}
\newcommand{\CL}{{\cal L}}
\newcommand{\CM}{{\cal M}}
\newcommand{\CN}{{\cal N}}
\newcommand{\CO}{{\cal O}}
\newcommand{\CP}{{\cal P}}
\newcommand{\CQ}{{\cal Q}}
\newcommand{\CR}{{\cal R}}
\newcommand{\CS}{{\cal S}}
\newcommand{\CT}{{\cal T}}
\newcommand{\CU}{{\cal U}}
\newcommand{\CV}{{\cal V}}
\newcommand{\CW}{{\cal W}}
\newcommand{\CX}{{\cal X}}
\newcommand{\CY}{{\cal Y}}
\newcommand{\CZ}{{\cal Z}}

% Theorem Stil

\theoremstyle{plain}
\newtheorem{lem}{Lemma}
\newtheorem{Satz}[lem]{Satz}

\theoremstyle{definition}
\newtheorem{defn}{Definition}[section]

\theoremstyle{remark}
\newtheorem{bem}{Bemerkung}    %[section]



\newcommand{\card}{\mathop{\rm card}\nolimits}
\newcommand{\Sets}{((Sets))}
\newcommand{\id}{{\rm id}}
\newcommand{\supp}{\mathop{\rm Supp}\nolimits}

\newcommand{\ord}{\mathop{\rm ord}\nolimits}
\renewcommand{\mod}{\mathop{\rm mod}\nolimits}
\newcommand{\sign}{\mathop{\rm sign}\nolimits}
\newcommand{\ggT}{\mathop{\rm ggT}\nolimits}
\newcommand{\kgV}{\mathop{\rm kgV}\nolimits}
\renewcommand{\div}{\, | \,}
\newcommand{\notdiv}{\mathopen{\mathchoice
             {\not{|}\,}
             {\not{|}\,}
             {\!\not{\:|}}
             {\not{|}}
             }}

\newcommand{\im}{\mathop{{\rm Im}}\nolimits}
\newcommand{\coim}{\mathop{{\rm coim}}\nolimits}
\newcommand{\coker}{\mathop{\rm Coker}\nolimits}
\renewcommand{\ker}{\mathop{\rm Ker}\nolimits}

\newcommand{\pRang}{\mathop{p{\rm -Rang}}\nolimits}
\newcommand{\End}{\mathop{\rm End}\nolimits}
\newcommand{\Hom}{\mathop{\rm Hom}\nolimits}
\newcommand{\Isom}{\mathop{\rm Isom}\nolimits}
\newcommand{\Tor}{\mathop{\rm Tor}\nolimits}
\newcommand{\Aut}{\mathop{\rm Aut}\nolimits}

\newcommand{\adj}{\mathop{\rm adj}\nolimits}

\newcommand{\Norm}{\mathop{\rm Norm}\nolimits}
\newcommand{\Gal}{\mathop{\rm Gal}\nolimits}
\newcommand{\Frob}{{\rm Frob}}

\newcommand{\disc}{\mathop{\rm disc}\nolimits}

\renewcommand{\Re}{\mathop{\rm Re}\nolimits}
\renewcommand{\Im}{\mathop{\rm Im}\nolimits}

\newcommand{\Log}{\mathop{\rm Log}\nolimits}
\newcommand{\Res}{\mathop{\rm Res}\nolimits}
\newcommand{\Bild}{\mathop{\rm Bild}\nolimits}

\renewcommand{\binom}[2]{\left({#1}\atop{#2}\right)}
\newcommand{\eck}[1]{\langle #1 \rangle}
\newcommand{\gaussk}[1]{\lfloor #1 \rfloor}
\newcommand{\frack}[1]{\{ #1 \}}
\newcommand{\wi}{\hspace{1pt} < \hspace{-6pt} ) \hspace{2pt}}
\newcommand{\dreieck}{\bigtriangleup}

\begin{document}

\pagestyle{empty}

\begin{center}
{\huge IMO Selektion 2005} \\
\medskip erste Pr�fung - 7. Mai 2005
\end{center}
\vspace{8mm}
Zeit: 4.5 Stunden\\
Jede Aufgabe ist 7 Punkte wert.

\vspace{15mm}

\begin{enumerate}
\item[\textbf{1.}] Die beiden Folgen $a_1 > a_2 >\ldots >a_n$ und $b_1 <b_2<\ldots <b_n$ enthalten zusammen jede
der Zahlen $1,2,\ldots ,2n$ genau einmal. Bestimme den Wert der Summe
\[|a_1-b_1|+|a_2-b_2|+\ldots+|a_n-b_n|.\]

\bigskip
\bigskip

\item[\textbf{2.}] Finde den gr�sstm�glichen Wert des Ausdrucks
\[\frac{xyz}{(1+x)(x+y)(y+z)(z+16)},\]
wobei $x,y,z$ positive reelle Zahlen sind.

\bigskip
\bigskip

\item[\textbf{3.}] Sei $n \geq 1$ eine nat�rliche Zahl. Ein regul�res $4n$-Eck der Seitenl�nge 1 sei
irgendwie in endlich viele Parallelogramme zerlegt.
\begin{enumerate}
\item[(a)] Beweise, dass mindestens eines der Parallelogramme in der Zerlegung ein Rechteck ist.
\item[(b)] Bestimme die Summe der Fl�chen aller Rechtecke in der Zerlegung.
\end{enumerate}

\end{enumerate}
\vspace{20mm}

\pagebreak


\begin{center}
{\huge IMO Selektion 2005} \\
\medskip zweite Pr�fung - 8. Mai 2005
\end{center}
\vspace{8mm}
Zeit: 4.5 Stunden \\
Jede Aufgabe ist 7 Punkte wert.

\vspace{15mm}

\begin{enumerate}

\item[\textbf{4.}] Seien $k_1$ und $k_2$ zwei Kreise, die sich im Punkt $P$ �usserlich ber�hren. Ein dritter Kreis $k$ ber�hre $k_1$ in $B$ und $k_2$ in $C$, so dass $k_1$ und $k_2$ im Innern von $k$ liegen. Sei $A$ einer der Schnittpunkte von $k$ mit der gemeinsamen Tangente von $k_1$ und $k_2$ durch $P$. Die Geraden $AB$ und $AC$ schneiden $k_1$ bzw. $k_2$ nochmals in $R$ bzw. $S$. Zeige, dass $RS$ eine gemeinsame Tangente von $k_1$ und $k_2$ ist.


\bigskip
\bigskip

\item[\textbf{5.}] Sei $p>3$ eine Primzahl. Zeige, dass $p^2$ ein Teiler ist von
\[\sum_{k=1}^{p-1} k^{2p+1}.\]

\bigskip
\bigskip

\item[\textbf{6.}] Sei $T$ die Menge aller Tripel $(p,q,r)$ von nichtnegativen ganzen Zahlen. Bestimme alle
Funktionen $f:T \rightarrow \BR$ f�r die gilt
\[f(p,q,r)= \left\{ \parbox{9.5cm}{$\;\;0$\\[3mm] 
\begin{tabular}{r@{}l}
$1+\frac{1}{6}\{$ & $\phantom{+}f(p+1,q-1,r)+f(p-1,q+1,r)$\\
& $+f(p-1,q,r+1)+f(p+1,q,r-1)$\\
& $+f(p,q+1,r-1)+f(p,q-1,r+1)\}$
\end{tabular}
}\parbox{3cm}{f�r $pqr=0$,\\[3mm] $\mbox{}$\\ sonst.\\ $\mbox{}$} \right.\]
\end{enumerate}

\pagebreak


\begin{center}
{\huge IMO Selektion 2005} \\
\medskip dritte Pr�fung - 14. Mai 2005
\end{center}
\vspace{8mm}
Zeit: 4.5 Stunden\\
Jede Aufgabe ist 7 Punkte wert.

\vspace{15mm}

\begin{enumerate}
\item[\textbf{7.}] Sei $n\geq 2$ eine nat�rliche Zahl. Zeige, dass sich das Polynom
\[(x^2-1^2)(x^2-2^2)(x^2-3^2)\cdots (x^2-n^2)+1\]
nicht als Produkt von zwei nichtkonstanten Polynomen mit ganzen Koeffizienten schreiben l�sst.

\bigskip
\bigskip

\item[\textbf{8.}] Betrachte einen See mit zwei Inseln darin und sieben St�dten am Ufer. Die Inseln und St�dte
nennen wir im Folgenden kurz \textit{Orte}. Zwischen genau den folgenden Paaren von Orten besteht eine
Schiffsverbindung:
\begin{enumerate}
\item[(i)] zwischen den beiden Inseln,
\item[(ii)] zwischen jeder Stadt und jeder Insel,
\item[(iii)] zwischen zwei St�dten genau dann, wenn sie nicht benachbart sind.
\end{enumerate}
Jede dieser Verbindungen wird von genau einem von zwei konkurrenzierenden Schiffsunternehmen angeboten. Beweise,
dass es stets drei Orte gibt, sodass zwischen je zwei dieser Orte Schiffsverbindungen desselben Unternehmens
existieren.

\bigskip
\bigskip

\item[\textbf{9.}] Sei $A_1 A_2 \ldots A_n$ ein regul�res $n$-Eck. Die Punkte $B_1,\ldots,B_{n-1}$ sind wie
folgt definiert:
\begin{itemize}
\item F�r $i=1$ oder $i=n-1$ ist $B_i$ der Mittelpunkt der Seite $A_i A_{i+1}$;
\item F�r $i \neq 1, i \neq n-1$ sei $S$ der Schnittpunkt von $A_1 A_{i+1}$ und $A_n A_i$. Der Punkt $B_i$ ist
dann der Schnittpunkt der Winkelhalbierenden von $\wi A_i S A_{i+1}$ mit $A_i A_{i+1}$.
\end{itemize}
Beweise, dass gilt
\[\wi A_1 B_1 A_n + \wi A_1 B_2 A_n+ \ldots + \wi A_1 B_{n-1} A_n = 180^\circ.\]

\end{enumerate}

\pagebreak


\begin{center}
{\huge IMO Selektion 2005} \\
\medskip vierte Pr�fung - 15. Mai 2005
\end{center}
\vspace{8mm}
Zeit: 4.5 Stunden\\
Jede Aufgabe ist 7 Punkte wert.

\vspace{15mm}

\begin{enumerate}
\item[\textbf{10.}] Sei $ABC$ ein spitzwinkliges Dreieck mit H�henschnittpunkt $H$ und seien $M$ und $N$
zwei Punkte auf $BC$, so dass $\overrightarrow{MN} = \overrightarrow{BC}$. Seien $P$ und $Q$ die Projektionen
von $M$ bzw. $N$ auf $AC$ bzw. $AB$. Zeige, dass $APHQ$ ein Sehnenviereck ist.

\bigskip
\bigskip

\item[\textbf{11.}] Finde alle Funktionen $f:\BN \rightarrow \BN$, sodass $f(m)^2+f(n)$ ein Teiler ist von $(m^2+n)^2$
f�r alle $m,n \in \BN$.

\bigskip
\bigskip

\item[\textbf{12.}] Sei $A$ eine $m \times m$-Matrix. Sei $X_i$ die Menge der Eintr�ge in der $i$-ten Zeile und
$Y_j$ die Menge der Eintr�ge in der $j$-ten Spalte, $1 \leq i,j \leq m$. $A$ heisst \textit{cool}, wenn die
Mengen $X_1,\ldots,X_m,Y_1,\ldots,Y_m$ alle verschieden sind. Bestimme den kleinsten Wert f�r $n$, sodass eine
coole $2005 \times 2005$-Matrix mit Eintr�gen aus der Menge $\{1,2,\ldots,n\}$ existiert.

\end{enumerate}

\end{document}
