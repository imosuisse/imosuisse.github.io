\documentclass[12pt,a4paper]{article}

\usepackage{amsfonts}
\usepackage[latin1]{inputenc}
\usepackage[centertags]{amsmath}
\usepackage{german}
\usepackage{amsthm}
\usepackage{amssymb}

\leftmargin=0pt \topmargin=0pt \headheight=0in \headsep=0in \oddsidemargin=0pt \textwidth=6.5in
\textheight=8.5in

\catcode`\� = \active \catcode`\� = \active \catcode`\� = \active \catcode`\� = \active \catcode`\� = \active
\catcode`\� = \active

\def�{"A}
\def�{"a}
\def�{"O}
\def�{"o}
\def�{"U}
\def�{"u}







% Schriftabk�rzungen

\newcommand{\eps}{\varepsilon}
\renewcommand{\phi}{\varphi}
\newcommand{\Sl}{\ell}    % sch�nes l
\newcommand{\ve}{\varepsilon}  %Epsilon

\newcommand{\BA}{{\mathbb{A}}}
\newcommand{\BB}{{\mathbb{B}}}
\newcommand{\BC}{{\mathbb{C}}}
\newcommand{\BD}{{\mathbb{D}}}
\newcommand{\BE}{{\mathbb{E}}}
\newcommand{\BF}{{\mathbb{F}}}
\newcommand{\BG}{{\mathbb{G}}}
\newcommand{\BH}{{\mathbb{H}}}
\newcommand{\BI}{{\mathbb{I}}}
\newcommand{\BJ}{{\mathbb{J}}}
\newcommand{\BK}{{\mathbb{K}}}
\newcommand{\BL}{{\mathbb{L}}}
\newcommand{\BM}{{\mathbb{M}}}
\newcommand{\BN}{{\mathbb{N}}}
\newcommand{\BO}{{\mathbb{O}}}
\newcommand{\BP}{{\mathbb{P}}}
\newcommand{\BQ}{{\mathbb{Q}}}
\newcommand{\BR}{{\mathbb{R}}}
\newcommand{\BS}{{\mathbb{S}}}
\newcommand{\BT}{{\mathbb{T}}}
\newcommand{\BU}{{\mathbb{U}}}
\newcommand{\BV}{{\mathbb{V}}}
\newcommand{\BW}{{\mathbb{W}}}
\newcommand{\BX}{{\mathbb{X}}}
\newcommand{\BY}{{\mathbb{Y}}}
\newcommand{\BZ}{{\mathbb{Z}}}

\newcommand{\Fa}{{\mathfrak{a}}}
\newcommand{\Fb}{{\mathfrak{b}}}
\newcommand{\Fc}{{\mathfrak{c}}}
\newcommand{\Fd}{{\mathfrak{d}}}
\newcommand{\Fe}{{\mathfrak{e}}}
\newcommand{\Ff}{{\mathfrak{f}}}
\newcommand{\Fg}{{\mathfrak{g}}}
\newcommand{\Fh}{{\mathfrak{h}}}
\newcommand{\Fi}{{\mathfrak{i}}}
\newcommand{\Fj}{{\mathfrak{j}}}
\newcommand{\Fk}{{\mathfrak{k}}}
\newcommand{\Fl}{{\mathfrak{l}}}
\newcommand{\Fm}{{\mathfrak{m}}}
\newcommand{\Fn}{{\mathfrak{n}}}
\newcommand{\Fo}{{\mathfrak{o}}}
\newcommand{\Fp}{{\mathfrak{p}}}
\newcommand{\Fq}{{\mathfrak{q}}}
\newcommand{\Fr}{{\mathfrak{r}}}
\newcommand{\Fs}{{\mathfrak{s}}}
\newcommand{\Ft}{{\mathfrak{t}}}
\newcommand{\Fu}{{\mathfrak{u}}}
\newcommand{\Fv}{{\mathfrak{v}}}
\newcommand{\Fw}{{\mathfrak{w}}}
\newcommand{\Fx}{{\mathfrak{x}}}
\newcommand{\Fy}{{\mathfrak{y}}}
\newcommand{\Fz}{{\mathfrak{z}}}

\newcommand{\FA}{{\mathfrak{A}}}
\newcommand{\FB}{{\mathfrak{B}}}
\newcommand{\FC}{{\mathfrak{C}}}
\newcommand{\FD}{{\mathfrak{D}}}
\newcommand{\FE}{{\mathfrak{E}}}
\newcommand{\FF}{{\mathfrak{F}}}
\newcommand{\FG}{{\mathfrak{G}}}
\newcommand{\FH}{{\mathfrak{H}}}
\newcommand{\FI}{{\mathfrak{I}}}
\newcommand{\FJ}{{\mathfrak{J}}}
\newcommand{\FK}{{\mathfrak{K}}}
\newcommand{\FL}{{\mathfrak{L}}}
\newcommand{\FM}{{\mathfrak{M}}}
\newcommand{\FN}{{\mathfrak{N}}}
\newcommand{\FO}{{\mathfrak{O}}}
\newcommand{\FP}{{\mathfrak{P}}}
\newcommand{\FQ}{{\mathfrak{Q}}}
\newcommand{\FR}{{\mathfrak{R}}}
\newcommand{\FS}{{\mathfrak{S}}}
\newcommand{\FT}{{\mathfrak{T}}}
\newcommand{\FU}{{\mathfrak{U}}}
\newcommand{\FV}{{\mathfrak{V}}}
\newcommand{\FW}{{\mathfrak{W}}}
\newcommand{\FX}{{\mathfrak{X}}}
\newcommand{\FY}{{\mathfrak{Y}}}
\newcommand{\FZ}{{\mathfrak{Z}}}

\newcommand{\CA}{{\cal A}}
\newcommand{\CB}{{\cal B}}
\newcommand{\CC}{{\cal C}}
\newcommand{\CD}{{\cal D}}
\newcommand{\CE}{{\cal E}}
\newcommand{\CF}{{\cal F}}
\newcommand{\CG}{{\cal G}}
\newcommand{\CH}{{\cal H}}
\newcommand{\CI}{{\cal I}}
\newcommand{\CJ}{{\cal J}}
\newcommand{\CK}{{\cal K}}
\newcommand{\CL}{{\cal L}}
\newcommand{\CM}{{\cal M}}
\newcommand{\CN}{{\cal N}}
\newcommand{\CO}{{\cal O}}
\newcommand{\CP}{{\cal P}}
\newcommand{\CQ}{{\cal Q}}
\newcommand{\CR}{{\cal R}}
\newcommand{\CS}{{\cal S}}
\newcommand{\CT}{{\cal T}}
\newcommand{\CU}{{\cal U}}
\newcommand{\CV}{{\cal V}}
\newcommand{\CW}{{\cal W}}
\newcommand{\CX}{{\cal X}}
\newcommand{\CY}{{\cal Y}}
\newcommand{\CZ}{{\cal Z}}

% Theorem Stil

\theoremstyle{plain}
\newtheorem{lem}{Lemma}
\newtheorem{Satz}[lem]{Satz}

\theoremstyle{definition}
\newtheorem{defn}{Definition}[section]

\theoremstyle{remark}
\newtheorem{bem}{Bemerkung}    %[section]



\newcommand{\card}{\mathop{\rm card}\nolimits}
\newcommand{\Sets}{((Sets))}
\newcommand{\id}{{\rm id}}
\newcommand{\supp}{\mathop{\rm Supp}\nolimits}

\newcommand{\ord}{\mathop{\rm ord}\nolimits}
\renewcommand{\mod}{\mathop{\rm mod}\nolimits}
\newcommand{\sign}{\mathop{\rm sign}\nolimits}
\newcommand{\ggT}{\mathop{\rm ggT}\nolimits}
\newcommand{\kgV}{\mathop{\rm kgV}\nolimits}
\renewcommand{\div}{\, | \,}
\newcommand{\notdiv}{\mathopen{\mathchoice
             {\not{|}\,}
             {\not{|}\,}
             {\!\not{\:|}}
             {\not{|}}
             }}

\newcommand{\im}{\mathop{{\rm Im}}\nolimits}
\newcommand{\coim}{\mathop{{\rm coim}}\nolimits}
\newcommand{\coker}{\mathop{\rm Coker}\nolimits}
\renewcommand{\ker}{\mathop{\rm Ker}\nolimits}

\newcommand{\pRang}{\mathop{p{\rm -Rang}}\nolimits}
\newcommand{\End}{\mathop{\rm End}\nolimits}
\newcommand{\Hom}{\mathop{\rm Hom}\nolimits}
\newcommand{\Isom}{\mathop{\rm Isom}\nolimits}
\newcommand{\Tor}{\mathop{\rm Tor}\nolimits}
\newcommand{\Aut}{\mathop{\rm Aut}\nolimits}

\newcommand{\adj}{\mathop{\rm adj}\nolimits}

\newcommand{\Norm}{\mathop{\rm Norm}\nolimits}
\newcommand{\Gal}{\mathop{\rm Gal}\nolimits}
\newcommand{\Frob}{{\rm Frob}}

\newcommand{\disc}{\mathop{\rm disc}\nolimits}

\renewcommand{\Re}{\mathop{\rm Re}\nolimits}
\renewcommand{\Im}{\mathop{\rm Im}\nolimits}

\newcommand{\Log}{\mathop{\rm Log}\nolimits}
\newcommand{\Res}{\mathop{\rm Res}\nolimits}
\newcommand{\Bild}{\mathop{\rm Bild}\nolimits}

\renewcommand{\binom}[2]{\left({#1}\atop{#2}\right)}
\newcommand{\eck}[1]{\langle #1 \rangle}
\newcommand{\wi}{\angle}

\parindent0pt

\begin{document}

\pagestyle{empty}

\begin{center}
{\huge OMS - Tour final 2006} \\
\medskip premier examen - 31 mars 2006
\end{center}
\vspace{8mm}
Dur�e: 4 heures\\
Chaque exercice vaut 7 points.

\vspace{15mm}

\begin{enumerate}
\item[\textbf{1.}] Trouver toutes les fonctions $f: \BR \rightarrow \BR$ telles que pour tout $x,y \in \BR$ on a
\[yf(2x)-xf(2y)=8xy(x^2-y^2).\]

\bigskip

\item[\textbf{2.}] Soit $ABC$ un triangle �quilat�ral et soit $D$ un point � l'int�rieur du c�t� $BC$. Un cercle touche $BC$ en $D$ et coupe les c�t�s $AB$ et $AC$ aux points int�rieurs $M,N$ respectivement $P,Q$.
Prouver qu'on a
\[|BD|+|AM|+|AN| = |CD|+|AP|+|AQ|.\]

\bigskip

\item[\textbf{3.}] Calculer la somme des chiffres de
\[9 \times 99 \times 9999 \times \cdots \times \underbrace{99\ldots 99}_{2^n},\]
o� le nombre de neufs double pour chaque facteur.\\

\bigskip

\item[\textbf{4.}]  $3n$ points coupent un cercle de circonf�rence $6n$ en $n$ arcs de longueurs 1, 2 et 3 respectivement. Montrer qu'il existe toujours deux parmi ces points qui sont diam�tralement oppos�s sur le cercle.\\

\bigskip

\item[\textbf{5.}] Le cercle $k_1$ est � l'int�rieur du cercle $k_2$ et le touche dans le point $A$.
Soient $B$ respectivement $C$ les autres points d'intersection d'une droite passant par $A$ avec $k_1$ respectivement $k_2$.
La tangente � $k_1$ qui passe par $B$ coupe $k_2$ aux points $D$ et $E$. Les tangentes � $k_1$ passant par $C$ touchent $k_1$ aux points $F$ et $G$.
Prouver que $D,E,F$ et $G$ sont sur un cercle.\\

\end{enumerate}

\vspace{1.5cm}

\begin{center}
Bonne chance!
\end{center}

\pagebreak


\begin{center}
{\huge OMS - Tour final 2006} \\
\medskip deuxi�me examen - 1er avril 2006
\end{center}
\vspace{8mm}

Dur�e: 4 heures \\
Chaque exercice vaut 7 points.

\vspace{15mm}

\begin{enumerate}
\item[\textbf{6.}] Au moins trois joueurs ont particip� � un tournoi de tennis. Chaque joueur a jou� exactement une fois contre tous les autres et chaque joueur a gagn� au moins un match. Montrer qu'il existe trois joueurs $A,B,C$ tels que $A$ a gagn� contre $B$, $B$ a gagn� contre $C$ et $C$ a gagn� contre $A$.\\

\bigskip

\item[\textbf{7.}] Soit $ABCD$ un quadrilat�re inscrit avec $\wi ABC = 60^\circ$. 
Supposons $|BC|=|CD|$. Prouver qu'on a
\[|CD|+|DA| = |AB|.\]

\bigskip

\item[\textbf{8.}] Des gens venant de $n$ pays diff�rents sont assis autour d'une table ronde tels que pour deux personnes du m�me pays leurs voisins de droite viennent de deux pays diff�rents. Quel est le nombre maximal de personnes qui peuvent s'asseoir � la table?\\

\bigskip

\item[\textbf{9.}] Soient $a,b,c,d$ des nombres r�els. Prouver qu'on a
\[(a^2+b^2+1)(c^2+d^2+1)\geq 2(a+c)(b+d).\]

\bigskip

\item[\textbf{10.}] D�cider s'il existe un entier $n>1$ avec les propri�t�s suivantes:
\begin{enumerate}
\item[(a)] $n$ n'est pas premier.
\item[(b)] Pour tout entier $a$, $a^n-a$ est divisible par $n$.
\end{enumerate}
\vspace{1.5cm}

\center{Bonne chance!}

\end{enumerate}
\end{document}
