\documentclass[12pt,a4paper]{article}

\usepackage{amsfonts}
\usepackage[centertags]{amsmath}
\usepackage{german}
\usepackage{amsthm}
\usepackage{amssymb}

\leftmargin=0pt \topmargin=0pt \headheight=0in \headsep=0in \oddsidemargin=0pt \textwidth=6.5in
\textheight=8.5in

\catcode`\� = \active \catcode`\� = \active \catcode`\� = \active \catcode`\� = \active \catcode`\� = \active
\catcode`\� = \active

\def�{"A}
\def�{"a}
\def�{"O}
\def�{"o}
\def�{"U}
\def�{"u}







% Schriftabk�rzungen

\newcommand{\eps}{\varepsilon}
\renewcommand{\phi}{\varphi}
\newcommand{\Sl}{\ell}    % sch�nes l
\newcommand{\ve}{\varepsilon}  %Epsilon

\newcommand{\BA}{{\mathbb{A}}}
\newcommand{\BB}{{\mathbb{B}}}
\newcommand{\BC}{{\mathbb{C}}}
\newcommand{\BD}{{\mathbb{D}}}
\newcommand{\BE}{{\mathbb{E}}}
\newcommand{\BF}{{\mathbb{F}}}
\newcommand{\BG}{{\mathbb{G}}}
\newcommand{\BH}{{\mathbb{H}}}
\newcommand{\BI}{{\mathbb{I}}}
\newcommand{\BJ}{{\mathbb{J}}}
\newcommand{\BK}{{\mathbb{K}}}
\newcommand{\BL}{{\mathbb{L}}}
\newcommand{\BM}{{\mathbb{M}}}
\newcommand{\BN}{{\mathbb{N}}}
\newcommand{\BO}{{\mathbb{O}}}
\newcommand{\BP}{{\mathbb{P}}}
\newcommand{\BQ}{{\mathbb{Q}}}
\newcommand{\BR}{{\mathbb{R}}}
\newcommand{\BS}{{\mathbb{S}}}
\newcommand{\BT}{{\mathbb{T}}}
\newcommand{\BU}{{\mathbb{U}}}
\newcommand{\BV}{{\mathbb{V}}}
\newcommand{\BW}{{\mathbb{W}}}
\newcommand{\BX}{{\mathbb{X}}}
\newcommand{\BY}{{\mathbb{Y}}}
\newcommand{\BZ}{{\mathbb{Z}}}

\newcommand{\Fa}{{\mathfrak{a}}}
\newcommand{\Fb}{{\mathfrak{b}}}
\newcommand{\Fc}{{\mathfrak{c}}}
\newcommand{\Fd}{{\mathfrak{d}}}
\newcommand{\Fe}{{\mathfrak{e}}}
\newcommand{\Ff}{{\mathfrak{f}}}
\newcommand{\Fg}{{\mathfrak{g}}}
\newcommand{\Fh}{{\mathfrak{h}}}
\newcommand{\Fi}{{\mathfrak{i}}}
\newcommand{\Fj}{{\mathfrak{j}}}
\newcommand{\Fk}{{\mathfrak{k}}}
\newcommand{\Fl}{{\mathfrak{l}}}
\newcommand{\Fm}{{\mathfrak{m}}}
\newcommand{\Fn}{{\mathfrak{n}}}
\newcommand{\Fo}{{\mathfrak{o}}}
\newcommand{\Fp}{{\mathfrak{p}}}
\newcommand{\Fq}{{\mathfrak{q}}}
\newcommand{\Fr}{{\mathfrak{r}}}
\newcommand{\Fs}{{\mathfrak{s}}}
\newcommand{\Ft}{{\mathfrak{t}}}
\newcommand{\Fu}{{\mathfrak{u}}}
\newcommand{\Fv}{{\mathfrak{v}}}
\newcommand{\Fw}{{\mathfrak{w}}}
\newcommand{\Fx}{{\mathfrak{x}}}
\newcommand{\Fy}{{\mathfrak{y}}}
\newcommand{\Fz}{{\mathfrak{z}}}

\newcommand{\FA}{{\mathfrak{A}}}
\newcommand{\FB}{{\mathfrak{B}}}
\newcommand{\FC}{{\mathfrak{C}}}
\newcommand{\FD}{{\mathfrak{D}}}
\newcommand{\FE}{{\mathfrak{E}}}
\newcommand{\FF}{{\mathfrak{F}}}
\newcommand{\FG}{{\mathfrak{G}}}
\newcommand{\FH}{{\mathfrak{H}}}
\newcommand{\FI}{{\mathfrak{I}}}
\newcommand{\FJ}{{\mathfrak{J}}}
\newcommand{\FK}{{\mathfrak{K}}}
\newcommand{\FL}{{\mathfrak{L}}}
\newcommand{\FM}{{\mathfrak{M}}}
\newcommand{\FN}{{\mathfrak{N}}}
\newcommand{\FO}{{\mathfrak{O}}}
\newcommand{\FP}{{\mathfrak{P}}}
\newcommand{\FQ}{{\mathfrak{Q}}}
\newcommand{\FR}{{\mathfrak{R}}}
\newcommand{\FS}{{\mathfrak{S}}}
\newcommand{\FT}{{\mathfrak{T}}}
\newcommand{\FU}{{\mathfrak{U}}}
\newcommand{\FV}{{\mathfrak{V}}}
\newcommand{\FW}{{\mathfrak{W}}}
\newcommand{\FX}{{\mathfrak{X}}}
\newcommand{\FY}{{\mathfrak{Y}}}
\newcommand{\FZ}{{\mathfrak{Z}}}

\newcommand{\CA}{{\cal A}}
\newcommand{\CB}{{\cal B}}
\newcommand{\CC}{{\cal C}}
\newcommand{\CD}{{\cal D}}
\newcommand{\CE}{{\cal E}}
\newcommand{\CF}{{\cal F}}
\newcommand{\CG}{{\cal G}}
\newcommand{\CH}{{\cal H}}
\newcommand{\CI}{{\cal I}}
\newcommand{\CJ}{{\cal J}}
\newcommand{\CK}{{\cal K}}
\newcommand{\CL}{{\cal L}}
\newcommand{\CM}{{\cal M}}
\newcommand{\CN}{{\cal N}}
\newcommand{\CO}{{\cal O}}
\newcommand{\CP}{{\cal P}}
\newcommand{\CQ}{{\cal Q}}
\newcommand{\CR}{{\cal R}}
\newcommand{\CS}{{\cal S}}
\newcommand{\CT}{{\cal T}}
\newcommand{\CU}{{\cal U}}
\newcommand{\CV}{{\cal V}}
\newcommand{\CW}{{\cal W}}
\newcommand{\CX}{{\cal X}}
\newcommand{\CY}{{\cal Y}}
\newcommand{\CZ}{{\cal Z}}

% Theorem Stil

\theoremstyle{plain}
\newtheorem{lem}{Lemma}
\newtheorem{Satz}[lem]{Satz}

\theoremstyle{definition}
\newtheorem{defn}{Definition}[section]

\theoremstyle{remark}
\newtheorem{bem}{Bemerkung}    %[section]



\newcommand{\card}{\mathop{\rm card}\nolimits}
\newcommand{\Sets}{((Sets))}
\newcommand{\id}{{\rm id}}
\newcommand{\supp}{\mathop{\rm Supp}\nolimits}

\newcommand{\ord}{\mathop{\rm ord}\nolimits}
\renewcommand{\mod}{\mathop{\rm mod}\nolimits}
\newcommand{\sign}{\mathop{\rm sign}\nolimits}
\newcommand{\ggT}{\mathop{\rm ggT}\nolimits}
\newcommand{\kgV}{\mathop{\rm kgV}\nolimits}
\renewcommand{\div}{\, | \,}
\newcommand{\notdiv}{\mathopen{\mathchoice
             {\not{|}\,}
             {\not{|}\,}
             {\!\not{\:|}}
             {\not{|}}
             }}

\newcommand{\im}{\mathop{{\rm Im}}\nolimits}
\newcommand{\coim}{\mathop{{\rm coim}}\nolimits}
\newcommand{\coker}{\mathop{\rm Coker}\nolimits}
\renewcommand{\ker}{\mathop{\rm Ker}\nolimits}

\newcommand{\pRang}{\mathop{p{\rm -Rang}}\nolimits}
\newcommand{\End}{\mathop{\rm End}\nolimits}
\newcommand{\Hom}{\mathop{\rm Hom}\nolimits}
\newcommand{\Isom}{\mathop{\rm Isom}\nolimits}
\newcommand{\Tor}{\mathop{\rm Tor}\nolimits}
\newcommand{\Aut}{\mathop{\rm Aut}\nolimits}

\newcommand{\adj}{\mathop{\rm adj}\nolimits}

\newcommand{\Norm}{\mathop{\rm Norm}\nolimits}
\newcommand{\Gal}{\mathop{\rm Gal}\nolimits}
\newcommand{\Frob}{{\rm Frob}}

\newcommand{\disc}{\mathop{\rm disc}\nolimits}

\renewcommand{\Re}{\mathop{\rm Re}\nolimits}
\renewcommand{\Im}{\mathop{\rm Im}\nolimits}

\newcommand{\Log}{\mathop{\rm Log}\nolimits}
\newcommand{\Res}{\mathop{\rm Res}\nolimits}
\newcommand{\Bild}{\mathop{\rm Bild}\nolimits}

\renewcommand{\binom}[2]{\left({#1}\atop{#2}\right)}
\newcommand{\eck}[1]{\langle #1 \rangle}
\newcommand{\gaussk}[1]{\lfloor #1 \rfloor}
\newcommand{\frack}[1]{\{ #1 \}}
\newcommand{\wi}{\hspace{1pt} < \hspace{-6pt} ) \hspace{2pt}}
\newcommand{\dreieck}{\bigtriangleup}

\parindent0mm

\begin{document}

\pagestyle{empty}

\begin{center}
{\huge IMO Selektion 2006} \\
\medskip erste Pr�fung - 29. April 2006
\end{center}
\vspace{8mm}
Zeit: 4.5 Stunden\\
Jede Aufgabe ist 7 Punkte wert.

\vspace{15mm}

\begin{enumerate}
\item[\textbf{1.}] Im Dreieck $ABC$ sei $D$ der Mittelpunkt der Seite $BC$ und $E$ die Projektion von $C$ auf $AD$.
Angenommen es gelte $\angle ACE = \angle ABC$. Zeige, dass das Dreieck $ABC$ gleichschenklig oder rechtwinklig
ist.

\bigskip
\bigskip

\item[\textbf{2.}] Sei $n\geq 5$ eine ganze Zahl. Bestimme die gr�sste ganze Zahl $k$, sodass ein Polygon mit $n$
Ecken und genau $k$ inneren $90^\circ$-Winkeln existiert? (Das Polygon muss nicht konvex sein, der Rand darf
sich aber nicht selbst �berschneiden.)

\bigskip
\bigskip

\item[\textbf{3.}] Sei $n$ eine nat�rliche Zahl. Jede der Zahlen $\{1,2,\ldots,n\}$ ist weiss oder schwarz gef�rbt.
Man kann nun wiederholt eine Zahl ausw�hlen und diese, sowie alle zu ihr nicht teilerfremden Zahlen umf�rben.
Anfangs sind alle Zahlen weiss. F�r welche $n$ kann man erreichen, dass irgendwann alle Zahlen schwarz sind?

\end{enumerate}

\pagebreak


\begin{center}
{\huge IMO Selektion 2006} \\
\medskip zweite Pr�fung - 30. April 2006
\end{center}
\vspace{8mm}
Zeit: 4.5 Stunden\\
Jede Aufgabe ist 7 Punkte wert.

\vspace{15mm}

\begin{enumerate}

\item[\textbf{4.}] Die positiven Teiler der nat�rlichen Zahl $n$ seien $1=d_1<d_2< \ldots <d_k=n$. Bestimme alle
$n$, f�r die gilt
\[2n=d_5^2+d_6^2-1.\]

\bigskip
\bigskip

\item[\textbf{5.}] Sei $ABC$ ein Dreieck und $D$ ein Punkt in dessen Inneren. Sei $E$ ein von $D$ verschiedener
Punkt auf der Geraden $AD$. Seien $\omega_1$ und $\omega_2$ die Umkreise der Dreiecke $BDE$ bzw. $CDE$.
$\omega_1$ und $\omega_2$ schneiden die Seite $BC$ in den inneren Punkten $F$ bzw. $G$. Der Schnittpunkt von
$DG$ und $AB$ sei $X$, und der Schnittpunkt von $DF$ und $AC$ sei $Y$. Zeige, dass $XY$ parallel zu $BC$ ist.

\bigskip
\bigskip

\item[\textbf{6.}] Finde alle Funktionen $f: \BR \rightarrow \BR$, sodass f�r alle $x,y \in \BR$ die folgende
Gleichung gilt
\[f(f(x)-y^2)=f(x)^2-2f(x)y^2+f(f(y)).\]

\end{enumerate}


\pagebreak


\begin{center}
{\huge IMO Selektion 2006} \\
\medskip dritte Pr�fung - 13. Mai 2006
\end{center}
\vspace{8mm}
Zeit: 4.5 Stunden\\
Jede Aufgabe ist 7 Punkte wert.

\vspace{15mm}

\begin{enumerate}
\item[\textbf{7.}] Das Polynom $P(x)=x^3-2x^2-x+1$ besitze die drei reellen Nullstellen $a>b>c$.
Finde den Wert des Ausdrucks
\[a^2b+b^2c+c^2a.\]


\bigskip
\bigskip

\item[\textbf{8.}] L�ngs eines Kreises stehen die Zahlen $1,2,\ldots,2006$ in beliebiger Reihenfolge. Es k�nnen nun
wiederholt zwei auf dem Kreis benachbarte Zahlen miteinander vertauscht werden. Nach einer Folge solcher
Vertauschungen steht jede der Zahlen diametral gegen�ber ihrer Anfangsposition. Beweise, dass mindestens einmal
zwei Zahlen mit Summe 2007 vertauscht wurden.

\bigskip
\bigskip

\item[\textbf{9.}] Sei $ABC$ ein spitzwinkliges Dreieck mit $AB \neq AC$ und H�henschnittpunkt $H$. Der Mittelpunkt
der Seite $BC$ sei $M$. Die Punkte $D$ auf $AB$ und $E$ auf $AC$ seien so, dass $AE = AD$ ist und $D,H,E$ auf
einer Geraden liegen. Zeige, dass $HM$ und die gemeinsame Sehne der Umkreise der beiden Dreiecke $ABC$ und $ADE$
rechtwinklig zueinander liegen.

\end{enumerate}


\pagebreak


\begin{center}
{\huge IMO Selektion 2006} \\
\medskip vierte Pr�fung - 14. Mai 2006
\end{center}
\vspace{8mm}
Zeit: 4.5 Stunden\\
Jede Aufgabe ist 7 Punkte wert.

\vspace{15mm}

\begin{enumerate}
\item[\textbf{10.}] Seien $a,b,c$ positive reelle Zahlen mit $\frac{1}{a}+\frac{1}{b}+\frac{1}{c}=1$. Beweise
die Ungleichung
\[\sqrt{ab+c}+\sqrt{bc+a}+\sqrt{ca+b} \geq \sqrt{abc}+\sqrt{a}+\sqrt{b}+\sqrt{c}.\]

\bigskip
\bigskip

\item[\textbf{11.}] Finde alle nat�rlichen Zahlen $k$, sodass $3^k+5^k$ eine Potenz einer nat�rlichen
Zahl mit Exponent $\geq 2$ ist.

\bigskip
\bigskip

\item[\textbf{12.}] Eine Raumstation besteht aus 25 Kammern, und je zwei Kammern sind mit einem Tunnel
verbunden. Es gibt insgesamt 50 Haupttunnel, die in beide Richtungen benutzt werden k�nnen, die restlichen sind
alle Einbahntunnel. Eine Gruppe von vier Kammern heisst \textit{verbunden}, falls man von jeder dieser Kammern
in jede andere gelangen kann, indem man nur die sechs Tunnel verwendet, welche diese Kammern untereinander
verbinden. Bestimme die gr�sstm�gliche Anzahl verbundener Vierergruppen.


\end{enumerate}


\end{document}
